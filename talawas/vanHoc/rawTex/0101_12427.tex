\documentclass[../main.tex]{subfiles}

\begin{document}

\chapter{Báo Tiếng dân nói sai, tôi không hề công kích Thơ Mới}

\begin{metadata}

\begin{flushright}29.2.2008\end{flushright}

Phan Khôi

Nguồn: Dân báo, Sài Gòn, s. 627 (23 Juillet 1941). Lại Nguyên Ân sưu tầm và biên soạn.

\end{metadata}

\begin{multicols}{2}

\textbf{Vài lời của người sưu tầm} 
 
\textit{Khoảng năm 2006, khi viết một tham luận để dự hội thảo về văn học Quốc ngữ Nam Bộ, thông tin vắn tắt kết quả tìm hiểu hoạt động báo chí của Phan Khôi ở Sài Gòn những năm 1920-30, tôi đã thuật theo nguồn tư liệu gia đình rằng từ cuối năm 1937, Phan Khôi lại vào Sài Gòn, nhưng không biết có cộng tác với tờ báo nào tại đây hay không, do đó chưa rõ ông có một thời kỳ thứ ba làm việc với văn chương báo chí Sài Gòn hay không, và tôi đã “ngờ rằng không có” thời kỳ thứ ba ấy! } 
 
\textit{Điều “ngờ rằng” ấy hoá ra lại sai!} 
 
\textit{Cho đến hôm nay, những ngày đầu năm 2008, tôi đã có căn cứ về sự tham dự của Phan Khôi vào báo chí Sài Gòn ở thời gian sau năm 1937, cụ thể là đã tìm thấy dấu hiệu của việc ông viết trên tờ  }Dân báo\textit{ ở Sài Gòn trong năm 1941. Như thế, dù chưa thể nói tường tận về hoạt động báo chí của Phan Khôi kể từ cuối năm 1937 (khi ông bán lại tờ }Sông Hương\textit{ cho nhóm ký giả cộng sản do Phan Đăng Lưu đại diện và từ biệt Huế vào Sài Gòn dạy học tại trường Chấn Thanh − trường này do một người quê Quảng Nam chủ trương), nhưng đã có thể nói Phan Khôi có một thời kỳ thứ ba góp mặt với báo chí Sài Gòn.} 
 
\textit{Về tờ  }Dân báo\textit{, theo Huỳnh Văn Tòng }(Báo chí Việt Nam từ khởi thuỷ đến 1945\textit{, Tp HCM, 2000: Nxb TPHCM, tr. 499) thì đây là một tờ báo ngày, ra từ  19/5/1939, tồn tại đến 1944. Và điều cần lưu ý là người làm chủ nhiệm báo này từ 1939 đến 1943 là Bùi Thế Mỹ, một người vốn gần gũi với Phan Khôi cả trong quan hệ họ hàng lẫn trong nghề báo. } 
 
\textit{Chưa rõ Phan Khôi bắt đầu viết cho tờ này từ thời điểm nào. } 
 
\textit{Theo nguồn tư liệu tôi đã nắm được cách đây khá lâu thì trên báo }Nước Nam\textit{ ở Hà Nội tháng 5/1941, tác giả Trúc Khê cho đăng nhiều kỳ mấy bài đáp lại những nhận xét của Phan Khôi về cuốn truyện ký }Nguyễn Trãi \textit{của ông; theo Trúc Khê}, \textit{những nhận xét ấy của Phan Khôi đã đăng }Dân báo\textit{ ở Sài Gòn tháng 4/1941. Tuy vậy, do bộ sưu tập }Dân báo \textit{ở Thư viện Quốc gia Hà Nội hiện chỉ có một lượng rất ít; tôi mới chỉ có thể tiếp cận sưu tập báo này thuộc các tháng 6, 7 và 8/1941. } 
 
\textit{Có thể thấy rõ: Phan Khôi có bài đăng trên tờ này khá đều đặn, bút danh Thông Reo được dành riêng cho mục hài đàm có tiêu đề “Chuyện hằng ngày”, họ tên thật Phan Khôi dùng để ký dưới các bài báo khác. } 
 
\textit{Dưới đây tôi giới thiệu hai bài của Phan Khôi liên quan đến thái độ của ông đối với Thơ Mới. } 
 
\textit{Việc Phan Khôi là người mở đầu cho phong trào này là điều nhiều người đã biết. Tuy vậy, sau cái khởi đầu ấy, việc ông có tiếp tục tác động đến phong trào này không, cụ thể là có những phát ngôn ra sao, bày tỏ thái độ thế nào xung quanh diễn tiến của phong trào thi ca ấy, trong sự sống đương thời của nó, thì lâu nay hầu như chưa ai trong giới sưu tầm nghiên cứu nêu ra được tài liệu hoặc nhận định nào đáng kể. Vì vậy hai bài báo nêu sau đây, là tài liệu thực sự mới, sẽ xác nhận thêm vai trò Phan Khôi đối với phong trào Thơ Mới, hơn thế, còn cho thấy cả sự giới hạn của ông trong tầm nhìn trước các hiện tượng mới lạ, khác lạ trong thơ. Việc Phan Khôi cho loại hình câu thơ tám từ, mà ông gợi ý gọi là thơ “bát ngôn”, như kết quả rõ nhất của Thơ Mới, đương nhiên là hẹp hơn so với những đánh giá của giới nghiên cứu hậu thế về thành quả của phong trào Thơ Mới.} \textit{Ở khía cạnh khác, việc Phan Khôi không nhận ra giá trị của thơ Bích Khê, Hàn Mặc Tử… rõ ràng là biểu thị cái giới hạn của ông về tầm nhìn trước viễn tượng biến đổi của thi ca. Nhưng điều đó không ngăn cản ông vững tin vào cái mới mà theo ông, Thơ Mới đã đem lại cho thi ca tiếng Việt.} 
 
Lại Nguyên Ân 
                                                                      \begin{center}
*\end{center}
 
Báo \textit{Tiếng dân} gần đây có hai bài trong hai số tiếp nhau tuyên bố lên rằng ông Phan Khôi, là tôi, đã bắt đầu công kích lối Thơ Mới. Tôi thấy mà rất lấy làm ngạc nhiên. Tôi không hề có khi nào phản đối lối Thơ Mới cả, cũng chưa từng bắt đầu nghĩ đến việc ấy, sao người ta lại hô lên như vậy? 
 
Tôi là người đề xướng ra Thơ Mới, vì bài “Tình già” của tôi ra đầu hết; nếu lui một bước, tôi không nhận lấy cái danh người đề xướng thì ít nữa tôi cũng là một người trong những người đề xướng Thơ Mới, há có lẽ nào mới giáp mười năm mà tôi đã quay lại nó mà phản đối hay sao? 
 
Sợ cho anh em trong làng Thơ Mới không rõ đầu đuôi, tin lời báo \textit{Tiếng dân} rồi chưởi tôi là thằng phản phúc, nên cực chẳng đã tôi phải viết bài nầy đính chánh, − đáng thương hại cho ngòi bút của tôi cứ luôn luôn là đính chánh. 
 
Trong số 1594 báo \textit{Tiếng dân} viết rằng: “Ông Phan Khôi trước có hùa vui viết Thơ Mới một đôi bài… nay thấy trong làng Thơ Mới của bọn trẻ có lắm bài vô nghĩa, trong \textit{Dân báo }ông Thông Reo (hiệu ông Phan Khôi) có bài dưới mục ‘Chuyện hằng ngày’ (số ra ngày 25/5/1941) cho Thơ Mới là một cái tai nạn của văn học, xem đó đủ thấy giá trị Thơ Mới ngày nay là thế nào”. 
 
Đoạn đó ở trong “Lời nói đầu” của \textit{Việt ngâm thi thoại} đăng ở số báo nói trên, dưới ký là Minh Viên.  
 
Tiếp số sau, 1595, ra ngày 12/7/1941, nơi mục “Chuyện đời”, Chuông Mai viết: “Nhà túc học và tay đàn anh trong làng báo là ông Thông Reo (tức Phan Khôi) trước kia giữa phong triều Thơ Mới, nhớ như ông có viết một bài về cái đề “mua sò trên xe lửa” … Nhưng \footnote{
Đoạn nầy vì tôi có bỏ bớt nguyên văn nên có thêm một vài chữ cho chạy ý, như chữ “nhưng” này (nguyên chú của Phan Khôi)}  mới đây ông kinh hoảng mà la lớn: “Một tai nạn trong văn học” (bài nầy trong \textit{Dân báo} ra ngày 25/6/1941) trong bài nầy ông chỉ vạch những câu vô nghĩa trong Thơ Mới rất là rành rẽ. Xem đó đủ thấy trưng triệu đổ sụp của Thơ Mới. Chuông Mai rất biểu đồng tình với… bạn Thông Reo mà hô lớn rằng: Xứ ta còn sản xuất thứ Thơ Mới là một điều vô phúc cho làng văn nước nhà”. 
 
Xem đó, bạn đọc thấy \textit{Tiếng dân} nói rõ ràng rằng tôi cho Thơ Mới là một tai nạn của văn học và Chuông Mai tỏ ý muốn cùng tôi đánh đổ Thơ Mới. Mà nói như thế, báo \textit{Tiếng dân} lấy chứng cứ ở đâu? Chỉ lấy ở bài đăng dưới “Chuyện hằng ngày” trong \textit{Dân báo} của Thông Reo. 
 
Thông Reo, báo \textit{Tiếng dân} nói là hiệu của tôi, điều đó rất là vô lý, tôi không nhận. Tuy vậy, cho đi rằng Thông Reo tức là Phan Khôi nữa, thì cũng nên xem lại thử bài ấy Thông Reo nói những gì. 
 
Nguyên văn bài ấy ra ngày 25/6/1941 dưới mục “Chuyện hằng ngày” trong \textit{Dân báo} là như vầy: 
\begin{blockquote}
 
		\textbf{Một tai nạn của văn học} 
		\textit{(đó là cái đề) } 
 
Nền văn học Việt Nam mới gầy dựng lên vài chục năm nay, đến nay bỗng dưng gặp một tai nạn lớn. \textit{(lược vài đoạn không quan hệ) }Cái tai nạn gì thế? Chẳng có gì lạ, chỉ là viết văn không nghĩa. \textit{(lại lược một câu) } 
 
Phàm văn, khoan cầu hay đã, trước phải cầu cho có nghĩa. Phải có nghĩa đã, rồi sau mới nói đến hay hay dở. Nhưng hiện nay có một hạng văn sĩ, hình như họ chỉ cầu cho hay, còn có nghĩa hay không, họ không cần. Bởi vậy thường có những câu vô nghĩa trong văn họ mà có lẽ họ gọi là hay đó.  
 
Một tập thơ xuất bản đã lâu, nhan là \textit{Tinh huyết}, tác giả là Bích Khê, mà đến ngày nay tôi mới đem ra chỉ trích cũng hơi muộn. \textit{(lại lược một câu)} 
 
Một bài đề là “Hoàng hoa” trong có những câu như vầy: 
	 
\textit{Lam nhung ô! Màu lưng chừng trời,} 
\textit{Xanh nhung ô! Màu phơi nơi nơi.} 
\textit{Vàng phai nằm im ôm non gầy,} 
\textit{Chim yên eo mình nương xương cây. } 
 
“Lam nhung” là gì? “Xanh nhung” là gì? “Chim yên” là gì? “Xương cây” là gì? Chẳng có nghĩa gì cả. \textit{(lược bỏ nhiều câu ở dưới vì đều thế cả)} 
 
Không hơi đâu mà kể cho hết cái vô nghĩa của họ… Họ điên chăng? Nếu thế, chỉ có người điên mới hiểu mà thôi. 

\end{blockquote}
 
Coi như trên đó, Thông Reo có hề công kích Thơ Mới không? Nếu Thông Reo có công kích Thơ Mới thì tôi cũng xin nhận là tôi − Phan Khôi − có công kích Thơ Mới. 
 
Không hề! Thật là không hề! Thông Reo chỉ công kích sự viết văn vô nghĩa mà thôi, chứ có hề công kích Thơ Mới đâu? 
 
Nguyên văn, Thông Reo nói: “Cái tai nạn gì thế? Chẳng có gì lạ, chỉ là viết văn không có nghĩa”. Thông Reo nhận cho sự viết văn không có nghĩa là một tai nạn của văn học. 
 
Thế mà đến báo \textit{Tiếng dân}, báo ấy nói rằng: “Ông Thông Reo (hiệu ông Phan Khôi) cho Thơ Mới là cái tai nạn của văn học”! 
 
Thế là \textit{Tiếng dân} nói sai. Tôi xin cải chính. 
 



\end{multicols}
\end{document}