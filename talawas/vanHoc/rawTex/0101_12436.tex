\documentclass[../main.tex]{subfiles}

\begin{document}

\chapter{Sau khi đính chính báo Tiếng dân: Ý kiến tôi đối với Thơ Mới}

\begin{metadata}

\begin{flushright}1.3.2008\end{flushright}

Phan Khôi

Nguồn: Dân báo, Sài Gòn, số 628 (24 Juillet 1941). Lại Nguyên Ân sưu tầm và biên soạn.

\end{metadata}

\begin{multicols}{2}

Cũng trong ngày cho đăng bài đính chánh mấy lời sai lầm của báo \textit{Tiếng dân}\footnote{\url{http://www.talawas.org/talaDB/showFile.php?res=12427&rb=0101}} ở một số trước, tôi có nhận được thơ của người bạn ở Huế, ông Nguyễn Đức Nguyên, bút tự Hoài Thanh, gởi mua một năm \textit{Dân báo} mà dặn rằng phải bắt đầu gởi cho ông ấy từ số nào có bài bàn về Thơ Mới. 
 
Việc riêng của giữa hai người với nhau, xin lỗi ông Hoài Thanh cùng bạn đọc nữa, tôi đem nói vào đây là vì chính nó có sự quan hệ: Bởi báo \textit{Tiếng dân} nói sai nên ông Hoài Thanh mới có chỗ hiểu lầm. 
 
Trong \textit{Dân báo} lâu nay, ở dưới tên ký của người nào cũng chẳng hề có bài nào luận về Thơ Mới cả. Thế thì sao ông Hoài Thanh lại bảo bắt gởi từ số có đăng bài ấy? Tôi chắc rằng bởi ông ấy thấy hai bài trong báo \textit{Tiếng dân} mà tôi đã nói đến trong số trước, rồi tưởng rằng gần nay trên \textit{Dân báo} chắc thế nào tôi cũng có bài phản đối hay công kích Thơ Mới, cho nên \textit{Tiếng dân} mới cứ theo mà thuật lại. Thế rồi, theo ông Hoài Thanh nghĩ, tôi từ trước là người có chưn trong làng Thơ Mới, sao quay lại trở mặt toan “bỏ làng”? Nghĩ thế rồi ông nhứt định đòi cho được số báo ấy để xem thử cái luận điệu của tôi ra sao. 
 
Hôm nay, đọc bài đính chánh của tôi đã ra trước và bài nầy, ông Hoài Thanh, cùng hết thảy bạn đọc nữa, đều hiểu rằng tôi chưa hề công kích hay phản đối Thơ Mới; tôi chỉ công kích sự viết văn không có nghĩa mà báo \textit{Tiếng dân} nói sai đi. Tôi nói sự viết văn vô nghiã là cái tai nạn của văn học mà \textit{Tiếng dân} lại thuật lộn rằng ông Phan Khôi nói Thơ Mới là cái tai nạn của văn học! Chỉ có vậy mà sanh chuyện! 
 
Vả tôi chỉ trích những câu vô nghĩa trong tập \textit{Tinh huyết} của Bích Khê; tập \textit{Tinh huyết } dầu có là Thơ Mới đi nữa, báo \textit{Tiếng dân} cũng không được phép cứ theo đó mà nói rằng tôi công kích Thơ Mới. Cái lẽ ấy cực kỳ dễ hiểu. Cũng như khi cụ Huỳnh Thúc Kháng nếu có chỉ trích những câu vô nghĩa trong một bổn tiểu thuyết nào, tôi há có thể vin theo đó mà nói cụ công kích lối văn tiểu thuyết hay sao? 
 
“Thơ Mới là cái tai nạn của văn học” − báo \textit{Tiếng dân} nói ông Phan Khôi nói như thế, − tôi đọc tới mà ngẩn người ra, vì tôi không hề nói như thế, tôi không hề phản đối hay công kích Thơ Mới. 
 
Vậy thì ý kiến tôi đối với Thơ Mới như thế nào, luôn tiện tôi cũng nên bày tỏ ra ở đây. 
 
Chữ “mới” trong cái danh từ “Thơ Mới”, có người cho là lạm, không xứng đáng, vì nó không phải mới gì, nó lấy nguồn ở thơ cổ phong hay từ khúc mà ra. Dầu vậy, cái danh từ ấy cũng thành lập được. Vì cái thể thơ như của Thế Lữ, Xuân Diệu, xưa nay chưa có ai làm nhiều, mà bây giờ có, thế là chúng ta nhận cho nó thành một thể được rồi. Thể thơ ấy phần nhiều dùng tám chữ làm một câu. Hoặc giả sau nầy không gọi là “Thơ Mới” nữa mà gọi là “thơ bát ngôn”, như ngũ ngôn, thất ngôn đã có rồi, cũng chẳng có gì trái với nguyên lý của văn học cả, cũng chẳng có gì làm hại, kêu bằng tai nạn cho văn học cả. Thế thì việc gì mà người ta yểm ố nó, cự tuyệt nó cho đành? 
 
Ai công kích Thơ Mới, tôi dám bảo người ấy chỉ tỏ mình ra là hẹp lượng và thiếu sự thấy xa biết suốt. 
 
“Ông cứ việc mà đánh đổ Thơ Mới đi, rồi sau nầy Thơ Mới nó cũng cứ đứng vững và phát đạt như thường”. Câu nói ấy bất luận ra từ miệng người nào, sau đây vài ba mươi năm, nó sẽ được truyền tụng như câu của một nhà thiên văn học ngày xưa: “Mặc dầu các ông phản đối cái thuyết địa viên, trái đất cũng cứ việc quay và quay chung quanh mặt trời!” 
 
Chắc lắm, không hồ nghi gì nữa, cái lẽ nó phải như thế. Sau tứ ngôn, ngũ ngôn, thất ngôn, lục bát gián thất, thêm một thể nữa gọi là “bát ngôn” làm giàu thêm cho văn học Việt Nam mà không trật ra ngoài con đường rầy tiến hoá, thì là sự chúng ta nên hoan nghinh, chứ lại điên gì mà yểm ố và cự tuyệt? 
 
Lý thuyết thì như thế, mà cứ xem hiện trạng thì chính tôi cũng nhận thấy vài ba năm nay Thơ Mới đã ra nhiều và nhảm lắm. Cái nhảm đó là do vài bốn kẻ tuổi trẻ, kém học thức mà lại tưởng mình có thiên tài, rồi cứ thuận mồm ngâm vịnh bừa đi. Một vài kẻ có những ý tứ lời lẽ quá nông nổi và non nớt như những bài đăng trên các báo về mục “Vườn thơ”; một và kẻ cố ý đặt cho mắc mỏ, tưởng thế là hay, chẳng dè thành ra vô nghĩa, như tập \textit{Tinh huyết} đó. Đã thế rồi còn có một vài kẻ tự coi mình là đàn anh, như ông Hàn Mặc Tử, theo mà tưng bốc, − chính ông ấy đã tưng bốc tập \textit{Tinh huyết}, là thứ thơ không có nghĩa, lên chín từng mây! Như thế bảo Thơ Mới còn biết kiêng nể ai mà không đâm ra nhảm? 
 
Ấy thế mà đừng lo. “Sông có khúc, người có lúc” thì văn học hay là thơ cũng cũng có thời kỳ của nó. Thời kỳ nầy chỉ là thời kỳ hỗn độn của Thơ Mới, nó phải đi cong quanh một lúc rồi trở lại theo đường thẳng mà tiến lên, làm vẻ vang cho nền văn học của chúng ta cho mà xem!  
 
Sao mà dám chắc thế? 
 
Vì một thể thơ đã sản xuất được những bài như của Xuân Diệu, của Thế Lữ thì không có thể nào nửa chừng bỗng tiêu diệt đi được, chết đi được, tha hồ cho ai ra sức đánh đổ nó, đang tay bóp cổ nó! 
\end{multicols}
\end{document}