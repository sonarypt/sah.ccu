\documentclass[../main.tex]{subfiles}

\begin{document}

\chapter{Về việc làm thơ và thơ nữ trẻ đương đại}

\begin{subtitle}

(Tham luận tại Hội thảo “Thơ Việt Nam đương đại”, Đại học Khoa học Xã hội & Nhân văn, TP Hồ Chí Minh, 19.02.2008)

\end{subtitle}

\begin{metadata}

\begin{flushright}5.3.2008\end{flushright}

Ngô Thị Hạnh



\end{metadata}

\begin{multicols}{2}

\textbf{Tôi làm thơ tôi tồn tại} 
 
Ai cũng nghĩ làm thơ là phải bay bổng, tưởng tượng và mơ mộng. Nhưng với tôi, thì dường như không phải vậy. Thơ tôi xuất phát từ những điều nhỏ nhất trong cuộc sống cũng như trong tâm hồn mình. Tôi tập quan sát, suy nghĩ về tất cả, và lấy bản thân mình ra làm đối tượng được bình phẩm hay cắt nghĩa đầu tiên. Để làm được điều này tôi phải tập hàng ngày, hàng giờ, thậm chí cả trong lúc ngủ.  
 
Lúc mới đến cùng thơ, có thể tôi cũng là người hay mơ mộng, tả cảnh tả tình một không gian nào đấy. Nhưng khi sống với thơ lâu hơn, tôi đã phải học cách chắp nhặt từng chi tiết nhỏ, từng ý nghĩ thoáng qua trong đầu, nuôi dưỡng chúng và chọn lựa chúng để đưa vào thơ của mình. Cảm xúc và ý tưởng được nuôi dưỡng, đến một lúc nào đó từ nó sẽ bộc lộ, lúc ấy, ta không thể dùng ý chí để hỏi: có nên làm thơ nữa hay không.  
 
Khi làm thơ, tôi thức tỉnh mọi giác quan từ thân đến ý của mình, và không biết giấu mình vào đâu được, đành bộc lộ nó bằng thơ. Lúc làm thơ, điều gì làm mình đau nhất, yêu nhất mình đều muốn diễn đạt ra bằng ngôn từ. Nếu mình có nhiều vốn từ phong phú, thơ mình cứ theo đó mà tuôn trào, ý tuôn ra thành tứ, tứ diễn đạt bằng hình ảnh, ngôn ngữ, thậm chí cả thanh âm… đều rộn ràng tuôn ra như suối, và khó lòng kiểm soát được từ nào dùng đúng, từ nào dùng sai. Có lẽ cũng vì vậy mà có người bảo rằng làm thơ như bị nhập, ai nhập mình? Chữ hay ý? Chẳng thể phân biệt được, mọi thứ đều trộn lẫn, và nhào lặn thành một thực thể: THƠ. 
 
Mỗi ngày, tôi nuôi dưỡng cảm xúc cho mình, sống với con người thực của mình trong từng khoảnh khắc. Tôi cố tạo ra một khoảng cách với chính tôi, dù là nhỏ nhất, để theo dõi và miêu tả chính mình. Tôi nhìn thấy tôi đi đứng, nói năng, làm việc và đôi khi sa đọa… Tôi nghe tôi thở gấp hay thở êm, giận dữ, hiền lành hay kiêu căng. Để tạo được khoảng cách với chính mình như thế, tôi đã phải học từ chính tôi, khổ cực vô cùng. Tôi nhìn thấy mình ghen, nhìn thấy mình giận, thấy mình xấu xa, nhỏ mọn... Bởi thế, khi viết, tôi viết về tất cả những điều tôi đang sống trong tôi, cố gắng diễn dạt được trạng thái của chính mình và mọi người (bởi trong mình có mọi người và ngược lại). Tập thơ \textit{Rơi ngược} cũng ra đời như thế, nhẹ nhàng mà đớn đau (bởi mình nghĩ về mình là sung sướng nhất mà cũng khổ đau nhất). Thế nên khi chắp bút, tôi viết khá dễ dàng, đôi khi như thở, nhưng để thở được nhịp nhàng, nhẹ và sâu làm thăng hoa sức khỏe thì cần phải học, học từ bất cứ đâu, bất cứ lúc nào. Học, đọc và hành để biết mình tồn tại.  
 
 
\textbf{Sự nổi loạn trong thơ nữ trẻ} 
 
Đầu tiên tôi muốn nói về sự nổi loạn tưởng tượng. Bởi thơ ca là sự sáng tạo, như bất cứ loại hình nghệ thuật nào khác, đó là sản phẩm của sự tưởng tượng. Tác nhân tưởng tượng là con người có cảm xúc mãnh liệt, nên đôi khi sự nổi loạn trong thơ mà độc giả nhìn thấy được là sự tưởng tượng của riêng họ. Những hình ảnh dữ dội, dào dạt, xô đẩy, sắc nhọn khôn cùng là sản phẩm của sự tưởng tượng phong phú và đáng quý.  
 
Nhà thơ Ly Hoàng Ly viết:  
\begin{blockquote}
 
\textit{Người phụ nữ tự trói mình 
Người phụ nữ bảo mọi người này anh này chị ơi hãy trói tôi lại } 
\textit{Trong tư thế trói gô 
Người phụ nữ tự mỉm cười thỏa mãn vì bị trói gô 
Rồi cười sặc sụa chảy nước mắt 
Rồi bỗng mếu rồi bỗng khóc 
Rồi giật đùng đùng 
Rồi gào lên ấm ức 
Rồi rú lên tuyệt vọng 
Gục xuống  
Giẫy giẫy 
Tắt ngấm } 
\textit{Những bức chân dung nhòe nhoẹt trên tường 
Bỗng trắng toát } 
\textit{Trong tư thế trói gô 
Người phụ nữ không tìm thấy xác mình 
Chỉ thấy rêu xanh lét chân tường 
Chỉ thấy đêm đầm đìa nước mắt…”} 
 		        
(trích “Performance photo”) 

\end{blockquote}
 
Vậy người phụ nữ làm thơ Ly Hoàng Ly muốn gì khi viết bài thơ này? Khi viết bài thơ này, trong cô thật sự muốn nổi loạn không mà những con chữ đung đưa gào thét, xô đẩy kiếm tìm rồi lại không kiếm tìm gì cả. Cô có muốn mình bị trói gô? Cuối cùng \textit{nước mắt} là \textit{đêm} hay \textit{đêm} là \textit{nước mắt}, nhà thơ là \textit{người phụ nữ} trong thơ hay \textit{người phụ nữ} trong thơ là nhà thơ? Điều này không thể tách bạch, chúng không có ranh giới. 
 
Thơ là sản phẩm của sự tưởng tượng, và sự tưởng tượng đó gắn chặt với người tưởng tượng ra nó. Là phụ nữ làm thơ, những cơn giông, cơn khát, nỗi khắc khoải của họ gắn chắt với ngôn từ mà họ viết ra. Nhưng thật ra, người ta chỉ nổi loạn khi bị ức chế, vậy nhà thơ nữ nổi loạn khi họ phải chịu đựng nỗi u uất trong lòng. Sự tưởng tượng trong thơ cũng một phần là nỗi ám ảnh, nỗi bất lực hay ẩn ức trong đời sống thật của nhà thơ. Như nữ thi sĩ Hồ Xuân Hương, do đời sống gia đình không được hạnh phúc, hai lần chồng chết, cuộc sống đẩy bà đến tận cùng của sự cô đơn. Bà bất lực, ẩn ức tình dục nên nổi loạn thét gào trong thơ. Bà tuyên ngôn: 
\begin{blockquote}
 
\textit{“Chôn chặt văn chương ba thước đất}  
\textit{Tung hê hồ thỉ bốn phương trời!}” 

\end{blockquote}
 
Và từ tuyên ngôn ấy, bà viết những bài thơ mang đậm yếu tố tính dục. Bà chửi hầu hết những người trong thiên hạ từ quân tử đến nhà sư, từ học trò đến quan chức trong triều… Nhìn sự vật nào bà cũng có thể tưởng tượng, nói cạch, nói khóe đến vấn đề tính dục: “Vịnh cái quạt”, “Quả mít”, “Hang Thánh Hóa”… Sự nổi loạn trong thơ bà không chỉ là kết quả của sự tưởng tượng mà còn thể hiện ẩn ức của người phụ nữ làm thơ không đạt được hạnh phúc đời thường như những người phụ nữ khác.  
 
 
\textbf{Người ta nổi loạn khi còn trẻ} 
 
Có dạo, trong làng văn nghệ Sài Gòn xôn xao với sự xuất hiện của nhóm thơ nữ trẻ Ngựa Trời. Nhóm đã khiến nhiều người chú ý đến thơ nữ hơn khi cho ra đời tập \textit{Dự báo phi thời tiết}. Tập thơ tập hợp tác phẩm của năm tác giả trong nhóm. Nếu hiểu theo nghĩa rộng của từ nổi loạn, tôi nghĩ đây cũng là hành động nổi loạn của những người làm thơ trẻ. Trong việc in ấn thơ, những tác phẩm in chung hàng hà sa số, nhưng để tạo được dư luận như \textit{Dự báo phi thời tiết} thì quả là từ trước giờ chưa thấy.  
 
Dù ai có chê trách, dè bỉu hay có ý kiến phản đối họ, thì họ cũng là những người dũng cảm, đã dấn thân vào con đường viết lách với tất cả lòng đam mê. Thơ họ mang hơi thở hiện đại, bề mặt và rát bỏng. 
\begin{blockquote}
 
\textit{“Có hay ho gì tay cầm quyển sách dày mà lòng yêu kém chữ nghĩa 
Dục vọng bao nhiêu lần tro bay bụi tan 
Đam mê cho lắm rồi cuối cùng ngượng ngập 
Chính mình đây cũng tán thưởng sự kiên nhẫn của mình } 
\textit{Chưa kịp úp mặt vào đôi vai hứa hẹn sẽ giữ lấy ta 
Thì đã vội mất đà bởi một vô tâm sắp đặt 
Này người, nếu được hóa thân làm kiếp vật 
Cũng xin quay lại một lần làm loài nhện độc 
Mà kiêu hãnh giăng tơ”} 
        
(trích “Phía trước cũng là đêm”) 

\end{blockquote}
 
Lập luận “\textit{nếu được…}” của Thanh Xuân quả là dũng cảm, giăng tơ trong kiêu hãnh, làm việc với tất cả thân xác, tâm hồn mình và kiêu hãnh vì nó. Nhưng “\textit{xin quay lại một lần làm loài nhện độc”} thì chỉ có là cách nói của người trẻ, đây dũng khí, đầy thách thức với thương đau.  
 
Trong khi đó, Phương Lan lại tìm cho mình một lối diễn đạt mới, thể hiện cơn khát cuồng nộ của người đàn bà “Đen”: 
\begin{blockquote}
        
\textit{“Đêm thộc vào}        
\textit{Chiếc lưỡi đầy cát}        
\textit{Cày xới những đường điên rồ}        
\textit{Bầu trời sáng sủa đã sập xuống}        
\textit{Lấp lánh ngàn mảnh ánh sáng không kịp tẩu thoát}        
\textit{Vụn âm còn lại của tiếng nói}        
\textit{Vụn ánh còn lại của bóng tối}        
\textit{Vụn máu còn lại của cuộc người}        
\textit{Ngày trổ cơn mưa sót}        
\textit{Dự định lang thang đổi chiều}        
\textit{Thèm rực một ngẫu cuồng cắn nghiến trống trải bấn loạn}        
\textit{Mồ hôi rộp khít khắp vòm tối}        
\textit{Đậu đen đen”}        
        
(trích “Đen”) 

\end{blockquote}
 
Trong năm tác giả nữ trẻ của nhóm Ngựa Trời, tôi chỉ xin trích đọc của hai tác giả trên, để thấy họ bạo liệt, dám sống và dám viết. Họ nổi loạn ngôn từ và ngôn từ xô rạt, rát bỏng, kiêu hãnh.  
 
Những người phụ nữ trẻ làm thơ, họ sẵn sàng phơi mở cái tôi mạnh mẽ, xé toạc bức màn truyền thống đầy nghi kỵ và những định kiến xã hội. Trần Lê Sơn Ý đã thể hiện sức trẻ của mình như sau:  
\begin{blockquote}
       
\textit{“Thức dậy đi hỡi chú ngựa non của cánh đồng ngực trẻ        
Thức dậy và tung bờm cất vó        
Phóng như điên        
Chỉ cơn điên mới cứu khỏi nỗi sợ hãi        
…        
Thức dậy để uống sương mai        
Đón mặt trời mỗi sớm        
Thức dậy đi ơi chú ngựa        
Đã ngủ sâu trong đáy tim nhiều năm tháng”         
}        
(trích “Bài ca ngựa non”) 

\end{blockquote}
 
Hay chính tôi khi bức bối đã viết:         
\begin{blockquote}
        
\textit{“Em muốn delete tình yêu}        
\textit{Tuỳ anh có hiểu hay không hiểu vì sao em làm thế}        
\textit{Con rắn trong em trườn qua xác chết, trườn qua hết thảy mọi định kiến}        
\textit{Con rắn ngóc đầu nguyền rủa dòng máu đỏ}        
\textit{Em vẫn yêu thương đến nhu nhược vì anh.}        
\textit{Em muốn format kỷ niệm}        
\textit{Tuổi trẻ chẳng cần nương tựa vào đâu}        
\textit{Con ngựa sáu chân uốn cong mình trốn chạy} 
\textit{Vẫn là nỗi bất an phía trước – lưỡi dao ngước nhìn em, rình rập hiểm nguy…”} 
        
(trích "Em muốn”) 

\end{blockquote}
 
Người trẻ, khi họ muốn được là chính mình, giữ được chứng kiến của mình họ bắt buộc phải mạnh mẽ - phải sử dụng cơn điên của mình. Không còn cách nào khác, bởi họ chưa học được cách kiên nhẫn, bởi họ không thể chịu đựng được khi bị coi thường. Nên sự nổi loạn trong thơ thông thường chỉ có nơi người trẻ, người nữ trẻ thì khởi phát sự nổi loạn ấy bằng chính những giờ khắc yếu đuối của mình. Bởi “\textit{Chỉ cơn điên mới cứu khỏi nỗi sợ hãi”, }bởi \textit{“lưỡi dao ngước nhìn em, rình rập hiểm nguy…”} 
 
 
\textbf{Tôi yếu đuối -  tôi làm thơ} 
 
Nghe có vẻ nghịch lý khi bảo người nữ trẻ nổi loạn trong thơ lại yếu đuối. Nhưng theo tôi đó là sự thật. Ai đó bảo làm thơ là dũng cảm, nhưng nguồn gốc của hành động cầm bút lên viết những vần thơ lại là yếu đuối. Khi tôi không thể làm gì khác, khi tôi chất ngất những cơn đau, chất ngất lời tâm sự thì không ai có thể chia sẻ với tôi ngoài thơ và nước mắt… Và tôi sống trong thơ, thơ sống trong tôi, tôi nổi loạn nhờ thơ và tôi mạnh mẽ. Nguồn gốc của sự mạnh mẽ lại chính là sự yếu đuối của mình.  
\begin{blockquote}
        
\textit{“Hạt mưa dĩ nhiên là rơi xuống}        
\textit{em rơi ngược} 
\textit{cô đơn như rừng chỉ còn một chiếc lá…} 
        
\textit{Đầu ngón tay chai vì cầm bút }        
\textit{vẫn không trải hết lòng mình}        
\textit{như nỗi ám ảnh về tình yêu và cái chết } 
\textit{em thét gào, em lại bảo em hãy lặng câm”} 
        
(trích "Mưa II“) 

\end{blockquote}
 
Người phụ nữ làm thơ, nổi loạn mà không nổi loạn, chửi rủa mà như than khóc. Bởi với tôi, ham muốn vượt bậc của người phụ nữ muôn đời vẫn là hạnh phúc gia đình và tình yêu. Có tình yêu họ có tất cả. Thế nên, từ năm mười sáu tuổi, nữ nhà thơ Vi Thùy Linh đã viết:  
\begin{blockquote}
 
\textit{“Chúng mình ở hai miền 
Ngày nào em cũng khóc... 
Anh yêu của em ơi 
Em yêu anh điên cuồng 
Yêu đến tan cả em 
Ào tung kí ức 
Ngày dài hơn mùa 
Em mong mỏi 
Em (có lúc) như một tội đồ nông nổi 
... Em nghe thấy nhịp cánh êm ái ân 
Một làn gió thổi sương thao thác 
Đêm run theo tiếng nấc 
Về đi anh! 
Cài then tiếng khóc của em bằng đôi môi anh 
Đưa em vào giấc ngủ nồng nàn, quên đi những đêm chập chờn trĩu nặng 
Ngày nối ngày bằng hi vọng 
Em là người dệt tầm gai... 
Em nhẫn nại chắt chiu từng niềm vui  
Nhưng lại gặp rất nhiều nỗi khổ 
Truân chuyên đè lên thanh thản 
Ôi, sự trái ngược - những sợi tầm gai! 
Không kỳ vọng những điều quá lớn lao 
Em lặng lẽ dệt hạnh phúc từ nỗi buồn - những sợi tầm gai - không ai nhìn thấy 
Gai tầm gai đâm em đau đớn 
Em chờ anh mãi... 
Tưởng chừng không vượt nổi cái lạnh, em đã khóc trên hai bàn tay trầy xước…”} 
        
(trích "Người dệt tầm gai")        
 

\end{blockquote}
 
Thế nên, sự nổi loạn trong thơ cũng như trong đời sống của người nữ làm thơ lại thể hiện những niềm khao khát, đáng yêu và đầy nữ tính của người làm thơ.  
 
Là người trong cuộc, tôi chỉ có bấy nhiều lời về thơ và người làm thơ. Như được nói về chính mình qua bài viết này, chắc chắn có yếu kém, rất mong quý vị lượng thứ.  
 
\textit{TP Hồ Chí Minh, tháng Giêng 2008} 
 
© 2008 talawas 
\end{multicols}
\end{document}