\documentclass[../main.tex]{subfiles}

\begin{document}

\chapter{Hành trình tinh thần của một nhà thơ}

\begin{subtitle}

(Tham luận tại Hội thảo “Thơ Việt Nam đương đại”, Đại học Khoa học Xã hội & Nhân văn, TP Hồ Chí Minh, 19.02.2008)

\end{subtitle}

\begin{metadata}

\begin{flushright}7.3.2008\end{flushright}

Lê Tâm



\end{metadata}

\begin{multicols}{2}

\textbf{1. “Người thơ phong vận như thơ ấy” - Hành trình thơ và đời Hoàng Hưng} 
 
Ròng rã suốt một năm đi dọc sông Dương Tử để quên đi bệnh tật và lo âu, Cao Hành Kiện đã hoàn tất \textit{Linh Sơn}, kiệt tác đưa ông đến giải thưởng Nobel danh giá. Trước đó gần ba trăm năm, lang thang trên con đường sâu thẳm tìm kiếm vẻ đẹp tinh thần Nhật Bản, Basho trở thành \textit{hành giả - thi nhân}, người mà đến nay vẫn được xem là biểu tượng cho sức sống kỳ diệu của thơ Haiku, của hồn thiền Nhật Bản.  
 
Hành trình của mỗi nhà thơ đều ghi dấu những suy tư dấn thân và được đúc lại trong mỗi con chữ, trang thơ. Tôi hình dung Hoàng Hưng cũng là một trường hợp như thế; nhất là khi đọc \textit{Hành trình} của ông, tác phẩm ra đời vào năm 2005 và được giải thưởng thơ của Hội Nhà văn Hà Nội năm 2006.  
 
\textit{Hành trình }mang trong nó một tư chất Thiền rõ rệt. Vì thế, tập thơ gợi cho tôi cảm giác an lạc mà vẫn “chưa xong”, hài hòa mà “vẫn đang”…; tập thơ quá nhiều nỗi niềm, nhiều thành tựu đậm đặc và cũng có những suy tư dang dở.  
 
Chất “hành hương” trong tập thơ mới nhất của Hoàng Hưng tạo ra một mạch ngầm cảm thụ cho độc giả. Vì thế, có người thấy tập thơ \textit{Hành trình} là “khúc cuối của ba cuộc hành trình song song”: làm thơ, dịch thơ và đọc thơ vòng quanh thế giới. Nguyễn Thụy Kha thì nghĩ: “Hành trình thơ Hoàng Hưng là hành trình bền bỉ và kiên định suốt hơn 40 năm qua”. Có người lại đọc thấy trong thơ Hoàng Hưng một “hành trình tâm linh” tìm đến với giấc mơ “tràn ánh sáng”… Tôi cho rằng riêng với nhà thơ Hoàng Hưng, \textit{Hành trình} là cách gọi tên sự sống – sự sống muôn màu bên ngoài và sự sống thẳm sâu trong cõi tinh thần. Phải chăng vì thế mà tập thơ đã hé lộ chiều sâu triết lý mà ít tập thơ đương thời nào có thể chia sẻ?  
 
Những gì có trong \textit{Hành trình, }theo tôi, cũng có trong những tập thơ trước của Hoàng Hưng. Tập thơ này không phải là lời tổng kết một giai đoạn sáng tác, không phải mở ra một cái gì khác cho riêng nhà thơ, không phải chuyển từ trạng thái “ngựa biển” ào ạt sang cái thì thầm sâu lắng nơi cửa Thiền, không phải đi từ hướng ngoại tới hướng nội,… Tất cả chỉ đơn giản là lời trò chuyện về những chuyến đi. Có những chuyến đi đầy háo hức và đợi chờ. Có những chuyến đi chỉ toàn là ác mộng và mất mát. Có những chuyến đi mơ màng và đau đớn của tình yêu. Cũng có những chuyến đi rất nhiều dấu hỏi…  
 
Thơ Hoàng Hưng, khởi từ \textit{Đất nắng} (1970) đến \textit{Ngựa biển} (1988), \textit{Người đi tìm mặt} (1994) và mới đây là \textit{Hành trình}, chỉ có một phẩm chất: đó là khả năng “đi cùng” với thơ ca một cách chân thành, sâu sắc. Cái “đi cùng” ấy, tôi thích hình dung nó giống như sự “dấn thân”, “thấm thía”, “đầy ứa”. Những người bạn trong thơ ông, nói cho cùng, là bạn đời, bạn tình, bạn thơ; những câu chuyện của ông, nói cho cùng, là chuyện thơ, chuyện tình, chuyện đời. Có gì khác nữa đâu. Ông dành cả đời cho thơ, và đời người cũng lẫn vào đời thơ. Và những gì ông kể lại, thì thào với người đọc, \textit{chính là những ám ảnh thơ ca trên mọi nẻo đường đời.}  
 
Một bài thơ không đề của ông trong tập \textit{Ngựa biển }có những câu thơ tình điệu rất sâu:        
\begin{blockquote}
        
\textit{Đường phố hôm nay mùa đông} 
\textit{Sao áo em mùa hạ?} 
        
\textit{Những sọc áo xanh cuộn sóng} 
\textit{Em mang trên ngực biển đầy.} 
        
\textit{Biển những ngày hè đẹp lắm} 
\textit{Ngày nào tìm biển ta say.} 
        
\textit{Nhưng mùa hạ đã ra đi } 
\textit{Chân trời xa không ngấn nắng} 
        
\textit{Sao em còn mang áo mỏng} 
\textit{Có còn mùa hạ nữa đâu.} 
        
\textit{Sao em làm lòng ta đau} 
\textit{Nhớ ngọn lửa hè đã tắt. } 

\end{blockquote}
 
Những câu thơ ngắn giàu ám ảnh như thế vẫn đeo bám đời thơ Hoàng Hưng. Số phận những câu thơ giàu cách tân đó tiếp tục nối dài những tranh cãi và hứa hẹn cho những chân trời thơ khác. Sau này, đến tập \textit{Hành trình, }chất ám ảnh ấy càng đầy đặn hơn, đến nỗi được người ta xem là biểu hiện“thơ Thiền hiện đại”; nghĩa là trong ý thức sâu xa, thơ ông đã tương đắc một cách đặc biệt với những sáng tác của các thiền sư cách chúng ta cả ngàn năm. Nhưng ám ảnh của các tu sĩ năm xưa là lời nhắc nhở chứng ngộ, là cách con người tham dự vào sự giải thoát cho chính mình. Còn ông, chỉ đơn giản là một nhà thơ chắt lọc ám ảnh đời mình thành sợi tơ vàng sáng tạo. Con kén quằn quại trong tinh thần ông là những câu hỏi không có điểm dừng với đời sống, với thân phận. Một trong những bài thơ được yêu mến nhất của Hoàng Hưng trong tập \textit{Người đi tìm mặt }cũng nằm trong dòng chảy ám ảnh kỳ lạ đó:         
\begin{blockquote}
        
\textit{Người về từ cõi ấy} 
\textit{Vợ khóc một đêm, con lạ một ngày} 
        
\textit{Người về từ cõi ấy} 
\textit{Bước vào cửa người quen tái mặt} 
        
\textit{Người về từ cõi ấy} 
\textit{Giữa phố đông nhồn nhột sau gáy} 
        
\textit{Một năm sau còn nghẹn giữa cuộc vui}        
\textit{Hai năm còn mộng toát mồ hôi}        
\textit{Ba năm còn nhớ một con thạch sùng}        
\textit{Mười năm còn quen ngồi một mình trong tối.}        
\textit{Một hôm có kẻ nhìn trân trối} 
\textit{Một đêm có tiếng bâng quơ hỏi. } 
        
\textit{Giật mình một cái vỗ vai. }        
("Người về") 

\end{blockquote}
 
Bài thơ đã lấy đi nhiều giấy mực của văn giới. Nhà thơ cũng tâm sự: bạn đọc trong và ngoài nước chia sẻ với ông rằng, bài thơ có thân phận của nhiều con người ở nhiều xứ sở khác nhau, bài thơ là nỗi bất an của thời đại. Dày đặc hơn nỗi cô đơn, cao rộng hơn nỗi tủi buồn, bài thơ không dừng lại ở tâm tình số phận. Nó là cái gì sâu hơn cảm xúc; cái tứ thơ này, nói như Hoài Thanh bàn về thơ Hàn Mặc Tử, khen hay chê cũng đều nhẫn tâm. Bởi đó chính là những nỗi niềm sống trải của nhà thơ đột ngột lên tiếng từ một chuyến đi kỳ lạ đau đớn của cuộc đời. Tù đày, hay hơn thế nữa, cũng neo đậu lại nơi tâm hồn con người vết thương của tồn tại, vết thương ở \textit{xứ loài người. } 
 
Đúng như lời bộc bạch của nhà thơ, \textit{hành trình} thơ ông chỉ chăm chú kiếm tìm, gọi tên những bất an ở tầng sâu của tâm hồn. Một tiếng ắc-coóc chiều Moskva cũng lay động tiếng khóc một thời tuổi trẻ. Một tiếng quạ kêu ở Calcutta đủ kéo quá khứ thiêng liêng vào trong cõi đời chật hẹp náo động. Những ngày mưa Bangkok gợi lên cảm tưởng về những chuyến đi vô nghĩa lý… Nỗi niềm bất an ấy còn hiện diện trong những câu thơ về sự sống, cái chết. Đó là một ngày sinh nhật giữa “vườn thú xác xơ thu”, giản dị và buồn bã. Đó là dự cảm miên man về cái chết, từ cái chết tinh thần đến sự kết thúc thân xác:        
\begin{blockquote}
        
\textit{Còn tôi sẽ chết cách nào đây}        
\textit{Chết mòn chết mỏi}        
\textit{Trước màn hình tivi?}        
\textit{Chết dần mỗi sáng trong bài múa tham sinh tập thể?}        
\textit{Chết nghìn lần trong mắt em?} 
\textit{Không. Cái chết ấy tôi không chịu nổi.} 
        
\textit{Tôi ước mình chết trong một chuyến đi…}        
("Cái chết") 

\end{blockquote}
 
Chết trên hành trình, điều ấy có khác gì tâm niệm “đi cùng thơ cho đến chết”. Nhiều tác gia lớn đã chết trên hành trình. Đó là Basho, là Tolstoi, là Nguyễn Bính…, những tâm hồn đi cùng mưa gió của đời, chết trong rét mướt bất ngờ và sống sót mãi với văn chương của họ. Vậy nên, chết trên hành trình cũng là một ẩn dụ đẹp. Đó là tư tưởng của thơ, chí khí của thơ.  
 
Những ám ảnh mới trong thơ Hoàng Hưng lại tiếp tục xô đẩy nhau trong tập thơ dày đặc những chuyến đi của ông: chỉ một cái “máy mắt” cũng trở thành “một đời chớp đông”, những đỉnh cao đối nghịch nhau đầy ẩn dụ: dãy Himalaya tuyết vàng kim – nơi ẩn tu nghìn năm bên cạnh những cao ốc chọc trời đảo Manhattan, New York chứa đầy hy vọng và hiểm nguy; sông Hằng linh thiêng hằn lên bóng xác con trâu mộng trong dàn thiêu quá khứ rồi đột ngột xuất hiện trong cảnh tình chen chúc mới:        
\begin{blockquote}
        
\textit{Tràn xuống sông bầy người ngũ sắc}        
\textit{Xin nước sông rửa sạch tội tình}        
\textit{Lão du-già sát đầy mình tro tử thi vừa nguội} 
\textit{Ướp xác phàm bằng hương liệu sắc – không.} 

\end{blockquote}
 
Tôi nghĩ rằng đây là những câu thơ hay nhất trong bài "Sông Hằng". Chuyến đi đầy tính chất hành hương về cõi Phật của nhà thơ Hoàng Hưng hẳn đã kịp ghi tạc nhiều biến chuyển tinh thần và làm “vụt hiện” nhiều ý tưởng tài hoa, sâu sắc của ông. Hình ảnh những tu sĩ thời hiện đại “ướp xác phàm” bằng thứ tro sinh tử vô thường khiến tôi cảm nhận khả năng “ký sự”, “tốc ký” của nhà thơ về các bí ẩn văn hóa từ góc nhìn đời thường, nóng bỏng. Khi ấy, thơ ca có được tiếng nói hồn nhiên sâu thẳm của nó trước thực tại - điều mà người đọc nhiều thế hệ đã bắt gặp và ngỡ ngàng khi đối diện với những trang viết quá sức đau đớn và chân thành của Boris Pasternak, của Apollinaire, hay Emilly Dickinson…  
 
Viết bằng ám ảnh, tin vào cảm giác, sẵn sàng mở rộng, học hỏi không ngừng, người thầy giáo dạy văn Hoàng Hưng năm xưa, anh lính hồn nhiên năm xưa, kẻ bị lưu đày hay “thọ nạn nghề nghiệp” vào một buổi chiều, nhà thơ “lang thang” trên những lục địa nhiều màu da… , tôi hy vọng, vẫn tiếp tục nhẫn nại mang đến cho người yêu thơ những tâm tình mới, suy tư mới và cả những thể nghiệm ngang tàng, bạo liệt chỉ có ở những nhà thơ tin sâu thân phận thơ ca của mình.  
 
 
\textbf{2. Sức sống tinh thần trong hành trình sáng tạo của một nhà thơ – trí thức } 
 
Tôi đọc thấy những cụm từ như thế này trong các bài bình luận thơ Hoàng Hưng: “giác quan chính trị”, “kiến giải”, “thời sự lớn”,… Nhà thơ Hoàng Hưng còn được biết đến với vai trò một dịch giả tài hoa, có uy tín. Vốn liếng chừng ấy: những quan sát tinh tường, những dự cảm thiên phú, khả năng ngôn ngữ, ngoại ngữ vững vàng, tất cả tạo nên một Hoàng Hưng nhà thơ, dịch giả, nhà báo. Và sâu sắc hơn, tôi nghĩ, ông là một nhà thơ – trí thức.  
 
Ít nhất trong thơ ông, tôi đọc thấy nhiều thao thức chỉ có ở những kẻ “ăn phải bả thiêng liêng”, quen cật vấn mình, quen hỏi trời hỏi đời, quen không khí đối thoại da diết với những tâm hồn lỗi lạc, quen thấy mình bất lực trước chân trời xa, trước biển cả, quen với không gian lý tưởng và tuyệt vọng não nề, quen với đòn chí mạng của cuộc đời đánh vào những kiếm tìm “xa xỉ” của tinh thần. Phẩm chất trí thức trong thơ Hoàng Hưng tạo nên đẳng cấp một nhà thơ hiện đại, đúng hơn là nhà thơ đương đại. Ông chẳng ngần ngại tưởng nhớ Lorca và Ginsberg trên phố phường Times Square, âm thầm gọi hồn Apollinaire giữa nước Pháp, và mai mỉa cả Nữ thần Tự do nơi biển trời New York:         
\begin{blockquote}
        
\textit{Trăm con tàu lượn đến chào nàng nhưng không đến gần nàng được}        
\textit{Tự Do! Tự Do! Một đời khao khát phút này nàng vẫn cách xa. } 

\end{blockquote}
 
Ông kể về chuyến đi ở nhà trọ Rovaniemi để chờ ánh sánh Bắc cực. Bài thơ gọn gàng đơn sơ nhưng cái kết cục chẳng hề bình thường, nếu không nói là lạ lùng.         
\begin{blockquote}
        
\textit{Người ta bảo ánh sáng thiêng rửa sạch tâm hồn} 
\textit{Bắc Cực Quang chỉ hiện ra cho những người tốt số} 
        
\textit{Nhưng đời anh rủi lắm hơn may} 
\textit{Bắc Cực Quang chờ suốt đêm không gặp.} 
        
\textit{Anh đã ngủ thiếp đi } 
\textit{Và giấc mơ anh tràn  ánh sáng. } 

\end{blockquote}
 
Những câu thơ cuối cùng có cái gì đó rất gần gũi với câu chuyện \textit{Nhà giả kim} của  Paulo Coelho. Chàng trai trẻ Santiago theo đuổi giấc mơ tìm kho báu của mình trong một chuyến đi mộng ảo. Ngòi bút ma thuật của Paulo Coelho đột ngột đưa chàng trai trở lại miền đất cũ ban đầu – nơi kho báu đã nằm ở đấy từ rất lâu. Thật đâu có khác gì người đàn ông trong bài thơ tìm ánh sáng thiêng. Giấc mơ ấy của nhà thơ là điểm kết nối thực tại lồng lộng với không gian tinh thần. Cái giấc mơ tràn ánh sáng là quá đủ cho một Bắc Cực Quang, hay chẳng thể nào là Bắc Cực Quang?... Tôi thích cái phấn khởi đầy bi kịch này.  
 
Tôi đọc đi đọc lại thơ ông để tìm lấy những va đập mạnh nhất. Và tôi thấy, thơ Hoàng Hưng là mối giao hòa đặc biệt giữa thời sự và tâm linh, mơ mộng và thực tại, nóng bỏng và chán chường… Tôi hiểu thơ đi giữa nhiều đối cực, không riêng gì thơ Hoàng Hưng. Nhưng tôi tâm đắc cách giao hòa của thơ ông. Nói theo cách của Tam Lệ: “Ông đã để chính cuộc đời mình thành tấm lọc lớn. Như một thứ phim đặc biệt, chỉ ghi lại những gì từ một “bước sóng” riêng”:        
\begin{blockquote}
       
\textit{Giữa cánh rừng xêxan}        
\textit{Tôi bắt gặp lũ trẻ trong làng }        
\textit{Đùa vui trên đống rác thải du lịch}        
\textit{Những tràng hoa phoi bào trên tóc}        
\textit{Trên mình gấm vóc giấy màu}        
\textit{Chúng nhảy nhót hò reo} 
\textit{Như chưa từng đói khát.} 
        
\textit{Các em hãy tới bên ta}        
\textit{Nhảy múa trên những ưu phiền của ta}        
\textit{Trên mình ta rác rưởi phù hoa (…)}        
(“Trong rừng Xêxan”) 

\end{blockquote}
 
Trò chơi con trẻ ở một xứ sở nghèo túng trên đống rác thải được chụp lại qua ống kính Hoàng Hưng. Và rồi tiếng nhảy nhót hò reo ấy lại trong sáng hơn biết dường nào khi ông hạ bút “mình ta rác rưởi phù hoa”. Một phóng sự bé tí nhưng chất chứa… 
 
Bắt nhịp với đời sống đang sống, du nhập cho được cái khoảnh khắc có thật của cõi người, dọn ra đủ món đời thường, triết lý và phản biện, hồn thơ Hoàng Hưng không chỉ đồng vọng nỗi niềm “sầu sát đãng chu nhân” (buồn đến chết lòng người đi thuyền) của Lý Bạch mà còn cất cánh lên những chân trời gai góc hơn.         
\begin{blockquote}
        
\textit{Chó đen rin rít những điều khó hiểu}        
\textit{Hồn ai đang lang thang trong đêm?}        
\textit{(…)}        
\textit{Chó đen sùng sục suốt đêm}        
\textit{Nỗi ngứa ngáy tiền kiếp} 
\textit{Phát điên vì không nói được} 

\end{blockquote}
 
Sau này đến tập \textit{Hành trình, }con chó đen tội nghiệp ngứa ngáy hóa thành đàn chó rừng đầy hăm dọa và bí hiểm:        
\begin{blockquote}
        
\textit{Suốt đêm thao thức hồ nghi}        
\textit{Tiếng chó rừng có thật, không có thật?}        
\textit{Tiếng vô minh}        
\textit{Hú lên lừa mị trên đường ta đi tìm sự thật?}        
\textit{(…)}        
\textit{Bỗng mắt mắt mắt mắt}        
\textit{Chi chít mắt xanh} 
\textit{Nhìn ta trong bóng đêm.} 
 
\textit{Im lìm. } 
 	("Chó rừng") 

\end{blockquote}
 
Từ hình ảnh con chó ngứa ngáy tiền kiếp đến đàn chó hóa thành “mắt mắt mắt mắt”, tôi nghĩ thơ Hoàng Hưng đã gặp gỡ với thơ ca thế giới ở khả năng biểu tượng hóa mạnh mẽ, rất thời sự mà vẫn quyến rũ ẩn dụ. Không biết ông có chia sẻ với tôi rằng, đó là một thành tựu, là “thuật giả kim” mà ông học được từ các bậc thầy thơ ca, hay từ chính cuộc đời mà ông đã trải nghiệm một cách tan nát và nhọc nhằn. 
 
“Thơ hiện đại mà súc tích như cổ thi”, “trong sáng cổ điển”, “tả bóng ra hình”, “công lực công phu”… tôi rất muốn mượn những lời bình này của Vũ Quần Phương, Vân Long, và những bạn thơ khác viết về Hoàng Hưng để nói thêm phẩm chất Đông – Tây kết hợp trong hành trình sáng tạo của nhà thơ. Rõ ràng có hai khuynh hướng được phát triển khá tinh tế trong dòng mạch sáng tác thơ Hoàng Hưng. Một là chịu sự tác động của tinh hoa văn học nước ngoài như Pháp, Mỹ, hay châu Âu nói chung (phần này quá nhiều người bàn đến), khuynh hướng thứ hai thể hiện rất rõ chất thơ phương Đông, đặc biệt của Nhật Bản (phần này có vẻ mới). Tôi cảm nhận chất Haiku trong thơ Hoàng Hưng qua những bài thơ dài ngắn khác nhau, trong những tứ thơ đột ngột và gợi cảm, trong những ý tình và cách chắt lọc ý tưởng:         
\begin{blockquote}
        
\textit{Vươn ra nắng}        
\textit{Mọc những đôi cánh lá}        
\textit{Trong veo} 

\end{blockquote}
 
Hoàn toàn mang phẩm chất một bài Haiku: đẹp, đơn sơ, cụ thể, và cũng tràn trề không gian… 
 
Một bài khác tôi đặc biệt yêu thích:         
\begin{blockquote}
        
\textit{Thầy vào như hơi gió} 
\textit{Tăng đoàn rạng rõ tuệ quang.} 
        
\textit{An tịnh – mỉm cười}        
\textit{Đã về - đã tới} 
\textit{Bây giờ - ở đây} 
        
\textit{Tự do ngay phút này – hoặc không bao giờ nữa. }        
(“Bậc thầy”) 

\end{blockquote}
 
Nếu thơ “vụt hiện” trước đây trong tập \textit{Ngựa biển} vẫn mang nhiều dáng dấp “thí nghiệm”, “phác thảo” (theo tôi là vậy) thì chất “vụt hiện” ở một số bài thơ trong tập \textit{Hành trình, }tôi nghĩ, đã tìm thấy “căn cơ” trưởng thành, nhất là trong bài thơ vừa nêu. Bản thân tôi có quá trình tìm hiểu thơ thiền nên cũng cảm nhận hương hoa đôi chút về thể loại thơ ca đặc biệt này. Với bài \textit{Bậc thầy, }tôi chỉ có thể nói: ở một góc độ nào đó, bản thân nhà thơ cũng có tư chất “bậc thầy”, ít nhất là ở khả năng cảm hiểu sâu sắc và mãnh liệt cốt lõi vẻ đẹp tôn giáo trong thơ. Hồn bài thơ là hồn thiền. Đó là bài thơ về khoảnh khắc, “chơi” khoảnh khắc: khoảnh khắc bậc Thầy xuất hiện, khoảnh khắc an tịnh, khoảnh khắc chứng ngộ, và khoảnh khắc hiểm nguy chết người (không bao giờ nữa).  
 
Con đường tìm kiếm đầy đam mê và quyết liệt của nhà thơ Hoàng Hưng cũng đến hồi sáng tỏ. Thơ ông cho thấy lối đi mới rất riêng, chạm được phong cách sáng tác \textit{hiện đại châu Á}. Tôi nói điều này vì nghiệm thấy cảm nhận của tôi có thể chia sẻ được với Paul Hoover, nhà thơ Mỹ đương đại – người đã làm một phép so sánh con người tinh thần trong thơ Hoàng Hưng giống với người đàn ông trong tiểu thuyết nổi tiếng \textit{Người đàn bà trong cồn cát }của nhà văn Nhật Bản Abe Kobo. Tôi không rõ khi phát hiện phẩm chất \textit{hiện đại châu Á} của thơ Hoàng Hưng, ông Paul Hoover có hình dung thơ Hoàng Hưng còn gặp gỡ với Abe Kobo ở những chủ đề khác. Trong khi nhiều người phỏng vấn nhà thơ Hoàng Hưng đã đặt vấn đề “người đi tìm mặt” ra sao, thì trong văn học thế giới, mô hình “tìm mặt” đã là một chủ đề lớn. \textit{Người cười }của Victor Hugo là một ví dụ. Trở lại Abe Kobo, tôi thích so sánh khái niệm “tìm mặt” của Hoàng Hưng với khái niệm “Khuôn mặt người khác”, cũng là tên tác phẩm của Abe Kobo - tác phẩm đã góp phần đưa nhà văn Nhật hiện đại bậc nhất này lên hàng tác gia thế giới. Cuốn tiểu thuyết viết về người đàn ông loay hoay với sự biến dạng, thay xác (đổi mặt mình cho người khác) cùng những triết lý về ý nghĩ khuôn mặt mình trong cuộc sống, trong tình yêu. Điều này cũng nằm trong chuỗi suy tưởng mà nhà thơ Hoàng Hưng muốn gửi gắm ở tập thơ \textit{Người đi tìm mặt}:        
\begin{blockquote}
        
\textit{Đêm xuống rồi}        
\textit{Ta lẻn}        
\textit{Đi tìm mặt mình}        
\textit{Đi tìm mặt mình đi tìm mặt mình đi tìm mặt mình}        
(…) 
        
\textit{Mặt tôi trong gió cuốn}        
\textit{Mặt tôi trong nắng đốt}        
\textit{Mặt tôi trong lá ngón}        
\textit{Mặt tôi còi vọng cô liêu}        
\textit{Mặt tôi bàn tay ôm ấp}        
\textit{Mặt tôi đá núi im lìm}        
\textit{Mặt mình đi tìm mặt mình đi tìm mặt mình đi tìm (…)}        
(“Người đi tìm mặt”) 

\end{blockquote}
 
Ngoại trừ bài thơ rất rõ tứ này, tôi có cảm giác tập thơ “tìm mặt” dường như dừng lại ở ý thơ, ở niềm hưng phấn và phát hiện một chiều sâu nào đó; nó chưa được khai triển thành tứ, hay đúng hơn chưa phải là nòng cốt tư tưởng của tập thơ mà nó mang tên.  
 
Phải chăng, chất thơ Hoàng Hưng vẫn phải gắn với sự sắc sảo hồn nhiên và ấm nóng hơn là sự sắc bén lạnh lùng theo kiểu những nhà tư tưởng hiện sinh hay “phá phách”? Trong tâm thế này, tôi rất chú ý đến hai bài thơ: “Cửa sông” và “Tuyết sơn” của nhà thơ trong tập \textit{Hành trình}.  
 
Và tôi phát hiện \textit{cái “không thể” trong thơ Hoàng Hưng như một tiếng khóc sâu}. Mong ngắm Bắc Cực Quang mà không thể. Chờ ngắm núi tuyết mà không thể. Muốn lên đỉnh bài thơ mà không thể. Phấp phỏng chờ đợi đường đổi ngày trên máy bay cũng không thể… Tôi đã hiểu vì sao trong đêm trở về sau chuyến phiêu bạt bất ngờ, bên người bạn đời, nhà thơ đã viết: \textit{“Ước nằm nghe mưa rồi chết}” (“Mùi mưa hay bài thơ của M.”). Không còn thay đổi được. Không còn biết làm thế nào. Không biết phải chịu đựng điều gì. Không biết vì sao đã phải chịu đựng. Và cái cồn cát nơi cửa sông lại trào lên một ám ảnh ma rợn:         
\begin{blockquote}
        
\textit{Ta bước lên một chợ cá sắp tàn}        
\textit{Cồn cát trắng lửng lơ giữa biển}        
\textit{Đến hết cồn này mình sẽ thành con trẻ}        
\textit{Cởi ba lô vứt lại giữa những mảnh ván thuyền} 
\textit{Đến hết cồn này mình sẽ sang kiếp khác.} 
        
\textit{Em ngập ngừng một giọt lệ trên mi} 
\textit{Đời sống này buồn mà đẹp quá (…)} 
        
\textit{Ta cứ đứng phân vân trên cồn cát}        
\textit{Các bạn chài đã đi hết rồi}        
\textit{Những chiếc thúng rập rờn ngoài cửa biển}        
\textit{Còn hai chúng mình}        
\textit{Đi thôi}        
\textit{Về thôi.}        
(“Cửa sông”) 

\end{blockquote}
 
“Nỗi quằn quại của đời anh, ngòi bút anh đang nói với chúng ta đôi điều mới lạ về số phận con người” (Hoàng Cầm). Bài thơ “Cửa sông” lặng lẽ trôi qua tay tôi, nhưng giờ đã ở lại trong lòng tôi với một nỗi buồn “tiền kiếp”, một đam mê “tiền kiếp”. Buồn và đam mê cuộc đời này. Cửa sông cửa biển ấy là cửa ngõ của rất nhiều đôi lứa, nhiều định mệnh, đưa đẩy và úa tàn, chấm dứt và mênh mang. Tôi nhớ Herman Hesse với \textit{Shiddhartha }(được dịch ra tiếng Việt là \textit{Câu chuyện dòng sông)}. Thái tử Tất Đạt, người thành Phật dưới cội bồ đề hơn hai ngàn năm trước, được Herman Hesse mô tả trong những biến cố tan tành của dục lạc. Và con đường tâm linh của Người vang trong tiếng nói sâu thẳm nơi dòng sông nhân loại. Câu chuyện dòng sông là câu chuyện của đau khổ được giải thoát. Còn “Cửa sông” của Hoàng Hưng là giải thoát hay chấm hết, là “đi” hay “về”, là “sống” hay “chết”?... Có nhiều người đọc thơ đã yêu cái “bối rối” vô bờ ấy. 
        
        
*  
 
Về cuối đời, nhà thơ Chế Lan Viên viết:         
\begin{blockquote}
        
\textit{Nửa thế kỷ tôi loay hoay kề miệng vực} 
\textit{Leo lên những đỉnh tinh thần chất ngất…} 

\end{blockquote}
 
Tôi có niềm tin rằng những người sống còn với thơ ca trước sau vẫn là kẻ hành hương lên các đỉnh tinh thần. Cõi sáng tạo bắt họ hiến tế chính niềm vui nỗi buồn của mình, gửi đi biền biệt những thao thức xa xôi về bao nhiêu phi lý. “Cuốn sổ đoạn trường” ấy chắc là đã có tên nhà thơ Hoàng Hưng. Hành trình đời, hành trình suy tư gân guốc và tha thiết là tiếng nói thầm thì trong thơ ông. Tôi tin đó là sức mạnh bền bỉ hồn thơ Hoàng Hưng. Tận tụy và liều lĩnh, ông đã phó thác mình cho những chuyến đi vô tận của thơ ca và cuộc đời – những chuyến đi quá nhiều sóng gió rủi may, nhưng cũng không ít vàng ròng sáng tạo! 
 
\textit{TP. Hồ Chí Minh, đầu xuân 2008} 
 
 
\textbf{Tư liệu tham khảo } 
 
Các tác phẩm của Hoàng Hưng: \textit{Ngựa biển} (NXb Trẻ 1988), \textit{Người đi tìm mặt} (NXB Văn hóa Thông tin, 1994), \textit{Hành trình} (Hội Nhà văn, 2005) và một số bài phỏng vấn Hoàng Hưng trên các báo, tạp chí, … 
 \begin{enumerate}

item{“Hoàng Hưng đi tìm mặt” – Hoàng Cầm, trích trong \textit{Văn xuôi Hoàng Cầm}, NXB Văn học, 1999}

item{“Nỗi ngứa ngáy tiền kiếp trong một cuộc chuyển giao tài sản hư vô và cuộc truy tìm khuôn mặt Hoàng Hưng”, Nguyễn Đỗ, lời bình trong \textit{Người đi tìm mặt}, NXB Lao động, 1994}

item{“Vùng Hoàng Hưng”, Nguyễn Hữu Hồng Minh, đăng trên talawas 2003}

item{“Hành trình thơ Hoàng Hưng” – Nguyễn Thụy Kha, tạp chí \textit{Sông Trà}, số 14/ 2006}

item{“Người đi tìm mặt”, Thụy Khuê, RFI, 1994}

item{“Ngựa biển”, Thụy Khuê, RFI, 1988}

item{“Thơ đến với người và thơ đi tìm mình”, Phong Lê, in trong \textit{Văn học trong hành trình tinh thần của con người}, NXB Lao động, 1994.}

item{“Hành trình đến giấc mơ tràn ánh sáng” - Nhật Lệ}

item{“Vụt hiện của con thạch sùng” – Tam Lệ, in trên trang web Tiền vệ  
http://www.tienve.org/home/literature/viewLiterature.do?action=viewArtwork&artworkId=4362\footnote{\url{http://www.talawas.org/talaDB/http://www.tienve.org/home/literature/viewLiterature.do?action=viewArtwork&artworkId=4362}}
}

item{“Thời sự lớn từ một bài thơ nhỏ” - Vân Long}

item{“Hành trình Hoàng Hưng” – Vân Long, 2006}

item{“Người làm thơ khó tính”, Ngô Văn Phú, in trong \textit{Duyên nợ văn chương}, Hội Nhà văn, 2002}

item{“Người về” - Vũ Quần Phương}

item{“Người đi một cuộc hành trình” (tài liệu tác giả cung cấp)}

item{Người đi tìm mặt, người đi tìm… thơ” – Nguyễn Thị Minh Thái, trích trong \textit{Đối thoại mới với văn chương}, NXB Hội Nhà văn 1996}

item{“Người chỉ đếm đến một” Thanh Thảo, lời bình trong \textit{Người đi tìm mặt}, NXB Lao động, 1994}

item{“Người về - một bài thơ xuất sắc của nhà thơ Hoàng Hưng” - Nguyễn Anh Tuấn }

item{Thư của Paul Hoover (Nhà thơ, Trường Đại học Columbia, Chicago – Chủ biên tạp chí \textit{New American Writing}) gửi GS. Tom Nawrocki,  Khoa Anh ngữ, CCC, tháng 1-2003}

item{Thư của Robert Creeley, Nhà thơ Mỹ, đồng chủ tịch Viện Hàn lâm thơ Hoa Kỳ gửi Hoàng Hưng năm 1999 }

\end{enumerate}
 … 
 
 
Tác giả \textbf{Lê Tâm} là TS Ngữ văn, Đại học Khoa học Xã hội và Nhân văn TP. HCM 
 
© 2008 talawas   
\end{multicols}
\end{document}