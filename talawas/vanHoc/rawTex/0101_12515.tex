\documentclass[../main.tex]{subfiles}

\begin{document}

\chapter{Bình thơ Nguyễn Đức Tùng}

\begin{metadata}

\begin{flushright}10.3.2008\end{flushright}

Đỗ Quyên



\end{metadata}

\begin{multicols}{2}

\textbf{Bài “Chùa”} 
 
Tôi không nhìn ra "tích" nào, "sự" nào khiến Nguyễn Đức Tùng làm bài này\footnote{\url{http://www.talawas.org/talaDB/http://www.talachu.org/tho.php?bai=220}} (như một số bài thơ thời cuộc văn chương mà anh tặng các cây bút "có chuyện" làm nên nền văn học "có vấn đề" ở Việt Nam, như với các bài cho Nguyễn Khải, Nguyên Ngọc\footnote{\url{http://www.talawas.org/talaDB/showFile.php?res=12091&rb=12}}, Nguyễn Đình Thi...)  
 
Nhưng cái chuyện “lên chùa hái một cành sen” thì trúng quá, với bất kỳ người Việt thẩn thơ nào. Lại là Chế thi sĩ - người từng ưa dùng thi ca để chinh phục các di tích văn hóa; và ông thành công ở cuộc khai phá thành Hời khi đến với văn hóa bằng một trái tim thi ca, mà trên tay không cần cây bút thời cuộc (tôi chưa muốn dùng chữ “cây bút chính trị”). 
 
“Chùa” là một câu chuyện, cho người hành hương họ Chế. Khi chốn linh thiêng đã thành đời thường, nhà thơ luận lý đã không hiểu nổi vì sao. Và một sứ giả lành nhất, nhẹ nhất đã mách chỉ thi sĩ bằng tiếng hót của riêng mình.  
 
Từ khóa ở bài thơ là “lật trái lá sen”! Lại quá trúng, với một thi sĩ triết lý toan tính lật tứ tung cuộc đời bằng luận lý của thi ca, mà lại không bằng lòng thành của thi ca - trong khi thi ca không phải là cái kia; mà là cái này! 
 
Đây là một trong những bài thơ hay và có ích, của thi ca Việt Nam hiện đại, bởi góp phần “mở khóa” hồ sơ văn học hiện thực xã hội chủ nghĩa ở miền Bắc trước đây. 
 
Nó vẫn là một bài thơ hay, nếu không có lời đề tặng Chế Lan Viên. Như thế, sức mạnh của thi ca trong kịch bản này sẽ không sâu như trước, nhưng lan toả và tràn đầy Việt tính – nơi của chùa chiền và của những sự “ngẩn ngơ”; nơi không hợp với sự “lật trái” tự nhiên chỉ bằng vài con tự. 
 
 
\textbf{Bài “Ở Paris”} 
 
Hai chữ “con chuột” khiến bài thơ trở nên bàng hoàng. Và đến hai câu cuối thì sự rùng mình đến với toàn tâm trí tôi. Cái đau đớn bất ngờ được xoa dịu bằng… “cái bánh ngọt”. Đây là một hình tượng thơ đơn ảnh mà hiệu quả cao. (Hậu hiện đại không cần phải phiền tới các \textit{“ngọn đèn thắt cổ”, }những “\textit{đám mây mặc quần”}!) 
 
Trong tất cả các bài thơ ở mạch “Giễu nhại pha chất Thời cuộc” (Việt Nam) của Nguyễn Đức Tùng, tôi chưa bị rùng mình như thế. Nhẹ nhàng vẻ lịch sự Tây phương, cũng vẻ rất Tây phương: humor chun chút (về tu từ thì không Việt tính ào ào như thác đổ của Thanh Thảo trong “Bác Năm Trì dân Quảng Ngãi\footnote{\url{http://www.talawas.org/talaDB/http://www.talachu.org/tho.php?bai=210}}" trên talawas chủ nhật) 
 
“Ở Paris” khiến cho thi ca trở nên đau đớn một cách dịu dàng, khi cần công phá các huyền thoại và các quá khứ phiền toái cho hiện tại, giải tán các trung tâm (từ tư tưởng cho đến địa lý). Đó chính là phong cách Hậu hiện đại!  
 
Mở đầu bài là một giai thoại khá là sáo cũ – từ giấc mộng bên Văn Miếu Hà Nội của người đẹp Hollywood từng quậy tưng bừng nước Mỹ khi xảy ra chiến tranh Việt Nam. Tác giả đã tăng nhiều cấp cho giai thoại ở vị trí địa lý khác, và đẩy kịch tính ở sự hóa thân và phân tâm chia trí độc giả theo hai nhân vật. Song bài thơ có được tư tưởng lớn không phải ở các sự kiện xã hội, mà ở triết lý luân hồi Đông phương. Luân lý ở xứ Đông phương là một cái gì không cần giải thích, hoặc là nghe theo hoặc là không.  
 
Có hai kịch bản cho nhân vật “Anh”: Người Việt Nam; Không là người Việt Nam. Ở trường hợp đầu, “Anh” này trốn chạy khỏi thân phận Việt khi đến Kinh thành Ánh sáng, ngay trước mặt cả một “ông” cần tới “một cái bánh ngọt croissant”!) Nhưng, lòng yêu nước, tự hào dân tộc vẫn níu anh về với “kiếp trước”. Đau và hóm đến vậy! 
 
Nguyễn Đức Tùng – có lẽ lần đầu tiên và bằng thi ca – vạch ra hai mô hình chuyển hóa của loài người loài vật với Việt Nam làm “trung tâm”: 
 
Mô hình 1: Kiếp trước Người Việt Nam -→ Kiếp này: Khách ở Paris (có tiền bao “vô sản”) 
 
Mô hình 2: Kiếp trước con Chuột -→ Kiếp này Người Paris -→ Kiếp sau: Người Việt Nam! 
 
Bài thơ hiện ra hầu như không cần thủ pháp và kỹ thuật riêng, dù dễ nhận thấy có sự kết hợp giữa kỹ thuật mô tả, đối thoại (ở các phần đầu và thân) và hành động (“Anh mua cho ông một cái bánh ngọt / Croissant”). Nó nằm trong thi pháp giễu cợt đang trở thành mãnh liệt trong tay nhà thơ. Bài thơ vừa hay (hơi khó hiểu chút đỉnh!) như một bài thơ không có hậu ý, vừa “chiến” như một bài thơ đầy ác ý! 
 
Ở cái hay thứ hai, đây là một thành công của thi ca luận chiến, trong thời “hậu chiến”. Lại là Việt Nam hậu chiến… 
 
 
\textbf{Bài “Nếu” } 
 
Đây thực sự là \footnote{\url{http://www.talawas.org/talaDB/http://www.talachu.org/tho.php?bai=220}}một trong ba hay bốn bài hay nhất của Nguyễn Đức Tùng trong sự hôn phối hai dòng thơ của anh đang phát tài phát lộc trong một năm qua: dòng “Thời cuộc Việt Nam” (mà cụ thể là văn học hiện thực xã hội chủ nghĩa cũ) với dòng “Luận đề” của anh. Các bài “Chùa\footnote{\url{http://www.talawas.org/talaDB/http://www.talachu.org/tho.php?bai=220}}”, “Chim\footnote{\url{http://www.talawas.org/talaDB/showFile.php?res=12091&rb=12}}”, “Tình yêu nước thời thổ tả\footnote{\url{http://www.talawas.org/talaDB/showFile.php?res=11788&rb=0401}}”… cũng vậy. 
 
Việc phê phán dòng văn học hiện thực xã hội chủ nghĩa cũ rất khó làm cho nó hay. Nguyễn Đức Tùng có thuận lợi là tới với nó lần đầu, tươi mới, nhưng cũng có khó khăn là các vấn đề của nó đã bị nhiều người phê phán. 
 
Hình ảnh “nước trong xanh” quả đúng với nhân vật Hoàng Phủ Ngọc Tường; mặc dù nếu bỏ đi dòng chữ đề tặng, bài thơ vẫn nghiệm đúng cho văn học Việt Nam. 
 
Hình tượng nước trong xanh thật ra cũng rất cũ khi lấy nó như "tấm gương" soi người, xét đời. Nhưng ở kết của “Nếu”, nó đã tạo sự bất ngờ mà tác giả gài sẵn. Tôi cho là có hai lý do:  
 \begin{enumerate}

item{Nó hợp với các đối tượng "Anh" - các tác giả của một loại văn chương không thể được định hình một cách tường minh như văn học hiện thực xã hội chủ nghĩa cũ, mà Hoàng Phủ Ngọc Tường là một;  }

item{Nó gợi lên một lối ngơ ngẩn trong suy luận trước dòng... nước.  }

\end{enumerate}
 Tính luận đề khiến cho thơ - ở đây là bài “Nếu” - không có một lời đáp duy nhất. Và đó là thành công không chỉ của thơ. Nó càng đắc địa với văn học hiện thực xã hội chủ nghĩa cũ, và càng trúng với trường hợp Hoàng Phủ Ngọc Tường\footnote{\url{http://www.talawas.org/talaDB/showFile.php?res=12454&rb=0307}}. 
 
Tôi nghĩ là Hoàng Phủ Ngọc Tường sẽ phải "vất vả" với bài thơ “Nếu” hơn cả hàng trăm bài vở, sách báo... ở khắp nơi, hải ngoại và trong nước - kể cả Việt Nam thời văn học hiện thực xã hội chủ nghĩa - viết về anh, và còn tiếp tục viết về anh.  
 
 
\textbf{Bài “Bảy năm”} 
 
Bài “Bảy năm\footnote{\url{http://www.talawas.org/talaDB/http://www.talachu.org/tho.php?bai=220}}” là một ví dụ đẹp cho dòng thơ “Luận đề” của Nguyễn Đức Tùng: sâu sắc, giản dị, kết hợp Đông Tây. Chất humour lãng đãng như khói, đến mức phải ngẩn ngơ ra thì mới cười được. 
 
Bản thân nó là một bài thơ tuyệt hay. Cái hay nội tại trùng với cái hay ngoại vi. Đó cũng là nỗi băn khoăn lớn, là sự phân tâm của nhiều chủ thể, là sự tìm kiếm của con người về vị thế cá nhân trong xã hội. Và điểm hay nhất của bài thơ mà tôi nhìn thấy là ở chỗ đó: con người ta thường sai lầm, gần như suốt cả đời (7 năm văn học = 70 năm cuộc đời) để đi tìm và cho đến khi gần chết mới tìm thấy được vị thế của mình, và bi hài lại là ở chỗ, wrong position - chỗ ngồi trong bếp. 
 
 
\textbf{Bài “Tam tuyệt 1”} 
 
Lại một “loại hình” mới\footnote{\url{http://www.talawas.org/talaDB/http://www.talachu.org/tho.php?bai=220}} của thơ mà Nguyễn Đức Tùng đang thể nghiệm. Đây là một thể nghiệm thành công; và một thể loại mới cần nhiều thể nghiệm thành công tương tự. 
 
Lẽ thường, trong một thể nghiệm, cái Hay không cần ganh đua với cái Mới, và đôi khi cái Mới chịu nhường phiếu cho cái Lạ, mà lại là cái Lạ mang ý đồ Cách tân thì… xong ngay! Trong thời Hậu hiện đại, người đọc, và nhất là giới phê bình, càng rất cần mang đôi mắt thưởng thức và thẩm định đó. 
 
Tôi gọi là thể nghiệm, vì từng bài lẻ: “Mùa xuân”, “Kỷ niệm tuổi nhỏ”, nhất là “Buổi trưa”, cũng như cả cặp ba còn thiếu các yếu tố để thành một tác phẩm. Đó chỉ như những “mệnh đề” của toán học, mà chưa làm nên “định lý”, đó cũng chỉ như những “hiện tượng” của vật lý mà chưa làm nên “định luật”. 
 
Trong làng thơ Việt đương đại, hẳn ai cũng cho nằm lòng “mệnh đề” tuyên ngôn nổi tiếng “Tôi đứng về phe nước mắt” của Dương tiên sinh. Vì sự độc đáo - và nhất là mặt nhân văn - của nó nằm trong bối cảnh thời thế và một phần dựa lưng vào tên tuổi tác giả mà thành một “bài thơ”, dù nó còn thiếu thi tính và (một số) thành tố của tác phẩm.  
 
Trong hành trình thi ca thế giới, không thiếu các “bài thơ” như vậy. Và chúng không cho phép lặp lại.  
 
Các “tuyên ngôn” như thế thường được các nhà thơ tập hợp trong một chuỗi mà tôi không lục tìm trong kho để làm chứng. Trang tienve.org có một, hai tác giả Tây-Đông như thế. Vài năm trước Nguyễn Hữu Hồng Minh ra hẳn một tập thơ, \textit{Vỉa từ},\footnote{\url{http://www.talawas.org/talaDB/http://www.talachu.org/tho.php?bai=170}} coi “từ” là nguồn sáng tạo, và lấy các “mệnh đề” làm đơn vị bài thơ. 
 
Nguyễn Đức Tùng là một “luận lý gia thực hành”, ở chỗ anh không tuyên ngôn bằng tuyên ngôn. Tôi cho gần 50 bài thơ ngắn của anh là một trong những loại Haiku Việt. Haiku Việt của Nguyễn Đức Tùng có lối đưa ra một luận điểm bằng các hình tượng thơ và dấu ấn xã hội, sự kiện thơ (thời cuộc Việt Nam), được nuôi dưỡng trong phong cách giễu nhại. 
 
Nhiều bài thơ trong 50 bài đó trông qua, hay tính đếm số chữ, cũng bé nhỏ như 3 bài này; nhưng chúng thành một-bài-thơ là ở chỗ chúng là “định lý”, “định luật”, là ở chỗ chúng là một cây của rừng.  
 
Hoàng Ngọc Hiến – trong thư gửi cho Nguyễn Đức Tùng mà tôi được đọc - nhận xét kỹ mà bao quát: “\textit{Giọng Haiku nhưng rất khác Haiku. Có những bài khiến tôi nghĩ đến Brecht nhưng thanh thoát, nhẹ nhõm hơn}.” Cho tôi được phép thưa cùng Hoàng tiên sinh rằng, Mai Văn Phấn trong một lời dẫn cho loạt thơ Nguyễn Đức Tùng trên trang maivanphan.com có kể đến vài tiêu chí của Haiku Nhật, và có lẽ chúng ta nhìn các thơ-ngắn Việt Nam theo cái nhìn đó. Tôi muốn bổ sung cho Mai Văn Phấn rằng, tính mô phỏng phải là đặc trưng số một của Haiku Nhật và các loại thơ ngắn của Việt Nam mà số tác giả nhiều năm nay đã thành đàn.  
 
Hoàng Ngọc Hiến “\textit{Hy vọng sẽ có một thể loại mới của thơ Việt Nam ra đời}” sau khi ông “\textit{thích thú đọc những bài thơ ngắn của Nguyễn Đức Tùng}”. Còn sở dĩ ông thấy loại thơ ngắn của Tùng “\textit{thanh thoát, nhẹ nhõm hơn}” Brecht, thì có lẽ là vì chất luận đề của Tùng ẩn xuống, nhường chỗ cho các hình tượng, các kỷ niệm tưởng như mông lung, phân tâm. Trong khi Brecht dằn dữ, gọi tên ngay hiện tượng bằng các hành động thơ có một không hai. “\textit{Giải tán Nhân dân}”, chẳng hạn. Nói thêm: tôi nhìn thơ ngắn Brecht không có chất Haiku, chí ít ở sự không hề mô phỏng giữa các bài, không hề lặp lại từ cấu tứ cho đến hình tượng. Càng… ít thiên nhiên trong đó! Cái lặp lại duy nhất ở chúng là sự dữ dội, miệng đọc lên là mắt thấy Brecht! Còn nữa, nếu so với thơ Tùng, thơ Brecht không có tính luận đề đa nghĩa. 
 
Ba “cây thơ” nêu trên là ba hình ảnh, sự kiện cuộc sống. Chúng chẳng nên “non” thi ca như một bài thơ, vì đó chỉ là những mệnh đề. Chụm ba cây lại, cũng chỉ là ba mệnh đề. Có cây mà chưa nên rừng là thế. 
 
“Kỷ niệm tuổi nhỏ” là khổ thơ khá nhất, và quả là nó mang không khí rờn rợt nhân văn Brecht, như ở bài “Trong công viên”: 
\begin{blockquote}
        
\textit{Đêm nay}        
\textit{Những lứa đôi gặp gỡ }        
\textit{Để ngày mai}        
\textit{Ra đời } 
\textit{Những đứa trẻ mồ côi} 
 
 \end{blockquote}
 Cuối cùng, sự bàn thảo nghiêm túc và sự đọc nghiêm túc của chúng ta cũng là một chứng chỉ cho thành công của thể nghiệm mà Nguyễn Đức Tùng đang làm. Tôi tin là anh sẽ biến được các cây của mình như thế này thành cây rừng; như gần một năm trước với các thể nghiệm, để nay có 50 bài Haiku Việt của tác giả Nguyễn Đức Tùng. 
 
 
\textbf{Bài “Tam tuyệt 2”} 
 
Đây là một thử nghiệm thành công\footnote{\url{http://www.talawas.org/talaDB/http://www.talachu.org/tho.php?bai=220}}; thành công hơn thử nghiệm đầu tiên.  
 
Tôi tin rằng mình sắp “chế tạo” ra tên đặt cho loại hình này, loại thơ… \textit{“Ba cây chụm lại”. }Nếu Nguyễn Đức Tùng đưa loạt \textit{“Ba cây chụm lại}” này ra trước tiên\textit{, }chắc là ít người nhìn ra chất thử nghiệm của nó. Nội dung ở các bài này không mới trong mạch thơ Nguyễn Đức Tùng. Anh chỉ muốn nâng lên trong hình thức mới.  
 
Khác cụm “Mùa xuân, Kỷ niệm tuổi nhỏ, Buổi trưa” là các mệnh đề, cụm “ba cây” này là các sự việc; tính tuyên ngôn - dễ làm loại thơ ngắn trở thành giáo điều - vì thể giảm đi tới mức triệt tiêu. “Ba cây” này mang dư âm ở loại thơ Luận đề của cùng tác giả. Hình thức Haiku nhất. 
 
Trang daumau.org vừa mới cống hiến một định nghĩa Haiku qua thơ của “cận Noble Nam Hàn” Ko Un: “Cái thể thơ tưởng chừng chỉ phô bày thuần túy sự hiện hữu của cái-bề-mặt, thậm chí là những cái-bề-mặt rất nhỏ nhặt, bình thường. Thế nhưng chính cái bề mặt đấy lại khiến ta bất ngờ bởi những triết lý hàm ẩn sâu xa.”  Định nghĩa này đúng cho “ba cây” trên. 
 
Xin dẫn 2 trong 5 bài minh họa đó của Ko Un. Cái này tôi chịu là hay: 
\begin{blockquote}
 
\textit{Trên mảnh sân của một nhà nghèo   
Trăng sáng quá   
Xuyên thủng cả những chiếc bánh gạo}.   

\end{blockquote}
  
Cái này thì “tuyên ngôn”, “giáo điều”: 
\begin{blockquote}
 
\textit{Hãy tự kiếm một người bạn   
Rồi sẽ hiểu một kẻ thù   
Hãy tự kiếm một kẻ thù   
Rồi sẽ hiểu một người bạn   
  
Đây là kiểu trò chơi gì thế?} 

\end{blockquote}
 
Nguyễn Đức Tùng đã thành nhà thơ Haiku Việt với độ trí tuệ cao nhất theo kiểu Tây phương; cũng có nghĩa ít chất Thiền Đông phương nhất. Về thể loại, thi sĩ này cũng như khá nhiều thi sĩ khác, ở Việt Nam. Như Ko Un ở Nam Hàn, như bao thi sĩ khác ở Đông phương, cùng muốn nhận Haiku Nhật làm một nhánh thi ca của văn hóa mình. 
 
Với hai thử nghiệm \textit{“Ba cây chụm lại”} này, liệu Nguyễn Đức Tùng có đưa ra một “thể loại” mới, hay chỉ là một sự sắp đặt mới cho Haiku Việt, cũng như Haiku nói chung? 
 
© 2008 talawas 
\end{multicols}
\end{document}