\documentclass[../main.tex]{subfiles}

\begin{document}

\chapter{Thơ Viên Linh của thời lưu vong thất tán}

\begin{metadata}

\begin{flushright}24.3.2008\end{flushright}

Huỳnh Hữu Ủy

Nguồn: Tạp chí Văn số 95 & 96, tháng 11&12, 2004

\end{metadata}

\begin{multicols}{2}

Trong một bài viết trước, chúng ta đã có dịp đọc lại một số thơ của Viên Linh thời trước 1975, từ những bước chân mạnh mẽ, táo bạo nhưng trầm tĩnh của thời trai trẻ, với thơ tự do, rồi đến dòng chảy của nhịp đập êm đềm lục bát. Trước khi chuyển qua khảo sát giai đoạn mới của thời thơ Viên Linh khi lưu lạc ra nước ngoài sau năm 1975, có lẽ chúng ta cũng cần nắm vững lại một vài điểm. 
 
 
\textbf{Thái độ và cách sống với thơ } 
 
Trước tiên, hãy nói qua đôi dòng về Viên Linh, thái độ và cách sống với thơ, nghĩa là cung cách sáng tác và làm việc của anh. Viên Linh là một thi sĩ cần mẫn, sống hết đời mình cho thơ (anh cũng có viết truyện, viết kịch, làm báo, nhưng cái thiết cốt của đời chữ nghĩa vẫn chỉ là thơ mà thôi), gần như Lục Du vào thế kỷ XII bên Trung Hoa, không thấy thơ ba ngày đã thấy buồn thiu (vô thi tam nhật khước kham ưu). \footnote{
Dẫn lại trong \textit{Lý luận văn học nghệ thuật cổ điển Trung Quốc} của Khâu Chấn Thanh, Mai Xuân Hải dịch, Nxb Văn Học, Hà-Nội, trang 64.}  Thơ viết ra là để trang trải cho đời, nhưng cũng là để cất giấu trong lòng. Một người sống với thơ như vậy, xem thơ như tôn giáo của mình, thì tất nhiên anh nghiêm cẩn, chu đáo, tươm tất với thơ, chăm chút trên những dòng thơ anh viết ra, đâu có cách nào khác. Mỗi chữ anh lựa chọn chính là một đồ vật cụ thể, phù hợp với toàn bộ công trình thiết kế là bài thơ của anh. Nhiều lúc, anh phải đục đẽo cái dư thừa, cái không ăn khớp, thay thế cái không vừa ý bằng cái vừa ý hơn, thường xuyên tìm cách sửa đổi cái không cân xứng hay không hợp nhãn. Có lúc, anh còn đập phá hết cái cũ, để dựng lên một cái khác mới mẻ hoàn toàn. Như anh từng nói, những thứ anh viết ra có lúc là đá quý nhưng có lúc cũng chỉ là gạch ngói mà thôi. Tôi rất đồng ý với anh, gạch ngói tất là cần thiết cho đời sống, nhưng nếu phải lập một sưu tập trân châu mã não thì không thể xếp gạch ngói vào đó được. 
 
Có người, như Lê Huy Oanh, Thanh Nam, Võ Phiến than phiền là Viên Linh sửa thơ kỹ quá, cái cầu kỳ nhiều lúc thay thế cái tân kỳ, làm bài thơ mất đi cảm giác tươi mát ban đầu, có lúc lại còn làm biến mất cả một bài thơ hay đã từng được nghe. \footnote{
Võ Phiến, \textit{Văn học miền Nam, }Thơ, Văn Nghệ, California, 1999, trang 3157-3158.}  Riêng tôi, tôi thích thú và trân trọng cung cách đó của Viên Linh. Tôi nhớ đến Giả Đảo khi nghe thấy những lời phàn nàn trên. Giả Đảo để đời với giai thoại "thôi xao", cũng đã từng viết: \textit{Nhị cú tam niên đắc / Nhất ngâm song lệ lưu / Tri âm như bất thức / Qui ngọa cố sơn thu}. Hẳn là Viên Linh phải khoái trá với mấy ý tưởng này: Ba năm mới làm được chỉ hai câu thơ, khi ngâm ngợi lên thì hai hàng lệ nhỏ xuống, mà nếu kẻ tri âm không hiểu thấu, thì người viết hai câu thơ này chỉ còn cách, giữa mùa thu, đi về nằm một góc trong núi cũ mà thôi. 
 
Tôi đã từng được xem bản thảo một bài thơ của Viên Linh, sửa chữa chằng chịt, chữ này trên chữ khác, dòng này trên dòng kia. Tôi quý trọng cung cách ấy. Ý thơ đến đúng là một tia chớp, người nào làm thơ tất cũng đã trải qua kinh nghiệm này, nhưng sau đó thì phải là cả một quá trình tạo dựng, phải nói là lao động với bài thơ. Người đời thường bị ám ảnh chuyện Vương Bột hay Lý Bạch uống rượu, trùm mền ngủ say, rồi tỉnh dậy, viết một hơi là xong, không sửa một chữ nào, nhưng đó chỉ là giai thoại lưu truyền làm cho đẹp cái cõi văn chương mà thôi. Sự nghiệp chữ nghĩa được kiến tạo trên thiên tài trời đất ban tặng chỉ là một phần, mà chủ yếu là phải đặt nền trên học vấn và kinh nghiệm rút tỉa từ cuộc đời. Đọc nhiều, đi nhiều, rồi sau cùng là sự làm việc trì chí, bền bỉ, lâu dài. Đi một vạn dặm đường, đọc một ngàn pho sách, để chỉ viết được một bài thơ ngắn mấy câu, chính là có nghĩa như thế. Vậy nên, việc làm thơ, rồi sửa chữa cẩn thận, sửa đi sửa lại, sửa cho đến bao giờ đạt tới cái hay mới dừng lại là một chuyện đáng tán dương. 
 
Một nhà nghiên cứu thơ Đường đời Minh là Hồ Chấn Hanh, tổng kết kinh nghiệm sáng tác của những nhà thơ đời Đường, đã cho rằng thơ không sửa thì không thể nào hay được (thi bất cải bất công). Hồ Chấn Hanh nhắc lại một lời nói của Đỗ Phủ: "Bởi vì tính con người ta chỉ thích những câu thơ hay, cho nên lời thơ của ta mà chưa làm cho người ngạc nhiên thì tới chết ta vẫn chưa thôi sửa chữa." \textit{(Vị nhân tính tịch đam giai cú, ngữ bất kinh nhân tử bất hưu)}. Nói đến kinh nghiệm viết của Bạch Cư Dị, Viên Mai trong \textit{Tuỳ viên thi thoại} cũng có ý kiến tương tự cho thấy công việc sáng tác của Bạch Cư Dị thực gian nan khổ ải và kiên tâm biết bao: "Thơ của ông Hương Sơn họ Bạch dường như bình dị, nhưng khi xem tới những di cảo còn lưu lại, mới thấy những chỗ sửa chữ rất nhiều, thậm chí có bài sửa lại hoàn toàn không còn một chữ." \footnote{
Lý luận văn học nghệ thuật cổ điển Trung Quốc, Sđd, trang 341-345.}  
 
Tôi viện dẫn đến các cây bút đại gia ngày xưa để nói về một thi sĩ thời nay của chúng ta, tất sẽ có nhiều người phàn nàn tôi lắm lời, nhưng lòng tôi thành thực nghĩ như thế, nên cũng nói ra như thế mà thôi. 
 
Đã nói qua đôi chút về cung cách của Viên Linh đối với thơ, giờ thì hãy tiếp tục cuộc hành trình khám phá lại thế giới thơ của Viên Linh trước 75, vì nơi bài viết "Viên Linh trên những chặng đường thơ" \footnote{
"Viên Linh trên những chặng đường thơ", Văn, California, Số tháng..... 2004.} , chúng tôi cũng chưa kịp trình bầy cho đủ ý. Tất nhiên, thơ Viên Linh, cũng như bất kỳ thứ thơ nào khác và bất kỳ xuất xứ từ đâu, cũng đều phải xây dựng trên nền tảng của một sự hài hoà của âm thanh, cảm giác và hình tượng, là sự hài hoà của một thứ ngôn ngữ vượt ra ngoài ngôn ngữ. Để có thể dễ dàng bước vào thế giới thơ riêng biệt ấy, chúng ta cần chiếc chìa khoá mở cửa, nghĩa là cần nhớ đến một số hình ảnh, ý tưởng cốt lõi đã trở thành nỗi ám ảnh của nhà thơ. Lần mò theo những mối ám ảnh ấy, tức là chúng ta đang dựng lại chiều sâu hay cấu trúc lại khung sườn của thế giới thơ Viên Linh. Như vậy, dưới ánh sáng của cách nhìn Weber, tức nghiên cứu đề tài bị ám ảnh và phê bình chủ đề, chúng ta đã có thể thấy ở Viên Linh ba chủ điểm dưới đây. 
 
 
\textbf{Ba chủ điểm hay các nỗi ám ảnh } 
 
Thứ nhất, là nỗi ám ảnh của một tâm hồn cô độc, tâm hồn ấy chỉ hoà hợp giữa nhịp điệu của một đời sống cô liêu, vắng lặng, như chúng ta đã gặp thấy hình ảnh con chim ưng kiêu mạn sống trên núi cao, bên bờ một vực sâu. Hay đá tảng muôn nghìn năm không lời, bên cây cối cũng trầm buồn chỉ cất lời với gió. 
 
\textit{Trên núi cao cây cối thì buồn} 
\textit{Đá ở không hàng muôn nghìn năm} 
("Hoá thân", trang 74-75) 
 
Đó là tiếng chim đập cánh, muộn màng ngang trời giữa tiếng gió hú. 
 
\textit{Như trớt chim về lỡ} 
\textit{Tung cánh lật ngang trời} 
\textit{Ngàn khô nghe tiếng mãi} 
\textit{Gió hú bên kia đồi} 
("Hóa thân", trang 60) 
 
Không khí chung toả ra khắp nơi luôn luôn là một nỗi vắng lặng bất tận, quanh ta bỗng lẩn sương mù nghìn năm, như thế, trong cái cô tịch thiên thu của chính hồn mình, người thơ đã phải mãi hoài đối đầu với cái bóng của chính mình. Từ kinh nghiệm và ám ảnh siêu hình riêng, Viên Linh dựng nên thế giới thơ của mình bằng một sự nối kết chặt chẽ giữa ký hiệu trừu tượng và đời sống cụ thể, đôi lúc qua môi giới của những ẩn dụ. Để dẫn đến một không gian cô liêu, cô tịch, có chút gì rờn rợn. Ví dụ, cảnh chiều tà hiện đến là những cánh dơi đang lưới cả hoàng hôn đẫm máu và lệ, để tất cả đều rũ xuống dưới một sức nặng tan rã, mệt mỏi, và u buồn, đủ độ cho một tâm hồn đang sống với nỗi cô độc của riêng mình. 
 
\textit{Đời ôi thể phách hao dần} 
\textit{Hoang mang tín mộ, linh thần vụt bay} 
\textit{Ta rơi nằng nặng từ đây} 
\textit{Trong không bụi cũng trôi đầy mộng mê} 
 
\textit{Ngoài kia dơi lưới chiều về} 
\textit{Vây muôn vũng lệ trời tê máu hồng} 
\textit{Lưới mau đáy nặng hoàng hôn} 
\textit{Chân tay mỏi rủ tâm hồn mỏi theo} 
("Bản thân, Hoá thân", trang 19) 
 
Thứ nhì, là nỗi ám ảnh về những cơn mưa như có lần Võ Phiến gợi ý \footnote{
Võ Phiến, \textit{Văn học miền Nam, }Thơ, Sđd, cùng trang đã dẫn.} . Những cơn mưa tầm tã trên thơ Viên Linh. Viên Linh thích những cơn mưa. Mưa là một ám ảnh, cũng có thể là ham muốn, đam mê. Những cơn mưa sầm sập và ào ạt ngoài công trường, đại lộ. Những cơn mưa rào rạt âm thầm nơi một con hẻm nhỏ. Mưa xa cách, mưa nối kết, mưa của thuở thiếu thời, mưa trên những mối tình, mưa nơi cõi âm ty, mưa trong cuộc đời, mưa ngoài cuộc đời. Như cơn mưa chúng ta đã từng gặp ở bên trên \footnote{
Xin xem lại "Viên Linh trên những chặng đường thơ", Tạp chí Văn, đã dẫn ở trên.} . 
 
\textit{Nhớ em rồi Cúc Hoa xưa} 
\textit{Đêm nay dưới ngói trời mưa tầm tầm.} 
 
Hay là, khi đứng trước cơn mưa tầm tã nơi phương trời lưu lạc, cơn mưa ào ạt của Sàigòn ngày nào được liên tưởng, được nhớ lại, mà chúng ta sẽ gặp trong một hơi thơ lạ, rất âm trầm, cổ kính, và tao nhã. 
 
\textit{Cơn mưa chia biệt tháng ngày} 
\textit{Vẫn rơi tầm tã lòng này đêm đêm.} 
 
Thứ ba, là nỗi ám ảnh về một cõi âm hồn, địa phủ. Viên Linh tin có trời, có quỷ, và có địa ngục \footnote{
Viên Linh trả lời phỏng vấn Nguyễn Nam Anh, Văn, Sàigòn, số 198, đặc biệt về thơ, ngày 15.3.72, trang 90.} . Chúng ta không biết niềm tin có tính tôn giáo ấy bắt nguồn từ đâu, nhưng đó là sự thật chi phối cả cuộc đời Viên Linh, và ý tưởng chủ yếu đó bao trùm chân trời thơ Viên Linh. Thơ thường chỉ nên đọc và khám phá trong chính văn bản, nhưng trong kinh nghiệm riêng của tôi, tôi thích nối liền thơ với cuộc đời nhà thơ, cũng ít nhiều có phần gần với công việc của một người viết sử văn học. Khi biết về cuộc đời nhà thơ, chúng ta sẽ hiểu và cảm thơ của thi sĩ nhiều hơn. Huống hồ là với một người làm thơ như Viên Linh, luôn luôn đào sâu vào cõi thơ từ sinh động cuộc đời, chứ không phải như một cuốn tiểu thuyết, chỉ là sản phẩm của tưởng tượng. Thơ Viên Linh, đặc biệt với những bài thơ về cõi âm ty, địa phủ qua hình bóng Cúc Hoa, khi lại gần Viên Linh, tôi có cảm giác đó hoàn toàn là cuộc đời thực của anh. Như ngày xưa Nguyễn Du viết Long Thành Cầm Giả Ca, đó là sự kiện và cảnh đời nhà thi hào đã sống và từng trải. Cúc Hoa là thực và mộng, chẳng còn biên giới giữa cõi này và cõi kia, như bướm và Trang. Nên khi Viên Linh nhìn vào tấm gương soi, và anh thấy đằng chân trời xa tắp là hồn ma của Cúc Hoa thì anh cũng muốn, như bậc quân vương ngày nào, đập cổ kính ra tìm lấy bóng, để tìm lại nàng vương phi yêu kiều, nhưng Viên Linh đập vỡ kiếng theo kỹ thuật của những nhà thơ tượng trưng và siêu thực ngày nay, của André Breton, Jean Cocteau..., chẻ mặt nước thành lối để đi về địa phủ. Những hồn ma, bóng quế, những nấm mồ, thây chết, địa phủ, âm hồn, là một cái gì quen thuộc, thân thiết với cõi thơ Viên Linh. Nhưng rất khác với Đinh Hùng trước đây, bởi vì Đinh Hùng dựng nên \textit{Mê hồn ca} chỉ bằng tưởng tượng và hư cấu. Viên Linh thì hoàn toàn ngược lại, địa ngục và hồn ma là thế giới siêu hình ảo hoá anh sống trong tâm tưởng hàng ngày. Đó là cái cõi ủ rũ ưu sầu, mù mịt tối tăm, và chỉ lấp lánh những đốm lửa ma; đó là một vị trí hữu hảo để nối liền con đường đi về địa phủ mà tìm lại bóng hình xưa. Trong ba chủ điểm thơ của Viên Linh, chủ điểm hay chủ đề thứ ba này có lẽ là cốt yếu. Hãy đọc lại vài đoạn thơ trong chủ điểm này. 
 
\textit{Hồn còn giữa cuộc trầm luân} 
\textit{Que diêm đốm lửa bàn chân kẻ về} 
("Hoá thân", trang 45) 
 
\textit{Gương này bóng ấy mình ta} 
\textit{Kẻ thân biền biệt hồn ma cuối trời} 
("Hoá thân", trang 86) 
 
\textit{Ngồi lên khẽ thấy bàng hoàng} 
\textit{Nghe trong viễn mộ đôi hàng ngựa ra} 
\textit{. . .} 
\textit{. . .} 
\textit{Ý tôi giăng chết hồn chiều} 
\textit{Tình tôi tự đó tiêu điều những thây} 
 
\textit{Ngồi lên sửa dáng hao gầy} 
\textit{Từ đây đất rộng, thân này biệt ly} 
("Hoá thân", trang 128) 
 
Đến với thế giới thơ Viên Linh, có lẽ chúng ta cần nắm chắc mấy chủ điểm vừa được đề cập bên trên: Mối ám ảnh tự thân của một tâm hồn cô độc, ám ảnh về những cơn mưa, và ám ảnh về một cõi âm hồn, địa phủ. Đó là những điểm hoặc ẩn tàng, hàm súc, hoặc nổi bật lên, chúng ta có thể xem là ba cột mốc chính yếu trên dặm trường thơ Viên Linh. 
 
Thêm một sự kiện này nữa tôi cũng muốn nhắc lại dưới đây, bởi vì nó sẽ là cửa ngõ hay là một gợi ý để người đọc thơ Viên Linh có thể hiểu anh hơn, nhờ vậy mà cảm được thơ anh nhiều hơn. Khi được hỏi bài thơ nào được anh ưa thích nhất, Viên Linh cho biết đó là bài "Sông Lấp" của Tú Xương, anh thích vì đây là tiếng than thở ngậm ngùi hiếm thấy ở một thi sĩ thường cười đùa, phúng thích, châm biếm. \footnote{
Viên Linh trả lời phỏng vấn Nguyễn Nam Anh, đã dẫn ở trên, trang 87.}  
 
\textit{Sông kia giờ đã nên đồng} 
\textit{Chỗ làm nhà cửa chỗ trồng ngô khoai} 
\textit{Đêm nghe tiếng ếch bên tai} 
\textit{Giật mình còn tưởng tiếng ai gọi đò.} 
 
Khi thích một điều gì ở ngoài mình thì điều ấy cũng chính đã là mình. Đó chính là trường hợp Viên Linh với "Sông Lấp". Giữa cảnh biển dâu của hiện thực, tiếng vang trong tâm hồn vẫn luôn là một tiếng gọi đò chấp chới, cùng một nỗi ngậm ngùi vô hạn. Viên Linh nói điều ấy với Nguyễn Xuân Hoàng cách đây 32 năm, nhưng tôi vẫn tưởng rằng đó là tiếng gọi đò kỳ lạ của đời Viên Linh từ tiền kiếp nào. Cùng với nó là nỗi buồn sâu thẳm về những mất mát không thể nào tìm lại được. Tiếng gọi đò ấy thực lãng mạn làm sao, mông lung ngân vang trong cõi vô cùng. Với tất cả những điểm vừa phân tích bên trên, cùng với tiếng than thở ngậm ngùi và tiếng gọi đò ngân vang ấy, là hành trang để Viên Linh tiếp tục cuộc viễn du của mình. Năm 1975, Viên Linh vượt qua chiếc cầu lửa đỏ rực chưa từng thấy của lịch sử, giữa cơn bão lốc hung tợn đang thổi tốc trên toàn cõi miền Nam đất nước. Tâm hồn cô tịch, vắng lặng, nỗi buồn thiên thu, những cơn mưa tầm tã liên miên, những mùa địa ngục rực rỡ, và tiếng gọi đò vang vọng, với tất cả những thứ ấy, Viên Linh tiếp tục đời thơ của mình nơi đất khách gần 30 năm qua, và tiếp tục trong những ngày hôm nay, càng lúc càng rực rỡ hơn. 
 
 
\textbf{Bước đầu của thời thơ lưu vong } 
 
Ở buổi giao thời đầu thế kỷ XX, có một tiếng gọi đò chấp chới vọng vào không gian mênh mông, còn vang vọng mãi cho đến ngày nay, thì cuối thế kỷ XX, kỳ lạ thay, sau những tan rã khốc liệt của nửa phần đất nước, sau những giao tranh xung đột dữ dội của một cuộc chiến ý thức hệ kéo dài 30 năm, cũng có một tiếng âm vang, nhưng không phải là tiếng gọi đò, mà là một tiếng chuông vang vọng, dội lên thật trầm hùng và cuồng nộ. Lịch sử đổi thay quá tàn khốc tất phải sinh ra một tiếng nói khác hơn, dữ dội hơn, và khốc liệt hơn. Tiếng nói ấy, đúng hơn là tiếng chuông ấy, khuấy động cả một không gian mênh mông của thế kỷ. Đó là tiếng chuông vang ra từ lầu chuông ở một thiền viện chỉ còn trong trí tưởng, nơi mà em trai nhà thơ trước đây đã tá túc dưới chiếc áo thiền sư một thời gian dài. Thực hết sức thơ mộng, nhưng phải thấy ra đó chính là một biểu tượng của thời đại, thì mới nghe được tiếng chuông kỳ lạ và khốc liệt ấy âm u dội vang như thế nào qua bầu trời thế kỷ. Viên Linh lưu lạc ra nước ngoài sau cơn bão dữ, anh lắng nghe tiếng chuông ấy, và anh đã để lại một buổi chiều đầy âm thanh trầm hùng u uất cho nền văn học của chúng ta. 
 
\textit{Mưa đưa tôi lại Sài Gòn} 
\textit{Trán căng nhiệt đới hồn còn Đông Dương} 
\textit{Gặp em trở lại lầu chuông} 
\textit{Dang tay nện xuống hư không một chày.} 
 
\textit{Chuông không tiếng đã bao ngày} 
\textit{Nghe quen em tưởng chiều đầy âm thanh.} 
 
Vì "Lầu chuông" là một áng văn quý và điển hình của 30 năm văn học lưu vong (chưa đến 30 nhưng cũng xin nói như vậy cho thuận lời), chúng ta hãy thử đọc lại toàn bộ cả bài thơ này. 
 
\textit{Lầu chuông} 
 
\textit{Nhận tin em một năm rồi} 
\textit{Thành xưa đã đổi con người đã thay} 
\textit{Cơn mưa chia biệt tháng ngày} 
\textit{Vẫn rơi tầm tã lòng này đêm đêm.} 
 
\textit{Mưa lầy con phố bôi tên} 
\textit{Em chôn tầm vóc thanh niên giữa đời} 
\textit{Nhớ em biển sách làm khơi} 
\textit{Thả thân trôi giạt với lời muôn phương.} 
 
\textit{Nhớ em đêm tựa lầu chuông} 
\textit{Rung con tim nhỏ nghìn đường âm thanh} 
\textit{Em yêu lá ở trên cành} 
\textit{Yêu chim trong gió yêu thành vắng quân.} 
 
\textit{Em yêu miếu mộ linh thần} 
\textit{Yêu đầu không mũ yêu chân lột giày} 
\textit{Yêu người không thiết đi dây} 
\textit{Yêu nhà văn hoá đi Tây lại về.} 
 
\textit{Em yêu lòng trúc ý tre} 
\textit{Yêu kinh vô tự như bè yêu sông} 
\textit{Em yêu Camus lạnh lùng} 
\textit{Đạt Ma qua biển Ngộ Không giữa trời.} 
 
\textit{Yêu anh phóng đãng lầm nơi} 
\textit{Văn chương sai lúc thân dơi lộn chiều} 
\textit{Em yêu cuộc sống em yêu} 
\textit{Lầu chuông gác sách mộng điều tuổi xanh.} 
 
\textit{Hôm nay túi vải bên mình} 
\textit{Em tôi bán dạo trong thành phố quen} 
\textit{Ầm vang trong trí cơn điên} 
\textit{Ném thân anh giạt tới miền hư sinh.} 
 
\textit{Hơn ba mươi mộng tan tành} 
\textit{Tay xương quét lệ quanh trong mắt mờ} 
\textit{Thấy em lầm lũi hơn xưa} 
\textit{Loanh quanh ngõ dưới dật dờ lối trên.} 
 
\textit{Em tôi không sách không đèn} 
\textit{Một đầu tư tưởng bôi lem nghĩa đời} 
\textit{Đêm nay tầm tã mưa rơi} 
\textit{Tỉnh ra tôi thấy mặt trời trắng tinh.} 
 
\textit{Thấy trăng mọc lúc bình minh} 
\textit{Thấy người lưu xứ lênh đênh quê nhà} 
\textit{Thấy tôi đập kính soi hoa} 
\textit{Trên cây nhân thế la đà trái đen.} 
 
\textit{Thấy tôi nguyền rủa Thánh Hiền} 
\textit{Cầm dao giết Phật giả điên đốt chùa} 
\textit{Nhớ mưa xưa nhớ mưa xưa} 
\textit{Tháng tư úng thủy đầu mùa máu tuôn.} 
 
\textit{Mưa đưa tôi lại Sài Gòn} 
\textit{Trán căng nhiệt đới hồn còn Đông Dương} 
\textit{Gặp em trở lại lầu chuông} 
\textit{Dang tay nện xuống hư không một chày.} 
 
\textit{Chuông không tiếng đã bao ngày} 
\textit{Nghe quen em tưởng chiều đầy âm thanh}. \footnote{
In trong phần "Ngoại Vực”, \textit{Thuỷ Mộ Quan}, Thời Tập, California, 1992, trang 78-82.}  
 
Võ Phiến khá tinh tế khi đọc bài thơ này: "...cái âm hưởng gây nên do một chày nện xuống hư không ở lầu chuông nọ ngân nga mãi, thấm thía mãi thật lâu bền. Tôi có cảm tưởng đó là nhờ ở cái tử công phu, cái quá kỹ của nhà thơ, có cảm tưởng rằng tứ thơ độc đáo đã được hàm dưỡng, ấp ủ nhiều ngày đến chín muồi trong tâm tưởng và được diễn đạt trong từng chữ cân nhắc thận trọng, chứ không thể là kết quả của một phóng bút nhanh nhẹn,"tự nhiên."" \footnote{
Võ Phiến, \textit{Văn học miền Nam, }Thơ, Sđd, trang 3157.}  
 
Tôi hoàn toàn đồng ý với nhận xét ấy, nhưng đồng ý theo cái nghĩa đã đề cập ngay ở phần mở đầu của bài viết này khi nói đến kinh nghiệm của Bạch Cư Dị. Có nghĩa là: làm thơ phải bằng hết cả tấm lòng, phải vận động tất cả sức mạnh của trí tuệ, rồi còn phải kiên trì qua thời gian; phải tử công phu, phải hàm dưỡng, và phải cân nhắc thận trọng. Nghệ thuật nói chung, thơ nói riêng, phải dựa trên cái tự nhiên, nên người xưa, như Lý Chất đời Minh đã cho rằng tự nhiên là đẹp \textit{(Dĩ tự nhiên chi vi mỹ)}. \footnote{
\textit{Lý luận văn học nghệ thuật cổ điển Trung Quốc}, Sđd, trang 121.}  Nhưng tự nhiên không có nghĩa là không hàm dưỡng, không công phu, không đắn đo chọn lựa, nên \textit{Tây sương ký} là văn chương bắt được của trời mà \textit{Tỳ bà hành} cũng đưa ta đến cảnh giới kỳ lạ, tuyệt đẹp của văn chương. 
 
Chính trong cách xử sự như vậy với thơ, rồi với biết bao xúc động dồn dập của những ngày đầu lưu lạc thất tán, mất hết tất cả, chỉ còn lại một tấm lòng với sự hồi tưởng, nhà thơ chọn một số hình ảnh thân thiết của hiện thực và biến chúng thành ẩn dụ của thi ca. Hẳn rằng đây là một trong vài bài thơ hiếm hoi điển hình nhất của giòng văn học lưu vong kể từ 30 tháng 4,1975. Riêng tôi, nếu có người hỏi tôi thích nhất bài thơ nào của giòng văn học này thì tôi sẽ không ngần ngại trả lời chính là “Lầu chuông”. 
 
Chính trong cái giòng mạch của tiếng ngân nga vang vọng ấy, là một thứ tình hoài hương hay vọng tưởng quê nhà, Viên Linh có một cái nhìn về đất nước, về những ngày hôm qua, về ngày mai phải tới trong một cách nhìn đã trầm tĩnh, lắng đọng sâu sắc để tạo nên chiều dày sâu thẳm trong những dấu hỏi u buồn, anh thấy mình như một người cư tang, hỏi han chuyện cũ với sách xưa, để trách cứ chính anh và những sai lầm ngày trước, hay là anh đang trách cứ cả một thời kỳ lịch sử, \textit{Đêm nay sầu bút mực / Nắm lưng sách hỏi han / Sách biết gì tủi nhục / Chuyện cũ kể đầy trang}. Và như thế, trước mặt, trên bàn viết, là trang giấy trắng mở ra sẵn sàng cho anh, đợi những giòng chữ người thơ sẽ viết xuống. \textit{Xưa một kẻ u cư / Bốn mươi đầu bạc trắng / Ta thân xác phần thư / Viết gì trong cuộc nạn?  /.../ Viết rằng trang giấy trắng / Đang đợi người cư tang}. Bài thơ theo thể cổ phong ngũ ngôn Chuyện vãn cùng sách cũ, mặc dù không được toàn bích nhưng có những đoạn rất đẹp. Một thứ tâm tình u uất lạ kỳ, phải gửi gắm vào trong văn chương, trong thi ca, thì mới đủ ngôn ngữ để nói. Chúng ta nên đọc lại vài đoạn của bài thơ này; tôi trích lại nửa phần đầu của bài thơ, tôi rất tiếc là nếu tác giả chỉ dừng lại ở nửa phần đầu này thì bài thơ đã như một viên ngọc quý. 
 
\textit{1.} 
 
\textit{Xưa một kẻ u cư} 
\textit{Bốn mươi đầu bạc trắng} 
\textit{Ta thân xác phần thư} 
\textit{Đời tro than nguội lạnh.} 
 
\textit{Chữ nghĩa đã hàm oan} 
\textit{Tâm kiệt cùng mực cạn} 
\textit{Ẩn mật chút men trong} 
\textit{Cất lòng sầu vô hạn.} 
 
\textit{Hầm tối tháng ngày qua} 
\textit{Nghe hạc vàng nhớ bạn} 
\textit{Lưu lạc nơi xứ người} 
\textit{Sách cùng ta chuyện vãn.} 
 
\textit{2.} 
 
\textit{Chuyện ta mùa hạ đỏ} 
\textit{Mưa máu đẩy thuyền ma} 
\textit{Về đâu trời đất tận} 
\textit{Tìm không một mái nhà.} 
 
\textit{Mấy năm rồi ngóng đợi} 
\textit{Bằng hữu biệt muôn phương} 
\textit{Có chiều ta xén cỏ} 
\textit{Lệ rơi trong góc vườn.} 
 
\textit{Có chiều thương bút mực} 
\textit{Bàn viết như mồ hoang} 
\textit{Yên nằm hồn lệ quỉ} 
\textit{Chờ ý xuống hộ tang.} 
 
\textit{3.} 
 
\textit{Quỉ ơi đời giấy trắng} 
\textit{Chờ ngươi đã nhiều năm} 
\textit{Có nghe nghìn xác sóng} 
\textit{Tìm nhau ngoài hư không.} 
 
\textit{Xa nhau bờ Nam Hải} 
\textit{Gặp nhau lòng Biển Đông  } 
\textit{Sinh ly thà thủy biệt} 
\textit{Quê hương thà lưu vong.} 
 
\textit{Chuyện ta hờn chí nhỏ} 
\textit{Tâm ta ừ tâm tang.} 
 
 


\end{multicols}
\end{document}