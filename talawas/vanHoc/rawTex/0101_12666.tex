\documentclass[../main.tex]{subfiles}

\begin{document}

\chapter{Thơ Viên Linh của thời lưu vong thất tán}

\begin{metadata}

\begin{flushright}24.3.2008\end{flushright}

Huỳnh Hữu Ủy

Nguồn: Tạp chí Văn số 95 & 96, tháng 11&12, 2004

\end{metadata}

\begin{multicols}{2}

\textbf{Thơ của một thời điêu tàn chưa từng thấy: Thuỷ Mộ Quan} 

Lưu lạc đất khách và phải nhập vào cuộc sống mới để tồn tại, nhưng trái tim người lưu vong vẫn đập nhịp thổn thức với quê nhà. Nên khi những đợt người ngày càng nhiều, tiếp tục ra đi trên những chiếc thuyền mong manh, thách thức với mọi nỗi gian truân, thách thức với định mệnh, giữa cái sống cái chết chỉ còn là một sợi tơ mong manh, biển cả thực là bất nhân và con người cũng thực là bất nhân, thì lúc bấy giờ người lưu vong ấy, người thơ ấy, càng là người nhạy cảm nhất để sống với nỗi đau thương vô cùng tận đó. Anh gõ cửa trái tim mà hỏi lại nhiều điều. Không còn là riêng tư nữa, mà đã là nỗi đau chung, là cộng nghiệp, là tan nát và đau thương. Biển Đông trở thành nấm mồ vĩ đại của đồng bào anh, những cảnh tượng bạc ác xẩy ra hàng ngày ngoài biển chấn động cả trời đất, đụng vào nỗi thương cảm sâu xa nhất nơi lương tâm con người. Đó là thời kỳ anh phải dồn hết tất cả sức lực để cô đúc chữ nghĩa, nhà thơ làm nhân chứng của một thời đại tàn khốc. 

Khổ đau chồng chất như núi, những kiếp người trầm luân giữa cảnh tử sinh. Khổ đau đúc kết thành nghệ thuật, trái đau khổ chín đỏ trên cây nghệ thuật xanh tươi. Dù để nói về sự đau khổ đã lên đến tận mấy tầng trời, người thơ cũng phải vận dụng kỹ thuật và những qui luật của thơ. Vậy nên, viết về cái đau thương thì cũng phải viết thành lời đẹp đẽ, trau chuốt, có như thế mới làm thành văn chương, mới cảm được lòng người. Đó là thời Viên Linh viết những vần thơ trác tuyệt nhị thập bát tú, tên gọi của Vũ Hoàng Chương để chỉ thể thơ 7 chữ, viết trong 4 dòng. Thể thơ này đòi hỏi sự tinh luyện, cô đọng và hàm chứa. Từ thể thất ngôn, Vũ Hoàng Chương chỉ dừng lại ở bốn câu mà dựng nên bầu trời thơ với những cụm 28 vì sao lấp lánh. Tiếp tay Vũ Hoàng Chương là Viên Linh, đặc biệt với \textit{Thuỷ Mộ Quan}, để gửi thêm vào bầu trời của thi bá họ Vũ những chùm sao kỳ lạ lấp lánh. Có lẽ cũng nên nhân đây, nhắc thêm một người khác nữa là Mai Thảo, với \textit{Ta thấy hình ta những miếu đền \footnote{
Mai Thảo, \textit{Ta thấy hình ta những miếu đền}, Văn Khoa, California, 1989.} }. Mai Thảo tiếp bước theo Vũ Hoàng Chương và Viên Linh để mở rộng chân trời nhị thập bát tú đến một cõi định hình, thành giòng nhị thập bát tú cuối thế kỷ XX. 

Trong bối cảnh như vậy, để hoàn tất \textit{Thuỷ Mộ Quan} với 171 bài thất ngôn tứ tuyệt và một bài kết thúc sau cùng là "Gọi hồn" theo thể tự do, Viên Linh đã phải đi ngược lại lịch sử, tìm về lại nơi những trang sách xưa, đặc biệt là huyền sử và thời sơ sử để phần nào, tự trong vô thức, như một giải thích về cái nghiệp hiện nay, về những cảnh tượng đang xẩy ra, cùng lúc nhà thơ ghi nhận về những điều đang là thảm kịch. Như nhà nghiên cứu và nhận định văn học Trần Văn Nam đã tóm lược trong một cái nhìn tổng thể: "\textit{Thuỷ Mộ Quan} của Viên Linh là những cảm hứng về biển Đông huyền ảo có giải đất rất huyền sử, rất đẹp dù suốt tập thơ là bóng tối của đáy vực Thái Bình Dương, là một Thuỷ Mộ bao la của người Việt ra đi bằng vượt biển." \footnote{
Trần Văn Nam "Văn Học Hải Ngoại như một món quà cho quê hương," trong \textit{Đóng góp với văn chương}, bản in hạn chế từ \textit{computer} riêng của tác giả, Walnut, California, 2004, trang 89.} 

Viên Linh đọc lại và suy gẫm những \textit{Lĩnh Nam Chích Quái, Việt Điện U Linh Tập, Tang Thương Ngẫu Lục, Vân Đài Loại Ngữ, Vũ Trung Tuỳ Bút}, rất đặc biệt là bộ sử quý \textit{An Nam Chí Lược} và vô số tài liệu khác. Quá khứ xa xăm và hiện tại trước mắt nhập lại thành một, điều này sẽ đưa tới một kết quả như Lê Huy Oanh từng nhận xét: "\textit{Thuỷ Mộ Quan} gồm hai sắc diện của Biển Đông, một sắc diện phiếu diễu mơ màng thắm tươi rực rỡ, nơi phát sinh và diễn tiến nhiều ngàn năm văn hiến của dân tộc Việt-Nam, đối tượng của một lịch sử vừa êm đẹp vừa oai hùng; sắc diện khác của nó, sắc diện mới, là một cảnh ghê sợ của các thuyền nhân, ... ..., trên đường vượt biên đi tìm tự do." \footnote{
Lê Huy Oanh, “Đọc thơ Viên Linh”, tức bài “Phê bình Thuỷ Mộ Quan” đăng ở báo Đồng Nai số ra ngày 25.8.1983, California, in lại trên Khởi Hành, Số 74, tháng 12.2002, California, trang 23.} 

Ngay ở bài thơ đầu tiên của \textit{Thuỷ Mộ Quan}, người đọc đã được dẫn vào một cõi trời nước mênh mông, trầm lặng, tĩnh mịch, mới trông thì có vẻ mộng ảo, nhưng nhìn xoáy vào thì sẽ thấy chứa đầy những nỗi hiểm nguy khôn lường. Giữa lòng đêm tối, quê nhà chỉ còn là một điểm đen mờ xa, và con đường đi tới sẽ là oan khiên, trầm luân, sẽ chồng chất mãi để chỉ còn những tiếng ma vang vọng dội lên. Kinh nghiệm và nỗi ám ảnh sâu thẳm nhất ở Viên Linh là bóng dáng Cúc Hoa, một hình ảnh ma, thì ngày nay trước nấm mồ bao la dưới đáy biển, sẽ càng như là một vùng màu mỡ vô hạn cho Viên Linh thuận tiện gieo trồng, đi tìm lại, hay khám phá cái cõi yêu ma địa phủ ấy, anh muốn đi qua mấy tầng địa ngục như Dante của thế kỷ XIII-XIV trước đây. Mặc dù cũng từng đã có Đinh Hùng với "Bài thơ chiêu niệm", với \textit{Mê hồn ca}, xa hơn nữa với "Văn tế tướng sĩ" rằm tháng bảy hay "Văn tế thập loại chúng sinh", chúng ta cũng dễ nhận ra rằng Viên Linh đã dựng nên một thế giới ma rất lạ của riêng anh. Tuy thế, ở đấy, vẫn có một cái gì luôn nối vào cuộc sống thực, như một hình ảnh thân yêu cũ, một quê nhà đã mất. \textit{Thuỷ Mộ Quan} được đánh số từ 1 đến 171, chúng ta thử đọc lại vài bài, số 1, 3, 125, 129, 139, và 160. 

\textit{1.} 

\textit{Một biển trôi xa nghìn đảo lặn} 
\textit{Trời mây vần vũ thủy mang mang} 
\textit{Xung quanh màu bạc, trong lòng tối} 
\textit{Điểm cuối quê nhà, một góc đen.} 

\textit{3.} 

\textit{Nằm mộng đêm nay vào Hoả Ngục} 
\textit{Kiếm người oan thác đã trầm xanh} 
\textit{Dưới hiên mưa vắng hồn khua nước} 
\textit{Thả chiếc thuyền con ngược bến không.} 

\textit{125.} 

\textit{Ngần ngại tìm em lúc cuối năm} 
\textit{Xuân sang le lói ý đèn nhang} 
\textit{Nửa đêm trừ tịch sầu ma quỉ} 
\textit{Năm cũ còn chong đuốc trước thềm.} 

\textit{129.} 

\textit{Đến cầu ao cũ ngắt rau xanh} 
\textit{Nhìn ngược hình dung đến hoảng kinh} 
\textit{Vẫn tưởng Quê Nhà tìm lại được} 
\textit{Quê Nhà đã mất lúc u minh.} 

\textit{139.} 

\textit{Từ đáy sâu trầm giạt tiếng chuông} 
\textit{Gọi người dương thế giúp âm công} 
\textit{Gọi ma bốn biển về chung sức} 
\textit{Gom góp san hô dựng giáo đường.} 

\textit{160.} 

\textit{Quê ta trầm thống nỗi đau dài} 
\textit{Xum họp chiều hôm biệt sớm mai} 
\textit{Đáy nước chia lìa sơn cốc tận} 
\textit{Miếu đường chuông đổ mộ hồn ai.} 

Chỉ với mấy bài thơ trên, chúng ta cũng đã thống kê được một số chữ đặc biệt về ma quỉ và địa ngục, để thấy cánh cửa mở vào và con đường dẫn qua \textit{Thuỷ Mộ Quan} sẽ như thế nào: Hoả ngục, oan thác, trầm xanh, hồn khua nước, ý đèn nhang, sầu ma quỉ, u minh, âm công, ma bốn biển, mộ hồn ai. Có thể nói đó là ánh sáng huyền ảo toát ra từ thi phẩm \textit{Thuỷ Mộ Quan}, một kết tinh đặc biệt của tâm hồn nhà thơ và thời đại bất thường với tiếng kêu than u uất của ma quỉ ngập đầy ngoài biển Đông. Tôi bỗng nhớ đến Vương Ngư Dương, cũng với bài thơ chỉ có hai mươi tám từ đề từ cho bộ sách vĩ đại của Bồ Tùng Linh, mà hai câu sau: \textit{Liệu ưng yếm tác nhân gian ngữ / Ái thính thu phần quỉ xướng thi }đã được Lê Đạt chuyển dịch sang Việt ngữ cực kỳ thần tình: 

\textit{Ngôn ngữ nhân gian chừng đã chán} 
\textit{Thèm nghe mộ vắng quỉ bình thơ} 

Nói chuyện ma của Vương Ngư Dương có cái gì mạnh mẽ, quỉ quái, và ngang ngược. Ở Viên Linh thì khác, thơ mộng và thần bí. Đúng là mỗi cảnh đời một khác, mỗi hồn người là một chỗ riêng tư. 

\textit{Thuỷ Mộ Quan} xây dựng tập trung trên bối cảnh Biển Đông nhưng đề cập đến nhiều vấn đề, nhiều góc cạnh, từ huyền sử, lịch sử, đến xã hội, văn hoá, tâm lý, văn học... Chúng ta hãy nói đến một trong những khía cạnh đó. Vì thể thơ nhị thập bát tú, cũng như thơ hài cú của Nhật Bản, hết sức là cô đọng nên đòi hỏi người viết phải vận dụng bút lực, rất khổ công, bài thơ ít chữ mà nói nhiều, tế nhị và hàm súc, có thể đạt đến những điều mênh mông đến không cùng. Nhị thập bát tú có nhiều lúc gần với kệ của các bậc đại tăng, như một công án thiền, hay ngay cả với sấm ký, cũng có lúc gần như một bài minh, bài trâm, hay bài tán. \footnote{
Có thể xem lại định nghĩa Minh, Trâm, Tán trong \textit{Văn phạm Việt Nam} của Trần Trọng Kim, Bùi Kỷ, Phạm Duy Khiêm, Nxb Tân Việt, bản in lần thứ tám, không ghi năm xb, trang 184-185.} \textit{Thuỷ Mộ Quan }cũng vậy, có một số bài gần với không khí ấy, đáng kể là thành công. Hãy đọc thử lại vài bài. 

\textit{55.} 

\textit{Xuân nào lụt lội khắp trung châu} 
\textit{Nước rút đầm hoang Phật xuất đầu} 
\textit{Phật nổi từ xưa là Phật gỗ} 
\textit{Hèn chi Phật có đắm chìm đâu.} 

\textit{111.} 

\textit{Tầm sư học đạo bốn mươi năm} 
\textit{Thân thế gian nan chữ nghĩa cùng} 
\textit{Mạt lộ lần lưng tìm bí kíp} 
\textit{Một tờ giấy nhảm viết lung tung.} 

\textit{156.} 

\textit{Không biết bao giờ. Biết có chăng.} 
\textit{Biết đâu ngày tháng. Biết đâu năm.} 
\textit{Hoàng hôn nào biết. Đêm sao biết.} 
\textit{Thời khắc lưu cầm. Phút hỗn mang.} 

\textit{171.} 

\textit{Cái chết nhiều khi thấy thật xa} 
\textit{Chết từ trong lửa chết trong hoa} 
\textit{Chết trong trầm ải trong trăng tịch} 
\textit{Và có đâu ngờ trong chính ta.} 

Khởi từ một kinh nghiệm cá nhân, \textit{Thuỷ Mộ Quan} đi tới thảm kịch lớn của dân tộc trong một thời điểm rất đặc biệt của đất nước. Giấc mộng, kỷ niệm, và niềm đau riêng của nhà thơ được nhập vào giấc mơ và bi kịch chung của đất nước. Một số thơ chung quanh chủ đề lịch sử và hiện thực ngoài biển Đông được nhiều người khen ngợi, như Lê Huy Oanh cho rằng Viên Linh đã góp vào kho tàng thi ca Việt Nam một loại thơ với những hình ảnh mới lạ chưa hề có trước 1975. \footnote{
Lê Huy Oanh, Khởi Hành, số 74, đã dẫn ở trên, trang 24.} Tôi không trích dẫn lại những bài thơ ấy ở đây vì thấy có phần không toàn hảo: Vì Viên Linh nuôi một ý tưởng quá lớn khi dựng lại bầu khí \textit{Thuỷ Mộ Quan} nên ở đôi chỗ anh phải hy sinh chất thơ để đạt cho được ý đồ kết cấu. Nó chỉ còn là một thứ quốc sử diễn ca chứ không phải là thơ nữa. Về chủ điểm lịch sử, có đôi chỗ Viên Linh rớt vào bệnh dân tộc chủ nghĩa, là căn bệnh dường như không còn được thích hợp trong ánh sáng nhân bản của thời đại mới, như căn bệnh Eurocentrism, lấy phương Tây làm trung tâm và thước đo của văn minh nhân loại, ngày nay hầu như đã bị loại bỏ hoàn toàn. 

Bỏ qua những nhược điểm, bỏ qua những bài thơ không hay vì làm được thơ hay thì đâu phải dễ, cứ lật truyện\textit{ Kiều} ra cũng đủ thấy, biết bao nhiêu là thơ dở trong đó, huống hồ \textit{Thuỷ Mộ Quan} đánh số đến 171 thì có vài chục bài hay, hoặc chừng mười bài hay, thậm chí chỉ vài bài hay thì cũng đã là thành tựu lớn rồi. 

Chúng ta hãy đọc thêm vài bài khác nữa. 

\textit{7.} 

\textit{Lúc nhỏ anh em thường đánh lộn} 
\textit{Bây giờ sông núi nhớ thương nhau} 
\textit{Ngó xem vết sẹo bàn tay trái} 
\textit{Bên phải đầu tôi bỗng nhói đau.} 

\textit{110.} 

\textit{Tuổi trẻ nghe mưa mộng hải hồ} 
\textit{Mộng đi bốn biển sống phiêu du} 
\textit{Hôm nay mưa tuyết quê người lạnh} 
\textit{Ta mộng quay về ngõ hẻm xưa.} 

\textit{137.} 

\textit{Em có hai chân đẹp tựa men} 
\textit{Hai tay như ngọc tiếng như chim} 
\textit{Em yêu như mãn gào trên ngói} 
\textit{Tuy vậy em cần một trái tim.} 

\textit{117.} 

\textit{Thăm thẳm trời cao thăm thẳm sâu} 
\textit{Mênh mông sông nước mênh mông sầu} 
\textit{Nhỏ nhoi một chiếc thuyền không lái} 
\textit{Không biết về đâu không biết đâu.} 

\textit{108.} 

\textit{Cửa ngục A Tỳ ở biển Đông} 
\textit{Xưa kia Phật doạ rộng vô chừng} 
\textit{Nào hay Phật chỉ mơ hồ biết} 
\textit{Ngục ấy nay to gấp vạn lần.} 

\textit{78.} 

\textit{Sinh ở đâu mà giạt bốn phương} 
\textit{Trăm con cười nói tiếng trăm giòng} 
\textit{Ngày mai nếu trở về quê cũ} 
\textit{Hy vọng ta còn tiếng khóc chung.} 

Tất cả kỷ niệm, huyền sử, lịch sử, tất cả đau thương và địa ngục kia, nói cho cùng, cũng đều là vốn liếng và tài sản của đất nước, bởi vì tất cả những điều ấy đã cùng tập hợp thành chiều sâu và sức mạnh tâm linh của dân tộc, là kinh nghiệm và quá trình của hôm nay để chuẩn bị cho ngày mai phải tới. Bầy chim bỏ xứ bay đi khắp trời đất, rồi cũng đến lúc phải trở về chốn cũ, quên đi những ngày mưa tuyết quê người, trăm con sẽ cùng giòn giã trong một tiếng cười giọng nói chung, và tiếng khóc chung. Lời “Gọi hồn” của \textit{Thuỷ Mộ Quan} phần nào có gần gũi với “Văn tế thập loại chúng sinh”, nhắc đến tập tục cầu siêu thoát cho mười loại cô hồn uổng tử, nhưng khác hẳn với bản văn tế ấy, vì nó không phải chỉ là lời cầu khi lập đàn cho các hồn ma bóng quế vật vờ, mà “Gọi hồn” là một lời kêu cầu sum họp trong truyền thống đại gia đình Việt tộc. Tôn giáo chính yếu của người Việt là thờ cúng tổ tiên, nên ở những cuộc họp mặt thiêng liêng, như trong mấy ngày lễ tết, thì mọi người trong gia tộc bao giờ cũng ngưỡng vọng lên bàn thờ ông bà, như người khuất bóng đang có mặt và đang linh thiêng chứng giám đời sống của đàn con cháu. 

“Gọi hồn” viết theo thể thơ tự do, dù nhạc điệu cũng là yếu tố được chú tâm, nhưng không quá thê thiết và trầm buồn như bài ngâm khúc “Văn tế thập loại chúng sinh”. Tôi được nghe Phạm Duy hát bài ca “Gọi hồn” do chính ông phổ nhạc; tiết tấu nghe chừng như dồn dập và hân hoan, niềm vui nỗi buồn pha trộn, dưới sự tiết chế và bao trùm bởi một thứ ánh sáng rất trí tuệ, để nói về cuộc đoàn viên đẹp đẽ của đại gia đình dân tộc. Tôi tưởng như đó là lời bổ túc cho \textit{Tổ khúc Bầy chim bỏ xứ} trước đây, mới chỉ nói về sự trở về của mấy triệu người sinh cơ lập nghiệp phương xa, kéo nhau đi về quê cũ, mà quên nói về những người vắng mặt đang chìm đắm nơi một cõi âm u nào. 

\textit{Thấp thoáng trần gian} 
\textit{Mịt mù bóng đảo} 
\textit{Trôi về Tây về Bắc về Đông} 
\textit{Trôi về đâu bốn bề thuỷ thảo} 
\textit{Về đâu kiếp đắm với thân trầm.} 

\textit{Hồn ơi dương thế xa dần} 
\textit{Hồn đi thôi nhé thuỷ âm là nhà.} 
\textit{Hồn về trong cõi hà sa} 
\textit{Sống không trọn kiếp chết là hồi sinh.} 
\textit{. . . . .} 
\textit{. . . . .} 
\textit{Hồn vẫn ở la đà Nam Hải} 
\textit{Hồn còn trôi mê mải ngoài khơi} 
\textit{Hồn còn tầm tã mưa rơi} 
\textit{Tháng Tư máu chảy một trời xương tan.} 
\textit{. . .} 
\textit{. . .} 
\textit{. . .} 
\textit{. . .} 
\textit{Về đâu đêm tối} 
\textit{Hương lửa lung linh} 
\textit{Những ai còn bóng} 
\textit{Những ai mất hình} 
\textit{. . .} 
\textit{. . .} 
\textit{. . .} 
\textit{Ta vào lục địa ta hồi cố hương} 
\textit{Cùng nhau ta dựng lại nguồn} 
\textit{Chẻ tre đẵn gỗ vạch mương xây đình.} 
\textit{. . .} 
\textit{. . .} 
\textit{. . .} 
\textit{. . .} 
\textit{. . .} 
\textit{Năm nghìn năm lại bắt đầu} 
\textit{Chim nào tha đá người đâu vá trời.} 
\textit{Chúng ta rời bỏ xứ người} 
\textit{Loài chim trốn tuyết qui hồi cố hương.} 


\textbf{Sau thời Thuỷ Mộ Quan cho đến bây giờ }

Thơ Viên Linh như chúng ta đã khảo sát ở bên trên, từ thời tuổi trẻ với những bước chân hăng hái tiến vào cõi văn chương, nồng nhiệt, hăm hở, muốn đi tìm một cái gì thực mới, thực khác, đập phá càng tốt, cần phải chống lại những giá trị cũ, chống lại truyền thống, nghĩa là phải hiện ra trong một cung cách nổi loạn, chống đối, khác người. Nhưng Viên Linh đã rất mau chóng tìm lại được con đường của mình, tìm được sự ổn định trong tư tưởng. Anh vượt qua nhanh chóng những cơn sóng gió, bão táp phù phiếm của chữ nghĩa, để dựng nên thi giới của mình. Anh mỉm cười và dường như chẳng cần biết đến, chẳng lưu tâm chút gì những thứ gọi là hiện đại, hay đằng sau, sau nữa của cái hiện đại ấy. 

Hình như từ sau thời 25 tuổi cho mãi đến ngày nay, anh đã bước đi rất vững chắc trên con đường văn chương. Anh đến gần với các hiền giả phương Đông, anh mê Trang Tử, Lão Tử, và càng ngày càng nghiệm ra được nhiều điều kỳ lạ vô cùng ở kinh Phật. Anh đọc lại, ngẫm nghĩ kỹ lưỡng và gậm nhấm từng hình ảnh, ý nghĩa, thi tứ, tiếng vang trầm và sâu của chữ nơi các bậc tiền hiền từ bao nhiêu đời trước, những Nguyễn Phi Khanh, Nguyễn Trãi, Nguyễn Bỉnh Khiêm, Chu Văn An, Nguyễn Khuyến... 

Nếu người xưa cho rằng giản dị là cảnh giới tận cùng của văn chương, thì mỗi ngày bước tới là mỗi ngày Viên Linh càng ham muốn tiến đến và sống với sự thật ấy. Muốn đạt được điều ấy thì không phải dễ, nhưng hãy cứ sống, cứ rèn luyện và tiến bước. 

Thời của \textit{Thuỷ Mộ Quan} cũng đã giản dị rồi, nhưng bởi vì chồng chất bao nhiêu tầng đau khổ ở đấy, nên tự nó đã toả ra một cái gì đó còn u ám, tối tăm. Chuyển qua thời tiếp theo, Viên Linh đi hẳn vào một không khí rất giản dị của cái đẹp, khai thác tất cả sở trường của mình, phát huy được sức mạnh nội tâm, tình yêu đất nước, nền văn hoá thâm sâu của dân tộc. Một thế giới thi ca đầm ấm, trang nhã được xây dựng từ nền tảng đó, hy vọng sẽ góp được một nhành lộc mới cho đất nước đến muôn thu. Sống ở nước ngoài đã mấy mươi năm, anh vẫn thấy mình là một kẻ lạc đường. Đi trên chuyến tàu từ Paris qua Frankfurt, cũng kể là một chuyến đi thơ mộng cho những người ưa thích giang hồ, chỉ với chiếc túi da đã cũ đã sờn, leo lên toa tàu là đi, nhưng cái thời mê giang hồ đã không còn, con đường ngày nay phải là hành trình qui cố hương, nên khi tàu đến và tiến vào sân ga, anh tự nhận ra mình chỉ là một kẻ xa lạ, chỉ là thứ hành lý gửi lầm đến một nơi không đúng chỗ. Đấy là chưa nói đến những khía cạnh khác về mặt xã hội, nhu cầu hội nhập, đời sống của một người cầm bút lưu vong, con đường của một nền văn học lưu vong v.v... và v.v... Đã đành rằng thế giới ngày nay đã thu hẹp lại, nhất là khi ngà ngà trong hơi men và chất rượu cay, thì đâu cũng là nhà. Nhưng tự thâm sâu, khi người ta đánh mất nguồn cội của mình, đánh mất tình yêu của mình, mà với anh tình yêu cũng chính là tổ quốc, và hình bóng tình yêu chỉ còn là ảo vọng, thì chuyến tàu ấy sẽ không còn nơi để dừng lại và anh sẽ lạc lõng hoài giữa những sân ga. 

\textit{Trên tàu hoả Paris-Frankfurt} 

\textit{1.} 

\textit{Thiếu em, thơ thiếu một dòng,} 
\textit{Lời ca thiếu nhịp, trong lòng thiếu vui} 
\textit{Tàu đi, tiếng sắt bùi ngùi} 
\textit{Đáy toa gió giật bóng người lùi nhanh.} 

\textit{Thiếu em, lan thiếu một nhành} 
\textit{Tay dư mười ngón, bóng hình dư gương} 
\textit{Bánh lăn, trục cuốn chiếu giường} 
\textit{Một nghìn cửa sổ thiếu đường tìm em.} 

\textit{Tầu êm, rượu rủ vào đêm} 
\textit{Ly men rót mãi cũng mềm lòng ga} 
\textit{Rượu say, đâu cũng là nhà} 
\textit{Hai thanh đường sắt thế mà gặp nhau.} 

\textit{2.} 

\textit{Chim bay từ Bắc sang Nam} 
\textit{Mặt trời đang lặn, nỗi hàn đang xa.} 
\textit{Em ơi từ lúc phôi pha} 
\textit{Mặt trăng càng tỏ sân ga càng gần.} 

\textit{Con tàu lặng lẽ vào sân} 
\textit{Anh là hành lý gửi lầm đến đây.} 

Cảm hứng chính của Viên Linh càng về sau càng cho chúng ta thấy rõ, nền tảng là cội rễ dân tộc, và mặt khác, là ngọn lửa tình yêu riêng tư từ trong bản thân, ngọn lửa ấy cũng có thể chính là mùa hoa địa ngục rực rỡ qua hình bóng Cúc Hoa. Hai nguồn cảm hứng ấy ngày càng làm phong phú đến vô hạn thi giới Viên Linh. Viết đến đây, tôi không thể kiềm chế mình để không trích dẫn thêm một bài khác nữa Viên Linh viết về Cúc Hoa trong những năm sau này. 

\textit{Đêm khuya nghe tiếng gió lùa} 
\textit{Lắng trông ngoài cửa mơ hồ bước ma} 
\textit{Phải chăng em? Hỡi Cúc Hoa} 
\textit{Nửa đời tan tác một nhà nhớ mong} 

\textit{Em đi, đã chục năm ròng} 
\textit{Bánh xe lăn vội, chuyện lòng chìm mau} 
\textit{Tình ta, kìa đáy giếng sâu} 
\textit{Mạn thuyền sông Hậu, nhịp cầu Tiền Giang \footnote{
Câu này tác giả đã sửa lại, -nhưng rồi lại thôi- song tôi vẫn in như lần bài thơ xuất hiện đầu tiên trên Khởi Hành số 93. Câu sửa là: Mạn thuyền Địa phủ, nhịp cầu Dương gian.} }

\textit{Em đi, trời đất bàng hoàng} 
\textit{Cơn mưa tầm tã, cũ càng gối chăn.} 
\textit{Đã nửa đời. Đã bao năm?} 
\textit{Mái xưa anh vẫn tìm thăm bóng người.} 

\textit{Tấm hình em, thuở chia đôi} 
\textit{Bước chân em,} 
\textit{có phải người ngoài hiên?} 
\textit{Đêm nay, tiếng động ngoài thềm} 
\textit{Phải chăng em đã tới miền hoá sinh?} 

Thơ Viên Linh càng về sau càng giản dị mà rực rỡ và đẹp lạ lùng. Cái giản dị đó cho chúng ta biết là nhà thơ đã dồn cả đời mình vào mà tập luyện, đẽo gọt, điêu khắc với chữ, cho đến lúc chữ đến như ma quỉ chỉ đường, như tiếng sét đánh ngang trời, và chữ tuôn tràn ra thành từ khúc, vần điệu. Cách đây mấy hôm, Viên Linh sao chụp lại từ mấy trang bản thảo của anh để tặng tôi bốn câu thơ ngắn nói về điều ấy, một loại định nghĩa về cảm hứng và sáng tạo, như Hồ Dzếnh với “Phút linh cầu” hay Hàn Mạc Tử "Tôi làm thơ? - Nghĩa là tôi nhấn một cung đàn, bấm một đường tơ, rung rinh một làn ánh sáng." 

\textit{Không biết câu thơ tới lúc nào} 
\textit{Chỉ nghe trong gió chút âm hao} 
\textit{Lung linh giọt lệ đêm khuya dậy} 
\textit{Khóc lặng mà không hiểu tại sao.} 

Cảm hứng đến như một tia chớp, một tiếng sét, có lúc chỉ như cơn gió nhẹ thoảng qua, hay giọt lệ lung linh giữa đêm khuya chỉ trong một sát na, để truyền đi tất cả sức mạnh vô bờ của sự sáng tạo. Ở Viên Linh, cái sát na đó là trái chín của sự tích lũy, tu dưỡng của bao nhiêu năm tháng, rồi đến một lúc nào đó bất thần hiện ra, và nhà thơ sẽ còn phải "thôi xao" cho đến lúc nó hiện ra trong cái toàn vẹn của chính nó mới thôi. Ở Viên Linh, làm công việc ấy có nghĩa là phải đạt cho được cái đẹp của sự giản dị, tất nhiên để đạt đến cái giản dị đó thì phải cực kỳ công phu và hàm dưỡng. 

Tôi chưa thấy một người làm thơ nào nuôi một tứ thơ đến 20 năm, và đến mấy ngày gần đây mới viết ra được với những cảm xúc đặc biệt vẫn chôn giữ trong lòng suốt hai thập niên qua. Làm thơ nói riêng, hoạt động nghệ thuật nói chung, là phải rèn tập, nuôi dưỡng, chứ chẳng thể nào ăn xổi ở thì. Tựa như Trịnh Bản Kiều, một hoạ sĩ đời Thanh, chuyên vẽ lan và trúc suốt hơn 50 năm, rèn luyện chuyên tâm cho nên lúc cầm bút vẽ thì tự nhiên đã thành tre trúc. Nhìn trúc của Trịnh Bản Kiều, chúng ta biết ngay là trúc mùa hè, mùa thu, hay mùa đông. Làm thơ cũng thế, lấy tỉ dụ với lục bát của Viên Linh. Mới trông thì tưởng dễ bởi vì thể lục bát đã có sẵn âm thanh, vần điệu, chỉ cần ráp hình ảnh mới lạ vào là được, nhưng đâu có phải như vậy. Hãy chậm rãi đọc lại và thưởng thức hai bài lục bát “Chữ nghĩa” và “Tạp Thi I” của Viên Linh, để thấy rằng lục bát của Viên Linh, một khía cạnh của thơ Viên Linh, đã đi tới cái đẹp giản dị mà vô cùng thanh tú và tao nhã, và như vậy, viết cho thành một bài lục bát hay cũng đâu phải là chuyện dễ, cũng như nét tre trúc của Trịnh Bản Kiều vậy, đã là ma quỉ hiện hình với tất cả cái thần tướng linh diệu của nó. 

\textit{Chữ nghĩa }

\textit{Đêm qua thơ hỏi ta rằng} 
\textit{Người ơi vần điệu vô hằng còn không?} 
\textit{Trái tim người có còn hồng} 
\textit{Nhánh cây đau khổ có trồng vườn ai?} 

\textit{Trái tim ta đã ở ngoài} 
\textit{Vườn ta thảo mộc u hoài từng cây.} 
\textit{Sáng nay chữ hỏi câu này} 
\textit{Người ơi Ý Tứ còn đầy hay vơi?} 

\textit{Chân phương Ý ở trong đời} 
\textit{Hoài nghi Tứ đã ra lời này kia.} 
\textit{Chữ ta từ Nghĩa ra đi} 
\textit{Tâm ta chỉ hiểu phân ly là nhà.} 

\textit{Chập chờn trong sách là ma} 
\textit{Tấm chân diện mục là hoa trái mùa.} 

\textit{Đêm qua tầm tã cây mưa} 
\textit{Văn chương vô mệnh hoang sơ lắm rồi.} 
\textit{Hỏi ta đừng hỏi bằng lời} 
\textit{Một cây rụng lá vườn trời không bay. \footnote{
"Lục Bát Viên Linh", Khởi Hành, Xuân Quí Mùi, số 75-76, 2003, trang 21.} }

\textit{Tạp thi I} 

\textit{1.} 

\textit{Sừng sững như núi như rừng} 
\textit{Mênh mông trang sách cánh đồng cổ xưa} 
\textit{Thiên thu một mối mơ hồ} 
\textit{Bao nhiêu mùa gặt chưa vừa bụng ta.} 

\textit{2.} 

\textit{Nửa đêm nghe động ngoài thềm} 
\textit{Thắp đèn mở cửa ngó mình trân trân} 
\textit{Xóm người, ma quỉ nào thăm} 
\textit{Trở vào đã thấy bóng trăng giữa nhà.} 

\textit{3.} 

\textit{Văn là đẹp vẽ là văn} 
\textit{Sử truyền ta vẽ từ năm xuống thuyền} 
\textit{Vẽ con cá sấu lên mình} 
\textit{Đôi khi nghệ thuật mạo hình quỉ ma. \footnote{
"Lục Bát Viên Linh," Khởi Hành, Xuân Quí Mùi, số 75-76, 2003, trang 21.} }

Sống với thơ như Viên Linh có lẽ cũng là chuyện hiếm trên đời, đó cũng là trường hợp của những Nguyễn Đức Sơn, Bùi Giáng, Trần Tuấn Kiệt, Tô Thùy Yên, Phan Nhiên Hạo, Đỗ Kh., Nguyễn Đăng Thường... Viên Linh đánh đổi cuộc đời mình cho thơ, anh mãi hoài đi tới với thơ, mơ mộng với thơ, kiên gan với thơ, và chẳng bao giờ chịu đứng lại. 

\textit{Tôi đợi nhiều năm chẳng thấy người} 
\textit{Nói làm chi nữa. Nói sao nguôi} 
\textit{Cái bay lồng lộng ngoài muôn dặm} 
\textit{Cái đứng chôn vùi bia mộ thôi.} 
("Cái đứng", bản thảo chưa xuất bản) 

Viên Linh sống với thơ, nghĩa là anh đào sâu mãi vào những bí mật của sự sống và của chính bản thân mình, sống trong sự cô đơn vô cùng tận, để sẵn sàng chộp bắt ánh lửa của sự sáng tạo. Đó là phong cách của Viên Linh từ thuở đầu đến với thơ cho mãi đến ngày nay. "Sống ở đời, đầy sự nghiệp, nhưng sống như một thi sĩ", ý thơ của Holderlin mà Heidegger sử dụng để chú giải về hữu thể, chúng ta cũng có thể dựa vào nền tảng đó để khám phá lại Viên Linh; một dòng thơ mênh mông đại hải chảy qua gần nửa thế kỷ cùng với chiều dài đất nước. Dòng sông thi ca ấy vẫn còn tiếp tục chảy ngang đời sống chúng ta. 

Và hôm nay, nó uốn khúc tới cảnh giới tận cùng của văn chương: ngày càng giản dị, sáng sủa, đơn thuần mà tiềm ẩn một vẻ rực rỡ mênh mông vô hạn. 

Thành phố Vườn, 
Tháng 8.2004 


\textbf{Vài nét về Huỳnh Hữu Ủy}, theo báo \textit{Khởi Hành} 

Huỳnh Hữu Ủy sinh năm 1946 ở Huế, quê quán Hiền Lương, Phong Điền, Thừa Thiên. 

Trước 1975, đi lính, tòng sự tại Khối Quân Sử/PS/Bộ Tổng Tham Mưu Sài Gòn, từng tham gia biên soạn một số sách Chiến sử và Lịch sử quâ n lực Việt Nam Cộng hoà. 

Sở thích đặc biệt nhất là viết về mỹ thuật, khởi đầu với tiểu luận "Đường bay của nghệ thuật" in trên Tạp chí Văn của Nguyễn Đình Vượng và Trần Phong Giao, số 93, \textit{Đặc biệt về hội hoạ}, 1967, Sài Gòn. Từ đó liên tục viết về Mỹ Thuật, có nhiều bài viết công phu về Nghệ thuật Việt Nam đương đại. Sách đã xuất bản: \textit{Nghệ thuật tạo hình dân gian Việt Nam}, Hồng Lĩnh, Calif., 1993. \textit{Mấy nẻo đường của nghệ thuật và chữ nghĩa}, Văn nghệ, Calif., 1999. 


\textbf{Đọc thêm}: Huỳnh Hữu Ủy: "Viên Linh trên những chặng đường thơ\footnote{\url{http://www.talawas.org/talaDB/http://vanmagazine.saigonline.com/HTML-H/HuynhHuuUy/HuynhHuuUyVanVienLinhTrenNhungChangDuongTho.php}}" 




\end{multicols}
\end{document}