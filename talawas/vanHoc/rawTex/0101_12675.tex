\documentclass[../main.tex]{subfiles}

\begin{document}

\chapter{Nghĩ về đổi mới thơ từ trường hợp của Hàn Mặc Tử}

\begin{metadata}

\begin{flushright}25.3.2008\end{flushright}

Trần Thiện Khanh



\end{metadata}

\begin{multicols}{2}

\textbf{1. Hàn Mặc Tử: một đỉnh núi lạ} 
 
Từ địa hạt thơ Đường bước sang lãnh địa \textit{thơ lãng mạn} rồi \textit{thơ tượng trưng}, Hàn Mặc Tử đã có đóng góp không nhỏ cho công cuộc cách tân thi ca Việt Nam. 
 
Thơ của Hàn Mặc Tử không chỉ mới ở thi tứ và ngôn từ, mà còn mới ở cách thức giải phóng yếu tố cá nhân trong những giấc mơ vô thức, ở sự thể hiện "vũ trụ tinh thần" bí ẩn hoàn toàn siêu nghiệm, siêu linh. Hàn Mặc Tử cùng với nhiều nhà thơ khác trong \textit{Trường thơ Loạn} và nhóm \textit{Xuân Thu nhã tập} đã đổi mới phương thức trữ tình bằng cách \textit{kéo gần thơ tới âm nhạc}. Thi sĩ "dùng chiếc sáo của mình, chơi những điệu mình thích" (Mallarmé), biến \textit{nhạc thơ} thành một thứ \textit{nhạc chiêu hồn}, gợi lên những sắc thái tinh tế nhất của tâm trạng và những cảm niệm mơ hồ, kì lạ. Vườn thơ của Hàn Mặc Tử "rộng rinh không bờ bến". 
 
Nhưng vườn thơ của Hàn Mặc Tử có phải được dựng lên một cách dễ dàng? Hoài Thanh kể: "Đương thời người ta mạt sát Hàn Mặc Tử nhiều lắm, họ bảo: Hàn Mặc Tử thơ với thẩn gì, toàn nói nhảm." Còn Xuân Diệu, sau khi tuyên bố vẻ chắc chắn: "Hàn Mặc Tử không phải hạng "chân thi sĩ"", đã thẳng thắn đề nghị: "Người thơ ấy tốt hơn cứ tỉnh táo mà "yên lặng sống"." Chưa ai công bằng khi đứng trước tài thơ, nguồn thơ lạ lùng của Hàn Mặc Tử. Chưa ai công nhận những câu thơ siêu linh-mới cho đến tận hôm nay. 
 
Chỉ có Chế Lan Viên sớm nhìn ra tài thơ, con đường thơ của thi sĩ họ Hàn. Ông nói: "Mai kia, những cái tầm thường, mực thước sẽ mất đi, còn lại chút gì đáng kể của thời này, đó là Hàn Mặc Tử." Lời tiên đoán ấy, ngoài Chế Lan Viên, không ai viết nổi. Phải can đảm lắm, Chế Lan Viên mới viết lời giới thiệu xác quyết mạnh mẽ nhường đó. 
	 
Thì ra công cuộc đổi mới thơ nào cũng đầy thử thách, đòi hỏi người nghệ sĩ phải dám \textit{dấn thân}. Có niềm say mê, khát khao thôi, chưa đủ, mặc dù điều đó rất đáng quí. Tài năng ư? Dĩ nhiên cần, nhưng chưa xong. Đổi mới thơ sẽ trở thành câu chuyện phù phiếm, viễn tưởng nếu người nghệ sĩ thiếu đi phông văn hoá cần thiết, thiếu đi bản lĩnh giải phóng tư tưởng của mình và tư tưởng của con người nói chung ra khỏi những "điều cấm kị" vốn đang trở thành thiết chế khắc nghiệt nhất đối với kẻ cầm bút. 
 
Tiền đề của đổi mới thơ, phải chăng bắt nguồn từ sự khám phá ra một thế giới văn hoá trong thế giới nhân sinh, thế giới của sự tự do dân chủ. Câu chuyện cách tân văn chương đến nay và mai sau vẫn luôn xoay quanh vấn đề tư tưởng, quan điểm của nghệ sĩ đối với thực tại, đối với sự sống.  
 
Chẳng bao giờ có nhà nghệ sĩ lớn, nếu anh ta không được \textit{sáng tạo tự do} - trong ý nghĩa nghiêm ngặt và đời thường nhất của nó. Thử hình dung thế này: một người "cổ đeo gông, chân vướng xiềng" thì sản phẩm của anh ta cả khi còn trong trứng nước lẫn khi chào đời - khao khát được sống với đời sống riêng của nó, sẽ không thể vượt quá giới hạn thực tế cho phép. Chừng nào tư tưởng, ngôn ngữ còn bị gông xiềng trói buộc thì chừng đó còn có nhiều bi kịch. Một số độc giả thích "sự nổi loạn" trên tất cả các cấp độ của nghệ thuật ngôn từ. Nhưng người thơ ít tạo ra "sự nổi loạn" đáp ứng mong mỏi của họ. Không hiểu sao tôi thích \textit{Trường thơ Loạn}, thích "sự điên" của người làm thơ. Phải chăng vì trong sự điên ấy - theo cách nói của Hàn Mặc Tử - những bí mật của con người được phơi bày ra đầy đủ nhất, chân thành nhất. Phải chăng nhờ "sự điên cố ý" ấy, tôi và các bạn đọc khác được biết đến một thế giới khác - thế giới của vô thức, siêu linh, thế giới của linh hồn, ý niệm. Chứ không hẳn tôi tò mò, vì điều đó sẽ chóng qua đi. Cũng có thể hiện tượng "điên loạn cố ý" của người cầm bút đã tạo ra \textit{cảm giác lạ, nhận thức lạ} trong khi những người giáo điều, bảo thủ không thể đem lại điều đó. Đến đây thì không hẳn tôi ủng hộ "người phá phách ngôn từ", vì tôi biết sự vô lối thường yểu mệnh. Tôi chợt nghĩ, nhà thơ đôi khi phải "đồng bóng" một chút, ngôn từ phải ma mị một chút. 
  
Đọc thơ Hàn Mặc Tử, Hoài Thanh cho rằng nhiều lúc thi sĩ \textit{lạc vào thế giới đồng bóng}. Hàn Mặc Tử  lạc tới một cõi thơ, một miền thơ ít được người đời biết đến. Hàn Mặc Tử thường nói tới khu vực bí ẩn  chứa mọi sự tương giao. Ở đó không còn chỗ đứng cho nếp tư duy cũ kĩ, sáo mòn. Thi sĩ thành thực bày tỏ: ""Thế giới kì dị" của tôi được "tạo ra khi máu cuồng rền vang dưới ngòi bút"." Chính ở "thế giới đồng bóng" ấy, sự tự do của người thơ mới được thể hiện trọn vẹn, đầy đủ nhất. Thi sĩ xuất hiện giữa làng thơ, sắm vai một \textit{người khách lạ}, trụ vững trong làng Thơ mới với tầm vóc \textit{một đỉnh núi lạ}.  
 
Tôi nghĩ, ý thức đổi mới thơ biểu hiện rõ rệt ở khát khao phá bỏ những thành trì kiên cố đang ngự trị trong đời sống văn hoá tinh thần của tộc loại. Từ đó mở ra những con đường mới mà ý thức phong bế, lệ thuộc không làm được. Con đường thơ ấy có thể dài rộng tuỳ theo điều kiện văn hoá chính trị cho phép, có thể ngắn ngủi đến không ngờ. Biết bao nhà thơ phải lao tâm khổ tứ cả khi sống, lẫn khi sáng tạo. Thậm chí phải trả giá đắt, vì muốn có được một chuyến đi xa trọn vẹn cho riêng mình. Theo đuổi một lối thơ đến kiệt cùng, đâu có dễ gì. Tạo ra một lối thơ mới, càng khó khăn hơn. Huống chi khi chập chững bước vào nghề, đã bắt đầu chịu ảnh hưởng một lối thơ nào đó rồi, mà muốn có thành tựu gì đáng kể, nếu không phải người có tầm vóc tư tưởng lớn lao thì đâu có thể vượt lên nổi. Những người "theo đóm ăn tàn" chắc chắn sẽ bị chính lối thơ có vẻ tân kì kia nhấn chìm, đè bẹp. Trường hợp của Hàn Mặc Tử thì sao? Cứ theo hành trạng thơ thì thấy: thi nhân đã phải rẽ ngang ở đoạn đường nào đó. Văn chương cũng cần lắm, sức mạnh khai sơn phá thạch của người thơ. 
 
Tôi nghĩ mọi cuộc cách mạng, trong đó có thơ ca, để nảy sinh, phải hội đủ những điều kiện nào đó. Ví dụ, ở phương diện chủ quan, phải tính tới ý thức cá nhân cá tính, \textit{ý thức về sự tự do, dân chủ trong sáng tạo}. Ở phương diện khách quan, nên quan tâm tới bối cảnh văn hoá chính trị đã chi phối tới \textit{sự viết, sự sống} của kẻ cầm bút. Nghĩ thế, có phần xa rời thực tế. Vì hầu hết những thử nghiệm, cách tân thơ ca ở ta đều bắt nguồn từ sự \textit{tiếp biến tư tưởng văn hoá phương Tây}, chứ ít khi có cuộc \textit{cách tân nội bộ}. Người thơ luôn luôn đến muộn, muộn so với người mấy chục năm, chừng hàng trăm năm. 
 
Công bằng, không phải nhà thơ Việt "chậm chạp" đổi mới, mà thực ra những điều kiện văn hoá xã hội nào đó chưa chín muồi, chưa tạo ra những điều kiện cần thiết để ý thức đổi mới văn học đơm hoa kết trái. Một số "cánh chim đầu đàn" chưa mạnh dạn theo đuổi đường bay mới. Số ít táo bạo hơn trong cách nghĩ, cách làm thì gặp không ít trở ngại, thậm chí "bị thương". Kẻ hậu sinh cầm cây bút lên, thấy vết thương cũ của người năm ấy chưa lành, vết thương mới lại xuất hiện, thì cũng dè dặt lắm. 
 
Thơ Việt Nam giai đoạn 1930-1945 chứng kiến: nhiều thi nhân tìm đến Baudelaire, Mallarmé, Verlaine chẳng khác gì tìm kiếm một lối thoát cho những bế tắc về tư tưởng, về nghệ thuật biểu hiện. Số còn lại đón nhận nồng nhiệt Baudelaire để tiếp sức cho công cuộc cách tân thơ bền bỉ. Thế Lữ, người đầu tiên tuyên bố \textit{cuộc sống thoát li} cũng tìm đến Baudelaire hòng giữ địa vị bá chủ của mình trong Thơ mới. Xuân Diệu, Huy Cận đều tiếp nhận dè dặt tinh thần sáng tạo của Baudelaire – "ông tổ tượng trưng" và Verlaine, một đại biểu xuất sắc của trào lưu đó. Chịu ảnh hưởng đậm nét của Baudelaire, Edgar Poe, Mallarmé, Valéry phải kể đến: Hàn Mặc Tử và Chế Lan Viên. Xem ra, cách tân thơ liên quan mật thiết với "con người tư tưởng". 
 
Hàn Mặc Tử đến với thơ tượng trưng từ bao giờ? Năm 1936, tập \textit{Gái quê} ra đời. Thi sĩ họ Hàn trút bỏ phong vận Đường thi từ đấy (\textit{Lệ Thanh thi tập}). Cùng năm đó, \textit{Trường thơ Loạn} được thành lập, Hàn Mặc Tử giữ vai trò chủ soái. Tập thơ \textit{Gái quê} với tính cách tượng trưng của nó đã đóng vai trò như một bước đệm trong hành trình sáng tạo của Hàn Mặc Tử. Như vậy, có một bài học sáng tạo ở đây: nhà thơ cần làm mới con người tư tưởng ở mình, trước khi muốn làm mới văn chương. Để làm mới được, dĩ nhiên không thể thiếu bản lĩnh. 
 
Cách tân thơ càng trở nên có ý nghĩa và tạo thành "vệt đậm", thành "trường phái" khi có một nhóm người cầm bút cùng nhau theo đuổi một lối viết. Sự "cùng nhau" này, nhiều lúc do \textit{ngẫu nhiên}. Đúng hơn, ở những điều kiện nhất định, \textit{tất yếu} phải thế. Số phận của công cuộc cách tân thơ một phần phụ thuộc vào "cánh chim đầu đàn", phần nữa do các thành viên cùng chí hướng quyết định. 
 
Ta thấy, mọi ý đồ cách tân thơ đều không mấy dễ dàng thành công. Ban đầu, "người thơ" thường chịu sự ghẻ lạnh, hắt hủi của người đời, vì cái mới-cái lạ kia phá vỡ trạng thái lặng lẽ sống, lặng lẽ viết của họ. Sau nữa, giả định khuynh hướng sáng tác mới chứng minh được "lí do tồn tại tất yếu của mình", nó sẽ có chỗ đứng đáng kể trong sân thơ chật hẹp nhường ấy. Hiển nhiên, nếu thiếu ý thức tranh đấu quyết liệt cho sự tồn tại của khuynh hướng thơ tích cực thì ý đồ cách tân thơ nào đó sẽ nhanh chóng thất bại. Hơn nữa, theo tôi, chính nội lực sáng tạo dồi dào, tài hoa của người viết sẽ quyết định đường hướng thơ, số phận thơ của họ. Lấy trường hợp Hàn Mặc Tử làm ví dụ. Tập thơ \textit{Đau thương}, một tập thơ đậm tính cách tượng trưng nhất của Hàn Mặc Tử, được soạn từ năm 1937 và chỉ một năm sau thì hoàn thành. Song sinh với \textit{Đau thương}, có \textit{Điêu tàn} của Chế Lan Viên (1937). \textit{Tinh huyết} của Bích Khê ra đời muộn hơn (1939). Tập thơ \textit{Tinh huyết} lại do chính Hàn đề tựa, sau khi  ông đã giới thiệu Chế Lan Viên trên báo \textit{Tràng An} (1936), và \textit{Xác thu} của Hoàng Diệp (1937). Tại thời điểm \textit{Tinh huyết} chào đời, Hàn Mặc Tử đã đi qua lối thơ tượng trưng và bắt đầu đặt chân lên mảnh đất siêu thực. Thi tài của Chế Lan Viên, Hàn Mặc Tử, Yến Lan, Bích Khê… được thừa nhận. Chúng ta không thể nhắc đến công sức của người này mà bỏ đóng góp quan trọng của người kia. 
       
     
\textbf{2. Hàn Mặc Tử với nhiều ngã rẽ} 
 
Phan Sào Nam tiên sinh từng hết lời ca ngợi thơ Đường luật của Hàn Mặc Tử. Tưởng Hàn Mặc Tử cứ phong vận đó đến với chúng ta. Ai ngờ thi sĩ họ Hàn kia đã sớm cởi bỏ y phục cũ kỹ, mặc "Âu phục" để bước vào làng Thơ mới. Từ năm 1936, Hàn Mặc Tử sánh vai với \textit{Gái quê} đi về cõi hư linh, bay lên với trăng sao, với hồn, nhạc… Thế giới thơ Hàn Mặc Tử thánh thiện và huyền diệu. Ở đó, hư thực không thể phân biệt rõ ràng. Hàn Mặc Tử trở thành một "điềm lạ", một hiện tượng thơ phức tạp và còn nhiều bí ẩn.  
 
Đọc Hàn Mặc Tử lâu nay, xem trọng tinh thần lãng mạn, ít chú ý tới yếu tố tượng trưng và yếu tố siêu thực - cái làm nên \textit{bản sắc thơ} của một tài năng kì lạ và "đau thương tột cùng" này. Trong bài "Đôi nét về Hàn Mặc Tử", Quách Tấn, bạn tâm giao với thi sĩ sớm nhận thấy: "Ngay từ tập \textit{Thơ điên}, Hàn Mặc Tử đã "đi từ lãng mạn đến tượng trưng". Từ \textit{Xuân Như ý} đến \textit{Thượng thanh khí}, thơ Tử lần lần \textit{từ địa hạt tượng trưng đến địa hạt siêu thực}". \footnote{
\textit{Hàn Mặc Tử thơ và đời} (Lữ Huy Nguyên, sưu tầm, tuyển chọn). Nxb Văn học, 2000, tr 180.}  Thật hiếm có trường hợp nào, chỉ trong vài năm, đã làm ba cuộc cách mạng thơ ca như Hàn Mặc Tử.  
 
Hàn Mặc Tử không biến mình thành "cây đàn độc điệu", không chịu buông neo một chỗ. Ông tìm mọi cách tự vượt mình trong nhiều lối thơ tân kì. Thơ Hàn Mặc Tử không vẽ vời hình thức thơ ca, mà đổi mới từ trong cốt tuỷ. Không ai giống Hàn Mặc Tử trong bản hoà âm độc đáo ấy. Tôi xem thơ Hàn Mặc Tử \textit{hiện đại nhất, dị thường nhất}. Vương Trí Nhàn nói: "Trước mắt chúng ta có một giọng thơ độc đáo không chia sẻ âm hưởng với ai hết". \footnote{
Vương Trí Nhàn, \textit{Những kiếp hoa dại}. Nxb Hội Nhà văn, tr 98.}  Thơ Hàn Mặc Tử đại diện cho một khuynh hướng thơ độc đáo, với nhiều tìm tòi táo bạo. Có tìm thể thấy điệu thơ của Xuân Diệu, Vũ Hoàng Chương, Thế Lữ, Đinh Hùng… trong hồn thơ Hàn Mặc Tử. Nhưng để tìm thấy một bản sao nguyên cảo "lối thơ điên" nữa, thì  thật khó thay! 
       
 
\textbf{3. Hàn Mặc Tử tiếp nhận để cách tân thơ} 
 
Không phải ngẫu nhiên, khi Thơ mới nở rộ, đạt nhiều thành tựu cao, thì trường phái thơ tượng trưng được chào đón nồng nhiệt hơn cả. Baudelaire trở thành "đường viền" của sáng tác thơ ca. Ngôi sao Thế Lữ bị lu mờ, bởi "nguồn thơ Thế Lữ đã cạn không đi kịp thời đại" (Hoài Thanh). Thế Lữ đến với Baudelaire khá muộn. "Nguyễn Bính chỉ còn thiếu một hiểu biết Tây học nên không thành nổi nhà thơ đầu đàn." \footnote{
Phan Ngọc, \textit{Ảnh hưởng của văn học Pháp tới văn học Việt Nam trong giai đoạn 1932-1940} / Tạp chí \textit{Văn học} số 4-1993, tr. 25.}  Như vậy, có trường hợp \textit{tiếp nhận để cách tân thơ}. Nhưng, sự tiếp nhận với ý nghĩa này, luôn đòi hỏi phải có bản lĩnh và tiền đề văn hoá cần thiết. Muốn tiếp nhận được khuynh hướng sáng tác mới, nhà thơ phải có trữ lượng sáng tạo dồi dào, phải có năng lực làm mới mình, làm mới cái được tiếp nhận. Dù vậy, không hiếm trường hợp tiếp nhận lối viết mới khá sớm mà chẳng bao lâu, người thơ bị cùn bút ngay, vì nguồn năng lượng sáng tạo kia không thể phù trợ cho cái tạng sáng tác có phần riêng biệt của tác giả. Tiếp nhận khuynh hướng sáng tác thơ chỉ thực sự có ý nghĩa cách tân khi, xét về phương diện chủ quan, nhà thơ có đầy đủ các tố chất cần thiết đảm bảo cho nó nảy nở và phát triển theo qui luật đặc thù. Mọi hành động tiếp nhận sáng tạo thi ca, sẽ tạo ra các tác phẩm đơn điệu và nhàm chán, nếu nhà thơ không có nhu cầu đổi mới tư duy văn học. 
 
Hàn Mặc Tử tiếp nhận những gì? Thơ Mallarmé gắn bó với âm nhạc. Thơ Hàn Mặc Tử cũng có bản hoà âm huyền ảo của: "ánh sáng (…) tiếng suối (…)". Thi pháp của Apollinaire gắn bó với hội hoạ. Thi sĩ họ Hàn thường lấy chất liệu màu sắc để tạo nên thế giới thơ. Chủ nghĩa tượng trưng cho rằng: sáng tạo thơ ca tương đồng với sự sinh sôi của tạo hoá. Thi sĩ Hàn Mặc Tử cũng muốn nắm được cái huyền diệu của thơ, của tạo vật. Nhà thơ hăm hở "đi khơi mạch thơ ở Đức Chúa Trời" ("Quan niệm thơ"), và coi nghệ thuật là "tác phẩm của trời đất" ("Nghệ thuật là gì?"). 
 
Theo tôi, đỉnh cao thơ Hàn Mặc Tử, đóng góp lớn nhất của thi sĩ là ở mảng thơ \textit{tượng trưng và chớm siêu thực}, tạo nên vũ trụ thơ Hàn Mặc Tử đặc sắc nhất, vẻ vang nhất, "kì dị" nhất bắt đầu từ \textit{Đau thương}. Ngay từ \textit{Đau thương}, kiến trúc ngôn từ đã đồng nhất với cảnh chiêm bao vô thức. Thi sĩ  "siêu hoá những ước mơ không được thoả mãn": 
 
\textit{Ai đi lẳng lặng trên làn nước} 
\textit{Với lại ai ngồi khít cạnh tôi} 
\textit{Mà sao ngậm cứng thơ đầy miệng} 
\textit{Không nói không rằng nín cả hơi?} 
("Cô liêu") 
 
\textit{Ta là ta hay không phải là ta?...} 
\textit{Hồn vội thoát ra khỏi bờ trí tuệ} 
("Siêu thoát") 
				 
\textit{Tôi còn ở đây hay ở đâu?} 
\textit{Ai đem tôi bỏ dưới trời sâu?} 
\textit{Sao bông phượng nở trong màu huyết} 
\textit{Nhỏ xuống lòng tôi những giọt châu} 
 
Dù trong thời kì đầu và chặng cuối con đường, thơ của Hàn Mặc Tử trong sáng, nhưng về cơ bản, Hàn Mặc Tử không có vóc dáng lí tưởng của một thi sĩ  lãng mạn thuần nhất. Tôi nhấn mạnh: Từ tập \textit{Gái quê} trở về trước, Hàn Mặc Tử sáng tạo ra "thơ hội hoạ". Sau nó nghiêng hẳn về "thơ âm nhạc",  "thơ điên". Tập \textit{Thơ điên} minh chứng cho con đường đi riêng của thi sĩ về nhịp, nhạc, về khả năng biểu hiện bản giao hưởng của tâm hồn. Chính Hàn Mặc Tử, trước khi vào nhà thương Quy Hoà, đã từng dặn Quách Tấn: "Nếu Chúa ban phước cho tôi lành mạnh, tôi sẽ đốt tập \textit{Thơ điên}... Không nên để cho người đời thấy những \textit{bí ẩn của lòng mình}." Tôi thấy quan niệm thơ khá thú vị của Hàn Mặc Tử trong câu nói có vấn đề này: Sáng tạo thơ đồng nghĩa với khám phá và biểu hiện con người thứ hai trong mình. Con người trong thơ thuộc về thế giới ẩn ức, tiềm thức đầy bí ẩn. Con người trong thơ được tự do sống với bản lai diện mục của mình. Trong khi sáng tạo, nhà thơ sống với cảnh giới mà mình chưa hề biết, với trạng thái mà mình chưa trải qua, với thời gian, không gian phi hiện thực. Tất cả đều bí ẩn đối với người viết và đối với người đọc. 
 
Hàn Mặc Tử yêu cầu thơ ca phải phát ra tiếng kêu than rền rĩ: 
 
\textit{Tôi muốn hồn trào ra đầu ngọn bút } 
\textit{Mỗi lời thơ đều dính não cân ta} 
\textit{Bao nét chữ quay cuồng trong máu vọt} 
\textit{Cho mê man chết điếng cả làn da } 
\textit{Cứ để ta ngất ngư trong vũng huyết } 
\textit{Trải niềm đau trên mảnh giấy mong manh} 
\textit{Đừng nắm lại hồn thơ ta đang xiết } 
\textit{Cả lòng ai trong mớ chữ rung rinh...} 
(“Rướm máu”) 
              
Nếu xem điên là một \textit{trạng thái sáng tạo mãnh liệt, giây phút sáng láng} \textit{của hồn thơ}, thì thực chất bài thơ "Rướm máu” khẳng định: Thơ ra đời từ một trạng thái “quay cuồng”, “ngất ngư” không gì kiềm chế nổi. Thơ khởi phát từ \textit{trạng thái xuất thần}, từ "đáy tâm linh”. Ngôn ngữ tâm linh, ngôn ngữ nội tâm trong cảnh giới sáng tạo của thi sĩ hoá thân tự nhiên thành ngôn ngữ thơ. Chính cảnh ngộ đau thương hiện thực và tâm thức cái chết đương liền kề đã đem lại "cái rung động sung sướng" cho thi sĩ (“Nghệ thuật là gì?”). Trạng thái “điên” trong thơ Hàn Mặc Tử gần với khoảnh khắc “quên” kì diệu của thơ Thiền. Người làm thơ “không có thì giờ nghĩ về mình”, anh ta như bị thôi miên, lạc vào cõi huyền diệu, khám phá ra “cái siêu tôi”. Hàn Mặc Tử khẳng định: “Tôi làm thơ... nghĩa là tôi phản lại tất cả những gì mà lòng tôi, máu tôi, hồn tôi đều hết sức giữ bí mật... tôi mất trí, phát điên” (Tựa \textit{Thơ điên}). Thi sĩ họ Hàn coi trọng tiềm thức, vô thức, chủ trương một \textit{lối viết tự động}. Thi sĩ “để mặc cho giai âm rên rỉ”, khẩn khoản với mọi người: “Đừng nắm lại hồn thơ ta đang xiết...” rồi dứt khoát khẳng định “không ai ngăn cản được tiếng lòng tôi”. Theo tôi, \textit{lối viết tự động} ở Hàn Mặc Tử khá gần gũi với lối viết tự do đã được André Breton đề xướng từ năm 1929 trong “Tuyên ngôn thứ nhất” và bản “Tuyên ngôn thứ hai của chủ nghĩa siêu thực” \footnote{
Xem thêm: tạp chí \textit{Văn học nước ngoài}, số 5-2004.} .  
 
 
\textbf{4. Muốn cách tân thơ, trước hết phải đem đến một quan niệm mới về thể loại} 
  
Bằng chứng đáng tin cậy nhất của sự sáng tạo đổi mới chính là diện mạo của tác phẩm trong đời sống văn học. Còn quan niệm của nhà thơ về thể loại được phát biểu rải rác ở đâu đó sẽ trở thành tôn chỉ, mục đích sáng tạo của người nghệ sĩ, nếu như quan niệm đó thôi thúc nhà thơ cầm bút để khẳng định nét riêng của mình. Muốn cách tân thơ, nhà thơ cần hình thành cho mình một quan niệm thơ mới mẻ trước đã. Quan niệm về thể loại chẳng mới mẻ gì, thì chẳng bao giờ tác giả tạo được cho thơ ca một khuôn mặt mới. 
 
Với Hàn Mặc Tử, khi sáng tạo, một mặt nhà thơ khai thác những dữ kiện trực tiếp của ý thức cá nhân, mặt khác thi nhân sẽ “quên cả thói quen phân tích của tư duy lô gíc... để cho trực giác của tâm linh trỗi dậy”. Thơ “đưa chúng ta vào một trạng thái tâm lí bất ổn” (Béc-xông). Nhà thơ cố gắng nắm bắt những cảm xúc tột cùng của con người, “những cái trừu tượng đang vận động”. Thơ chợt về với nghệ sĩ ở những giây phút \textit{máu cuồng} và \textit{hồn điên}. 
 
Hàn Mặc Tử không giấu những \textit{đau thương}, thi sĩ cứ muốn ở mãi trong \textit{đau thương}: \textit{Thơ tôi thường huyền diệu} (“Cao hứng”), \textit{lời thảm thương rền khắp nẻo mơ} (“Trút linh hồn”). Hàn Mặc Tử nhận thấy “nhà nghệ sĩ bao giờ cũng điên”. Muốn \textit{phát điên}, anh ta phải “sống mãnh liệt và đầy đủ”, muốn bay tới địa hạt \textit{huyền diệu}, anh ta phải “mộng”, phải có \textit{trí tưởng tượng} dồi dào, đặc biệt phải sành âm nhạc và màu sắc. Nhà thơ muốn đến bến bờ tượng trưng cần “có đôi mắt rất mơ, rất mộng, rất ảo, nhìn vào thực tế thì sự thực sẽ trở thành chiêm bao...” 
 
 
\textbf{5. Vũ trụ thơ của Hàn Mặc Tử: kì dị và lạ thường} 
 
Kết quả của sự cách tân thơ, sau cùng phải đem lại cho người đọc một thế giới nghệ thuật mới, một hình thức mới của cái nhìn nghệ thuật. Không có thế giới nghệ thuật mới lạ thì coi như chưa đổi mới thơ. Vậy, Hàn Mặc Tử đã sáng tạo ra thế giới nghệ thuật nào? 
 
Trong bài viết “Nghệ thuật là gì?” năm 1935, Hàn Mặc Tử nhấn mạnh: nhà thơ cần có “năng lực mạnh mẽ về tinh thần, thứ năng lực ấy nó làm cho con người thêm hứng khởi đi tìm \textit{cái sự lạ}”. Hàn Mặc Tử  “đi tìm \textit{cái sự lạ}” “ở chốn xa xăm, thiêng liêng và huyền bí”. Nhà thơ “nhấn một cung đàn, bấm một đường tơ, rung rinh một \textit{làn ánh sáng}”, thơ Hàn có một nguồn “sáng lạ”, lời thơ và tâm thế của người thơ rất kì dị. 
 
Đọc thơ Hàn Mặc Tử, nhà phê bình Hoài Thanh có cảm nhận mình như lạc vào “cái thế giới kì dị”, “đi trong mờ mờ”, thấy nguồn thơ của thi sĩ nảy nở thật \textit{lạ lùng}. ““Xuân như ý” có những câu thơ đẹp một cách lạ lùng”; cảnh vật trong “Máu cuồng và hồn điên” “...không thấy có tí gì giống với cảnh trước mắt. Trời đất này thực của riêng Hàn Mặc Tử (...) trong văn thơ cổ kim \textit{không có gì kinh dị hơn}.” 
 
Trong cái nhìn nghệ thuật của Hàn Mặc Tử, \textit{cái sự lạ} kia biểu hiện như một cảnh thực, thứ hiện thực ảo. Cái sự lạ trong vũ trụ thơ ấy xuất hiện cùng với tâm trạng ngạc nhiên, ngỡ ngàng của chủ thể trữ tình. 
 
\textit{Tiếng động sau vùng lau cỏ mọc \footnote{
Toàn bộ thơ được trích dẫn ở đây, căn cứ vào cuốn: \textit{Hàn Mặc Tử, tác phẩm, phê bình và tưởng niệm} (Phan Cự Đệ tuyển) và \textit{Hàn Mặc Tử  thơ} (Chế Lan Viên tuyển chọn và giới thiệu).} } 
\textit{Tiếng ca chen lấn từ trong ra...} 
\textit{Áo quần vo xắn lên đầu gối} 
\textit{Da thịt, trời ơi! Trắng rợn mình...} 
\textit{Nụ cười dưới ấy và trên ấy } 
\textit{Không hẹn, đồng nhau nở lẳng lơ...} 
(“Nụ cười”) 
 
\textit{Gió rủ nhau đi trốn cả rồi} 
\textit{Nhỏ to, câu chuyện, ô kìa coi} 
\textit{Trong lau như có điều chi lạ} 
\textit{Hai bóng lung lay thấy cọ mài...} 
(“Khóm vi lau”) 
 
\textit{Bỗng đêm nay trước cửa bóng trăng quỳ} 
\textit{Sấp mặt xuống uốn mình theo dáng liễu} 
\textit{Lời nguyện, gẫm xanh như màu huyền diệu} 
\textit{Não nề lòng viễn khách giữa cơn mơ} 
 
Nhà thơ đi tìm \textit{cái lạ} chưa đủ, anh ta cần phải chiếm lĩnh cho được \textit{cái kì dị}. Hai thứ đó đan xen với nhau tạo ra hứng thơ mạnh mẽ và vô tận. 
 
\textit{Lời thơ ngậm cứng, không rên rỉ} 
\textit{Và máu tim anh vọt láng lai} 
\textit{Thơ ở trong lòng reo chẳng ngớt} 
\textit{Tiếng vang tha thiết dội muôn nơi...} 
 
\textit{Tiếng thông vi vút như van lơn...} 
\textit{Mây buồn vơ vẩn bay đầu non... } 
\textit{Ngây tình, bóng liễu câm không nói} 
\textit{Trong khóm vi lau có tiếng than } 
(“Trên bờ”) 
 
Tất cả đường thơ mà thi sĩ họ Hàn đi qua, ngay cả “Đường thi” cũng đã \textit{trổ ra những ánh khác lạ \footnote{
Đỗ Lai Thuý: \textit{Mắt thơ}, Nxb Văn hoá thông tin, H, 2000,  tr. 214.} }. Mỹ học thơ Hàn có thể gói gọn trong hai phạm trù thẩm mỹ: \textit{kì dị và lạ thường}. Thơ Hàn Mặc Tử không bình dị và không đài các. Lối thơ thứ nhất, có tính cách phổ thông, chưa biết đến \textit{cái lạ}. Lối viết thứ hai thuộc cái thông bệnh của thi sĩ Hán học, nên không thể trở thành \textit{cái kì dị} được. Thơ Hàn Mặc Tử: \textit{kì dị và khác lạ}. Kì dị và khác lạ trước hết ở thi ảnh, thi cảm.  
 
Nhà thơ Baudelaire từng hết lời ca ngợi những người tự do, biết: “bay vào những \textit{trường sáng sủa} và thanh sạch...” (“Lên cao”), tôn vinh “người hiểu được ngôn ngữ của những sự vật câm lặng”. \footnote{
Dẫn theo: Phạm Văn Sĩ: \textit{Về tư tưởng và văn học hiện đại phương Tây}, Nxb Đại học và trung học chuyên nghiệp, H, 1986,  tr. 42.}  Theo Baudelaire, nguyên tắc mĩ học của thơ ca thuộc về nghệ thuật biểu tượng. ông nhấn mạnh chính “\textit{trí tưởng tượng} đã dạy cho con người cái ý nghĩa tinh thần của màu sắc, của đường nét, của âm thanh, của mùi hương, từ khởi thuỷ nó đã... tạo ra phép ẩn dụ”. \footnote{
Dẫn theo: Phạm Văn Sĩ: \textit{Về tư tưởng và văn học hiện đại phương Tây}, Nxb Đại học và trung học chuyên nghiệp, H, 1986,  tr. 46.}  Đọc thơ Hàn Mặc Tử, ta thấy lời thơ cũng đầy ánh sáng. Thi cảm,thi ảnh được “nuôi mãi trong nguồn \textit{ánh sáng thiêng liêng}”. Thi nhân “say sưa đi trong \textit{mơ ước}”, “đi đến \textit{cõi ước mơ hoàn toàn}”, “ọc ra từng \textit{búng thơ sáng láng}”. Thế giới thơ Hàn Mặc Tử có vẻ đẹp của một giấc mộng. 
 
Verlaine chủ trương giấc mơ hơn thực tại. Hàn Mặc Tử cũng nói nhiều đến giấc mơ, cảnh chiêm bao, tới thế giới không nhìn thấy. Theo Hàn Mặc Tử, ý thơ nảy sinh từ trời mộng, thơ diễn tả “những tiếng ca của tình cảm, của \textit{tưởng tượng, }của\textit{ mơ màng}” (“Không nên có luật thơ mới”, “Chiêm bao với sự thật”), thi sĩ bị ánh sáng của chiêm bao vây riết. Theo tôi, bài thơ “Đây thôn Vĩ Dạ” khá tiêu biểu cho khuynh hướng tìm tòi sáng tạo này. Vì rằng, để có được “Đây thôn Vĩ Dạ”, Hàn Mặc Tử đã phải đối thoại âm thầm với tấm bưu ảnh, đối thoại với đối tượng lặng câm, với tình yêu đơn phương vô vọng. Hình thức đối thoại ảo truyền tả được khát vọng được yêu, được sống mãnh liệt của nhà thơ. Nhà thơ phá vỡ thế độc thoại bên trong để tạo vẻ đối thoại ảo. Vẻ huyền ảo xa vời của thôn Vĩ hiện về trong tâm thức đau thương của một hồn thơ cô đơn. Thi sĩ tưởng tượng ra một cố nhân đang mong chờ mình, mời mình về thôn Vĩ. Thi sĩ mơ tiếng gọi thiết tha trìu mến của người thương, ao ước nghe thấy lời chào mời giục giã của cô gái ấy. Thế giới “Đây thôn Vĩ Dạ” tràn đầy ánh sáng, thực ảo chập chờn chuyển hoá lẫn nhau. Con thuyền thơ cứ chảy trôi trong thế giới mộng ảo, trong cõi mơ, con người Huế cũng chìm trong mộng ảo.  
 
Nếu thơ Xuân Diệu đề cập nhiều đến \textit{sắc và hương} thì thơ Hàn Mặc Tử nói nhiều về \textit{âm thanh và ánh sáng}. Chỗ mạnh của Hàn Mặc Tử là cảm nhận được ánh sáng và âm điệu của sự vật. Hàn Mặc Tử  quan niệm: đời sống bí mật riêng tư của sự vật nằm ở ánh sáng và âm điệu của nó. 
 
Hàn Mặc Tử lạc vào thế giới của cái kì dị và lạ thường, thế giới của âm thanh và ánh sáng lạ. Thế giới ấy có cấu trúc riêng, ý nghĩa riêng, quy luật vận động riêng. Chẳng phải vô cớ Hàn Mặc Tử luôn chú ý tới \textit{nắng}. \textit{Nắng} trong thơ thi sĩ họ Hàn trở thành tín hiệu báo mùa: 
 
\textit{Trong làn nắng ửng: khói mơ tan} 
(“Mùa xuân chín”) 
 
\textit{Nắng ửng} có vẻ riêng trong cái nhìn xuân tình của tác giả. \textit{Nắng ửng} không chỉ báo hiệu “bóng xuân sang” mà còn đánh dấu khoảnh khắc: \textit{mùa xuân bắt đầu chín}. \textit{Nắng ửng} gắn liền với tâm trạng rạo rực  xôn xao ở hồn người. Bài thơ “Mùa xuân chín” đọng lại cái \textit{nắng hắt ra từ cõi nhớ}. Nắng trong hoài niệm, thứ nắng hoài vọng chín theo sự chín của mùa xuân, tình xuân. Nắng chín dĩ nhiên đẹp, nhưng phảng phất buồn. Đẹp bởi cảnh xuân, tình xuân nồng nàn. Buồn bởi “có kẻ theo chồng bỏ cuộc chơi”. Trong tác phẩm “Ngủ với trăng”, nhân vật trữ tình “khao khát trăng gió” và “đi bắt \textit{nắng ngừng, nắng reo, nắng cháy}”. \textit{Nắng chang chang} đốt lòng người thực ra là hình ảnh phái sinh của kiểu \textit{nắng cháy}. Nhưng nếu \textit{nắng chang chang} loang ra \textit{dọc bờ sông trắng}, thì \textit{nắng ngừng, nắng reo, nắng cháy} ở đây lại xuất hiện trong một không gian khá đặc biệt: “trên\textit{ sóng cành, sóng áo} cô gì má đỏ hây hây”. \textit{Ngừng, reo, cháy} ứng với ba cung bậc tình cảm khác nhau của con người: \textit{lặng im, xao xuyến} và \textit{cuồng si}. Ba trạng thái tình cảm ấy đồng nhất với ba cảm xúc sáng tạo. Hoá ra, nắng biểu hiện thi hứng, thi cảm của nhà thơ. 
 
Nắng trong thơ Hàn Mặc Tử có “tuổi” và có tình. Người ta thường nói: trăng sáng, sao sáng, còn Hàn Mặc Tử lại cảm thấy \textit{nắng sao}. \textit{Nắng reo} đã lạ, \textit{nắng sao}, nắng trong đêm thì lại càng kỳ. Có lẽ thứ nắng ấy chỉ xuất hiện trong thế giới thi ca của Hàn Mặc Tử với một tâm thế trữ tình đặc biệt “buồn trong mộng” (“Buồn ở đây”). Nắng trong thơ Hàn thường gắn với hoài niệm, phảng phất duyên tình: \textit{nắng vàng con mắt thấy duyên đâu}. Nắng gắn với duyên phận, nắng mang nỗi niềm cô đơn: “không duyên hồ dễ mong theo nắng” (“Duyên kỳ ngộ”). Nắng, thứ ánh sáng đặc biệt trong thơ Hàn, biến ảo theo cường độ nỗi đau, nỗi nhớ. Biên độ nắng không có giới hạn, rộng mở theo không gian xa cách, theo “thế giới ảo huyền”. \textit{Nắng ửng} làm “khói mơ tan”, \textit{nắng dọi} làm “bài thơ cháy”. Ngay cả \textit{nắng mai} cũng “dìu dịu mối sầu vương” (“Duyên kỳ ngộ”). 
 
\textit{Nắng }là một loại ánh sáng đặc biệt, “ánh sáng của chiêm bao, huyền diệu” (“Chơi giữa mùa trăng”). \textit{Nắng} trở thành tín hiệu thẩm mỹ, báo hiệu \textit{mùa thơ đang chín} (“Kêu gọi”). \textit{Nắng} kích thích trí tưởng tượng của nhà thơ bay vào cõi mơ:  
 
\textit{Nắng càng cao lòng ta càng hừng hực} 
\textit{Thơ lên rồi bay quá giải nhàn vân...} 
(“Duyên kỳ ngộ”) 
 
\textit{Ôi chao thơ ngầm bay theo dải nắng} 
\textit{Lộng vào xiêm áo mỏng manh sao...} 
 
Sự vận động của \textit{Nắng} tạo ra thi giới của “cái tột cùng”. \textit{Nắng} vừa hoá giải đau thương vừa ràng rịt nỗi đau. \textit{Nắng} được nhìn qua lăng kính của hồn và xác. 
 
\textit{Nắng ơi, nắng có lên cao} 
\textit{Làm sao da thịt hồng hào thế kia} 
(“Duyên kỳ ngộ”) 
 
Nói đến \textit{hồn}, đến \textit{thơ }không thể không nhắc tới \textit{nắng}. \textit{Nắng} hoà quyện với hồn, với thơ. \textit{Nắng} \textit{và hồn ở trong thơ} - cái vũ trụ do Hàn Mặc Tử sáng tạo ra.  
 
Hàn Mặc Tử ít nói đến \textit{nắng thu, nắng hè}... thi sĩ có ấn tượng nhiều hơn với \textit{nắng xuân}. \textit{Nắng xuân} ám ảnh, quấn riết lấy thi sĩ. Xuân trong thi giới của Hàn Mặc Tử cũng khá lạ: “xuân mộng”,”xuân gấm” (“Xuân đầu tiên”) “xuân thơm” (“Nhớ thương”), “xuân lịch sự”. Hình tượng Xuân chẳng qua do con người hoá thân mà thành, nhưng không phải con người trần tục, trần thế mà một người “ngọc”, người của cõi mộng, cao quí thanh sạch (“Cô gái đồng trinh”). Tuổi xuân là \textit{Ngọc như ý}, tên xuân là \textit{Dạ lan hương}. Xuân gắn với mơ ước, \textit{xuân tắm nắng tươi} (“Tiếng vang”), nắng mới.        
 
Ánh sáng trong thơ thi sĩ họ Hàn có hình khối, hương sắc chiếm vị trí quan trọng trong thơ, gần như trở thành đơn vị đo đếm thế giới. Bên cạnh ánh sáng của nắng, Hàn Mặc Tử còn ưa tả ánh sáng của trăng. Hàn Mặc Tử thường tả ánh sáng trong trẻo của trăng rằm. “Trăng (...) \textit{tượng trưng cho một mùa ao ước} (...) và hơn nữa, hiện hình của một nguồn khoái lạc chê chán.” (“Chơi giữa mùa trăng”)  Trong trăng có hương thơm, có nhạc, có hơi thở và có tình. “Tình thoát ra ở điệu nhạc mênh mang trong bờ bến của \textit{chiêm bao}.” Trong chiêm bao, trong \textit{vùng mộng siêu thời gian}, đến gió cũng “phảng phất những tiếng kêu rên của thương nhớ xa xưa.” Thế giới ánh sáng thu hẹp ở hình tượng “trăng”. Thế giới trăng, thế giới của những ao ước nhớ thương hợp thành một thể thống nhất: thế giới nghệ thuật.  
 
\textit{Trăng nằm sóng soải trên cành liễu} 
\textit{Đợi gió đông về để lả lơi} 
\textit{Hoa lá ngây tình không muốn động} 
\textit{Lòng em hồi hộp chị Hằng ơi...} 
  
\textit{Trăng nằm}, thơ mộng, chông chênh và hư huyền quá. Mà lại \textit{nằm sóng soải} thì thật táo bạo, gợi tình. Cảm xúc thơ bừng lên, rạo rực men say ái tình. Cái khao khát “cuồng điên” của trăng biểu hiện trong  tư thế, tâm trạng, thậm chí cả trong cái ý nghĩ trần thế: \textit{để lả lơi}. 
 
Con người trong thơ Hàn Mặc Tử được bao bọc "bằng ánh sáng, bằng huyền diệu”, “say sưa và ngây ngất vì ánh sáng”, bầu trời càng sáng con người càng “hứng trí”. Thậm chí đi trong ánh sáng "đê mê, không biết là có mình và nhận mình là ai nữa.” Ánh sáng tạo ra ở chủ thể sáng tạo cảm giác siêu thoát hay hư vô. Ánh sáng với vẻ trắng trong, đồng trinh, thanh thoát của nó - trong cảm quan của Hàn Mặc Tử - là hiện thân của Đấng tối linh, của Đức Mẹ. Ánh sáng được ví với thứ \textit{ma lực vô song}, “xô thi sĩ đến bờ huyền diệu”. “Mùa trăng bát ngát... lòng tôi rực lên cảm hứng”, “từ sự thực đi tới bào ảnh, từ bào ảnh đi tới huyền diệu, và từ huyền diệu đi tới chiêm bao. Mông lung đã trùm lên sự vật và cõi thực, bị ánh sánh của chiêm bao vây riết..." (“Chiêm bao với sự thực”). Ánh sáng vừa vĩnh viễn vừa không vĩnh viễn. Có ánh sáng thực, ánh sáng mộng. Có thứ ánh sáng "tan thành bọt", có loại \textit{ánh sáng muôn năm} mà thi sĩ khao khát chiếm giữ được. Ánh sáng "giải thoát cái "ta" của tôi ra khỏi nơi giam cầm của xác thịt..." 
 
Trong cảm quan Hàn Mặc Tử, ánh sáng của các vì tinh tú giống như "châu ngọc", "hào quang", ánh sáng của sao, trăng hợp lại thành một "vùng trời mộng", "khí hạo nhiên". Biết bao nhiêu thứ ánh sáng, nổi bật là ánh trăng. Chỗ nào cũng trăng, "tưởng chừng như bầu thế giới… cũng đang ngập lụt trong trăng, đang trôi nổi bình bồng đến một địa cầu nào khác", "cả không gian đều chập chờn những màu sắc phiếu diễu…" Trên con đường sáng láng ấy, Hàn Mặc Tử đi "tìm Chân lý ngàn năm" ("Chiêm bao với sự thực"). 
 
Bên cạnh hình ảnh ánh sáng, thơ Hàn Mặc Tử cũng tràn đầy âm thanh. Đó là "tiếng thất thanh rùng rợn", là "giọng hờn đau trăm vạn nỗi niềm riêng". Thi sĩ bộc bạch: \textit{Nàng đánh tôi đau quá / Tôi bật ra tiếng khóc, tiếng gào, tiếng rú}. Hơn một lần thi sĩ nghe thấy âm thanh kì dị ở chốn âm u:  
 
\textit{Một khối tình nức nở giữa âm u} 
\textit{Một hồn đau rã lần theo hương khói} 
\textit{Một bài thơ cháy tan trong nắng rọi} 
\textit{Một lời run hoi hóp giữa không trung} 
("Trường tương tư") 
 
"Trường tương tư" tái hiện "tiếng nói siêu thực", tiếng nói dị thường. Cảm quan về sự tồn tại của cái lạ thường đã xui khiến Hàn Mặc Tử tìm đến thế giới Hư Vô, tới "cõi vô cùng".   
 
\textit{Mới hay cõi siêu hình cao tột bậc} 
\textit{Giữa hư vô xây dựng bởi trăng sao} 
("Siêu thoát") 
 
\textit{Cũng hình như, em hỡi, động Huyền Không} 
\textit{Mà đêm nghe, tiếng khóc ở đáy lòng} 
\textit{Ở trong phổi trong tim trong hồn nữa..} 
(“Trường tương tư”) 
 
Thi nhân nhạy cảm với mọi âm thanh, đặc biệt là âm thanh vang lên từ tư tưởng, âm thanh từ \textit{cõi mờ, cõi huyền} của cuộc sống.  
   
Xuân Diệu bồng bột, đôi mắt \textit{xanh non biếc rờn }nên nhìn mọi thứ đều tươi mới. Xuân Diệu \textit{không muốn đi, mãi mãi ở vườn trần / Chân hoá rễ để hút mùa dưới đất}. Còn Hàn Mặc Tử cứ đi mãi vào sâu thế giới tâm linh, thế giới huyền hoặc của hồn và máu. Hàn Mặc Tử thấy mọi vật đang ở chặng cuối cùng hoặc đương lao nhanh về ngày tận thế, nên ông thấy trước cả “thế giới âm u”. Hàn Mặc Tử  thường tạo ra một thế giới mênh mông, không giới hạn: 
 
\textit{Không gian dày đặc toàn trăng cả } 
\textit{Tôi cũng trăng và nàng cũng trăng} 
 
Nhà thơ của những "Hương thơm" và "Mật đắng" thường nắm lấy tính chất tượng trưng của mọi hiện tượng. Thi nhân đồng hoá Hữu Thể với Hư Vô: 
 
\textit{Đây là tất cả người anh tiêu tán} 
\textit{Cùng trăng sao bàng bạc xứ mơ say } 
 
Theo cách diễn đạt của Hàn Mặc Tử, thì Hư Vô là một thực tại đặc biệt, có thanh-sắc, hình hài:   
 
\textit{Ánh trăng mỏng quá không che nổi} 
\textit{Những vẻ xanh xao của mặt hồ} 
\textit{Những nét buồn buồn tơ liễu rủ } 
\textit{Những lời năn nỉ của Hư vô  } 
 
\textit{Mới hay cõi siêu hình cao tột bực } 
\textit{Giữa hư vô xây dựng  bởi trăng sao} 
\textit{Xa lắm rồi, xa lắm, hãi nhường bao} 
\textit{Ai tới đó chẳng  mê man thần trí} 
 
Hàn Mặc Tử viết bằng tưởng tượng và "giấc mơ" trọn vẹn của chính mình. Mọi thứ trong thế giới thơ Hàn Mặc Tử đều huyền ảo. "Cái huyền ảo luôn đẹp, bất kỳ cái huyền ảo nào cũng đẹp" (André Breton). Đọc thơ Hàn Mặc Tử, người đọc phải "tư duy và nhìn theo nhà thơ".  
 
Phong Châu 5-2006 
Hà Nội 3-2008 
 
© 2008 talawas 




\end{multicols}
\end{document}