\documentclass[../main.tex]{subfiles}

\begin{document}

\chapter{Thơ chân, thơ đít}

\begin{metadata}

\begin{flushright}8.4.2008\end{flushright}

Nguyễn Đăng Thường



\end{metadata}

\begin{multicols}{2}

Thơ là gì? Nào ai biết! Nhưng thơ, với Đoàn Cầm Thi, chắc chắn phải là cái đít và cái chân của Đỗ Kh. Tuy nhiên, nếu Đoàn Cầm Thi, tác giả bài luận văn “Đỗ Kh. và Đinh Linh: Hai kẻ lạ trong ngôn ngữ của mình\footnote{\url{http://www.talawas.org/talaDB/http://damau.org/index.php?option=com_content&task=view&id=3483&Itemid=10171}}” nghĩ rằng mỗi độc giả mỗi khán giả cũng nghĩ y hệt như bà thì hơi bị chủ quan đấy nhé. 
Nếu cái đít và cái chân của Đỗ Kh. là thi phẩm lớn đối với Đoàn Cầm Thi, thì cơ thể của chàng ắt phải là tác phẩm vĩ đại. Từ nay cho tới khi chết, Đỗ Kh. chỉ cần thỉnh thoảng tiếp tụp chụp thêm các bộ phận khác trên người chàng để trưng bày, vì "ngay cả những bức ảnh chưa chụp, sẽ chụp, cũng thuộc về tác phẩm" (Đoàn Cầm Thi). Ước mong kỳ tới sẽ là "thơ - photo" với các bộ phận kín đáo hơn trên cơ thể của Đỗ Kh., để xem Đoàn Cầm Thi rối rít ra sao. Mà này, nếu các tấm hình của Đỗ Kh. là "thơ - photo" thì các tấm ảnh của các nhiếp ảnh gia Vũ Cam Đàm, Henri Cartier-Bresson chẳng hạn, sẽ là "photo - thơ"? Sao ta không gọi tấm ảnh là ảnh như gọi con mèo là mèo cho nó giản dị? Thử hỏi: Với hai tấm ảnh của hai nhà nhiếp ảnh để cạnh nhau, làm thế nào để phân biệt cái nào là "ảnh - photo" và cái nào là "thơ - photo"? Hay là bất cứ tấm ảnh nào do Đỗ Kh. bấm cũng đều là "thơ -photo"? 
 
Thiển nghĩ của tôi là lối viết của Đoàn Cầm Thi rất mâu thuẫn và suy diễn quá tùy tiện, khẳng định nhưng không chứng minh, hay chỉ "đồng lõa" người mà mình muốn ca tụng với các tên tuổi lớn để minh chứng, và Đoàn Cầm Thi chủ quan đến mức không nghĩ rằng những câu hỏi mình đặt ra không hẳn phải là câu hỏi của độc giả. Do vậy mà Đoàn Cầm Thi lạm dụng đại danh từ "người ta" số nhiều để thay thế cho chữ "tôi" số ít. Chẳng hạn như trong các câu "nó làm \textit{người ta} cười", "trong tim \textit{người Việt nào} không văng vẳng", "\textit{người ta} không khỏi nghĩ tới" (tôi nhấn mạnh), mà cái đỉnh có thể là: "Nhưng đọc Đỗ Kh., người ta đặt ra những câu hỏi quyết liệt: thơ là gì? Thơ ở đâu? Khi nào có thơ?". Câu đáp quá dễ: Thơ là cái chân và cái đít của Đỗ Kh. Thơ đến từ cái chân và cái đít của Đỗ Kh. Khi nào Đỗ Kh. làm thơ hay bấm máy lúc ấy sẽ có thơ.  
 
Chữ "người ta" tất nhiên sẽ khiến vài độc giả thích "bới lá tìm sâu" nghĩ đến từ "nhân dân" của Đảng Cộng sản. Xin hỏi: Người ta nào? Vì ắt phải có khá nhiều "người ta" đọc thơ Đỗ Kh. mà không đặt ra câu hỏi nào cả, nói chi “quyết liệt”. Ngoài ra, nếu Đỗ Kh. thực sự nghĩ rằng cái đít và cái chân của chàng là tuyệt vời, ai xem cũng hít hà như bà Đoàn thì "thơ" Đỗ Kh. không chỉ là "trò đùa, giễu và tự giễu", mà là trò đùa hơi... dai đó. Nhảy vọt từ hình ảnh của cái chân, tạm gọi là cụ thể, qua chữ nghĩa trừu tượng "chân thật / chân thực" thì không phải là biên khảo / phê bình mà một trò chơi chữ siêu việt. Đoàn Cầm Thi không nghĩ rằng lúc xem cái chân của họ Đỗ, người khác không nghĩ đến "chân thật / chân thực" mà nghĩ đến những cái chân khác, hay nghĩ đến nhiều thứ khác? A, còn các ông Tây, bà Mỹ thì sao? Khi xem chân của họ Đỗ. họ sẽ liên tưởng đến cái gì nào? Vì tiếng Pháp và tiếng Mỹ dùng để chỉ cái chân là các từ "pied / jambe" và "foot / leg".  
 
Thiển nghĩ của tôi là cái chân của Đỗ Kh. chỉ là cái chân của Đỗ Kh. Nó không thể là "chân thật" hay "chân lý". Nó cũng không thể là cái chân của bất cứ ai, của một người da trắng, da đỏ, hay da đen. Nó có tính giống, tuổi tác, đặc thù quá rõ ràng: nó không là chân phụ nữ, hay của một đứa bé, hay của một người mắc bệnh phong. Trước và sau nó sẽ mãi mãi là một cái "chân dung \textit{tự chụp}" (tôi nhấn mạnh) nghĩa là nó không thể là "một cái chân ‘khách quan’, theo cách nói của Robbe-Grillet" mà Đoàn Cầm Thi đã trích dẫn quàng xiên để minh họa cho lập luận của mình. Hơn nữa, chỉ có Đoàn Cầm Thi mới tin chắc rằng cái nhìn "khách quan" của Robbe-Grillet là hoàn toàn khách quan. Dù sao, nếu Đỗ Kh. muốn tha nhân khi xem cái chân của chàng có thể nghĩ tới một cái chân khách quan thì chàng chỉ cần ghi hai chữ "chân dung" là quá đủ rồi.  
 
Đoàn Cầm Thi viết: "Đinh Linh và Đỗ Kh. đều bị cuốn hút bởi hình ảnh, hình ảnh \textit{thật}, do máy móc đem lại, chứ không phải hình ảnh \textit{tưởng tượng} của trí não (hay trái tim cũng vậy)." Xin để qua một bên hình ảnh thật do máy thu hình tự động ghi lại trên xa lộ chẳng hạn vì không phải là trường hợp ở đây, xin thưa: Chẳng có hình ảnh nào "thật" trăm phần trăm cả, từ tấm ảnh chụp trong tiệm hình đến tấm ảnh bấm nhanh trên đường phố. Các tấm ảnh phóng sự chỉ nói lên phân nửa hay một phần tư của sự thật trong chiến tranh Việt Nam vẫn còn đó để chứng minh. Các tấm hình của Đỗ Kh. tất nhiên đã được chàng tưởng tượng, hình dung trước, lựa chọn góc cạnh trước khi bấm máy. Nếu chẳng để gây sốc thì chí ít cũng để cho đẹp mắt, hay bắt mắt theo \textit{thẩm mỹ của Đỗ Kh.} (tôi nhấn mạnh) mặc dù Đoàn Cầm Thi đã vội vàng la toáng lên rằng "có lẽ trong ba tiêu chí Chân - Thiện - Mỹ người xưa để lại, Đỗ Kh. chỉ giữ mỗi chữ Chân. Thiện và Mỹ, anh quẳng đi đâu hết…" Các tấm hình của Đỗ Kh. tất nhiên không "tả chân" như Đoàn Cầm Thi đã khẳng định mà ngược lại, chúng đã được chụp dưới góc cạnh và với độ sáng mờ mờ để cốt gợi ra một cái không khí nào đó. Hơn bao giờ hết, ảnh, phim, video đang được tận dụng không để nói lên sự thật mà chỉ ngược lại, thí dụ như các tấm ảnh, phim video về cuộc biểu tình ở Tây Tạng vừa qua do Trung Quốc phổ biến. Thiển nghĩ của tôi là Đoàn Cầm Thi cần đọc thêm các tiểu luận về phim, ảnh trước khi đặt bút viết. 
 
Đoàn Cầm Thi tiếp tục: "Quay lại với tấm ảnh của Đỗ Kh. cụ thể chính xác đến từng chi tiết, màu sắc hình dáng. Không vui. Không buồn. Không ẩn dụ. Không tâm tình. Không siêu hình. Không khiêu khích. Không gây sốc." Ngần ấy những cái "không", tất nhiên đã được người viết suy diễn tùy tiện theo cảm nghĩ chủ quan, rồi từ đó khẳng định và tổng luận, mà không kèm theo một lời giải thích, mà không có một thí dụ để chứng minh. Xin hỏi: Màu sắc \textit{ở đâu} trong một tấm hình hoàn toàn đen trắng? Xin đừng phản biện rằng đen trắng cũng là màu sắc. Màu sắc trong một tấm ảnh, nếu như có, thì không bao giờ thực sự chính xác, không chính xác với thiên nhiên và có thể khác nhau ở mỗi ấn bản của cùng một tấm ảnh. Làm sao một người xa lạ tới xem ảnh có thể biết chắc rằng tấm ảnh đó ghi lại "thực tế chính xác" nghĩa là nó không bị / không được tút (sửa lại) trước khi đem đi trưng bày? Làm sao một người ngoại cuộc xem ảnh có thể biết chắc rằng đấy là một cái chân thiệt và của chính Đỗ Kh., chứ không phải là chân giả, hay là chân của ai đó khác? Máy móc tất nhiên không hẳn lúc nào cũng đem lại những hình ảnh trung thực như Đoàn Cầm Thi suy luận quá vội vàng, mà ngược lại, chúng có thể ngụy tạo ra vô số "ảo ảnh", nhất là với kỹ thuật vi tính tinh vi để cắt dán, lắp ráp như bây giờ. Hollywood đã sử dụng kỹ thuật này từ lâu rồi. 
 
Nếu đề tài của tấm hình mang tên \textit{"Buồn trong khách sạn: chân dung tự chụp"}, và nếu đúng như Đoàn Cầm Thi đã khẳng định, rằng đề tài đó chỉ cốt để giễu nhại "hai chủ đề (đã trở nên) cực \textit{sến} của thơ, từ Đông sang Tây: \textit{nỗi lòng} và câu hỏi \textit{siêu hình} tôi là ai?", nhưng nội dung thì trái ngược lại, như Đoàn Cầm Thi đã khẳng định, nghĩa là nó không chứa đựng hai "mô-típ" đó; và Đoàn Cầm Thi có thể xem tấm ảnh đó mà không thấy buồn, mà không đặt ra câu hỏi siêu hình tôi là ai, nhưng người khác, khi xem tấm hình đó vẫn có thể thấy buồn và vẫn đặt ra câu hỏi siêu hình tôi là ai, thì sao? Thiển nghĩ của tôi là người viết nên tránh tổng luận rồi khẳng định theo chủ quan, suy bụng ta ra bụng người. Bảo rằng tấm ảnh đó "không ẩn dụ" nhưng Đoàn Cầm Thi đã tức tốc nghĩ đến "chân thật / chân thực".  
 
Xin hỏi: "Thơ không có địa lý, dân tộc, màu da, thơ không có tính từ" là thế nào? Những cái "không" của "thơ - ảnh" Đỗ Kh. do Đoàn Cầm Thi kể ra đã định nghĩa rõ rệt "thơ - ảnh" của Đỗ Kh. là thế nào rồi, nghĩa là chúng có cá tính hẳn hoi. Độc giả không thể nhầm lẫn cái chân lông đen của Đỗ Kh. với cái chân trắng nõn của Nicole Kidman. Thơ Đỗ Kh. nếu không thế này thì nó cũng phải thế kia, nghĩa là nó luôn luôn phải có một sắc thái chứ? Đoàn Cầm Thi khẳng định rằng "thơ - photo" của Đỗ Kh. "không tâm tình". Không tâm tình thì khoe chân khoe đít của mình để làm gì, nhất là khi tấm hình có nội dung chế nhạo, nghĩa là nó muốn "tâm sự / tâm tình" cảm nghĩ riêng về thơ ca của người chụp ảnh? Chế giễu, chọc quê người khác nhưng không khiêu khích? Không ẩn dụ nhưng "chân ở đây được hiểu trong mọi cách. Chân như bộ phận của cơ thể, nhưng còn là chân… thực, chân… thật". Không gây sốc nhưng luôn luôn muốn làm khác thiên hạ. 
 
Đỗ Kh. tuyên bố, qua trích dẫn của Đoàn Cầm Thi: \textit{"Với tôi, nước Pháp là con đường dẫn đến thế giới"}. Đoàn Cầm Thi nhận định thêm về Đỗ Kh.: "Thấm đẫm văn hoá Pháp, theo nghĩa nhân học, [...] không chỉ trong trí não mà ở đời sống hàng ngày, nó định nghĩa [Đỗ Kh.] như một cá nhân"; "các nhân vật của [Đỗ Kh.] tung hoành khắp Paris, say sưa kể về từng khu phố, biết tận gốc rễ và biến chuyển của từng xóm nhỏ. Tác phẩm của Đỗ Kh. hiển hiện điều này: Pháp thực sự như quê hương thứ hai, đào tạo anh những năm đầu tuổi trẻ, là nguồn cảm hứng thi ca và giữ một vị trí đặc biệt trong thế giới sáng tạo của anh" (“Đỗ Kh. — người của bốn phương\footnote{\url{http://www.talawas.org/talaDB/http://www.tienve.org/home/literature/viewLiterature.do;jsessionid=8778CF3F71399CD4CFAE2DC544FC2644?action=viewArtwork&artworkId=6468}}”). "Còn tính hiện thực [của tấm ảnh "\textit{Buồn trong khách sạn: chân dung tự chụp"}] thì ngay cả các bậc thầy của văn học tả chân chắc cũng phải thèm." Xin miễn bình luận. 
 
Đoàn Cầm Thi trích dẫn Barthes, Deleuze, Robbe-Grillet, Proust để minh họa mà không cần suy luận thêm, như thể họ là đấng tối cao của một tôn giáo bất khả xâm phạm. Trần Dần có thể là anh hùng của \textit{Nhân văn - Giai phẩm} và là nạn nhân của Đảng Cộng sản. Như vậy không có nghĩa là mỗi câu thơ Trần Dần viết ra đều đáng được tôn sùng. Vì thấy lối viết của Đoàn Cầm Thi quá "tân kỳ" và quá "chân thật" như trong câu: "\textit{Thơ - video} và \textit{thơ -photo} không phải để ngâm để đọc mà là để xem" nên tôi mạo muội trình bày ra đây vài cảm nghĩ của mình. Câu hỏi chót: Các thi nhân nam lẫn nữ không có thân hình tuyệt mỹ, có nên đem các "thi phẩm tả chân" của mình ra phơi bày hay không? Thi nhân ngắm nghía chân, đít của mình có khác với kẻ chỉ nhìn vào cái lỗ rốn của mình hay không?  
 
Ảnh (photo) đã được chủ nghĩa siêu thực sử dụng và khai thác từ lâu rồi. Hình ảnh do computer ráp nối đầy dẫy trong các video-DVD ca nhạc hiện đại. "Thơ - video" và "thơ -photo" theo kiểu Đỗ Kh. là thơ của thế kỹ 21? Riêng tôi, trong lúc Trung Quốc ngang nhiên cưỡng chiếm Hoàng Sa và bắn giết ở Tây Tạng, mà lại có những buổi sinh hoạt, trưng bày, biên khảo "Thơ không có địa lý, dân tộc, màu da. Thơ không có tính từ" thì "câu hỏi quyết liệt" không phải là "khi nào có thơ", mà chỉ có thể là: "Mùa xuân của thi sĩ" hay là "Ngày tàn của thi ca"? 
 
© 2008 talawas 
\end{multicols}
\end{document}