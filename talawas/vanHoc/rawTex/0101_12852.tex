\documentclass[../main.tex]{subfiles}

\begin{document}

\chapter{Về thơ photo-video không “điạ lý, dân tộc, màu da”}

\begin{metadata}

\begin{flushright}12.4.2008\end{flushright}

Phan Nhiên Hạo



\end{metadata}

\begin{multicols}{2}

<p>Đoàn Cầm Thi\footnote{\url{http://www.talawas.org/talaDB/showFile.php?res=12829&rb=12}} viết: “Tôi không hiểu vì lý do gì mà cái tên của Đỗ Kh. và những tấm ảnh thi sĩ tự chụp có khả năng kích động độc giả Nguyễn Đăng Thường\footnote{\url{http://www.talawas.org/talaDB/showFile.php?res=12812&rb=0101}} đến như vậy?” 

Vài năm trước, Đỗ Kh. bày trò “thơ” chụp hình khoe rốn trên \textit{Tạp chí Thơ}, tôi nhớ trong cuốn sách \textit{Biên tà tà}, nhà văn Hoàng Mai Đạt dí dỏm coi trò này là biểu hiệu của khủng hoảng tâm lý tuổi trung niên (mid-life crisis). Thiểt nghĩ cái tên Đỗ Kh. và những tấm ảnh tự chụp chân đít của ông cùng lắm cũng chỉ khiến thiên hạ cười ruồi, chẳng việc chi phải “kích động”, nếu chúng đã không được phê bình gia Đoàn Cầm Thi\footnote{\url{http://www.talawas.org/talaDB/http://damau.org/index.php?option=com_content&task=view&id=3483&Itemid=10171}} trịnh trọng nâng cao, và như cách viết thường lệ của bà, kèm theo những tuyên ngôn trịnh trọng về thơ. 

Sử dụng video và photo đã trở thành trò thường trong nghệ thuật ý niệm Âu Mỹ hiện nay, không còn mới, nếu không muốn nói là đã phần nào trở nên “chính thống” và bị một số trường phái đương đại như Stuckism\footnote{\url{http://www.talawas.org/talaDB/http://en.wikipedia.org/wiki/Stuckism}} \footnote{
Xem Wikipedia:} chống đối. Chỉ thảm là với văn chương Việt Nam, trò này vẫn còn khiến những người như tiến sĩ Đoàn Cầm Thi xúc động phát hiện ra những điều “mà nhiều người khác có nằm mơ cũng không thấy được” \footnote{
Ý kiến của Nhỏ Thanh, talawas, 9/4/2008} . Xúc động mạnh đến nỗi, Đoàn Cầm Thi sẵn sàng phạm sai lầm căn bản trong phê bình là nhảy vào “lòng” các tác giả để suy diễn thay cho họ. Đoàn Cầm Thi viết: 

“Tôi xin quả quyết, chúng [video, photo] không được dùng để minh họa. Nói cách khác, trong quan niệm của Đinh Linh và Đỗ Kh., \textit{thơ-photo} và \textit{thơ-video} là một nghệ thuật hoàn toàn mới, trong đó hình ảnh và con chữ làm nảy sinh lẫn nhau’’. 

Với Đỗ Kh., trò này có mới hay không, tôi không biết. Nhưng về phần Đinh Linh, tôi không hiểu Đoàn Cầm Thi dựa vào đâu để “quả quyết” Đinh Linh quan niệm “thơ-photo và thơ-video là một nghệ thuật hoàn toàn mới”. Trong phỏng vấn trên đài Á châu Tự do\footnote{\url{http://www.talawas.org/talaDB/http://www.rfa.org/vietnamese/in_depth/2007/02/25/PoetAndWriterDinhLinhP2_Mthuy/}} cách đây không lâu, nhân bàn về thơ Tân hình thức, Đinh Linh nói: “Đây đúng là trường hợp cũ người mới ta (…) Mình có quyền bắt chước người ta nhưng không nên làm rầm rộ, trịnh trọng quá, thấy rất buồn cười” \footnote{
http://www.rfa.org/vietnamese/in_depth/2007/02/25/PoetAndWriterDinhLinhP2_Mthuy/\footnote{\url{http://www.talawas.org/talaDB/http://www.rfa.org/vietnamese/in_depth/2007/02/25/PoetAndWriterDinhLinhP2_Mthuy/}}} . Tôi không tin một người đã nhận định như vậy về thơ Tân hình thức, một người quen thuộc với sinh hoạt văn chương Mỹ như Đinh Linh lại cho rằng thơ video-photo là “hoàn toàn mới”. Quen biết cá nhân nhiều năm giữa Đinh Linh và tôi cũng cho phép tôi tin rằng tác giả này không thể là một người “nhà quê” như Đoàn Cầm Thi suy diễn. Đinh Linh là người không câu nệ, thích gì làm nấy, nhưng không phải dạng ảo tưởng mình đang cách mạng văn chương. 

Đoàn Cầm Thi có vẻ dễ xúc động trước những thứ mà bà muốn đề cao, đến nỗi thuờng phán những câu gần như ấu trĩ. Bàn về một tác phẩm “thơ-photo” của Đinh Linh, Đoàn Cầm Thi ca: “Còn tính hiện thực của nó thì ngay cả các bậc thầy của văn học tả chân chắc cũng phải thèm.” Bậc thầy nào thèm vậy? So sánh tính tả chân của ảnh chụp với mô tả văn chương, thật không gì ngớ ngẩn hơn. Nói kiểu Đoàn Cầm Thi, các nhà văn nên vứt bút vào sọt rác cho rồi, muốn tả gì chỉ việc chụp ảnh dán vào sách cho khoẻ. Các hoạ sĩ cũng không nên phí thời gian rị mọ vẽ chân dung hay phong cảnh, mua máy hình kỹ thuật số về sáng tác cho nhanh. 

Đoàn Cầm Thi trách “độc giả” (!) Nguyễn Đăng Thường chỉ nhắm vào Đỗ Kh. mà không nói đến Đinh Linh, trách cứ này chỉ là chiến thuật lôi kéo cá nhân trong tranh luận. Không phải Đoàn Cầm Thi viết về bao nhiêu người là “độc giả” nhất thiết phải tranh luận với bà về chừng đó người. Nhân vật nào có vấn đề thì mới cần đề cập đến thôi. Mặc dù Đoàn Cầm Thi cố đánh đồng Đỗ Kh. với Đinh Linh, hai tác giả này vẫn rất khác nhau. Đinh Linh đến Mỹ năm 11 tuổi, viết chủ yếu bằng tiếng Anh, sinh hoạt trong văn giới Mỹ. Đỗ Kh. đến Pháp năm 20 tuổi, viết chủ yếu bằng tiếng Việt, sinh hoạt trong “thế giới” người Việt. Có thể nói, đến Mỹ lúc 11 tuổi, Đinh Linh là người Mỹ gốc Việt, trong khi đến Pháp năm 20 tuổi, Đỗ Kh. là người Việt… gốc me. Xin lỗi về cách nói đùa này, nhưng sự thực, ở hải ngoại ai cũng biết nhập cư ở tuổi 11 và tuổi 20 khác nhau một trời một vực về khả năng ngoại ngữ và mức độ ảnh hưởng văn hóa. Đỗ Kh. từng được Đoàn Cầm Thi ca ngợi như người của bốn phương tám hướng gì đó, nhưng đọc ông, thấy rất rõ cái não trạng nông dân Việt Nam tinh ranh và lòng vòng, thiếu hẳn tinh thần thẳng thắn của văn hoá phương Tây. Những địa danh xa lạ và tiện nghi sinh hoạt được kể lể dông dài trong văn chuơng Đỗ Kh. tuy dễ tạo ấn tượng “quốc tế” với người đọc ít có dịp đi đây đó, nhưng đối với người hiểu rõ đời sống ngoài này, chúng chỉ là mớ đồ đạc loè loẹt trong căn phòng trưởng giả tỉnh lẻ thích khoe khoang. Tôi sẽ bàn thêm về văn chương Đỗ Kh. một dịp khác. Vấn đề tôi muốn nói ở đây là việc Đoàn Cầm Thi bỏ Đinh Linh vào chung rọ với Đỗ Kh. để nói rằng cả hai cùng làm thơ phi “địa lý, dân tộc, màu da”, tôi thấy không thuyết phục. Đoàn Cầm Thi chắc chưa đọc những suy nghĩ của Đinh Linh về xung đột văn hóa, chủng tộc, về đời sống Việt-Mỹ trong một phỏng vấn với Leakthina Chau-Pech Ollier đăng trong tuyển tập \textit{Of Vienam Identities in Dialogue \footnote{
\textit{Of Vietnam Identities in Dialogue}, Ed. By Winston and Ollier. Palgrave, 2001.} }. Đoàn Cầm Thi cũng sẽ nói gì về vô số những bài thơ liên quan đến Việt Nam của Đinh Linh? 

“Thơ photo-video” chỉ là một phần trong sáng tác của Đinh Linh gần đây, và tôi sẽ thẳng thắn nói rằng tôi không quan tâm đến chúng. Đinh Linh từng viết những truyện ngắn xuất sắc mà tôi đã chọn dịch \footnote{
Tuyển tập truyện ngắn \textit{Thư lạ}, Đinh Linh. Phan Nhiên Hạo chọn dịch. Văn Mới, 2007. \end{center}} , những bài thơ thông minh, ý nhị. Nhưng không phải tất cả sáng tác của một người viết nhiều viết nhanh như Đinh Linh đều hay. Những sáng tác photo-video của Đinh Linh, tôi nghĩ, có vẻ chỉ là trò giải lao giữa những tác phẩm đòi hỏi nhiều lao động nghệ thuật hơn. Không may, những tác phẩm này lại được Đoàn Cầm Thi sử dụng để tô vẽ chân dung thơ ca của Đinh Linh một cách rất phiến diện, dùng minh họa cho phát ngôn nghệ thuật kêu vang của bà. 

Tôi không biết quan niệm thơ phi “địa lý, dân tộc, màu da” mà Đoàn Cầm Thi đề cao có phải là một quan niệm nghệ thuật đang thịnh hành ở Pháp hay không, nhưng tôi có thể nói, ở Mỹ, một quan niệm như vậy chỉ có thể có thể được xem là biểu hiện của sự thiếu hiểu biết về xã hội đa văn hoá. Ở Mỹ, không gì lố bịch và gây kinh ngạc hơn việc một nhà văn gốc dân thiểu số tuyên bố rằng mình đang làm nghệ thuật vượt lên trên “địa lý, dân tộc, màu da”. Mà thật ra, liệu người ta có khả năng làm một thứ nghệ thuật vô tính như vậy không? Một cái chân trong ảnh tự nó đã là một cái chân da trắng, vàng hay đen, chụp ở Đức hay Hà Nội, trong khách sạn hạng sang hay rẻ tiền. Nó không thể phi địa lý, dân tộc, màu da. 

Đọc Đoàn Cầm Thi tôi thường tự hỏi, cái quán tính văn hóa nào đã khiến một người được đào tạo từ trường lớp phương Tây như bà - một trường hợp còn “quý hiếm” trong văn chương Việt hiện nay - vẫn không thể viết phê bình với phong độ trầm tĩnh của học giả, và nhất là không phải dùng đến những trò bùn đất như chụp mũ công an hay ám chỉ khuynh hướng tình dục của đối thủ. \end{center}

<p>© 2008 talawas

\end{center}



\end{multicols}
\end{document}