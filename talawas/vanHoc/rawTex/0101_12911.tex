\documentclass[../main.tex]{subfiles}

\begin{document}

\chapter{Trình diễn Thơ Việt, một chặng đường…}

\begin{metadata}

\begin{flushright}18.4.2008\end{flushright}

Yến Nhi



\end{metadata}

\begin{multicols}{2}

Cho đến nay Hà Nội và Thành phố Hồ Chí Minh đã tổ chức có trên dăm buổi trình diễn Thơ, không kể trong “Ngày Thơ” hàng năm nơi Văn Miếu. Người tham gia không ít, có người khen, có vị nghi ngờ, nhìn chung vẫn còn dè dặt, đó cũng là điều dễ hiểu khi một cái mới xuất hiện, dù là của nghệ thuật đầy tính sáng tạo hay gì gì chăng nữa! 
 
Trình diễn Thơ (người xem gọi là biểu diễn thơ) như trên đã nói thực ra không có gì là lạ lắm với công chúng yêu thơ Việt. Về nguyên tắc, đó là sự kết hợp thơ với một vài thủ thuật biểu diễn như nhạc, vũ, hoạ… mà trong truyền thống ta cũng đã gặp riêng lẻ đâu đó ở các buổi đọc (tấu, ngâm) thơ, hát thơ (ca trù), diễn thơ (trò Kiều)…  Cái kích thích người xem hôm nay là buổi diễn tận dụng được các phương tiện hiện đại của kỹ thuật âm thanh, ánh sáng sắc màu và vũ điệu tự do cùng kết hợp và luôn thay đổi, tạo nên một cái “lạ”, và đặc biệt là lúc nào cũng có sự đích thân xuất hiện và trình bày của tác giả. Các tác giả nhiệt tình tìm cách đổi mới Thơ, để Thơ đến với độc giả ngoài cách thức in trên giấy, với nhiều hiệu ứng thẩm mỹ, tựu trung không ngoài mục đích giúp người đọc cảm thụ hình tượng thơ tốt hơn. 
 
Trình diễn Thơ có cái giống với việc biểu diễn thơ ở việc phối kết các thể loaị nghệ thuật, nhưng cũng có cái khác. Trong nhiều thể nghiệm sự liên kết các loại hình nghệ thuật mà chủ nghĩa hậu hiện đại nêu lên thì sự pha trộn nhiều thể loại vào một tác phẩm, sự kết hợp đồng thời văn bản ngôn ngữ với màu sắc, âm thanh, hình khối… trong viêc sáng tạo được đề cao. Đấy là sự liên kết nội tại trong kết cấu hình tượng nghệ thuật mà không làm mất bản chất đặc trưng của loại thể. Thơ vẫn cứ phải là thơ, văn vẫn là văn, kịch, vẫn là kịch, hoạ vẫn là hoạ, nhạc vẫn là nhạc… chứ không phải làm một phép cộng tất cả để thở thành một loại nghệ thuật đa nguyên, nói là gì cũng được! 
 
Ở các nước, người ta cũng đã thực hiện từ lâu nhiều cuộc trình diễn thơ, đưa nó thành một thể tài nghệ thuât riêng gọi là “Thơ trình diễn” (Performance Poetry) có người còn gọi là “Thơ đa phương tiện”. 
 
Lý thuyết của các nhà thơ trình diễn có thể tóm lược như sau: “… Nếu như thơ ngôn ngữ tìm lối phát minh ra một tương lai thông qua văn bản viết, thì thơ trình diễn mang trong nó hoài cảm về một quá khứ hoàn hảo hơn, khi mà sự truyền miệng là căn bản”, khởi thuỷ là “những bài thơ nói (talk poem)”, sau đó “nó không còn dựa trên việc đọc một văn bản đã in ra cho một cử toạ nghe, mà đúng hơn là khuyếch trương văn bản đã được chuẩn bị, nếu như có một văn bản như thế, thông qua sân khấu và nghi thức… đưa thơ trở lại với gốc gác cộng đồng của nó. Thơ trình diễn ngầm thách thức sự quí giá của trang viết và khái niệm bài thơ như một hệ thống đóng.” (“Thơ trình diễn ở Mỹ” - Paul Hoover). \footnote{
Hoàng Hưng dịch – “Vài nét về thơ trình diễn” (Performance Poetry), Jack Bowman 2001 (theo tư liệu trên Internet).}  
 
Trong sự thổ lộ trên, ta thấy các nhà thơ trình diễn muốn đưa Thơ trở lại gốc gác cộng đồng (thơ ca diễn xướng dân gian) bằng phương thức truyền miệng cùng những nghi thức sân khấu. Thơ trình diễn không “đóng kín” trong trang viết mà “mở’ trong biểu diễn bằng giọng nói, bằng động tác giao lưu với công chúng. \footnote{
Tồn tại ban đầu của thơ ca dân gian là sự diễn xướng có tính nguyên hợp. Những câu ca xuất phát từ lao động, trình diễn trong lao động, hỗ trợ cho công việc. Lúc đó có sự hoà hợp giữa người lao động, công cụ và ngôn ngữ. Thí dụ, các bài ca “chặt cây”, bài hát “chèo đò”… được cất lên cùng lúc người ta làm những động tác chèo đò, chặt cây với rìu rựa hoặc bơi chèo, theo nhịp điệu có sẵn trong bài ca. Đó là cái triết lý mà các nhà thơ trình diễn muốn thực hiện để “quay trở lại nguồn gốc cộng đồng trong một quá khứ hoàn hảo”.}  Những tư tưởng trên của Thơ trình diễn dẫu sao cũng có một cơ sở triết lý nhân bản về ý thức hoà hợp cộng đồng, “hoài cảm về một quá khứ hoàn hảo” đáng trân trọng khi sự chia cách giữa con người và con người không quá mạnh mẽ, xa lạ! 
 
Hoài vọng thì thế, nhưng trải mấy mươi năm theo đuổi không mỏi mệt, các tác giả cũng chỉ tạo được một lối đi hẹp cho Thơ qua số buổi biểu diễn nhất định với các bài thơ chọn lọc thích hợp, chứ không phải thực hiện được đại trà như khởi thuỷ diễn xướng Thơ ca Dân gian hay lối đọc thơ khá thịnh hành trong bối cảnh hiện đại! Người tham dự chỉ thưởng thức như ghé qua một trò thể nghiệm của nghệ thuật nghe nhìn! 
 
Vì sao? Vì cái lý do để nó tồn tại không nằm trong quy chuẩn của loại hình. 
 
Tác phẩm nghệ thuật nói chung và thơ ca nói riêng, sức mạnh và lẽ tồn tại của nó nằm ở chính “hình tượng nghệ  thuật”. Hình tượng nghệ thuật thơ được taọ nên bởi ngôn ngữ. Nó có màu sắc cũng do ngôn ngữ, nó giàu âm thanh cũng nhờ ngôn ngữ, và có hình khối cũng bởi ngôn ngữ. Sự tồn tại và phát triển của nó không thể thoát ly khỏi ngôn ngữ! Ấy thế mà giờ đây cái chỉnh thể tồn tại của Thơ đã thay đổi, từ Hình-Tượng-Thơ-Ngôn-Ngữ, chuyển sang Tác-Giả-Thơ-Vũ-Nhạc. Liệu như vậy có còn gọi là phát triển, làm đẹp thơ hay không! Các nhà thơ trẻ thích trình diễn thơ vì họ muốn được tiếp cận độc giả trực tiếp, để độc giả trông thấy họ nhãn tiền như ca sĩ, như người mẫu, như vũ công… chứ cái gọi là Thơ lùi về sau mất rồi. Hình tượng Thơ bị xoá nhoà chính bởi cái nhan sắc, cái vũ điệu, cái âm thanh mà các tác giả tạo nên! Công chúng sau những buổi xem trình diễn khó mà nhớ được gì nhiều về những đặc sắc của hình tượng thơ mà chỉ nhớ cái sự phô diễn ngộ nghĩnh là lạ của các tác giả. Bài thơ hay dở không cần biết, chủ yếu là tác giả diễn Thơ có kích thích họ không? Vấn đề sức sống của bài thơ, của hình tượng Thơ chỉ còn thu hẹp vào trò diễn của nhà thơ. Chắc chúng ta đã từng hơn một lần được nghe các nhà thơ nổi tiếng tâm sự: Công chúng có thể chỉ nhớ bài thơ mà quên đi tác giả. Mà bài thơ tồn tại bằng chính ngôn ngữ của nó. Còn bây giờ thì sao? 
 
Bản chất của thơ, văn là nghệ thuật ngôn ngữ. Những phụ gia nào biến chúng thành thứ nghệ thuật khác chắc hẳn sẽ làm chúng ngả màu. Thơ chỉ ở thế thượng phong khi bám vào sức mạnh của ngôn ngữ khai thác trí tưởng tượng của người đọc, đi sâu vào địa hạt thầm kín tâm tư tình cảm của con người mà các thể loại khác không có được lợi thế như nó. Thơ ca dân gian ngày xưa chủ yếu sáng tác truyền khẩu gắn với diễn xướng là sản phẩm của một thời kỳ khoa học chưa phát triển, con người chưa có điều kiện suy ngẫm; nó gắn với lao động trực tiếp phục vụ lao động. Thơ ca ngày càng tách ra đi riêng con đường của nó, khai phá miền sâu kín tiềm ẩn trong từng cá thể, kể cả những miền ảo giác, vô thức. Khi nó nhờ vả nhiều hoặc muốn lấn sân thứ nghệ thuật khác thì nó sẽ mất dần sức mạnh cố hữu cùng nhiệm vụ cao quý, mà chỉ còn là thứ nghệ thuật mua vui tầm tầm, vì tác động vào công chúng bằng âm thanh thì kém xa âm nhạc, bằng ánh sáng và màu sắc không thể sánh với hội hoạ, điện ảnh… Thơ không in lên giấy thì cứ đọc cho thính giả nghe, đọc có ngữ điệu có thể có vài động tác minh hoạ, công chúng sẽ thưởng thức, sẽ lĩnh hội qua ngôn từ, từ chính ngôn từ làm lan toả khơi dậy bao tưởng tượng trong tâm tư người đọc, người nghe. Những thứ phụ gia khác không thể thay sức mạnh của ngôn từ. Trình diễn Thơ không nên biến thành tạp kỹ thơ, không nên làm công chúng quên mất lời thơ, chỉ nhớ các động tác uốn éo hào nhoáng, những âm thành cuồng nộ và mấy đoạn phim gợi tò mò… 
 
Trong sự phát triển của nghệ thuật hiện đại có nhiều cái mới, cũng có thể mới ở nơi này nhưng lại cũ nơi kia! Rất trân trọng các tìm tòi, đổi mới, sáng tạo của các tác giả mong tìm một con đường không nhàm chán để đến với công chúng một cách ấn tượng truyền cảm. Tuy nhiên, đừng vì cái lạ mà quên mất cái đẹp, cao nhã và thanh khiết mà hình tượng thơ tiềm ẩn, đừng đánh mất bản chất Thơ. Các thủ pháp âm thanh, màu sắc, hình khối, ánh sáng... của các thể loại kịch, múa, điện ảnh… khi liên kết với Thơ là một con dao hai lưỡi, cần chú ý đến liều lượng và ý nghĩa thẩm mỹ khi sử dụng, cái khó là nó phải kết hợp từ bên trong kết cấu hình tượng nghệ thuật, bên trong tác phẩm, chứ không phải bên ngoài như một phép cộng tầm thường. 
 
Từ các trò diễn của các tác giả thơ chúng tôi thoáng nghĩ, cũng như các kiểu chạy theo mode thời thượng khác của các tác giả trẻ, họ tiếp cận nghệ thuật mới thường ở ngọn, ở nhánh mà quên đi cái gốc của nó. Đó là sự sống là cơ sở triết học làm nảy sinh các trường phái nghệ thuật. Muốn có một đổi mới, đích thực các nghệ sĩ phải thấm nhuần sâu sắc cái bề sâu triết học nhân bản của nó, từ đó mới hiểu được những đặc điểm của các thủ pháp nghệ thuật mà nó mang theo để mà ứng dụng cho phù hợp hoàn cảnh của đất nước, của công chúng mình, chứ đừng dừng lại sự bắt chước vài biểu hiện đầu ngọn theo kiểu ham thanh chuộng lạ, nếu không lạc lõng thì e cũng chẳng bền lâu! 
 
Từ buổi trình diễn Thơ ban đầu đến nay, đoạn đường các tác giả trải qua không ít khó khăn, nhưng không phải không có những sàng lọc, lựa chọn, để đến nay ít nhiều cũng đã có chỗ đứng trong lòng một số công chúng yêu thơ. Từ những hình ảnh ồn ào kỳ khu hôm nào mà người diễn như những con rối đa năng, đưa đến cảm nghĩ người xem: “… Tôi cũng có vài lần đi nghe thơ bên Mỹ, nhưng chưa lần nào thấy như vậy. Tôi cũng không nghĩ mấy nhà thơ trong đêm trình diễn này sẽ làm cái gì ngoạn mục kiểu như thế, nhưng cũng tò mò muốn biết thơ sẽ ra sao khi trình diễn theo nghĩa mới, nhất là ở ta.” (theo Trịnh Lữ\footnote{\url{http://www.talawas.org/talaDB/showFile.php?res=9110&rb=0206}}) \footnote{
http://www.talawas.org/talaDB/suche.php?res=9110&rb=0206\footnote{\url{http://www.talawas.org/talaDB/../talaDB/suche.php?res=9110&rb=0206}}}  
 
Cho đến bây giờ đã thay đổi, đã có sự chọn lựa cân nhắc tạo đồng cảm ở công chúng. Xin hãy tham khảo một tiết mục trình diễn thành công gần đây: 
 
Tác giả đứng giữa sân khấu, phủ kín thân bằng một tấm vải trắng. Phía sau là một màn hình video với hình ảnh dòng nước chảy ào ạt. Người xem có cảm giác như đang hoà mình vào thác nước, “... xúc động nhất là khi [tác giả] ngồi cô độc trên ghế, đọc bài thơ "Người đàn bà và căn nhà cổ". Sự suy tưởng của một cô gái trẻ về những giá trị cổ xưa đang ngày càng biến mất khiến người nghe day dứt... Một vài người trong số cử toạ kín đáo lau nước mắt.” (theo Hà Linh) \footnote{
http://www.evan.com.vn/News/doi-song-van-nghe/2008/03/3B9ADD2F/\footnote{\url{http://www.talawas.org/talaDB/http://www.evan.com.vn/News/doi-song-van-nghe/2008/03/3B9ADD2F/}}}  
 
Trình diễn thơ muốn có những kết quả nhất định, trong những tìm tòi đổi mới có lẽ cần lưu ý là tránh sự lạm phát âm thanh, màu sắc, ánh sáng, vũ điệu… vì như lời khuyên của một tác giả từng cổ suý nhiều cho sự cách tân thơ: “... việc trình diễn vẫn lấy đọc diễn cảm là chủ đạo, phối hợp với động tác hình thể và âm nhạc. Bởi điều cốt yếu của trình diễn thơ vẫn là làm sao cho lời thơ âm vang và thấm sâu vào lòng khán giả.” \footnote{
Hoàng Hưng,}  
  
Và như vậy là đã trở lại điểm xuất phát như cha anh ta  đã  làm về Thơ, chỉ mong hậu sinh làm hay hơn, “cái gì của Caesar hãy trả cho Caesar” vậy! 
 
Người viết bài này vẫn mong cho sự thành công của các nhà thơ trình diễn, vì đó là một khuynh hướng đang có nhiều cố gắng, cũng đang thu hút được khán thính giả, có một vài tiết mục gây được ấn tượng với công chúng; cái còn lại trước mắt là các bạn trẻ đừng nóng vội. Cần thâm nhập và thấm nhuần sâu sắc hơn cái bề dày triết học của thứ nghệ thuật mới ở xứ người, đem ứng dụng cho có ích ở xứ ta. 
 
© 2008 talawas



\end{multicols}
\end{document}