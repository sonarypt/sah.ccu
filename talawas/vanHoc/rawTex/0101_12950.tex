\documentclass[../main.tex]{subfiles}

\begin{document}

\chapter{Lê Đạt tư duy về thơ}

\begin{metadata}

\begin{flushright}23.4.2008\end{flushright}

Trần Thiện Khanh



\end{metadata}

\begin{multicols}{2}

Người thơ ấy đã thiên di về một miền xa xôi và vĩnh cửu, sau mọi nỗ lực cách tân thơ và chủ trương tự do cho văn học. Thành bại đến đâu và vì sao... chỉ có lịch sử mới thấu hiểu và trả lời một cách công minh nhất. Tôi - kẻ hậu sinh trộm biết Lê Đạt chẳng thể nào thoát được kiếp văn nhân. Ngay từ thuở ông “biết nghĩ”, biết viết thơ, thân hữu của ông đã thấy ở ông lộ ra những dấu hiệu của bi kịch. 
 
Lê Đạt  từng ví thi phận mình giống với đời cô Kiều bị “\textit{ma đưa lối quỷ dẫn đường}” nên bỗng đâu “\textit{lại tìm những lối đoạn trường mà đi}”. Chẳng phải ngẫu nhiên, ông sắm vai “kẻ cu li”, nhọc lòng khai thông mạch chữ. Và có lẽ bây giờ ông đang “cõng chữ lên non”. Mong rằng trên non cao ấy, vào giây phút hạnh ngộ với chúa trời, sẽ có một làn gió ập đến thổi bay đi “mặc cảm tội lỗi” ở ông. Tôi tin được vậy, thì ông thanh thản lắm. Thanh thản để can đảm sống tiếp kiếp phù sinh khác.        
\begin{blockquote}
 
\textit{Chiều Âu Lâu } 
<span style="padding-left: 30px">\textit{                      bóng chữ động chân cầu}</span> 

\end{blockquote}
 
Bản thân Lê Đạt được nuôi dưỡng bằng bầu không khí hào hứng tái sinh văn nghệ. Cho nên, thơ Lê Đạt đến giờ, lại có dịp may thức giấc, khe khẽ thì thào, sau một đêm dài ngỡ chỉ còn xác chữ. Tự Lê Đạt mua oán chuốc sầu, thì trách làm chi cái công vớt lấp muộn mằn của người năm đó. Bâng khuâng lạc vào “bóng chữ” của ông \footnote{
Bài viết này chỉ điểm lại một số quan điểm thơ ca của nhà thơ Lê Đạt.} . 
 
Lê Đạt từng nói: trong các môn nghệ thuật, có lẽ “thơ chịu nhiều hiểu lầm hơn cả”. Mối nghi ngờ số phận hẩm hiu của thơ ở Lê Đạt có nguyên nhân sâu xa từ nỗi ám ảnh nhân sinh. Tôi đoán ông thừa biết cách viết tế nhị để diễn tả cho hay bi kịch của con chữ. Song ông không làm thế, vì e mình tước mất đời sống bí mật, phong phú của phát ngôn thơ. Lê Đạt thường sử dụng thủ pháp lạ hoá từ, lạ hoá tiếng nói quen thuộc. Nhiều đoạn thơ, nhiều bài thơ nhạc tính khá ấn tượng.          
\begin{blockquote}
 
\textit{Cô gái trộm hái sen }        
<span style="padding-left: 30px">\textit{   		về ủ tuổi}</span> 
\textit{Lỏng khuy cài } 
<span style="padding-left: 30px">\textit{   		 gió cởi }</span> 
<span style="padding-left: 60px">\textit{     			một dòng hương}</span> 
        
\textit{Ngực dự hương đêm thơm mùi tuổi chín }        
\textit{Mắt lá tre đằng ngâm mộng ba giăng} 

\end{blockquote}
 
Mảnh đất thơ của Lê Đạt không thật rộng \footnote{
Có bài chỉ có hai câu theo kiểu thơ Haiku: “Mimôza”, “Vũ ẻn”, “Xôlô”, “Bống bống”, “Tương tư”, “Áo trắng”, “Tuổi chín”, “Vải Thanh Hà”, “Tóc phố”, “Mộng cũ”, “Trêu ngươi”, “Mắt bão”...} , nhưng ở đó ẩn giấu cả một quặng nghĩa. Lê Đạt không vơ vào thơ tất cả xương cốt của chữ, tức xác chữ. Ông lấy cái hồn của chữ, cái bóng sáng của chữ để làm nên giá trị cho câu thơ, bài thơ. Những bước gập gềnh của câu chữ trong thơ Lê Đạt có gì đó gần với thăng trầm của đời ông. Lê Đạt bảo: Thơ trẻ còn nhiều lời mà nghĩa lại lộ thiên. 
 
Lê Đạt từng phê phán những nhà thơ biến sáng tác của mình thành một hoạt động thần bí. Nhưng ông lại rất tinh khi nhìn thấy sau cái phẩy bút thành thơ của thi nhân đời Đường một sự lao động cực nhọc. Mọi nhà thơ trước lúc trở thành cây đại thụ trên thi đàn, họ đều làm phu chữ. Lê Đạt tạo ra chữ, cốt sao cho nó sống được. Chữ sống thì nó sẽ sản sinh ra ý nghĩa, tư tưởng. 
 
Thơ Lê Đạt nặng nợ đời. Tâm tính ông không phù hợp với loại người chỉ đợi chờ ông trời ban cho cảm hứng. Ông nhấn mạnh: viết theo hứng thì thường đổ chữ ào ạt lên khuôn giấy. Chưa bao giờ Lê Đạt ưa hạng độc giả mà ngoài ca tụng khả năng ngoại cảm của người sáng tạo thì không còn hay biết gì nữa. Theo Lê Đạt, cuộc phiêu lưu của nhiều nhà lí luận về miền thơ cũng rơi vào tình trạng tuyệt mù ấy. Ông khăng khăng bảo cả nhà lí luận và nhà thơ tầm bậc trung đều đã từng mắc tội đối với thơ ca. 
 
Trong bài viết “Nghiệp thơ”, Lê Đạt bày tỏ thẳng thắn suy nghĩ của mình: “Theo tôi thơ là một nghề. Đã là một nghề thì phải có kỷ luật lao động. Không nên thụ động thắp hương chờ mà phải chủ động gọi hứng đến. Công việc này đòi hỏi một kỷ luật nghiệt ngã và gian khổ.” Vậy ra, Lê Đạt đề cao sự viết, sự năng động của chủ thể trong sáng tạo ngôn từ. Ông xem, mỗi nhà thơ phải đề ra cho mình một mô hình sáng tạo, một nguyên tắc tổ chức tác phẩm. Hơn nữa, phải kiên trì theo đuổi lối viết ấy mới hòng tạo dựng được phong cách cá nhân. Người viết, nếu không khắt khe với chính mình, không trọng con chữ, thì chẳng những dễ thất nghiệp mà còn yểu mệnh. 
 
Lê Đạt chọn điểm rơi cho từ ngữ một cách tài hoa. Ông chơi chữ, nghịch chữ và tạo cho thơ mình một điệu nhạc riêng, một “thiên lí chữ” không dễ lặp lại. Nếu chẳng phải người kiên trì ngậm ngải tìm chữ, Lê Đạt chẳng thể nào tạo ra được những hình ảnh thơ độc đáo nhường này: 
\begin{blockquote}
        
\textit{Cây gạo già         
lơi tình } 
\textit{lên hiệu đỏ}        
\textit{La lả cành         
cởi thắm } 
\textit{để hoa bay} 
        
\textit{Bước đệm }        
\textit{đưa tình }        
\textit{xanh khúc phố}        
\textit{Nốt chân xuân }        
\textit{đàn cò lạ } 
\textit{phím lùa} 

\end{blockquote}
 
Với Lê Đạt, lao động thơ rất khổ ải. Chỉ ngay việc ngày ngày nhà thơ tìm tòi chữ nghĩa, viết ra rồi xoá bỏ biết bao nhiêu lần cũng đủ thấy nhà thơ cực nhọc lắm mới có câu hay, từ đắt. Cái thú trong quan niệm của Lê Đạt ở chỗ: ông xem những từ ngữ do nhà thơ ngậm đắng nuốt cay, bạc đầu khổ luyện mới có được đều ẩn giấu một nền tảng văn hoá nhất định. Thước đo tài năng nằm ở khả năng tìm được các câu chữ lấp lánh văn hoá tư tưởng, chứ không phải ở sự viết nhanh, viết nhiều: “Tôi không thích những thần đồng. Tôi yêu những người lao động có tri thức, một nắng hai sương trên cánh đồng chữ”. Đây “bông chữ” mà nhà thơ Lê Đạt đem từ cánh đồng về:        
\begin{blockquote}
        
\textit{Nhớ liễu hồ }        
\textit{tới nhờ em xõa tóc}        
\textit{Em vắng nhà }        
\textit{bồ kết chửa đi xa}        
\textit{Cầu nước chảy }        
\textit{bóng chiều xuân tha thướt} 
\textit{Xanh Thanh minh em thổi liễu vô hình} 
        
\textit{Lúa con gái }        
\textit{làm rùng rình nổi gió} 
\textit{Lá hát tình} 
<span style="padding-left: 30px">\textit{       		nắng tỏ }</span> 
<span style="padding-left: 60px">\textit{        			bạch đàn chanh}</span>        
\textit{Nợ cũ khối xương rồng hoa trả đỏ        
Hương thắp gọi ba lần}        
\textit{không đáp lửa}        
\textit{Hồn có nhà} 
\textit{hay bát mộ đi xanh} 

\end{blockquote}
 
Có thể gọi Lê Đạt - người phu chữ; lại có thể ví ông với Thái Công vọng \footnote{
Dĩ nhiên không phải từ bài thơ này.} . Thái Công vọng còn gọi Khương Tử Nha - sống ở giai đoạn lịch sử cuối nhà Thương\footnote{\url{http://www.talawas.org/talaDB/http://vi.wikipedia.org/wiki/Nh%C3%A0_Th%C6%B0%C6%A1ng}}, đầu nhà Chu ở Trung Quốc. Tử Nha sinh ra chỉ đề phò Cơ Xương chống lại Trụ vương. Còn Lê Đạt không phò cá nhân nào hết. Ông chỉ biết tới câu chữ. Ông bị giời đày xuống trần vào đúng thời kì thơ tiền chiến lùi lại nhường chiếu thơ cho “thế hệ sau” \footnote{
Lê Đạt, Trần Dần từng bảo nhau: đào mồ chôn thơ tiền chiến.} . Ông hăng hái lãnh vác sứ mạng phò câu phò chữ khi chúng đang bơ vơ. Lê Đạt ngồi câu trên dòng sông chữ ngần ấy trời năm bằng cây bút thẳng đứng của mình. Người ta thấy Lê Đạt và một số khác khi xuất hiện thì chẳng khác gì một điềm quái trong văn học. 
 
Theo Lê Đạt có hai phẩm chất đáng quí, hai kinh nghiệm hay: đọc sách và tích lũy chất liệu thơ hàng ngày. Nhà thơ phải luôn luôn có ý thức về nghề nghiệp. Chừng nào người viết thơ thấu sự khó khăn của nghề chữ, biết luyện chữ, dùng chữ một cách tinh thông, thì chừng đó mới được câu chữ bầu cử làm nhà thơ đích thực. Mọi nhà thơ tồn tại cùng đời sống con chữ của họ. Trường hợp nào tự phong hoặc được sắc phong, bổ nhiệm làm nhà thơ đều vô nghĩa và chóng về hưu nhất. Có lẽ, vì sớm thức nhận và thấm thía sự đày ải của nghề chữ, nên Lê Đạt mới thông cảm được “với hoàn cảnh khó khăn của các nhà thơ trẻ” và “niềm nở ân cần đối với họ”. Ông viết “Phải thương yêu, giúp đỡ họ (tôi rất ghét từ nâng đỡ) chỉ rõ những ưu điểm, khuyết điểm cụ thể với một thái độ khoan dung... thế hệ sau... phải tìm ra được tiếng nói riêng... sống động và khác lạ...” Muốn vậy “phải lao động hết sức gian nan... và đòi hỏi có một đam mê mãnh liệt đến mức dũng cảm”. 
 
“Người sáng tác tìm được cái độc đáo rất khó”. Chừng nào người sáng tác nói được tiếng nói của mình thì chừng đó anh ta mới trưởng thành, và được “ra ở riêng”. Lê Đạt rút ra kinh nghiệm: Nhà thơ muốn sắm được chiếc áo mới mẻ, hiện đại để sang nước ngoài ngó nghiêng một lát thôi, thì cũng phải tự làm ra chữ nghĩa, không nên cứ “ăn nhờ ở đậu” bố mẹ mình \footnote{
Xem: “Nhà thơ Lê Đạt: giải thường này chỉ là một cử chỉ đẹp.”} . 
 
Tôi - kẻ hậu sinh chợt thấy trong nhiều cuộc trò chuyện \footnote{
Đọc một số bài phóng viên phỏng vấn Lê Đạt.}  Lê Đạt rất khoái nói về sự can đảm - can đảm biến khát vọng trở thành hiện thực, can đảm phá bỏ định kiến đứng trên vai thần tượng, can đảm phê phán, can đảm định giá truyền thống bằng đôi mắt của chính mình và can đảm vượt mình để không phải “ăn mày dĩ vãng”... Người ta nói: Lê Đạt có cái nhìn trẻ một cách lạ lùng. Thử đọc lại bài thơ sau: 
\begin{blockquote}
				 
\textit{Bầy em én} 
<span style="padding-left: 30px">\textit{                 tin xuân}</span>        
<span style="padding-left: 90px">\textit{                               tròn mẩy áo}</span> 
\textit{Hội kênh đầy}        
<span style="padding-left: 60px">\textit{                      chân trắng ngấn sông quê}</span> 
\textit{nắng mười tám}        
<span style="padding-left: 60px">\textit{                      nắng bờ đê con gái}</span> 
\textit{cây ải cây ai} 
<span style="padding-left: 90px">\textit{                        gió sải}</span> 
<span style="padding-left: 120px">\textit{                                   tóc buông thề}</span> 

\end{blockquote}
 
Một lần trả lời phỏng vấn báo \textit{Sinh viên Việt Nam} (2004), nhà thơ Lê Đạt cho biết: “Cái đẹp trong câu thơ kêu gọi sự cao thượng”, thơ ca thể hiện “cảm xúc mĩ học”. Do quan niệm cái đẹp có khả năng thanh lọc tâm hồn con người, làm cho con người sống đẹp đẽ hơn, nên Lê Đạt hăng hái đi tìm cái đẹp ấy. Xét trên phương diện này, có thể nói thơ Lê Đạt giàu tính nhân văn. 
 
Đọc Lê Đạt, nghe Lê Đạt nói, thoạt tiên ta thấy ông khá bi quan. Nhưng xét kĩ thì không phải vậy. Chẳng hạn, khi nhà thơ Nguyễn Đức Tùng \footnote{
Xem:}  hỏi thông điệp của ông “Gửi nhà thơ trẻ”, Lê Đạt nói: “Mỗi thế hệ đều có bi kịch riêng của mình. Vấn đề là anh có nhận ra cái bi kịch ấy không và có sống với nó hay không”. Lê Đạt bàn nhiều về năng lực và bản lĩnh của nhà thơ. Ông vừa khắt khe vừa cởi mở đối với nhà thơ trẻ. Khắt khe, vì ông trọng nghề và đam “mê chữ”. Cởi mở, bởi ông hiểu bất kì nền văn học lành mạnh, khoẻ khoắn nào cũng có nhu cầu cách tân, cũng cố gắng thoát khỏi sự khô khan cứng nhắc của  tình trạng chính trị hoá văn học. Lê Đạt và những người bạn cùng chí hướng với ông đều nhìn nhận tiến bộ xã hội có trước sẽ tạo ra một nền tảng tư tưởng thuận lợi cần thiết cho tiến bộ văn chương. 
 
Ông cũng đề cao sự đoàn kết, sẻ chia giữa các văn nghệ sĩ với nhau: “Văn nghệ sĩ ở chế độ nào cũng bị bạc đãi, thế thì chúng ta phải thương nhau. Từ kinh nghiệm cá nhân của tôi mà ra, tôi thấy văn nghệ sĩ mà không biết thương nhau chỉ chờ chính trị thương mình là một ảo tưởng quá lớn”. Trăn trở với chuyện đối xử tệ bạc ở lĩnh vực văn chương, Lê Đạt đề nghị thi nhân văn sĩ người Việt ở mọi nơi  phải hoà thuận với nhau, ai cũng phải tự giác ngộ và biết cách xoá bỏ khái niệm ranh giới trong tâm thức mình. 
 
Wolfgang Kubin, một Giáo sư Hán học người Đức từng tổng kết: “Nhà văn Trung Quốc trước năm 49 đều giỏi ngoại ngữ. Trương Ái Linh, Lâm Ngữ Đường, Hồ Thích... đều có thể viết văn bằng ngoại ngữ. Một số nhà văn giỏi nhiều ngoại ngữ, ví như Lỗ Tấn. Sau 1949, về cơ bản, bạn không thể tìm được một nhà văn Trung Quốc nào biết ngoại ngữ cả. Bởi vậy anh ta không thể nhìn lại tác phẩm của mình từ một hệ thống ngôn ngữ khác. Ngoài ra, anh ta cũng không thể đọc được các tác phẩm ngoại văn. Anh ta chỉ có thể tiếp cận với những tác phẩm nước ngoài đã được dịch sang tiếng Trung. Bởi vậy, sự tìm hiểu và hiểu biết của nhà văn Trung Quốc về văn học nước ngoài cực kỳ kém, kém vô cùng. Trước năm 1949, không ít nhà văn cho rằng, học ngoại ngữ để làm phong phú ngòi bút của mình. Nhưng nếu bạn hỏi một nhà văn Trung Quốc (hiện nay) vì sao không học ngoại ngữ, anh ta sẽ nói, ngoại ngữ chỉ có thể phá hoại tiếng mẹ đẻ của tôi. Tại sao sau năm 1949, Trung Quốc lại không có một nhà văn vĩ đại nào, tại sao những nhà văn này chắc chắn không thể sánh được với các nhà văn trước năm 1949?” 
 
Lê Đạt cũng sớm chủ trương nhà thơ “nên học thêm một, hai ngoại ngữ để có thể đọc được nguyên bản...” Ông khoe: “Thế hệ của tôi thường biết tiếng Pháp. Cũng có nhiều người đọc được cả tiếng Nga hay tiếng Anh, có người cũng đọc được chữ Hán. Ảnh hưởng từ văn học nước ngoài rất lớn, vì mình biết được người ta đã làm gì và đang làm gì. Học nhiều thứ. Học ở các bậc thầy cả về kĩ thuật làm thơ”. Vấn đề quan trọng của việc học ngoại ngữ, hiểu theo cách của Lê Đạt là: cái nhìn của nhà thơ sẽ không bị phong bế, con chữ của nhà văn sẽ có thêm một sức mạnh để không phải bị cầm tù hay bị quật ngã. 
 
Vậy đấy, vấn đề giao lưu tiếp biến văn hoá, học hỏi kĩ thuật sáng tác từ bên ngoài để hội nhập văn chương từ lâu các nhà văn làm rồi, đâu phải đợi đến ngày nay. 
 
© 2008 talawas



\end{multicols}
\end{document}