\documentclass[../main.tex]{subfiles}

\begin{document}

\chapter{Tiếng nói một người}

\begin{metadata}

\begin{flushright}7.5.2008\end{flushright}

Thanh Tâm Tuyền

Nguồn: Tạp chí Sáng Tạo, bộ mới, số 6, ra tháng 12-1960 và 1-1961, trích từ trang 95 đến 98. Chủ nhiệm: Mai Thảo. Quản lý: Đặng Lê Kim. Trình bày: Duy Thanh. Toà soạn và trị sự: 133B Ký Con, Sài Gòn. Giá: 15đ. Bản điện tử do talawas thực hiện.

\end{metadata}

\begin{multicols}{2}

\textbf{1. } 
 
Sự thật không bao giờ là sự thật khách quan. Nhớ lấy những kinh nghiệm thường ngày: cùng một điều việc kẻ này nói, làm được, kẻ khác là thấy chói tai chướng mắt. Chỉ có sự giả trá hư ngụy mới đội lốt khách quan, sự thật xuất hiện từ con người và xác minh làm kẻ khác chấp nhận là của người đã dám phát hiện ra nó. Và trong sự mong manh yếu hèn của kiếp người, nhiều kẻ đã phải mang tự do tối hậu duy nhất của một đời là cái chết để bảo đảm cho sự thật của đời mình được thành sự thật với những người khác. 
 
Thơ là tiếng thổn thức của con tim, đó là một sự thật tôi tìm thấy lại qua Trần Lê Nguyễn. Suốt tập thơ, Nguyễn chỉ nói về mình, nói rất nhiều về mình, nói quá nhiều về mình bằng thứ ngôn ngữ xô bồ, đôi lúc rối loạn, lảm nhảm, buồn cười. Và con tim Nguyễn phơi bày nguyên hình dáng, một khối thịt bầy nhầy bóp vào nở ra bất tận. Đó là hình ảnh của đời sống. 
 
Tiếng nói một người hay tiếng nói một đời, một kiếp? 
 
 
\textbf{2.} 
 
Trước hết thơ là một nỗ lực tinh khiết hoá thực tại. Các nhà thơ cổ điển đã làm công việc này. Bọn lãng mạn lầm tinh khiết hoá thực tại với ruồng bỏ thực tại, dùng nước mắt, tiếng rên la, sự xúc cảm nhầy nhụa làm vẩn đục thực tại, rồi trốn chạy vào ảo tưởng. Thơ ngày nay cũng là một nỗ lực tinh khiết hoá thực tại từ khởi điểm làm hiện hình nó, cái hình dáng thô sơ đã bị bọn lãng mạn mài nhẵn bằng nước mắt nước mũi. 
 
Thơ Nguyễn là cái ánh sáng lộ liễu khô khan chiếu vào thực tại, sự vật nổi lên còn đủ những góc cạnh sần sùi. 
\begin{blockquote}
 
\textit{Mỗi lần tôi mượn tiền bạn bè là mấy thằng chó chết chưa hề có con cười hô hố bảo là tôi lại sắp phịa chuyện đến nhà thương thăm con mới đẻ.} 

\end{blockquote}
 
Nếu anh hiểu được rằng người ta vẫn có thể mộng tỉnh thức chẳng cần phải tìm tới giấc ngủ hôn mê, anh sẽ nhận ra Nguyễn đang làm thơ. 
\begin{blockquote}
 
\textit{Đêm cưới em, anh sẽ không ghé câu lạc bộ mà vào Snack Bar uống rượu thật say (dĩ nhiên bằng tiền đánh bạc chứ không phải tiền viết văn) rồi không ghé đăng-xinh (dù biết rằng sắp bị đóng cửa) mà đi ngược về đường Duy Tân (một nhà vua cách mạng) hay dọc theo đại lộ Hai Bà Trưng (hai nữ anh hùng dân tộc) tìm gặp một “me” lính Pháp ra đi còn để lại.} 
        
\textit{Để suy ngẫm về cõi đời} 
\textit{và mừng em lấy chồng Mỹ} 

\end{blockquote}
 
Cái thế giới của Nguyễn là thực tế hằng ngày chúng ta đang sống, quay cuồng, hỗn độn, đầy khát vọng. Tại sao cứ đòi hư vô để mơ mộng? Hãy thử mơ mộng như Nguyễn xem sẽ thấy sự kỳ lạ của thế giới ấy. 
 
Và thơ là gì? Nếu không phải là sự khám phá mầu nhiệm bằng ngôn ngữ một thế giới vẫn trốn mặt ở quanh. Sự phơi mở ở thơ cho anh cảm giác tràn đầy hạnh phúc, tâm hồn đã nhập được một phần của sự sống bí ẩn còn thiếu sót. Những phút xâu dài như một đời. 
 
 
\textbf{3.} 
 
Tiếng nói một người là tiếng nói của tình yêu, tình bằng hữu. Trong cô đơn và đêm tối.        
        
 
\textbf{4.} 
 
Người sắp nói là một người bốn mươi tuổi. Người ta thường làm thơ vào những năm hai mươi. Vào tuổi ấy Nguyễn chỉ còn muốn viết tiểu thuyết, viết kịch. Hắn chỉ thấy cần làm thơ trong vài năm gần đây. 
 
Việc làm thơ của Nguyễn chứng nhận lời tiên tri của Lautréamont: Thơ không phải để một người làm mà để mọi người làm. Thơ là sự giải phóng, sự tự do, là quyền của mọi người, không bao giờ là đặc quyền của một bọn thi sĩ đầu bù tóc rối trí tưởng tượng như con gián bay quanh đèn, sự cảm xúc như tiếng động của thùng thiếc. 
 
 
\textbf{5.} 
 
Thơ Nguyễn kể lể ồn ào nhưng vẫn nghe đâu sự nín lặng trong cùng. 
 
Hạt nhân nín lặng, khép kín làm mỗi bài thơ tự đầy đủ, phân biệt thơ Nguyễn với thơ Prévert. Hai bên chỉ giống nhau ở điểm tưởng tượng, mơ mộng cùng thực tại. Còn Prévert đòi hỏi những đối tượng ở ngoài để phóng tới. 
 
 
\textbf{6.} 
 
Một người sống đến bốn mươi tuổi không làm thơ để mong thành thi sĩ. Thi sĩ! Thi sĩ! Thằng người đó đã tự sát. Tên của chàng bị cướp bị bôi nhọ. Ngày nay còn toàn một bọn nhái giọng người chết. Mấy tên thư lại luồn cúi nịnh hót cũng là thi sĩ. Mấy tên cán bộ làm thơ như những bản thỉnh nguyện xin tha mạng sống, xin thêm quyền lợi. Mấy tên thanh niên hiến thân làm tấm gối ôm trong khuê phòng. 
 
Nếu anh đọc thơ Nguyễn, anh nghĩ Nguyễn không phải là thi sĩ, anh nghĩ đúng. Nhưng coi chừng, anh đã bị đầu độc bởi bọn giả danh. Tôi nhắc lại: Thi sĩ đã tự sát. Và anh cũng như Nguyễn cũng như tôi được thừa hưởng cái gia sản của chàng cùng với mọi người, trừ bọn tự nhận là con cháu chàng. Chúng ta phải cướp lại tiếng nói sắp muốn tắt; mỗi người đều được quyền làm thơ như Nguyễn, như làm một hành động giải phóng. Đừng để bọn người nào độc chiếm thơ làm phương tiện áp bức. 
 
Một ngày thi sĩ sẽ hồi sinh. Chưa phải bây giờ. Nhưng hãy thổi những hơi thở mới vào mũi chàng, đuổi bớt những uế khí, ám khí, tử khí, đang ướp quanh chàng.  
 
 
\textbf{7.} 
 
Tiếng nói của Nguyễn chỉ là tiếng nói một người. Một người hèn mọn như loài run dế. 
 
Nhưng Nguyễn, như anh, biết rằng một người không có nghĩa là một. Nói một người là nói tới số đông. Mai kia hắn chết đi, cái chết bất cứ trường hợp nào cũng chỉ là sự lịm tắt của một khát vọng, là chết theo người yêu một đời của hắn, những bạn bè gần gũi, những mộng ước đau đớn, tuyệt vọng, nghĩa là một phần thế giới. 
 
Đọc mà xem, anh sẽ thấy hắn phải nói trong cô đơn để được gần anh. 
 
 
\textbf{8.} 
 
Mỗi bài thơ của Nguyễn là một nỗi đầy cô đơn. Nếu tôi nói hắn sống rất vui trong cô đơn, anh sẽ cười tôi. Bởi anh đã khổ vì cô đơn. 
 
Thực ra tôi phải nói là hắn bằng lòng trong cô đơn, vì nơi đó hắn được sống với anh, trong cái thế giới lạnh lẽo đáng sợ, hắn được chia sẻ với nhiều người, những người không được gặp nhau. Như hắn và người yêu của hắn:        
\begin{blockquote}
        
\textit{Anh đã yêu cùng cực}        
\textit{đến không còn em }        
\textit{............................................}        
\textit{Anh ôm thật chặt khoảng trống căn nhà hoang}        
\textit{như thấy em cả đời trọn vẹn } 
\textit{nửa đêm nào thức giấc} 

\end{blockquote}
 
Như hắn và một người bạn nào:        
\begin{blockquote}
        
\textit{Và những đêm nhìn trăng sao}        
\textit{dưới mái hiên nhà dây thép tôi nói với anh }        
\textit{về trời đất về ước vọng hai đứa mình ở đời.}        

\end{blockquote}
 
\textbf{9.} 
 
Đây là tiếng nói của một người nối kết những cô đơn.        
        
 
\textbf{10.}        
\begin{blockquote}
        
\textit{Nửa đêm những người yêu nhau nhảy slow }        
\textit{Kẻ hút “píp” đi một mình bờ đại lộ}        
\textit{vì không ai yêu mình}        
\textit{hay mình không yêu ai} 
\textit{ngậm tẩu như hôn người đàn bà một đời } 

\end{blockquote}
 
Không phải, Nguyễn yêu nhiều người quá và nghĩ nhiều người đang yêu hắn. 
 
 
\textbf{11.} 
 
\textbf{… }hắn thầm thì, hổn hển, khó nhọc soi sáng những sự thật thầm kín ấy, những sự thật người ta chỉ có thể diễn tả nổi với cố gắng vô biên – những sự thật hết sức tối tăm, hết sức gian nan – nhưng chính với những sự thật ấy, thế giới phải thay đổi toàn diện, một lần cho xong. (Virginia Woolf)        
 
 
 
\textbf{12.} 
 
Một sự thật nữa, hắn đã sống. 
 
Rất có thể ở ngoài đời hắn đã sống giả như chúng ta vậy. Lý do giản dị: cái giả hiện ra trong sự vận động mải miết của sự vật vượt qua mình, hôm nay từ chối hôm qua, và mình vẫn thích ứng tồn tại như không hề có gì xảy ra, tấn thảm kịch biến thành hài kịch. Chúng ta nhìn nhận mọi cảnh ngộ bằng cặp mắt chấp nhận tự nhiên, nỗi bất bình chìm sâu ở đáy thân bị nghiến nát không còn. Khi Nguyễn làm thơ (và nếu anh cũng làm thơ) Nguyễn phát hiện sự sống thực. 
 
Và khi đã sống thực người ta sẽ không hài lòng một chút nào trước cảnh ngộ.        
        
 
\textbf{13.} 
 
Phương,  
 
Tập thơ này Nguyễn viết cho em. Em là người yêu của hắn. 
 
Bài mở này anh viết hộ em. Anh viết những điều em biết về hắn nhưng không thể diễn được thành lời. Và em sợ nên em xa hắn. 
 
Thôi cũng xong. Để cho hắn làm thơ. 
 
\textit{13 và 14 tháng 8 năm 1962} 
Thanh Tâm Tuyền        
        
 
 
\textbf{Phụ lục 1} 
\textbf{Thơ Trần Lê Nguyễn } 
(Trích trong \textit{Tiếng nói một người})        
 
\textbf{Phương} 
 
Anh yêu em không ngủ đêm nay 
Từ có em 
người đàn bà một đêm trở nên vô nghĩa 
Hành động của yêu 
qua không gian tiếp nối thời gian 
như hơi thở có ngưng không bao giờ dứt 
phút sống ngập đầy 
ý nghĩa lứa đôi tìm thấy 
khi em không còn trong tay 
Anh đã yêu cùng cực 
đến không còn em 
sống với màu xanh quá khứ 
sương sớm nắng chiều 
bông hoa nở giữa hai ngành héo buổi em đi 
hơi thở nghẹn ngào 
mi khép ứa dòng nước mắt 
Anh ôm thật chặt khoảng trống căn nhà hoang 
như thấy em cả đời trọn vẹn 
nửa đêm nào thức giấc 
 
Anh chả bao giờ có em 
chỉ có bốn mắt nhìn nhau một chiều súng đạn 
Em có nhớ ra anh 
Anh có nhớ ra em 
Đường nắng không một bóng dừa 
một ngã ba hoang vắng 
Anh chả bao giờ có em 
chỉ có một đêm chớm lạnh 
mưa trên sông 
nghe tiếng thở dài của kẻ chung đôi 
cùng tiếng thở dài của người cô độc 
Anh chả có em nụ cười 
chỉ có em nước mắt 
Anh muốn giết em để đừng thấy lệ em rơi 
để đừng bao giờ em bỏ đi 
đừng bao giờ em tìm tới 
 
Tiếng em kêu thất thanh đêm nào 
anh nghe chính lời anh hấp hối 
Tại sao anh yêu em 
Tại sao em yêu anh 
Tình yêu rất hiếm một con đường xanh 
rất nhiều những con đường lội 
Em có nhớ những chiều ngoại ô 
buổi sáng ngồi xe thổ mộ 
tiệm nước bên dây quan tài 
hàng cây trong nghĩa địa  
Anh nhớ em gục vào vai anh ướt 
nước mắt mặn của môi 
chua chua miếng thơm em đem qua nhà giữa trưa rất nắng 
Có một hôm anh khen em đẹp 
người ta sung sướng nép đầu vào ngực tôi 
Em ơi em ơi em ơi em ơi 
anh yêu em không thể nào ngủ được. 
\textit{1957} 
 
 
\textbf{Ám ảnh} 
 
Tôi làm bài thơ giản dị 
đến không còn thơ 
để gửi anh nhớ ngày gặp gỡ        
 
Trận đói bốn mươi lăm 
đồng quê hết gạo chạy về thành phố 
Từng đoàn rũ trên đường 
chết không kịp chôn 
đổ chung một hố 
Đàn quạ đen tím cả hoàng hôn 
Chị cướp cơm em 
Mẹ bịt mũi con cho hết bú 
Những chiếc xe bò người kéo đầy thây 
bao chiều cửa ô lớp lớp đợi đây 
Tôi làm bài thơ bầm vết máu 
những người Nhật trói đầy nắng tháng năm 
Tôi đi cùng anh 
buổi chiều vĩ đại  
Hà Nội băm sáu phố phường 
tung năm cửa ô 
Cách mạng mùa thu Tháng Tám 
Đêm kịch nhà hát lớn 
tôi đọc thơ cho anh bẻ song tù 
sáng mai về quê thấy không còn bố 
- Thầy tôi ai bắt. 
- Cách mạng cần có trại giam 
Giọng anh lạnh 
Mặt anh đanh 
Tôi lặng người cay đắng 
Nước sông Hồng thôi đỏ phù sa 
nước sông Hồng màu đỏ chiến khu Phú Thọ 
anh không dùng đạn 
lưỡi lê 
trôi sông 
Máu anh Việt Quốc hòa nước sông Hồng  
Tôi thôi làm kịch 
son hậu trường như máu tanh tanh 
tôi đi giết giặc để quên thấy anh 
 
Tôi vào Quảng Ngãi nghe tiếng mõ khuya 
thấy rợn hồn trẻ thơ chết chưa kịp đẻ 
Diệt mầm phản động mai sau 
mã tấu 
anh chém cả con lẫn mẹ 
Sóng gào bãi biển Tam Quang 
dừa xanh hoang tàn thánh thất 
một xóm Cao Đài cùng khóc 
cha 
chồng 
anh 
con 
chết một ngày 
một giờ 
anh chôn sống chung một hố 
 
Ba lô tháng năm kháng chiến 
tôi đi trên những nẻo cùng 
lạc loài làm tên phi đảng 
mấy lần biên giới qua sông 
Tôi vô tới mũi Cà Mau 
nghe các anh dân chủ 
thời Nguyễn Bình 
xác nhận lòng kinh 
nghe các anh Đệ tứ 
xương bóng rừng cao su mông mênh 
 
Tôi muốn viết cho anh 
bài thơ không bằng chữ máu 
tôi muốn viết cho anh  
bằng những tâm tình 
tôi muốn viết cho anh 
như thuở ban đầu chiến đấu 
 
Tôi tìm lại về sân khấu 
có đêm khi bức màn buông 
tiếng gõ ba hồi xuống ván 
nghe như vồ đập áo quan. 
\textit{1957} 
 
        
\textbf{Khi yêu em}        
\begin{blockquote}
Của O.T.\end{blockquote}
 
 
Tôi bán quê hương lúc con một nửa 
Người đàn bà Việt Nam đuổi tôi khỏi tròng mắt đen 
của đôi mắt bồ câu 
đôi mắt dao cau 
Đục màu hạt dẻ 
hay xanh chân trời xa 
là đôi mắt xứ người chân tình nên phản bội  
tôi gặp Berlin 
Paris ở thủ đô tôi 
thủ đô thiếu một hồ gươm lịch sử  
một thư viện trên đường Trường Thi 
Như em không quê hương  
lấy kinh thành người làm kinh thành mình 
ra đời ở Trung Âu 
cư ngụ bên hồ Leman trời Thụy Sĩ 
 
Tôi đọc người “Nga-La-Tư” 
thấy nàng thiếu nữ 
khi chết cho chiến thắng 
hôn anh chiến hữu gửi lại người yêu  
Và người nữ chiến sĩ “Thông Hành Giả” 
không nhận thư tình 
bàn tay run run mở tung nút áo 
Vết sẹo hành hình in ngực  
- bàn tay kéo cao cổ áo đi mưa anh đồng chí một đêm sương mù 
Thần tượng ấy giết người yêu bé nhỏ 
đau bệnh lao quê tôi ở miền Bắc  
Lá thư cuối cùng một hồn người 
tôi không được đọc chỉ nghe kể lại 
ngày cách mạng thành công  
 
Cách mạng đưa con người tới đâu 
chỉ biết anh gặp em không hề chờ đợi 
Anh nghe nàng tiên hát bài hát tiếng Nga 
có tuyết có gác chuông 
buồn chia cách khi vừa gặp gỡ  
Anh yêu gió tung mái tóc 
để tay anh lùa trong tóc em 
Anh thù bóng đêm che sâu màu mắt 
chỉ thấy hư ảo một màu nhớ nhung 
Hư ảo như tay em trong tay anh 
như tình chúng ta  
của thời đại không còn được yêu được chết 
của thời đại chỉ gặp nhau xa nhau 
Nên em chỉ cho anh một nửa bàn tay  
Nên anh không dám hôn môi em 
như sợ truyền nhiễm bệnh lao 
dù anh chưa hề hôn một lần  
người yêu chờ chết bệnh lao miền Bắc 
Anh cảm ơn sự tàn bạo cho chúng ta gặp nhau 
Anh cảm ơn em cho anh sống lại tuổi hai mươi 
có một lần trong đời 
anh quên sống chạy theo cách mạng 
đi bắt bóng những dáng hình 
bỏ quên người yêu chết mười sáu tuổi 
không một cành hoa trắng cắm lên mồ  
Những trận mưa bom đồng minh trên quê hương anh 
Những tờ truyền đơn rơi trên lòng đồng bào anh câm nín 
 
Em cho anh sống những gì anh chưa sống 
không thể thiếu trong một đời người  
Anh không muốn nghĩ đấy là lý tưởng em đang đi tìm 
dù có một ngày nào nở hoa 
Anh sống với hình ảnh em 
Nhìn nghiêng lạnh và buồn như tượng 
Với giọng em hát nhỏ khi không đành nói lên lời 
Với căn phòng lữ quán cô đơn 
Với chiếc ban-công từng lầu ba dưới bóng me cổ thụ phố Sài Gòn 
giống như dưới bóng cây hạt dẻ thành Prague 
chiếc ban-công em gục xuống tay một mặt phút không đành khóc 
Và tiếng cửa cầu thang máy đóng lại gần sáng một đêm nào 
 
Em buồn ra riết trước ngày em đi 
và bảo anh nói quá một lần sự thật 
Em trách anh đến với em quá mau 
Anh biết làm sao 
khi không thể níu thời gian ngừng lại 
Vì anh gặp em như tự bao giờ 
qua năm tháng dài đấu tranh  
người nữ cán bộ cô đơn cười vui chiến đấu 
Vì em đến và em đi 
Anh muốn Việt Nam đón em với tất cả ân tình 
Người ta không sống cùng tài liệu mà bằng kỷ niệm 
 
Em đi chấm cuối hàng người trên sân bay 
Bao-lơn phi cảng một mình anh đứng lặng 
Hai đứa cùng giơ tay – 
Bàn tay giơ lần thứ nhất – lần đầu tiên – ở cửa lữ quán Sài Gòn 
Bàn tay dơ lần thứ hai – lần cuối cùng – ở trường bay Tân Sơn Nhất 
 
Áo em hồng đẹp nhất 
Mắt em buồn đẹp nhất những người ra đi 
Con chim hiếm bay 
Buổi sáng không vui như chiều đã đến 
Em hiện ở góc trời nào 
không một chữ 
không một tấm hình bưu thiếp 
Tôi không muốn nghĩ 
em đi tìm lãnh tụ đã dẵm lên hoa cỏ bên đường 
Tôi muốn nghĩ  
em sẽ là người đàn bà hiền hậu 
có chồng có con 
một đứa con gái hai đứa con trai 
như em hằng mơ ước 
Tôi đào ngũ khỏi lòng dân tộc một ngày nào đây 
Biết rồi sẽ nhớ đôi mắt bồ câu 
đôi mắt dao cau 
Tôi gói hình ảnh người đàn bà Việt Nam làm hành lý lên đường 
Tôi sẽ không bao giờ tìm em 
như bây giờ không viết một dòng thư 
chỉ làm bài thơ giấy giáp 
Hai đứa yêu nhau rất nhiều để không quên nhau 
để thôi nhớ làm lịch sử  
 
Có con bướm trắng vừa đậu trên tóc em 
Có bông hoa trắng mới nở trem mồ người yêu chết mười sáu tuổi 
 
Có cách mạng nào thành công 
Có tình yêu nào tan vỡ 
\textit{7-1959} 
 
 
\textbf{Sài Gòn mưa} 
 
Đế giày tôi lủng hai bên 
Những chiều Sài Gòn mùa mưa như chiều nay 
tôi đi bằng gót qua nhiều lề đường đọng nước 
Đĩa nhạc quay tròn quay tròn 
âm thanh nổi 
Tờ báo buổi chiều loan tin chiến sự xứ Lào 
 
Mưa lại rơi 
như tháng bảy mưa rơi ngoài Bắc 
Hà Nội không còn 
Sáng qua tôi gặp cô gái Hàng Ngang 
Ngoài kia đâu còn Hà Nội 
Có những người chết đi 
mất xác bên cầu Kiệu  
con đường hành quân thuở trước vắt qua Dốc Mỏ 
gặp mộ người nữ cứu thương 
“Thái-mortier” Tây bắn ở Tuy Hòa chết không hay đâu còn Hà Nội 
 
Tôi trú mưa đầu phố 
Giày tôi vào nước từ lâu 
Gió tạt quán rượu góc đường Charner thuở trước 
Ly rượu anh thủy thủ trên đất liền 
Người đàn bà Pháp chờ Taxi cô độc 
Không là đây Paris 
Sao tin chiến sự Vientiane làm nhớ thương Hà Nội 
Sao Hà Nội nhắc những người chết đi 
Tôi đi nhận lá thư không đến chiều nay 
sao lại gặp Sài Gòn mưa như tháng bảy trời mưa xứ Bắc 
 
 
\textbf{Đất nước tôi tình duyên tôi}        
 		\begin{blockquote}
\textit{Của V.L.}\end{blockquote}
        
 
Đường Tự Do mọc lên nhiều Snack-Bar 
Một chiếc chen thêm vào bên chỗ trú chân của những nhà văn hoá văn nghệ xứ mình        
 
Giữa hiện trạng ấy anh nghe tin em lấy chồng 
Một đồng đô-la giá chợ đen ăn chín mươi đồng Việt Nam  
Anh ước mong em lấy chồng Mỹ vì tình 
một chuyện tình như phim Mỹ chúng ta thường xem 
một chiến sĩ (cấp tá) bỏ vợ vì cuộc chiến tranh Cao Ly 
gặp người con gái Hàng Đào di cư mang tật nguyền máy bay Tây hồi giặc 
em bơ vơ sau bao cuộc tình duyên 
như người đàn ông ngoại quốc cô độc xứ người sau một đoạn đời 
một tình duyên rất cha con 
một hôn nhân rất anh em 
 
Anh vui và buồn cùng em như của chính anh 
Khi bỏ nghề viết kịch (đánh máy hai mươi trang rưỡi diễn đúng bốn mươi lăm phút, hai tháng sau được 300đ bản quyền tác giả) để đi đánh bạc quên đời 
Anh đã vừa viết, vừa đóng vừa đạo diễn cuốn phim \textit{Rizamer} 
Kịch tác gia giải thưởng văn chương toàn quốc của em như vậy đào đâu ra tiền cưới vợ nuôi con 
chưa kể về mặt tinh thần  
có xứng đáng với em một người con gái muốn thấy đời cao đẹp 
Cho nên anh mừng hay tin em lấy chồng 
lấy chồng trống trơn 
vì anh nghĩ đây là một chuyện khác biệt 
hoàn toàn tự do 
hoàn toàn bình đẳng 
không phải vấn đề đồng đô-la U.S. bên cạnh đồng bạc V.N.  
em lấy chồng có cưới xin theo lễ nghi xứ mình 
xóa được vết đen môi “kỹ nghệ” thời xưa 
viết được một chuyện tình quốc tế 
 
Đêm cưới em anh sẽ không ghé câu lạc bộ mà vào Snack-Bar uống rượu thật say 
(dĩ nhiên bằng tiền đánh bạc, không phải tiền viết văn) 
rồi không ghé đăng-xinh (dù biết rằng sắp bị đóng cửa) mà đi ngược về đường Duy Tân (một nhà vua cách mạng) hay dọc theo đại lộ Hai Bà Trưng (hai nữ anh hùng dân tộc) tìm gặp một “me” lính Pháp ra đi còn để lại 
để suy ngẫm về cõi đời 
và mừng em lấy chồng Mỹ 
để anh còn được là đàn ông của nước Việt Nam 
nay có đàn bà lấy chồng khác nước.        
\textit{11-5-1959}        
 
        
\textbf{Đã đi còn đi}        
 	\begin{blockquote}
\textit{cho Cung}\end{blockquote}
        
 
Đã đi chân không thuở mười sáu vào đời 
trong lò than đá mỏ Vàng Danh 
ánh đèn đất ma chơi soi đường hầm địa ngục 
Tình anh thợ mỏ yêu chị Nhà Sàng 
thân thiết như goòng than 
từ một lò đang phá 
kéo ra Uông Bí ra “boo” Rơ Đông 
xuống những con tàu đại dương ăn than 
dọc theo mạn Đông Triều Hòn Gai Cẩm Phả 
Còn thấy hố mắt không hồn buổi tan tầm 
Mỗi lần sập lò bao xác chết đen thui 
chôn một đời min mỏ 
 
Tôi nhớ Vàng Danh linh hồn bé nhỏ 
nhớ thường bạn đeo mìn 
những đường “tơơi” cực nhọc 
tiếng còi tầm xé ruột 
ngọn roi song xua thợ đi làm 
đá cắt gan bàn chân lạnh buốt 
 
Hai mươi mấy năm rồi tuổi trẻ vui tin 
nguyên vẹn tờ truyền đơn đòi cơm áo 
Hai mươi mấy năm rồi 
tôi đã đi thêm đoạn đường kháng chiến  
với dép vỏ xe hơi Bình Trị Thiên 
tôi đã đi thêm đoạn đường di cư 
với đôi giày đế lủng 
Tôi thấm mệt chiều nay nhớ anh người bạn đeo mìn thuở nhỏ 
 
Anh còn sống hay đã chết 
được thấy đổi đời chưa 
đèn đất anh dùng có bao lưới thép phòng ghi-du khỏi nổ 
còn có nạn ngập lò 
Muốn gì đi nắm cơm anh ăn vẫn toàn than bụi 
vẫn toàn mồ hôi 
mồ hôi của thợ mỏ hay của anh hùng công nhân cũng vẫn chỉ là mồ hôi 
nhiều chất mặn. 
 
Như tôi vẫn còn phải đi 
gọi là đi tìm tự do 
tôi hiểu nghĩa hai tiếng ấy trong xà lim hẹp 
qua những ngày thiếu ăn 
những lần trốn tiền nhà chủ phố 
và thấm mệt vẫn còn phải đi 
để chạy những thiên đàng đóng hộp 
 
tôi nhớ ngày đầu đi mỏ 
lũ loong-toong Tây ức hiếp cướp mất tích-kê 
anh đưa tôi ra chợ ăn cơm cởi áo thay tiền trả 
và những đêm nhìn trăng sao 
dưới mái hiên nhà dây thép tôi nói với anh về trời đất 
về ước vọng hai đứa mình ở đời. 
Có bao giờ không nhỉ 
con anh và con tôi 
chúng sẽ gặp nhau tình cờ như chúng ta đã gặp nhau 
ở bến Sáu Kho 
ở trường Đại học Sài Gòn 
hay ở một quê hương nào khác nữa. 
\textit{14-7-1957} 
 
 
\textbf{Khuôn mặt} 
 
Không thấy nổi khuôn mặt người yêu 
buổi chiều bệnh cái chết kéo về khoảng trời xanh ấu thơ 
 
Em hư ảo trăng mùa hạ cũ 
em không còn em 
chiếc áo hở tay bầy đom đóm hoa dạ lai hương 
em còn không em 
đôi má hồng người bệnh mùa xuân 
em không còn em 
bông hoa không nở trên mồ con gái chết mười sáu tuổi 
 
Không thấy nổi khuôn mặt bạn bè 
buổi sáng gục bên đường đói khát 
mặt trời hết bình minh 
Chúng mày ở nơi đâu 
cuộc sống mang đầy ung nhọt 
một đứa gục xuống súng chửa rời tay 
một đứa bước lên miệng còn thơm sữa 
Bây giờ mùa xuân không còn 
từng đứa lui vào dĩ vãng 
chị ngã bên đường xác em nằm đây 
mày vội chết đi, mặc tao còn sống  
Khi mặt trời chỉ là nắng cháy 
kẻ bại trận đi một mình 
bàn tay ngửa xin một hơi nước lạnh 
 
Không thấy nổi khuôn mặt mình  
một lần sống sót 
những ngày không người yêu bạn bè 
khuôn mặt ban đêm tiếng kèn già nua thảm thiết 
Tìm lại quê hương đã mất 
chiều lập đông vườn cải hoa vàng 
bóng mẹ già phơi áo 
khuôn mặt vỡ tan từng mảnh vụn 
như bàn tay mở lựu đạn  
liệng ngay khuôn mặt mình 
\textit{1963} 
 
 
\textbf{Sám hối} 
 
Tôi chưa hề sống 24 tiếng đồng hồ với một người đàn bà 
chỉ một đêm 
nửa đêm 
khoảng khắc trời mưa tại tiệm cà-phê nhìn chiều hè phố 
hay dài một cuốn phim 
xem lại buổi trưa nắng gắt 
một đàn bà 
những đàn bà 
của một lần gặp gỡ 
của cả một đời 
Tôi không nhớ hết tên họ dù không hề quên một người 
đôi ba lần nghe ai nói muốn có con với tôi 
Những đứa con 
chắc có trai có gái 
có đứa sống đứa bỏ đi 
để không một đứa bên mình 
người về già hay thương máu mủ 
 
Tôi gặp con tôi một trưa về thăm đầu đường sống 
những nàng Sáu Nhỏ 
Những nàng tôi gọi là nữ chiến sĩ ân tình 
Tất nhiên má của con tôi cũng là nữ chiến sĩ 
Và dĩ nhiên không thể rõ ai là cha 
Má nó bảo con ra đời thiếu tháng 
cho đúng ngày anh bố mạch lô đi biển trở về 
Đến đây tôi muốn mở một cái ngoặc đơn 
nói về những đôi vợ chồng rất là thương yêu rất là hoà thuận 
nhưng đồng lương chồng không đủ sở hụi gia đình 
nên anh đi mần 
em cũng phải đi làm 
Tôi nói rất là thương yêu nhau mới chịu được một chuyện như rứa ở đời        
 
(Lạy Chúa đã để anh mạch lô tin là con ra đời thiếu tháng) 
vì dù rất là thương yêu  
gã thủy thủ không thể kham được việc người gái chơi lại có con với một kẻ khác ngoài anh 
nên thay cho sự hòa thuận 
phải là án mạng 
vì dù là ghé bất cứ bến bờ nào 
anh cũng chỉ có một quê hương 
và nếu là nghề nghiệp bắt buộc 
không yêu thì có con sao được) 
Nghe kể vậy tôi lặng thinh nhìn 
đứa nhỏ cười  
(tôi vốn hay cười với bất cứ trẻ thơ nào) 
muốn thấy một chút gì tôi ở nó 
 
Chuyện không đâu ấy làm tôi thức đêm nay 
dù không phải đây là lần đầu 
(mỗi lần tôi mượn tiền bạn bè là mấy thằng chó chưa hề có con cười hô hố bảo là tôi lại sắp phịa chuyện đến nhà thương thăm con mới đẻ) 
Không phải nghĩ tôi cảm thấy rõ ràng sự thật  
sự thật còn thực hơn hai với hai là bốn 
sáng đẹp hơn mọi thứ chủ nghĩa tôi đã đi tìm hay người ta sắp giới thiệu với tôi 
một sự thực đơn giản không rõ có biết nhưng ai cũng làm 
riêng tôi không hay quá nửa đời người tiêu toàn bạc giả 
 
Nhớ lại mối tình lớn  
(yêu người ấy từ thủa mười lăm 
yêu người ấy qua gia đình đổ vỡ) 
tôi nghĩ đến mấy trăm người ta nhờ bác sĩ xoá giùm kỷ niệm 
và tác phẩm văn chương toàn quốc từng đem vinh quang cho đời cầm bút 
nếu đặt bên miệng cười con tôi 
chắc không hơn chồng giấy lộn 
 
Lần đầu tiên tôi cầu nguyện thành khẩn ở đời 
Con tôi vì là con gái nên không thể giống tôi 
nhưng chính vì là con gái nên xin đừng bao giờ còn là má nó.        
\textit{21-5-1962}        
 
        
\textbf{Có em hay không có em}        
 
Anh giã biệt em và phi trường và trời xanh và mây mù và quê hương mình khuôn mặt rỗ nham nhở bom đạn. Anh nhìn xuống từ trời cao mũi súng nào ngước lên từ chiến khu lẩn khuất. Bữa ăn “Hầm Rượu” ấy mùa thu cũ châu bản nào xa xưa. Có phải châu Lương Sơn? Cách mạng màu đỏ máu cô bé nhìn cha ngơ ngác vết chém khôn rời. Thầy anh qua đời băng đạn tiểu liên ngọt xớt bàn tay lạnh nảy cò người lính Bắc Phi đen. Sơn Tây kế cận Hòa Bình sát bên Hà Đông như buồn đau em gần gũi tủi nhục anh niềm đau khổ chung đất nước. Tám mươi năm đã qua. Và còn bao nhiêu năm nữa? 
 
Bữa ấy trời xanh cao. Anh hẹn em nơi phòng triển lãm. Và anh rủ em đi mua búp bê khi đứa bạn nói có bày bán ở lề đường. Anh không mua được – dù là một thứ búp bê bày bán ở lề đường – vì con đẹp nhất không còn nữa. Bao giờ anh cũng chỉ là người đến chậm. Rồi hai đứa đi ăn và má em ửng đỏ vì rượu chát như màu cánh hồng trên tay. Ra về anh ngừng lại bên quán hoa muốn mua tặng em một bông hồng khi ngoảnh lại đã thấy em đi mất. Anh bước theo và không bao giờ bắt kịp. Rồi trời mưa. Không hiểu vì sao trời hay mưa vào rất nhiều ngày trong đời anh? Và từ đấy là giận hờn xa lánh. Tại sao? Tại sao anh không được coi em như một người em gái? Các em gái anh ở ngoài miền Bắc và ngoài ấy mưa bom như giờ đây tại sao anh không được phép coi em như một người em gái? 
 
Giáng sinh năm nay không có lễ nửa đêm và anh vẫn một mình đến trước cửa Vương Cung Thánh Đường vào mười hai giờ đúng. Có tiếng súng nào từ đâu vọng đến không em? 
\textit{27-12-1966} 
 
(Nguyệt san \textit{Vấn Đề}, số 1 tháng 4-1967, từ trang 66 đến 93. Chủ nhiệm/sáng lập: Vũ Công Trực. Chủ biên: Vũ Khắc Khoan. Thư ký toà soạn: Thanh Tâm Tuyền. Địa chỉ: 129 Lê Văn Duyệt, Sài Gòn. Giấy phép số 6068 TBTTCH BCI 2-12-1966. Số K.D. 484/21/3137 B.T.T.C.H.. In xong ngày 22-3-1967 tại nhà in Vạn Hạnh. Giá 40 đồng. Bản điện tử do talawas thực hiện.) 
 
 
 
\textbf{Phụ lục 2} 
\textbf{Thơ Trần Lê Nguyễn } 
        
\textbf{Màu đen}        
 
Anh đến đêm qua 
sáng nay không tìm em 
đợi chiều về nắng không soi màu mắt 
anh gặp em hoàng hôn 
có ngày tàn nào không thắm 
Hai đứa sẽ ra sao 
khi tình yêu không có sáng mai 
chỉ có trưa nắng trên bờ biển vắng 
Có lẽ em sẽ không nhìn anh 
anh không nói với em 
như mặt trời không trối trăng trên rặng núi phía tây 
Anh phiêu lưu giữa màu mắt xứ người 
để sẽ thấy chiều nay mắt em đen nhất 
Nước mắt không nhoà được màu đen 
Nụ cười không phai được màu đen 
Màu đen về chiều đen nhất những màu đen 
 
Anh nghĩ màu đen chết 
một chiều lang thang gặp chồng em hè phố 
Chồng em cười vô tội 
như không có chuyện gì 
dù đứng trước anh một thây ma biết nói 
 
Thủ phạm bao giờ chẳng là vô tội 
Anh là quan toà lên án tử hình chính anh 
Trốn tránh vào màu xanh 
(những màu xanh chồng lên nhau) 
vẫn thấy một màu đen ám ảnh 
Nụ cười chồng em là sức phá hoại hiền từ 
Thơ anh là vết mai đào huyệt 
Chúng ta vốn yêu ai có nụ cười vui hơn kẻ chôn người chết 
Có thực em sống với nỗi buồn bên trong niềm vui bên ngoài 
Anh biết con gái em càng lớn mắt càng đen hơn mẹ 
 
Anh giảm khinh cho mình một lần tự tử 
vì trong đời đã có một lần yêu 
như anh yêu em 
như trời yêu biển 
dù lòng em như người đi nghỉ mát mùa hè 
không bao giờ nhớ biển mùa mưa 
Vì hồn anh là bãi cát 
hơn một lần nát vết chân 
hơn một lần phẳng lại 
Vì anh sinh ra để sống với màu đen 
khi biển không sao hoà với mây thành màu duy nhất 
ngôi sao là em dù không còn soi đời anh 
anh vẫn sống với màu đen 
màu đen con mắt 
màu đen cuộc đời 
như thấy không được quyền yêu màu đen cõi chết. 
 
 
\textbf{Nước biển} 
 
Tôi yêu người đàn bà không còn muốn thấy mặt tôi 
như không thể chán ghét mình tội lỗi 
Tại sao em không viết cho anh một lời dịu nhẹ 
Nước biển mặn từ ngày yêu nhau 
Kiếm đâu một dòng nước ngọt 
Anh không đành phản bội chính anh 
như em vẫn trung thành với em 
Anh là quán em nghỉ chân đoạn đường cháy nắng 
cho qua chuyện ngoại tình 
trái cấm người đàn bà nào không hái 
 
Nếu vứt được tình yêu như người ta thay áo tắm 
áo dài em vẫn trắng tinh anh 
vẫn đoan trang màu đen thầm lặng 
trọn nghĩa với chồng 
trọn tình với con 
và coi anh như một tên khốn nạn 
 
Anh không muốn nhớ 
người đàn bà một lần đòi chung sống 
(yêu người ấy từ thuở mười lăm 
yêu người ấy qua gia đình đổ vỡ) 
anh thấy những song sắt ngăn đôi 
anh nghe tiếng xiềng xích kêu than 
anh sợ đứa nhỏ phải nhìn kẻ đàn ông không phải cha sống bên má nó 
Anh đã muốn làm tên khốn nạn 
dù không được ai yêu như em yêu anh 
dù biết mất em anh mất cả đời trọn vẹn 
 
Anh làm thơ thấy nguyện ước thực hiện giữa đời 
vui sướng và đau xót 
Khi mặt trời mọc với em 
đêm tối đến cùng anh 
giữa màu đen đi yêu rất nhiều phản bội  
không một ngôi sao soi đường 
Tại sao em không viết cho anh lấy một lời dịu ngọt 
Nước biển mặn chát từ ngày quen nhau 
 
Tôi yêu người đàn bà không còn muốn thấy mặt tôi 
Như không thể chán ghét chính mình tội lỗi. 
\end{multicols}
\end{document}