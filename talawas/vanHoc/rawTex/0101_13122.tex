\documentclass[../main.tex]{subfiles}

\begin{document}

\chapter{Thơ Cao Thoại Châu, tình yêu và một góc nhìn đời}

\begin{metadata}

\begin{flushright}13.5.2008\end{flushright}

Lương Thư Trung



\end{metadata}

\begin{multicols}{2}

Tạp chí văn học nghệ thuật \textit{Thư Quán Bản Thảo}, số 30, tháng Giêng năm 2008, giới thiệu ba người viết cũ, Trần Huiền Ân, Mang Viên Long và Cao Thoại Châu. Riêng Cao Thoại Châu, được nhà văn Trần Hoài Thư, trong Ban Biên tập, giới thiệu như sau: 
 
“Nhà thơ Cao Thoại Châu tên thật Cao Đình Vưu, sinh năm 1939, tại Giao Thủy, Nam Định, Bắc Việt. Tốt nghiệp Đại học Sư phạm Sài Gòn. Cựu giáo sư trường Trung học Thủ Khoa Nghĩa và làm thơ từ năm 1963, hiện sống tại Long An, Việt Nam. Trước năm 1975, thơ anh xuất hiện rất nhiều trên các tạp chí thời danh bấy giờ như \textit{Văn, Nghệ Thuật, Khởi Hành, Văn Học, Thái Độ}. Sau 1975, không biết anh làm nhiều không, thỉnh thoảng bắt gặp vài bài thơ của anh trên Net. 
 
Vì trước năm 1975, thơ anh nhiều, lại hay, nên khó chọn một bài tiêu biểu. Chỉ chọn một bài có hơi hướm giống như bài thơ anh mới sáng tác. Để bạn đọc chia sẻ niềm vui là thơ anh, dù trong hoàn cảnh nào đi nữa, nhưng vẫn là thơ của Cao Thoại Châu dạo nào, không thay đổi từ lời đến ý. Bài thơ mới làm sau này được trích từ tập thơ chung khảo cuộc thi thơ Đồng bằng Sông Cửu Long năm 2006. Đây là một trong hai bài thơ của Cao Thoại Châu được chiếm giải nhất. Điểm đặc biệt ở đây là hai bài thơ tình.” \footnote{
Tạp chí văn học nghệ thuật \textit{Thư Quán Bản Thảo}, số 30, năm thứ 7, tháng 1-2008, giới thiệu ba người viết cũ là Trần Huiền Ân, Mang Viên Long và Cao Thoại Châu.}  
 
Bài thơ của Cao Thoại Châu, mà Trần Hoài Thư vừa nhắc, là bài: 
\begin{blockquote}
 
\textbf{Quán của người tên V} 
        
\textit{Đường sinh tử một vòng chưa khép}        
\textit{Tạt vào đây quán trọ đời em}        
\textit{Rót cho tôi chai nào cay đắng nhất} 
\textit{Hồn tôi là chiếc ly không.} 
        
\textit{Mái quán em tường xiêu giấy lợp}        
\textit{Hào phóng đời cho mượn ánh đèn}        
\textit{Tôi sẽ thắp giùm em thêm chút nữa} 
\textit{Dẫu chỉ là đom đóm trong đêm.} 
        
\textit{Bàn ghế nhựa làm sao rơi loảng xoảng}        
\textit{Rừng ở đâu cho phá đá cưa cây}        
\textit{Em chỉ đưa mượn tạm chiếc ly này} 
\textit{Không cho đập lấy gì phóng đãng.} 
        
\textit{Chủ quán ơi, hôm nay ngày tháng mấy}        
\textit{Nhân loại trừ tôi còn lại được bao người}        
\textit{Mái quán em thành trời cao vời vợi} 
\textit{Để cái nền làm vỡ chiếc ly rơi.} 
        
\textit{Trăm cơn sầu đang đổi cơn say}        
\textit{Tôi đốt quán, em đừng buồn tôi nhé}        
\textit{Mở giùm tôi chai nào cay đắng nữa} 
\textit{Ly vỡ rồi cứ đổ xuống thân tôi.} 

\end{blockquote}
 
Bài thơ thứ hai cũng đoạt giải nhứt trong cuộc thi này là bài: 
\begin{blockquote}
        
\textbf{Lỡ có xa đồng bằng} 
        
\textit{Cũng đành bứt sợi dây câu}        
\textit{Ra đi để lại một châu thổ buồn} 
\textit{(Dân gian)} 
        
\textit{Chân bước không nhấc hồn lên được}        
\textit{Ly rượu đầy không thể nhắp trên môi}        
\textit{Lửa và nếp đã làm nên rượu} 
\textit{Em làm nên tôi ngơ ngác giữa trang đời}        
\textit{         
Ai nào muốn chôn chân một chỗ}        
\textit{Cổ thụ già rồi mục mất mà thôi}        
\textit{Thì xin chọn làm cây cột điện} 
\textit{Ai quan tâm đứng đấy giữa ban ngày}        
\textit{         
Một lần xa chắc đâu là xa tạm}        
\textit{Chập choạng ánh đèn buông lưới ra khơi}        
\textit{Trong một mẻ có khi nhiều tôm cá} 
\textit{Biết đâu chừng lưới được cả hồn tôi}        
\textit{         
Tôi sẽ về như cá nằm trên thớt}        
\textit{Mùa này ruộng lúa cũng đang hong}        
\textit{Dưới chân rạ hằn sâu đôi vết nứt} 
\textit{Nhớ mài cho sắc lưỡi dao em}        
\textit{         
Xa sẽ nhớ dãy thềm rơi những nắng}        
\textit{Cỏ bình nguyên xanh mượt chân đê}        
\textit{Thương và xa, số phần tôi như thế}        
\textit{Đừng ai tin lời hẹn sẽ quay về.}        
(Trang nhà TSCD, trang thơ) 

\end{blockquote}
 
Qua hai bài thơ tác giả sáng tác và trúng giải nhất trong cuộc thi vùng Đồng Bằng Cửu Long vào năm 2006, tức lúc tuổi già chồng chất bên đời, với tứ thơ dạt dào, hồn thơ lâng lâng qua những ước hẹn của bao cuộc tình, để rồi đợi chờ và trách móc người trong mộng như một đuổi bắt một bóng hình trong vô vọng: \textit{Đừng ai tin lời hẹn sẽ quay về}, Cao Thoại Châu quả là thi sĩ của tình yêu đầy mộng mị cả một đời… 
 
Nhưng có lẽ cái chất lãng mạn ở nhà thơ vùng đồng bằng ấy nó đã ngầm cháy tự hơn bốn, năm mươi năm trước khi thi nhân mới bước chân chập chững vào đời qua bài thơ “Cảm ơn, và xin lỗi một người”, tác giả ghi \textit{gởi cho Ch \footnote{
Tuyển tập \textit{Thơ miền Nam trong thời chiến}, tủ sách văn chương Miền Nam, do Thư Ấn Quán xuất bản, Hoa Kỳ, năm 2006.} }, mà chúng tôi tình cờ bắt gặp, có đoạn: 
\begin{blockquote}
        
\textit{Người đã lỡ cho tôi ngó thấy}        
\textit{Một điều gì từa tựa chút yêu đương}        
\textit{Tôi bối rối và ngạc nhiên, hẳn vậy}        
\textit{Như bất thần cười mỉm trong gương.}        
… 

\end{blockquote}
 
Thế rồi, hai hình bóng non trẻ ấy gặp nhau trong vườn nhà ai giữa một đêm khuya dưới ánh trăng mờ ảo với một chút phân vân giữa mơ và thực: 
\begin{blockquote}
        
\textit{Trăng có tối trong vườn khuya đêm đó}        
\textit{Người có vì tự ái nên phân vân}        
\textit{Tôi cũng nhận ra điều khác lạ} 
\textit{Và thưa người, xin được mang ơn.} 

\end{blockquote}
 
Có lẽ vì cái lòng tự ái của nàng có cái tên “Ch” nào đó mà nhà thơ đã giữ lại trong tim và đề thư gửi tặng, với nỗi “phân vân” của tuổi còn bé bỏng, là nguyên nhân cho những bước chân nhà thơ  lúc ấy cũng mới vào đời nhận ra mình sáng sáng chiều chiều thơ thẩn cô đơn và tự ví mình qua bóng hình \textit{chiếc xe bò đã cũ} tự chở lấy đời mình: 
\begin{blockquote}
        
\textit{Tôi là chiếc xe bò đã cũ}        
\textit{Đường gập ghềnh tôi chở tôi đi}        
\textit{Đường gập ghềnh tôi chở tôi về} 
\textit{Trên một chiếc xe bò đã cũ.} 
 \end{blockquote}
 
Nhưng thi nhân không trách người mà còn cảm ơn đời, cảm ơn người: 
\begin{blockquote}
        
\textit{Xin cám ơn người như cám ơn tôi}        
\textit{như cám ơn cuộc đời}        
\textit{đã cho tôi chỗ ngồi để thở}        
\textit{đã cho tôi biết dùng nước mắt}        
\textit{thứ nước mắt không buồn không vui} 
\textit{vẫn hằng hằng chan chứa.} 

\end{blockquote}
 
Và thế rồi cuộc tình ấy rồi ra không bao giờ đậu lại bến bờ nào… Có lẽ đó là một cuộc tình \textit{so le ngôn ngữ}, một thứ ngôn ngữ của lễ giáo không dung chứa nổi cái chất lãng mạn giữa thầy và trò, giữa hai mảnh hồn trinh nguyên trong trắng dưới một mái trường ngày nào, và lúc bấy giờ, nó khó có thể được người đời chấp nhận vì nhân luân luôn đặt nặng phần lễ giáo, nặng tính kỷ cương, nặng phần thuần phong mỹ tục để giữ gìn cái mối giềng xã hội cho tốt đẹp vậy. Thi nhân không những không trách thế nhân về điều này, mà còn có lời cảm ơn và xin lỗi người thương và cuộc đời: 
\begin{blockquote}
        
\textit{Và xin lỗi người như xin lỗi tôi}        
\textit{Như cám ơn cuộc đời}        
\textit{Chúng ta sống so le cùng ngôn ngữ}        
\textit{Chúng ta sống một đời bấp bênh}        
\textit{Và trùng điệp đau buồn} 
\textit{Thứ đau buồn không tên để gọi.} 

\end{blockquote}
 
Tình yêu đầu đời của thi nhân như thế đó và nó cứ nằm êm trong tâm hồn nhà thơ qua gần năm mươi năm, để có lần Cao Thoại Châu tâm sự qua bài viết ngắn: “Châu Đốc, nơi đưa tôi chập chững vào đời”, được tác giả vén chút màn bí mật về bút hiệu Cao Thoại Châu, và người thưởng ngoạn thơ ông hiểu rõ hơn về những bước chân đầu đời của tác giả và ý nghĩa hai chữ “Ch” mà nhà thơ ghi tặng: 
\begin{blockquote}
 
\textit{Ai đã đặt cái tên Châu Đốc, và người cha mẹ nào đó đặt tên cô con gái khéo quá, chỉ chữ đệm của cô học trò nhỏ ghép thêm vào đủ cho tôi có cái tên để ký trên những bài viết ngày nay: Cao Thoại Châu! Bi kịch - nói cho có vẻ lớn - bắt đầu từ đây, từ mối tình trong trắng nhất và đáng tiếc nhất đời tôi. Một nhà thơ nữ mới đây tò mò hỏi về bút danh Cao Thoại Châu, khi được trả lời, đã bảo đại ý thơ anh hay nên anh phải trả cái giá bằng mối tình ấy! Và cái giá đó, tôi đã mượn không bao giờ trả cho Châu Đốc. \footnote{
Trang nhà \textit{Thất Sơn Châu Đốc}, trang bút ký, ngày 21-2-2008.} } 

\end{blockquote}
 
Dòng thơ của Cao Thoại Châu sau cuộc tình không thành ấy rồi cứ tiếp tục âm ỉ chảy trong hồn qua hơn bốn mươi năm qua rồi nhưng vẫn bền bỉ: 
\begin{blockquote}
        
\textit{Từ đấy chúng ta chia lìa đôi ngả}        
\textit{tôi trốn trong tôi thành con gấu ngủ đông}        
\textit{tôi hát lên đoản khúc giáng sinh buồn}        
\textit{em có thấy lời tôi không thành tiếng.}        
(“Vọng ngôn đêm giáng sinh”, \textit{TSCD}, trang thơ) 

\end{blockquote}
 
Với cái “thứ đau buồn không tên gọi” ấy chính là niềm cô đơn bất tận một đời:        
\begin{blockquote}
        
\textit{“ta đã nài xin, em hãy uống}        
\textit{chất men đời làm cháy mắt ta xanh}        
\textit{rượu cay đắng hay chén vàng tê tái} 
\textit{em chối từ không biết nói sao hơn.} 
        
\textit{em không uống nên ta lẻ bạn}        
\textit{vòng tay ôm hồ rượu thấy mênh mông}        
\textit{rượu đã hết hay mắt ta vừa cạn}        
\textit{hay hồn ta rung chuyển đến tang thương.”}        
(“Mời em uống rượu”, 26-12-1968, trong \textit{Nhà thơ miền Nam thời chiến}) 

\end{blockquote}
 
Vì đã từng lặn ngụp trong bể tình trường rồi không tròn cuộc hẹn nên Cao Thoại Châu rất sợ những buổi tiễn đưa nhau: 
\begin{blockquote}
        
\textit{Sẽ không đưa em qua bến đò}        
\textit{sông nước có mấy khi đúng hẹn}        
\textit{cơn nước dữ mùa đông ập đến}        
\textit{lạnh cả hai người, đi và tiễn.}        
(Trích bài “Sẽ”, trang nhà \textit{TSCD}) 

\end{blockquote}
 
Thế nhưng, Cao Thoại Châu vẫn không khỏi bị ám ảnh bởi một lần đưa tiễn mới ngày nào:        
\begin{blockquote}
        
\textit{Hôm nay buồn như vừa tiễn ai}        
\textit{dù sân ga chẳng có một người}        
\textit{tàu hoả bây giờ không chạy nữa}        
\textit{cho đêm đêm hồn vỡ theo ai.}        
(“Hôm nay buồn”, trong \textit{Thơ miền Nam trong thời chiến})  

\end{blockquote}
 
Nhưng rồi cũng đưa tiễn và trớ trêu thay, tác giả lại đưa tiễn chính mình khi chia tay với tuổi ba mươi, tác giả hồi tưởng lại những ngày mới vào đời qua cái nghề dạy học mà ngao ngán biết bao: 
\begin{blockquote}
        
\textit{năm hai mươi tuổi ta vào đời}        
\textit{tập đu đưa cùng miếng cơm manh áo}        
\textit{và áo cơm làm rạn nứt tâm hồn}        
\textit{khi mở mắt thấy vô cùng hoảng sợ.}        
\textit{…}        
\textit{tuổi ba mươi đã bỏ đi rồi}        
\textit{ta tự do như người đãng trí}        
\textit{ngày lại ngày dỡn đùa cùng chiếc ly}        
\textit{ít tờ giấy ta vẽ voi vẽ rắn}        
\textit{ta để rơi ta như những hạt lệ kia}        
\textit{những hạt lệ đã rơi thành khói…}        
(“Tiễn chân tuổi ba mươi”, trong \textit{Thơ miền Nam trong thời chiến}) 

\end{blockquote}
 
“Tiễn chân tuổi ba mươi” là vậy, nhưng khi “chào mừng tuổi năm mươi” nhà thơ của chúng ta cũng chẳng lấy gì làm vui sướng lắm: 
\begin{blockquote}
        
\textit{Xin vẫy tay chào tuổi năm mươi…} 
\textit{Những năm dài bão lửa chiến tranh} 
        
\textit{Chào em, người đưa tôi vào cuộc sống}        
\textit{Để trong nhau những vết thương bầm}        
\textit{Tôi thành người lỡ có tên riêng}        
\textit{Giữa đám đông ồn ào xa lạ…}        
(“Chào mừng tuổi năm mươi”, \textit{TSCD}, trang thơ) 

\end{blockquote}
 
Phải chăng đây cũng lại một lần nhắc đến “em”, người em “Ch” cho nhà thơ có cái tên riêng giữa biển trời văn chương để nhớ!  
 
Dù sợ tiễn đưa, nhưng rồi Cao Thoại Châu lại “… nhớ lúc Trâm đi xa” mà hồn cứ lãng đãng, bàng hoàng, giằng xé và tha thiết lắm: 
\begin{blockquote}
        
\textit{Hình như tôi vừa tiễn một người}        
\textit{có điều gì mất đi trong tôi}        
\textit{lúc qua đèo tôi nhủ mình như thế}        
\textit{lệ có bào mòn núi cũng không nguôi.}        
\textit{……}        
\textit{Tôi tiễn người để biết kẻ đi xa}        
\textit{đã mang theo hồn người ở lại}        
\textit{sao người không đi bằng sân ga}        
\textit{có ánh đèn cho mắt tôi vàng úa}        
\textit{đời buồn tênh sao người không đi ngựa}        
\textit{cho tôi nghe lóc cóc trên đường…}        
\textit{….}        
\textit{Có thật người đã đi chiều nay}        
\textit{hay tiễn đưa chỉ là ảo tưởng}        
\textit{hay chính tôi, tôi vừa khởi hành}        
\textit{vào trăm cõi nhớ nhung vô tận}        
\textit{….}        
\textit{Có người đi, sao chiều không mưa}        
\textit{có người đi, sao chiều không nắng}        
\textit{rất lãng mạn, sao tôi không buồn} 
\textit{chỉ hình như có nhiều đau đớn…} 

\end{blockquote}
 
Và rồi: 
\begin{blockquote}
        
\textit{Chuyện người đi đã là có thật}        
\textit{thôi cũng đành to nhỏ với hư không}        
\textit{tôi là núi sao người bỏ núi}        
\textit{tôi là thuyền sao người không qua sông…}        
(“Để nhớ lúc Trâm xa”, Pleiku 11-5-1969)  

\end{blockquote}
 
Những hình ảnh “tôi là núi sao người bỏ núi”, “tôi là thuyền sao người không qua sông” là hình ảnh chập chùng man man bất tận mang theo lời thống trách thương yêu vô cùng. Nó làm cho người nào có chút lòng là nghe ra tiếng thở dài trong gió trong mây của một người tuyệt vọng biết chừng nào! 
 
Dù có cố tránh thế nào nhưng rồi Cao Thoại Châu cũng phải đành “tiễn bạn” trong nỗi ngậm ngùi: 
\begin{blockquote}
        
\textit{Có bao giờ thích tiễn đưa ai}        
\textit{đưa tiễn vốn chỉ gây nhàm chán}        
\textit{đời trăm năm mập mờ như bóng nắng}        
\textit{đưa tiễn một người làm trống phía sau lưng}        
(“Tiễn bạn”, \textit{TSCD}) 

\end{blockquote}
 
Là một người nghệ sĩ với tâm hồn ngập đầy chất lãng mạn, khi ở tuổi ngoài bảy mươi, thi nhân vẫn để hồn mình trôi về một chốn nào, mơ tìm lại một bóng hình, khắc khoải chờ nghe đến mỏi mòn một giọng nói từ phương trời mây nước xa xăm: 
\begin{blockquote}
        
\textit{Tôi nhớ người vào lúc cuối năm} 
\textit{Gây gây một chút lạnh âm thầm…} 
        
\textit{Người ở đâu, hoa đã về rồi}        
\textit{Khắp phố xá bừng lên rực rỡ}        
\textit{Những hương ấy và người ơi sắc ấy} 
\textit{Làm lẻ một người, lẻ một người thôi!} 
        
\textit{Thêm một Tết, lại thêm tờ lịch}        
\textit{Đêm giao thừa điện thoại vẫn không reo}        
\textit{Hồn tôi hoá vô tri vô giác} 
\textit{Im lìm tan tác biết bao nhiêu} 
        
\textit{Cuối năm rồi, ở tận nơi đâu}        
\textit{Giọng nói ấy vẫn trong như nước lọc}        
\textit{Tự hỏi cuối năm trời và đất}        
\textit{Sao cứ đành biền biệt cách xa nhau...}        
(“Người ở đâu vào lúc cuối năm”, \textit{TSCD} và tạp chí \textit{Da Màu}, ngày 23-2-2008) 

\end{blockquote}
 
Qua những âm điệu dìu dặt mà tha thiết như \textit{những hương ấy và người ơi sắc ấy}, rồi \textit{thêm một Tết, lại thêm tờ lịch}, \textit{đêm giao thừa điện thoại vẫn không reo} làm cho người thưởng ngoạn liên tưởng đến nỗi chờ mong tha thiết biết bao nhiêu của tác giả và rồi thi nhân một mình một bóng chìm vào bóng đêm trừ tịch, cô đơn lặng lẽ biết dường nào... Điều này làm cho tôi nhớ lại có lần thi sĩ Cao Thoại Châu đã thổ lộ: 
\begin{blockquote}
        
\textit{em biết đó, tôi ngồi như chiếc bóng}        
\textit{đêm cũng sâu như tiếng hú quanh thềm}        
\textit{hỡi đoá hồng mà lòng tôi rất mến}        
\textit{thôi lỡ rồi, trễ hết cuộc yêu đương}        
\textit{tôi vẫn sống như là khói bếp}        
\textit{chiều đông nào ngại rét chẳng bay cao}        
\textit{vẫn bơ vơ như những toa tàu}        
\textit{ngại đưa tiễn nên khởi hành thật sớm.}        
(“Trả lời một đôi mắt”, \textit{TSCD} và \textit{TMNTC}) 

\end{blockquote}
 
Phải chăng, Cao Thoại Châu mãi mãi mang trong hồn mình một tâm cảm cô đơn tận cùng và mãi mãi muốn ôm hoài mộng mị dù đã trễ hết những cuộc tình: 
\begin{blockquote}
        
\textit{Có thể tôi hoá thân thành bóng}        
\textit{để yêu tôi khi bóng chiếu lên tường?}        
\textit{để gần tôi như một mùi hương} 
\textit{vẫn thoang thoảng trên đầu môi mỗi tối...} 
        
\textit{ôi mộng mị suốt đời không nỡ dứt}        
\textit{khi giật mình trễ hết những yêu đương.}        
(“Trả lời một đôi mắt”, trong trang nhà \textit{TSCD} và trong \textit{TMNTC}) 

\end{blockquote}
 
Dòng thơ Cao Thoại Châu, với những bài thơ ca ngợi cuộc tình ngang trái, dở dang làm người đọc nhiều lúc cũng phải nhận ra cái đẹp của những cuộc yêu đương không tròn, những đưa tiễn mặn bờ môi và cảm thông cùng tác giả đã triền miên chìm trong dòng chảy của một dòng sông tình ái, phải qua nhiều khúc rẽ của cuộc đời đầy vị cay và mặn.  
 
Thế nhưng, người đọc còn nhận ra thêm ở thơ Cao Thoại Châu cái ý nghĩa cao siêu của đời sống qua những lời “minh triết”: 
\begin{blockquote}
        
\textit{Hết ly này ta mời nhau ly nữa}        
\textit{Bên ngoài ly năm tháng đã vơi dần}        
\textit{Có điều chi bạn ta trầm ngâm thế} 
\textit{Để vô tình làm vỡ chiếc ly không.} 
        
\textit{Có điều chi bạn ta trầm ngâm thế}        
\textit{Năm lại đầy vào sáng mai thôi}        
\textit{Cơn say không thể chia hai nửa}        
\textit{Thì cơn buồn đâu sẻ được làm đôi}        
(“Lời minh triết”, trang nhà \textit{TSCD}, trang thơ) 

\end{blockquote}
 
Quả là những vần thơ thoát ra từ một tâm can già dặn với mấy mươi năm chìm nổi giữa dòng đời, cùng cái tứ thơ vừa cô đọng như một thứ cao hổ cốt, vừa đắng ở bờ môi, vừa tê ở đầu lưỡi và vừa ngọt lịm trong tim sau mấy mươi năm gần như cạn dòng máu tươi làm nên đời sống. Làm thơ tình yêu, Cao Thoại Châu làm với con tim tràn trề sinh lực với những “mộng mị một đời không nỡ dứt” bao nhiêu, thì làm thơ về cuộc đời thường, tác giả lại chắt lọc những chất liệu đời mà ông đã từng sống, từng trải, để rồi thơ toát ra từ một con người lão luyện đầy kinh nghiệm bấy nhiêu. Như một lời tâm sự hay một mối ưu tư? 
\begin{blockquote}
        
\textit{Hay bạn ta lỡ uống nhiều rồi}        
\textit{Uống tới bến cơn sầu chất ngất}        
\textit{Thân trượng phu rượu nào pha nước mắt}        
\textit{Đốm thuốc lập loè như đom đóm trong đêm…}        
(“Lời minh triết”, \textit{TSCD}, trang thơ) 

\end{blockquote}
 
Để rồi nhận ra cái giây phút tuyệt đỉnh của cuộc đời nào rồi cũng có cùng số phận như nhau trong một chu kỳ của dòng sống thế nhân: 
\begin{blockquote}
        
\textit{Trong cơn say lắm kẻ hay cười}        
\textit{Tưởng như rượu để mừng ngày hội}        
\textit{Tới nữa đi, bạn ta ơi hãy tới}        
\textit{Cái đĩnh nào chót vót cũng như nhau.}        
(“Lời minh triết”, \textit{TSCD}, trang thơ) 

\end{blockquote}
 
Còn đỡ khổ nếu gặp rượu ngon để nghe được tiếng cười trong cơn say bất tận, nhưng rủi gặp “rượu lạt” rồi thì , ôi thôi, đời còn chua chát biết bao nhiêu: 
\begin{blockquote}
        
\textit{Và lỡ uống phải ly rượu nhạt}        
\textit{Như một lời cảm thán buồn tênh}        
\textit{Rượu sóng sánh những lời minh triết}        
\textit{Bạn ta buồn có nhận ra không?}        
(“Lời minh triết”, \textit{TSCD}, trang thơ) 

\end{blockquote}
 
Dưới cảm nhận của người đọc, qua những vần thơ mà chúng tôi tình cờ bắt gặp, thi sĩ Cao Thoại Châu quả là một nhà thơ lãng mạn đúng mực, ông luôn luôn chung thủy với những cuộc tình, một thứ tình yêu làm nên thơ ông và làm nên cuộc sống đời ông. Thêm vào đó, thơ ông còn phảng phất đó đây một thứ nhân sinh quan mang đầy chất “triết lý” mà gần gũi, dễ nhận dưới cái nhìn của ông về cuộc đời, về nhân sinh và về chính mình sau 70 năm lăn lóc sống giữa dòng đời đầy biến thiên dâu bể này, thật vô cùng thâm thúy, ý nghĩa. 
 
\textit{Lấp Vò ngày 26-2-2008, đọc lại và bổ túc ngày 20-4-2008} 
 
© 2008 talawas 
 
 <hr>
\small{[1]}\footnote{\url{http://www.talawas.org/talaDB/#nr1}}Tạp chí văn học nghệ thuật \textit{Thư Quán Bản Thảo}, số 30, năm thứ 7, tháng 1-2008, giới thiệu ba người viết cũ là Trần Huiền Ân, Mang Viên Long và Cao Thoại Châu. 
 \small{[2]}\footnote{\url{http://www.talawas.org/talaDB/#nr2}}Tuyển tập \textit{Thơ miền Nam trong thời chiến}, tủ sách văn chương Miền Nam, do Thư Ấn Quán xuất bản, Hoa Kỳ, năm 2006. 
 \small{[3]}\footnote{\url{http://www.talawas.org/talaDB/#nr3}}Trang nhà \textit{Thất Sơn Châu Đốc}, trang bút ký, ngày 21-2-2008.
 
\end{multicols}
\end{document}