\documentclass[../main.tex]{subfiles}

\begin{document}

\chapter{Trần Dần, người im lặng}

\begin{metadata}

\begin{flushright}23.5.2008\end{flushright}

 Lữ



\end{metadata}

\begin{multicols}{2}

Tôi nhớ hoài hình ảnh nhà văn Phùng Cung ngồi trước nhà. Anh ngồi thẳng. Dáng ngồi đó, khiến tôi nghĩ anh có tập luyện khí công, để chống chọi với bệnh tật trong những năm tháng tù đày. Sau này, tôi mới biết ra, đó cũng là dáng ngồi của một nhà văn, chỉ còn vài tháng trời để sống. Mà không phải chỉ ngồi thẳng, anh còn có cái nhìn thật thẳng khi nói chuyện. Anh không ngó qua, ngó lại, mà đôi mắt chiếu thẳng về phía trước mà nói. Lời nói của anh khẳng định, rõ ràng. 
 
Anh nói thẳng. Đồng thời, anh nói lớn tiếng. Anh nói những điều mà một người Việt Nam phải e dè khi phát biểu ở trong nước, vào thời điểm đó và ngay cả bây giờ. Nhiều lần, anh nhắc tới “ông Hồ Chí Minh”. Rồi một lúc nào đó, anh kể sang nhà thơ Trần Dần. Anh nói: “Trần Dần đến nhà tôi, bắt tay. Nói rằng, mình bỏ qua mọi chuyện không vui với nhau đi. Tôi vui vẻ thôi. Người nghệ sĩ phải biết tha thứ, và thương yêu thì mới tiếp tục sự nghiệp sáng tạo của mình được.” 
 
Lúc đó, tôi không hiểu khó khăn giữa anh và nhà thơ Trần Dần. Tôi cứ ngỡ là mình nghe lầm, và bỏ qua chi tiết ấy. Sau này, đọc Phùng Hà Phủ\footnote{\url{http://www.talawas.org/talaDB/showFile.php?res=1619&rb=0102}}, con trai của nhà văn Phùng Cung, tôi mới biết là mình nghe đúng, và hiểu rõ hơn nỗi khó khăn, khổ đau của những cây bút hàng đầu trong phong trào \textit{Nhân văn - Giai phẩm}. Nhưng rồi, họ đến với nhau được. Chính anh Phùng Hà Phủ cũng viết: “mối quan hệ của bố tôi với các bạn cũ có phần cởi mở hơn.” \footnote{
\textit{Truyện và thơ}, Phùng Cung – Văn Nghệ (USA) xuất bản 2003}  Điều này làm cho tôi hết sức vui mừng, tin tưởng vào khả năng hoà giải của những người cầm bút chân chính. 
 
Ngược lại với anh Phùng Cung, khi tôi đến thăm, nhà thơ Trần Dần hoàn toàn im lặng. Suốt buổi ấy, người tiếp khách, nói chuyện là chị. Chị kể nhiều về người con trai đang theo đuổi ngành hội hoạ. Rồi chị nói về chồng mình: “Bây giờ, anh không nói gì nữa cả.” Tuy không nói, nhưng đôi mắt của anh vẫn sáng. Anh ngồi nhìn chúng tôi nói chuyện. Tôi cũng không nói gì nhiều, cũng nhìn. Tôi nhìn anh như tôi đã nhìn nhà văn Phùng Cung. 
 
Tôi hoàn toàn bị bất ngờ khi gặp Trần Dần. Trong tôi, Trần Dần là một nhà thơ trẻ. Tôi nghĩ mình đi gặp một chú bộ đội trẻ, làm thơ. Trên mặt ý thức, dĩ nhiên chúng ta hiểu đây là điều hết sức vô lý. Năm đó, nhà thơ đã 70 tuổi thì làm sao là một chú bộ đội trẻ cho được. Nhưng hãy tha thứ cho sự ngu ngơ của tôi, trước đó, tôi chỉ biết Trần Dần qua tác phẩm \textit{Trăm hoa đua nở trên đất Bắc}\footnote{\url{http://www.talawas.org/talaDB/showFile.php?res=9495&rb=08}} của cụ Hoàng Văn Chí mà thôi. Trong tác phẩm ấy, Trần Dần là một chú bộ đội trẻ, tác giả của bài thơ “Nhất định thắng”. 
 
Bài thơ “Nhất định thắng” thật hay, nhưng không được lòng nhà cầm quyền. Thời buổi nào cũng có những người không được lòng đám đông đang có quyền, vì dám nói và dám viết thật với lòng mình. Tôi thích thơ Trần Dần, thích luôn thái độ không chịu khuất phục của anh và những anh em khác trong phong trào \textit{Nhân văn -  Giai phẩm}. Tôi muốn xem tận mắt cái vết sẹo ở cổ do anh tự tay cứa lấy trong một cơn uất ức, nhưng không chết, đã trở thành biểu tượng của sự phản kháng của văn nghệ sĩ miền Bắc vào những năm cuối thập niên 50. Tôi đến tìm anh với lòng ngưỡng mộ và những ý niệm ngây thơ như vậy. Nhưng khi gặp nhà thơ Trần Dần, tôi ngạc nhiên đến sững sờ. Tôi không thấy trước mắt tôi một biểu tượng của sự phản kháng nào. Tôi chỉ thấy một con người. Một ông lão đầu tóc bạc phơ, ngồi nhìn tôi mà không nói một câu gì nữa cả. 
 
Phải ngồi trước một con người, nhà thơ Trần Dần ở tuổi 70, tôi mới thấy buồn cười cho những ý niệm sai lạc tôi có về anh và về phong trào \textit{Nhân văn - Giai phẩm}. Trước mặt tôi không phải là một biểu tượng, không phải là một sự phản kháng, mà cũng không phải là nạn nhân của chế độ độc tài, đảng trị như cách tôi đã nhìn trước đây. Trước mặt tôi là một nhà thơ, vậy thôi. Vượt qua khỏi những giằng co, mâu thuẫn, áp bức của cuộc đời, xã hội, nhà thơ Trần Dần vẫn tiếp tục sáng tạo. Và cũng như Phùng Cung, anh đã hái những bông hoa thật đẹp của sự sống, đưa vào thơ, mà sau này tôi được đọc thật nhiều. 
 
Khi tôi chào anh ra về, nhà thơ Trần Dần nắm tay tôi thật chặt và thật lâu. Cái nắm tay thay cho một lời nói, mà anh muốn gửi đến tôi. Và tôi hiểu lời của anh như vầy: “Cám ơn anh đến thăm tôi.” Tôi rất cảm động. Tôi thương cho anh như một con người. Chắc anh phải cô đơn lắm, sống giữa bao nhiêu nhãn hiệu mà người ta gắn lên anh. Đến với anh, người ta phải tò mò lắm về cái quá khứ \textit{Nhân văn - Giai phẩm}, về nền “văn học phản kháng”, mà ít ai chịu nhìn anh như nhìn một con người. Một con người đã gần đất xa trời, suốt đời đem hết tài hoa của mình để cống hiến những tác phẩm thật đẹp cho xã hội, cho con người. Một con người, cuối đời, càng chán ngán với những danh xưng phù phiếm, mà thấy cần một thứ thật là đơn giản: tình người. 
 
Nhà thơ Trần Dần đã đến bắt tay với anh Phùng Cung; gặp lại người bạn mà một thời gian lâu, không nhìn nhau được. Hai anh vui vẻ với nhau, như những ngày còn trẻ. Sau đó, hai anh rủ nhau ra đi, cùng trong một năm. Đó là năm 1997. Tôi biết rằng hai anh đã nắm tay nhau thật chặt, thật đầy tình người khi ra về với ông bà, tổ tiên. Cái tình người đó, tôi cảm nhận thật rõ ràng nơi anh Phùng Cung, và nơi anh Trần Dần. Bây giờ, nhìn lên trời xanh, tôi có thể thấy họ đang hoá thân thành những đám mây trắng, thong dong bay khắp nơi, tự do. 
 
Khi viết bài này, tôi thấy nhà thơ Trần Dần, với mái tóc bạc phơ, như đang nhìn tôi mà cười. Phải rồi, sau lần thăm viếng ấy, tôi tự hứa với mình là sẽ viết về tấm lòng của anh mà tôi đã cảm nhận được trong cái bắt tay thật chặt đó. Tôi chưa bao giờ quên lời hứa của mình. Nhưng mà thời gian trôi nhanh quá. Mới đó, mà hoa mùa xuân lại nở thêm một lần nữa rồi. Cám ơn anh. Cám ơn tấm lòng anh đã dành cho tôi. Đó là hành trang tôi đã mang theo bao nhiêu năm trời, hết sức là trân quí. 
 
\textit{Hoà Lan, 16-5-2008} 
 
© 2008 talawas 




\end{multicols}
\end{document}