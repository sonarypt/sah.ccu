\documentclass[../main.tex]{subfiles}

\begin{document}

\chapter{Á Nam Trần Tuấn Khải – nhà thơ của dòng văn học yêu nước trong những năm 1920}

\begin{metadata}

\begin{flushright}27.5.2008\end{flushright}

Lê Chí Dũng

Nguồn: Tham luận tại Hội thảo Á Nam Trần Tuấn Khải do Trường Đại học Khoa học xã hội và Nhân văn thuộc Đại học Quốc gia Hà Nội tổ chức năm 2006.

\end{metadata}

\begin{multicols}{2}

Sau Đại chiến thế giới lần thứ nhất, thực dân Pháp thực hiện cuộc khai thác thuộc địa lần thứ hai, thay chính sách đồng hoá (politique d’assimilation) bằng cái gọi là chính sách hợp tác (politique d’association), mở ra những lỗ thoát hơi cần thiết (soupapes nécessaires) về chính trị, kinh tế, xã hội, giáo dục, văn hoá, trong đó có việc cho những nhân viên văn hoá cổ động “xây đắp nền quốc văn” như một thứ “chủ nghĩa ái quốc bằng quốc ngữ”, mơn trớn, lôi kéo cả cựu học lẫn tân học, lái thanh niên, trí thức chỉ vào một nẻo đường: “Các nước Âu Mỹ trọng các nhà văn sĩ hơn các bậc đế vương vì cái công nghiệp tinh thần còn giá trị quý báu và ảnh hưởng sâu xa hơn là những sự nghiệp nhất thời về chính trị” \footnote{
\textit{Nam phong tạp chí, }số 10, décembre 1919.} . Một cuộc thay đổi mà bất cứ một cuộc bể dâu nào trước đây cũng không thể so sánh đã xảy ra ở Việt Nam. 
 
Tình hình như vậy trong những năm 1920 thúc đẩy mạnh mẽ việc hiện đại hoá văn học nước nhà theo hai con đường – con đường thay đổi dần dần văn học trung đại Việt Nam để tiến đến văn học hiện đại và con đường xây dựng ngay nền văn học hiện đại; trên hai con đường hiện đại hoá văn học đó văn học Việt Nam đều thu được những thành tựu. 
 
Trong những năm 1920 hồi quang rực rỡ của văn học các nhà chí sĩ soi sáng, dẫn dắt dư luận xã hội xung quanh các sự kiện: đả kích Khải Định; ca tụng Phạm Hồng Thái; chống những luận điệu lừa mị, có lợi cho trật tự của chế độ thuộc địa, rằng “vấn đề quan trọng bậc nhất trong nước ta hiện nay là vấn đề văn quốc ngữ” và rằng “chữ quốc ngữ ấy chính là cái bè từ cứu vớt bọn ta trong bể trầm luân vậy” \footnote{
\textit{Thượng Chi văn tập, }Editions Alexandre de Rhodes, Hanoi, 1943, tr. 55-56.} , đưa thanh niên, trí thức ra khỏi ảo mộng “ỷ Pháp cầu tiến bộ”, “Pháp Việt đề huề”; đấu tranh đòi ân xá Phan Bội Châu và để tang Phan Châu Trinh. Hồi quang rực rỡ của văn học các nhà chí sĩ đã kích hoạt sự trỗi dậy mạnh mẽ của dòng văn học yêu nước… 
 
Trong những năm 1920, những văn phẩm của Nguyễn Ái Quốc được viết bằng tiếng Pháp “chọc thủng lưới sắt” của thực dân về Việt Nam. Một lớp thanh niên, trí thức giác ngộ tư tưởng cách mạng xã hội chủ nghĩa sáng tác thơ ca không chỉ để thể hiện cái tôi trữ tình tươi mới, mãnh liệt của mình, mà còn để thuận lợi truyền bá một đường lối cứu nước tất thắng vào công nông, vào nhân dân đông đảo. Trong bối cảnh chính trị, xã hội, văn hoá, văn học nhiều chiều và phức tạp như vậy, thanh niên, trí thức, trong đó có những người cầm bút sáng tác văn chương, phân hoá theo nhiều ngả, “người sang Pháp, người sang Liên Xô, người chui vào bí mật, người xoay ra làm ăn theo lối bình thường” \footnote{
Trần Huy Liệu, “Nhớ lại ông già bến Ngự” trong sách\textit{ Nhà yêu nước và nhà văn Phan Bội Châu, }Nxb. Khoa học Xã hội, Hà Nội, 1970.} . 
 
Đặt Á Nam Trần Tuấn Khải trong bối cảnh như đã trình bày ở trên, nhà nghiên cứu có thể hiểu rõ và đánh giá đúng sáng tác thơ của ông trong những năm 1920: \textit{Duyên nợ phù sinh I} (in lần đầu, 1921; Hương Ký xuất bản, Hà Nội, 1928); \textit{Duyên nợ phù sinh II} (Chân Phương xuất bản, Hà Nội, 1923); \textit{Bút quan hoài I} (tức \textit{Duyên nợ phù sinh III}, được viết từ năm Nhâm Dần 1926; Hương Ký xuất bản, Hà Nội, 1934); \textit{Bút quan hoài II }(được viết trong những năm 1926, 1927; in năm 1927); \textit{Hồn tự lập I}, 1926; \textit{Hồn tự lập II, }1927; \textit{Trường thán thi} (10 khúc, được viết năm 1926; in trong \textit{Sách chơi năm Nhâm Thân, }1932) \footnote{
Á Nam Trần Tuấn Khải còn là tác giả của các tiểu thuyết \textit{Gương dâu bể I, }1922; \textit{Hồn hoa, }1925. Ông đã dịch \textit{Thủy hử}, 1925, \textit{Hồng lâu mộng}, 1934. Vở kịch của ông \textit{Mảnh gương đời }từng được công diễn ở Hà Nội, Hải Phòng.} . 
 
Sáng tác thơ của Á Nam Trần Tuấn Khải nằm trong dòng văn học yêu nước những năm 1920 mà đội ngũ của nó hết sức đông đảo: Đoàn Như Khuê, Tản Đà, Bùi Kỷ, Nguyễn Phan Lãng, Nguyễn Can Mộng, Võ Liêm Sơn, Phạm Tuấn Tài, Trần Huy Liệu, Phạm Tất Đắc, Nguyễn Xuân Lãm, Đạm Phương, Nguyễn Trung Khuyến, Vị Bắc, Giang Hồ Du Tử, Vũ Khắc Tiệp, Tao Đàn, Sầm Phố, Trần Ngọc Hoàn, Nguyễn Văn Áng, Nguyễn Hi Chu, Dương Bá Trạc, Đông Bình, Phạm Văn Cung, Lê Hoa, Giả Ẩn, Nguyễn Thúc Khiêm, Nguyễn Tử Siêu, v.v... và v.v... “Ông già bến Ngự” cũng hoà giọng của mình trong dòng văn học yêu nước này. 
 
Trên văn đàn lúc ấy nổi lên những bài thơ vịnh sử ngợi ca những anh hùng chống xâm lăng; những bài thơ vịnh vật phê phán bọn bán nước cầu vinh; những bài thơ phơi bày nỗi nhục nhã của người dân vong quốc nô; những tiểu thuyết lịch sử: \textit{Tiếng sấm đêm đông, Vua Bố Cái, Đinh Tiên Hoàng, Lê Đại Hành.} 
 
Trong phong trào rộng lớn sáng tác văn chương yêu nước như vậy, Á Nam Trần Tuấn Khải đã viết trên chục bài thơ vịnh sử, như “Chơi thành Cổ Loa”, “Qua nhà Giám”, “Đề đền vua Hùng”, “Thăng Long hoài cổ”, “Đề tượng vua Lê”, “Trường thán thi”, “Hai chữ nước nhà”… Đọc những bài thơ này của ông, độc giả ghi vào lòng những câu thơ gan ruột của thi nhân: 
\begin{blockquote}
        
\textit{Xẻ yếm may cờ dù thoả chí,}        
\textit{Kiếp này khỏi phụ với cha ông}        
(“Trường thán thi”) 
        
\textit{Tuốt gươm thề với thương thiên,}        
\textit{Phải đem tâm huyết mà đền cao sâu.}        
\textit{Gan tráng sĩ vững sau như trước,}        
\textit{Chí nam nhi lấy nước làm nhà.}        
\textit{Tấm thân xẻ với sơn hà,}        
[…]        
\textit{Nữa mai mốt giết xong thù nghịch,}        
\textit{Mũi Long Tuyền rửa sạch máu tanh.}        
\textit{Làm cho động đất trời kinh,}        
\textit{Bấy giờ quốc hiển gia vinh có ngày.}        
(“Hai chữ nước nhà”) 

\end{blockquote}
 
Á Nam Trần Tuấn Khải truyền cho người đọc, người nghe niềm tự hào về thắng cảnh, danh lam của đất nước, truyền cho ngươi đọc, người nghe cả niềm tin nữa: 
\begin{blockquote}
        
\textit{Rủ nhau thăm cảnh Kiếm hồ,}        
\textit{Thăm cầu Thê Húc, thăm chùa Ngọc Sơn.}        
\textit{Đài Nghiên, Bút Tháp chưa mòn,}        
\textit{Hỏi ai tô điểm nên non nước này?}        
(“Phong dao”) 

\end{blockquote}
 
Viết “Con hoàng anh”, “Mắng bù nhìn”, “Hỡi cô bán nước”, ông phê phán, cảnh tỉnh bọn người làm tay sai cho giặc. Ông nói lên nỗi nhục của người dân mất nước: 
\begin{blockquote}
        
\textit{Nô nức đua nhau hội với hè,}        
\textit{Văn minh Nam Việt tiến mau ghê!}        
\textit{Nhảy đầm, ăn tiệc, ông Tây sướng,}        
\textit{Liếm chảo, leo đu, đứa trẻ mê!!!}        
\textit{Trời nắng lợi riêng phường bán nước,}        
\textit{Bụi lầm khổ chết lũ buôn xe.}        
\textit{Anh mù nỏ biết trò chi cả,}        
\textit{Cứ bập bùng bung, cứ cò ke...}        
(“Xem hội Tây”) 

\end{blockquote}
 
Hồi ấy những bài thơ của Á Nam Trần Tuấn Khải, như “Gánh nước đêm”, “Tiễn chân anh Khoá xuống tàu”, “Mong anh Khoá”, là những bài thơ quen thuộc với công chúng ở thành thị và cả ở nông thôn, bởi những bài thơ này đã theo chân những người hát sẩm đến các nhà ga, bến xe, bến tàu và về các vùng quê. Cùng với \textit{Bể thảm} của Đoàn Như Khuê, những bài thơ ấy của ông đã gieo vào lòng người nỗi buồn “quốc phá gia vong”, tương lai mờ mịt, anh hùng tận lộ, nhưng đồng thời cũng vì vậy mà nhắc nhở mọi người Việt Nam không quên nước, thấm thía nỗi nhục mất nước... 
 
Đọc Á Nam Trần Tuấn Khải, độc giả bắt gặp trong thơ ông \textit{cái tôi nội cảm} (le moi intérieur). Cái tôi nội cảm này man mác trong những bài thơ thể hiện lòng yêu nước của thi nhân và nổi rõ trong những bài thơ bộc lộ cái nhìn ái ân phong tình của ông đối với con người và những hiện tượng trong thực tại, như trong bài thơ: 
\begin{blockquote}
        
\textit{Hiu hắt phòng thu nhớ cố nhân!}        
\textit{Nhớ cô hàng quạt chợ Đồng Xuân.}        
\textit{Tờ mây phong kín lời sơn hải,}        
\textit{Tin gió bay tàn lửa ái ân.}        
\textit{Hương hoả ba sinh tình khắc cốt,}        
\textit{Can tràng trăm đoạn lúc rời chân.}        
\textit{Thói đời nóng lạnh coi mà ngán,}        
\textit{Hiu hắt phòng thu nhớ cố nhân.}        
(“Nhớ cô hàng quạt”) \footnote{
Đọc “Nhớ cô hàng quạt” của Á Nam Trần Tuấn Khải, độc giả nghĩ đến “Nhớ chị hàng cau” của Tản Đà Nguyễn Khắc Hiếu: 
\textit{Ngồi buồn đâm nhớ chị hàng cau,} 
\textit{Khoảng mấy năm trời ở những đâu?} 
\textit{Khăn vải chùm hum lâu vắng mặt,} 
\textit{Chiếu buồm che giữ có tươi màu?} 
\textit{Ai đương độ ấy lăm dăm mắt,} 
\textit{Tớ đã ngày nay lún phún râu.} 
\textit{Bèo nước hợp tan người mỗi nẻo,} 
\textit{Cậy ai mà nhắn một đôi câu.}}  

\end{blockquote}
 
và trong những bài \textit{phong dao, }chẳng hạn 
\begin{blockquote}
        
\textit{Anh đi anh nhớ quê nhà,}        
\textit{Nhớ canh rau muống, nhớ cà dầm tương.}        
\textit{Nhớ ai dãi nắng dầm sương,} 
\textit{Nhớ ai tát nước bên đường hôm nao!} 

\end{blockquote}
 
Như vậy, cái tôi nội cảm ấy của Á Nam Trần Tuấn Khải vừa là “bút quan hoài”, vừa là “duyên nợ phù sinh”. 
 
Thơ của Á Nam Trần Tuấn Khải là sự thể hiện tràn đầy quan niệm nghệ thuật của ông:        
\begin{blockquote}
        
\textit{Đời không duyên nợ thà không sống,} 
\textit{Văn có non sông mới có hồn.} 

\end{blockquote}
 
Lúc bấy giờ Á Nam Trần Tuấn Khải, Tản Đà Nguyễn Khắc Hiếu và nhiều thi sĩ Việt Nam khác “đua nhau viết những bài thơ, bài ca thể thức dân gian: sa mạc, bồng mạc, hát sẩm, hát điên, những bài lục bát hay song thất lục bát… song song với những bài thơ theo lối cổ phong hay luật Đường” \footnote{
Đặng Thai Mai, \textit{Trên đường học tập và nghiên cứu}, tập III, Nxb. Văn học, Hà Nội, 1973, tr. 118.} . 
 
Thơ của Á Nam Trần Tuấn Khải và thơ của Tản Đà Nguyễn Khắc Hiếu man mác một hồn thơ dân gian, một tình điệu Việt Nam. 
 
Trong những năm 1920, thơ của Á Nam Trần Tuấn Khải và thơ của Tản Đà Nguyễn Khắc Hiếu đang đi trên con đường biến đổi dần dần thơ truyền thống Việt Nam; con đường ấy và con đường hiện đại hoá thơ Việt Nam của trí thức Tây học là cùng một hướng… 
 
\textit{Đà Lạt, 6/2006} 
 



\end{multicols}
\end{document}