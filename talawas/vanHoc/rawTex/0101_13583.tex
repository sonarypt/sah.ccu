\documentclass[../main.tex]{subfiles}

\begin{document}

\chapter{Thơ rất thiêng}

\begin{metadata}

\begin{flushright}1.7.2008\end{flushright}

Bùi Minh Quốc



\end{metadata}

\begin{multicols}{2}

Không biết đã có nhà nghiên cứu nào mò mẫm vào cái phạm trù đặc biệt này: tính thiêng của thơ? Phần tôi, bằng sự trải nghiệm của hơn năm mươi năm cầm bút, với tất cả sự dè dặt, chỉ xin giãi bày đôi chút cảm và nghĩ. 
 
Cảm và nghĩ này bắt đầu vụt loé trong tôi vào năm 1992, khi đọc \textit{Chân dung nhà văn}\footnote{\url{http://www.talawas.org/talaDB/showFile.php?res=3860&rb=08}}\textit{ }của Xuân Sách. Trước kia chỉ nghe truyền miệng, các “Chân dung” cứ trượt đi trong cái bầu khí bỗ bã cười đùa tếu táo, có bài nghe xong tôi còn thầm trách ông Sách ác khẩu. Giờ thì đọc đến đâu giật mình đến đấy. Và ngộ ra: thơ thiêng lắm! Năm ấy tôi ghi lại, bằng thơ, cảm nghĩ của mình: 
\begin{blockquote}
        
\textit{Thơ thiêng lắm người ơi}        
\textit{Phản thơ thì phải chết} 
\textit{Chẳng ai giết mình mà mình tự giết} 

\end{blockquote}
 
Cuộc \textit{\textbf{tự giết}} ấy đã được nhà thơ Chế Lan Viên tự thổ lộ vào lúc cuối đời (1989) trong bài “Trừ đi”: 
\begin{blockquote}
 
\textit{Cái cần đưa vào thơ, tôi đã giết rồi } 

\end{blockquote}
 
Quái lạ, sao vậy nhỉ? 
 
\textit{\textbf{Tôi}} (tức Chế Lan Viên) là người làm thơ, một thần đồng thơ từ 16 tuổi, một (trong không nhiều) thi sĩ hàng đầu của đất nước, mà sao \textit{tôi }lại giết thơ?  
 
Vì \textit{tôi} không còn là \textit{tôi} nữa. 
 
Cần phải hỏi tiếp: Vậy chớ vì sao \textit{tôi} không còn là \textit{tôi} nữa?  
	 
Tại vì… 
	 
Tại vì… 
	  
Có cà lăm mấy thì cũng phải tự nhận 2 cái “tại vì” này: 
\begin{itemize}

item{Tại vì có một sức mạnh ở bên ngoài \textit{tôi} đã tàn bạo đang tay vo tròn bóp méo \textit{tôi}. }

item{Tại vì ở ngay trong \textit{tôi }có một sức mạnh ma quỉ nào đó chỉ riêng \textit{tôi} biết xui khiến \textit{tô}i phải tự vo tròn bóp méo mình. }

\end{itemize}
 Các nhà triết học gọi đó là trạng thái tha hóa của con người, là đánh mất bản ngã, đánh mất cái tôi. Cũng có người gọi là bán linh hồn cho quỉ dữ. 
 
Năm 1960, sau một thời gian dài không xuất hiện, Chế Lan Viên bỗng tái xuất với tập thơ \textit{Ánh sáng và phù sa }được giới phê bình nêu bật như ngôi sao trên văn đàn miền Bắc trong tư thế một nhà thơ tiền chiến nhờ đi theo Đảng đã vươn tới góp phần quan trọng làm nên thành tựu thơ cho nền văn học cách mạng. Giữa bản hợp xướng tôn vinh ồn ã, nghe truyền miệng lạc dòng một giọng thơ thì thầm khép nép nhưng rất lạ và… “láo” - \textit{thơ chân dung}, của một anh lính nào đó có tên là Xuân Sách vừa từ dưới đơn vị chuyển về tạp chí \textit{Văn nghệ Quân đội}. Một câu thơ tâm ngẩm tầm ngầm truyền lan như điện giật: \textit{“Lựa ánh sáng trên đầu mà thay đổi sắc phù sa”\textbf{.}} Thế thôi, chỉ vẩy bút chấm phá sơ sơ thế thôi, đã thấy hiện ra mồn một cái bản lai diện mục ẩn kín của nhân vật đang rất “hoành tráng” bề ngoài. Ghê thật. Cái thủ pháp vẽ chân dung kiểu này quả là độc chiêu có lẽ ông trời phú riêng cho Xuân Sách. Bài thơ chân dung không nói là vẽ ai, nhưng mọi người nghe qua đã biết liền đấy là Chế Lan Viên, nhờ ngón lẩy chữ tài tình:\textit{“ánh sáng”}, \textit{“phù sa”}.  
\begin{blockquote}
 
\textit{Lựa ánh sáng trên đầu mà thay đổi sắc phù sa.} 

\end{blockquote}
 
Thế đấy, khi đã \textit{lựa ánh sáng trên đầu mà thay đổi sắc phù sa} thì cũng là bắt đầu tự giết rồi. Đấy là dùng thơ để đổi lấy cái gì đó ngoài thơ, phản thơ chứ gì nữa. Quá trình tự giết từ đấy tự vận hành bên trong con người mình, không cưỡng nổi bởi lực đẩy của những tham vọng ngoài thơ, phản thơ. Cứ thế, âm thầm, một mình mình biết. Và đinh ninh rằng chỉ một mình mình biết. Nhưng mà không, nhầm to. Có người khác biết. Người đó là Xuân Sách. Thế cho nên đến lúc có người đọc cho Chế Lan Viên nghe bài thơ chân dung Xuân Sách vẽ mình, Chế Lan Viên chỉ cười cười lặng lẽ, lảng lảng, không cãi, và những lần hội họp này khác có gặp Xuân Sách thì vẫn bắt tay bình thường, sự thể ấy cũng chính Xuân Sách đã kể với tôi. 
 
Chế Lan Viên không cãi. 
 
Nhưng Tố Hữu thì cãi. 
 
Ông bảo, Xuân Sách viết \textit{“máu ở chiến} \textit{trường hoa ở đây”} là Xuân Sách phịa (iời cãi này đã in trong một bài tường thuật đăng trên báo \textit{Văn nghệ}).  
  
Nhưng dù cố cãi thế nào thì Tố Hữu cũng không thể gạt ra khỏi đời thơ của mình mấy câu thơ này: 
\begin{blockquote}
        
\textit{Yêu biết mấy nghe con tập nói} 
\textit{Tiếng đầu lòng con gọi Xta-lin} 

\end{blockquote}
 
Theo nhận xét của riêng tôi, đây là những câu thơ vong bản nhất, cổ kim chưa từng có, trong thơ Việt. 
 
Hãy cứ tạm tin rằng khi đặt bút viết những câu thơ vừa dẫn trên, Tố Hữu chưa có thông tin gì về tội ác trời không dung đất không tha của Stalin đối với nhân dân Liên Xô, trí thức Liên Xô, và trước hết là đối với những đồng chí cộng sản thân thiết từng kề vai sát cánh cùng ông ta từ thuở nằm gai nếm mật, mặc dù các thông tin đó đã công bố không ít trên sách báo phương Tây. Và cũng tạm tin rằng tấm lòng kính yêu sùng bái của Tố Hữu đối với Stalin là thành thực. Nhưng dám vẽ ra cái cảnh con mình, một đứa bé Việt Nam cất tiếng đầu lòng không gọi “Mẹ” mà gọi “Xta-lin” thì thật là một sự bịa đặt gượng ép lố bịch quá quắt, đến nỗi tôi phải nghĩ rằng đó là của ai khác viết ra chứ không phải của tác giả “Nhớ đồng” (tên một bài thơ của Tố Hữu mà tôi rất mê trong tập \textit{Từ ấy} với những câu như \textit{“Gì sâu bằng những trưa thương nhớ/ Xao động} \textit{bên trong một tiếng hò”})\textit{. }Đâu rồi chàng trai yêu nước yêu dân yêu tự do trong \textit{Từ ấy}? Theo đà chín chắn hơn của tuổi tác, chất vong bản không nhạt đi mà vẫn y nguyên thế, nếu không nói là đậm hơn trong lời lẽ bóng bẩy hơn:\textit{“Mao Trạch Đông/ Bóng Người cao lồng lộng/ Ấm hơn một ngọn cờ hồng” }(Tố Hữu – “Đường sang nước bạn”). Lời tụng ca cất lên vào lúc tuy chưa có Cách mạng Văn hóa Vô sản nhưng đã có Cải cách Ruộng đất và Đại nhảy vọt mà Tố Hữu quá biết đã gây tai họa như thế nào. Ở đây, rung động thơ không còn thuần khiết cái trinh bạch của tâm hồn cá thể trữ tình nữa mà đã có sự chi phối của ý thức hệ giai cấp. (Thảm trạng ý thức hệ giai cấp lấn lướt và thôn tính ý thức hệ dân tộc diễn ra trong tâm thức đại đa số đảng viên, nhất là đảng viên trung cao cấp, đã đưa Đảng và toàn dân tộc sa vào một bi kịch khủng khiếp dai dẳng như thế nào và hiện vẫn còn là một vấn nạn lớn đã được một số nhà nghiên cứu có tư duy độc lập phân tích phê phán khá kỹ, xin tìm đọc Đào Phan, Hà Sĩ Phu, Lữ Phương, Lê Hồng Hà…). 
 
Những câu thơ nêu trên là cái dấu mốc cho thấy Tố Hữu đã tự vo tròn bóp méo mình, đánh mất mình, đã phản thơ.  
 
Lô-gích tất yếu của sự phản bội: Tố Hữu, một thi sĩ cách mạng hàng đầu, trở thành đao phủ thủ hàng đầu hạ độc thủ các đồng nghiệp \textit{Nhân văn}. 
 
\textit{“Người yêu người sống để yêu nhau”\textbf{ – }}Tố Hữu viết thế, có thật lòng không? Trong cái “yêu” ấy, có chỗ nào của cái “yêu” mà Tố Hữu đã ban cho anh em \textit{Nhân văn}? Phùng Quán cũng “yêu” – “\textit{Yêu ai cứ bảo là yêu/ ghét ai cứ bảo là ghét}.” Cũng một chữ “yêu”, ở hai nhà thơ cách mạng, một người nằm dưới lưỡi dao hành quyết của người kia, thì ai nói thật lòng? Chắc chắn chỉ là Phùng Quán – “\textit{dù ai cầm dao dọa giết/ cũng không} \textit{nói ghét} \textit{thành yêu”} - điều đó đã được định luận bằng cả cuộc đời “\textit{nhất quán tận can trường”} của ông.  
 
\textit{“Cái ghế quan trường giết chết thơ”}  – Xuân Sách viết thế, trong bài vẽ chân dung Chính Hữu. Nhưng đâu phải chỉ Chính Hữu. Nhìn kỹ lại, từ Tố Hữu đến các văn nghệ sĩ dưới quyền ông trong hệ thống phẩm trật quan trường hầu hết đều thế cả, nói cho chính xác thì không phải cái ghế nó giết mà cái lòng hám ghế nó thôi thúc đương sự tự giết mình, giết thơ. 
 
Tự giết mình đồng thời cũng không ngần ngại giết cả đồng nghiệp, tiêu biểu là Tố Hữu như nêu trên, và một người kế tục cũng khá tiêu biểu: Nguyễn Khoa Điềm, tác giả câu thơ nổi tiếng thời chiến tranh qua giai điệu của Trần Hoàn \textit{“mai sau con lớn làm người tự do”}, sang thời hậu chiến trở thành người “tự do” ném vào máy nghiền cuốn \textit{Chuyện kể năm 2000} của Bùi Ngọc Tấn mà nguyên trưởng Ban Văn hóa Văn nghệ Trung ương, cố nhà văn Trần Độ coi là \textit{“một tiểu thuyết hiện thực lớn”}. Bằng các cuộc giết ấy, Nguyễn Khoa Điềm leo nhanh lên những nấc ghế ngày càng cao, đến tận hàng ghế tối cao, ủy viên Bộ Chính trị, Trưởng Ban Tư tưởng Văn hóa Trung ương, nắm giữ guồng máy điều khiển gò siết trói buộc tư duy của toàn Đảng toàn dân.	 
 
Nếu là người thuộc loại chuyên chú dốc lòng theo nghiệp vua quan thì chắc chỉ có việc rung đùi trên ghế, nhưng ở những người đã trót tự nguyện mang lấy nghiệp thơ văn, thì cái sức mạnh thiêng liêng của thơ, của văn chương chữ nghĩa nó buộc phải đối mặt với vấn đề này: cái mà ta tự giết và giết đồng chí, đồng nghiệp để đổi lấy cái mà ta nghĩ rằng giá trị hơn \textit{(“cái ghế quan} \textit{trường”}), rốt cuộc có giá trị hơn thật không? 
 
Tố Hữu phải nói lại về anh em \textit{Nhân văn} những lời ngược hẳn trước kia (Nhật Hoa Khanh ghi\footnote{\url{http://www.talawas.org/talaDB/showFile.php?res=3214&rb=0305}}). Chế Lan Viết “Bánh vẽ”, “Trừ đi”, Nguyễn Đình Thi viết “Gió bay”, Nguyễn Khải viết “Đi tìm cái tôi đã mất\footnote{\url{http://www.talawas.org/talaDB/http://www.diendan.org/sang-tac/111i-tim-cai-toi-111a-mat}}”. 
 
Bài của Nguyễn Khải hiện đang gây chú ý cao độ đối với văn giới Việt Nam. Và không chỉ văn giới. Giáo sư Chu Hảo, giám đốc Nhà xuất bản Tri thức cho tôi biết, trong cuộc gặp ngày 19.5.2008 tại Hà Nội vừa qua giữa Ban Bí thư với một số trí thức tên tuổi, ông đã trao cho văn phòng Ban Bí thư bài “Đi tìm cái tôi đã mất” và đề nghị các ủy viên Bộ Chính trị, Ban Bí thư cần phải đọc trước khi ra nghị quyết về “Xây dựng đội ngũ trí thức…” 
 
Nhà thơ Dương Tường nhận xét\footnote{\url{http://www.talawas.org/talaDB/showFile.php?res=13373&rb=0102}}, trong một bài trả lời phỏng vấn: 
\textit{\textbf{	}} 
\textit{“Trong Khải, luôn có hai con người. Một Nguyễn Khải khôn khéo giả dối và một Nguyễn Khải thành thật trắng trợn. Một Nguyễn Khải hèn nhát và một Nguyễn Khải khinh ghét tay Nguyễn Khải hèn nhát kia. Và sự tranh chấp giữa hai con người ấy không bao giờ ngã ngũ.”} 
 
Nhà phê bình Vương Trí Nhàn\footnote{\url{http://www.talawas.org/talaDB/showFile.php?res=13428&rb=0102}}: 
 
\textit{“Gọi là ‘Đi tìm cái Tôi đã mất’ cho sang. Ở đây tác giả không định đi tìm cái gì cả... Thế tại sao Nguyễn Khải lại viết ‘Đi tìm cái Tôi đã mất’? Theo tôi, trường hợp này cũng giống như Chế Lan Viên viết ‘Di cảo thơ’, và Tố Hữu tâm sự với Nhật Hoa Khanh. Thực chất cái việc các ông ‘cố ý làm nhòe khuôn mặt của mình’ như thế này là cốt để xếp hàng cả hai cửa. Cửa cũ, các ông chẳng bao giờ từ. Còn nếu tình hình hình khác đi, có sự đánh giá khác đi, các ông đã có sẵn cục gạch của mình ở bên cửa mới (bạn đọc có sống ở Hà Nội thời bao cấp hẳn nhớ tâm trạng mỗi lần đi xếp hàng và không sao quên được những cục gạch mà có lần nào đó mình đã sử dụng).”  } 
\textit{\textbf{		}}	 
Giáo sư Nguyễn Huệ Chi\footnote{\url{http://www.talawas.org/talaDB/showFile.php?res=13460&rb=0102}}: 
 
\textit{“Bài viết của anh Vương Trí Nhàn\footnote{\url{http://www.talawas.org/talaDB/http://www.webwarper.net/ww/~av/www.talawas.org/talaDB/showFile.php?res=13428&rb=0102}} sắc sảo quá, nhưng như một số bạn bè trao đổi với nhau, cũng khí cay nghiệt quá.”} 
 
Vương Trí Nhàn có cay nghiệt quá không? 
 
Xin trình ra đây một tư liệu để chúng ta cùng tham chiếu: 
 
\textit{“Bỗng nhiên có một nhà làm chính trị, cũng là dân làm văn làm báo của Đảng từ trước cách mạng, nhưng đã mất ngôi, mất quyền, bèn đứng ra tổ chức một tờ báo cho những nghệ sĩ ham chuộng tự do được tự do bày tỏ nỗi niềm. Mình thì nói tự do về nghệ thuật, họ thì nói tự do về chính trị, họ muốn giành quyền, muốn đòi quyền, nhưng tự họ không thể làm được những chuyện đó, thân phận họ tầm thường, tài nghệ thì vớ vẩn, tập hợp thế nào được dư luận và công chúng, nhất là công chúng của chúng ta, mượn cả tiếng kêu thống thiết và cảm động đòi tự do để sáng tạo của chúng ta nữa. Nhà chính trị ấy là ông Nguyễn Hữu Đang, ông đó mới thật là linh hồn, kẻ xúi giục và tổ chức ra mọi sự của cái thời ấy, mưu mô bị vỡ lở, kẻ chủ mưu phải ngồi tù, mấy anh em mình không đi tù nhưng bị treo bút mất mấy chục năm còn đau đớn khổ cực hơn cả đi tù. Mấy ông chính trị thất thế, lắm tham vọng, lắm mưu mô, có đi tù tôi cũng không thương. Đã theo cái nghề ấy phải chịu cái nghiệp ấy, chỉ thương anh em mình lòng trong dạ thẳng, nông nổi thơ ngây, cứ nghĩ bụng dạ họ cũng như mình, nào ngờ họ lại nghĩ ngợi sâu xa đến thế.”} 
 
Đấy là tôi trích thư của nhà văn Nguyễn Khải, phó Tổng thư ký Hội Nhà văn Việt Nam viết ngày 01/9/1988 từ TP HCM gửi Hội nghị lần thứ 7 Ban chấp hành Hội. Thư này được bộ phận thường trực Hội chính thức sao gửi đến từng hội viên. Thời gian này đang chuẩn bị tiến tới Đại hội lần thứ tư của Hội. Những lời miệt thị tàn nhẫn mà Nguyễn Khải chĩa vào ông Nguyễn Hữu Đang thoạt tiên khiến tôi ngạc nhiên, nhưng bình tĩnh mà nghĩ mới thấy thực chất là nhằm vào ông Trần Độ, Trưởng Ban Văn hóa Văn nghệ. Chẳng lẽ, sâu xa trong lòng mình, Nguyễn Khải có thể nhẫn tâm đến thế với Trần Độ, một cán bộ lãnh đạo hiếm hoi mà Nguyễn Khải biết rất rõ là người luôn thật lòng yêu quý kính trọng văn nghệ sĩ và có công lớn qua việc chuẩn bị rất công phu để Bộ Chính trị cho ra được “Nghị quyết 05 về văn hóa văn nghệ” làm nức lòng toàn thể anh chị em văn nghệ? Nguyễn Khải cũng biết rất rõ Trần Độ là người đã khéo léo vận động để Tổng thư ký Nguyễn Đình Thi đề xuất đưa Nguyên Ngọc (bí thư Đảng đoàn Hội, bị thất sủng sau vụ “Đề dẫn”)\footnote{\url{http://www.talawas.org/talaDB/showFile.php?res=3144&rb=0102}} trở lại làm Tổng biên tập báo \textit{Văn nghệ}, và Nguyên Ngọc đã mau chóng đưa tờ báo tiến lên đứng ở tuyến đầu của cuộc chiến đấu cho đổi mới, đổi mới thật chứ không phải \textit{“giả vờ} \textit{đổi mới” }(chữ của Nguyễn Duy trong một bài thơ viết cũng trong thời gian ấy). Không, có lẽ đây chẳng qua là Nguyễn Khải dứt khoát vứt béng cái tôi của mình đi để tỏ rõ với cấp trên rằng trước kia dù mình có trót hăng hái đồng tình với Nguyên Ngọc và Trần Độ nhưng nay thì đã dứt khoát lập trường với hai người ấy. Với sự thính nhạy đặc biệt, Nguyễn Khải đã sớm thấy bên trên Trần Độ có người ngoài miệng hô “cởi trói” nhưng thâm tâm chỉ muốn đối với văn hóa văn nghệ thì mọi sự cứ phải trói chặt lại y như cũ. Và chính tại hội nghị ngày 11/9/1988, Ban chấp hành Hội Nhà văn đã ra nghị quyết giáng cho báo \textit{Văn nghệ} một chùy rất nặng, cũng tức là giáng cho Nguyên Ngọc và Trần Độ, dẫn đến việc Nguyên Ngọc, rồi Trần Độ phải rời khỏi chức vụ. (Nguyễn Khải khôn khéo tránh mặt không dự hội nghị, chỉ gửi bức thư mà tôi trích dẫn bên trên, nội dung thật là hiểm, tạo thêm nhân cốt cho quả chùy đang chuẩn bị vung lên.) 
 
Tố Hữu, Chế Lan Viên, Nguyễn Đình Thi, Nguyễn Khải đã ra người thiên cổ. Những gì ngòi bút các ông viết ra – nói như Gorki - rìu cũng không bổ được, còn nguyên đó. Các ông sống thế nào, mọi người dần dần rồi sẽ biết cả, kể cả những gì còn ẩn khuất. Những chữ nghĩa của các ông mà Xuân Sách lẩy thành \textit{thơ chân dung}, cũng còn nguyên đó. Hậu thế sẽ tiếp tục đọc và suy ngẫm, và chiêm nghiệm. 
 
Nguyễn Khoa Điềm thì hiện đang sống và viết ở Huế, sau khi thôi chức ủy viên Bộ Chính trị, Trưởng Ban Tư tưởng Văn hóa Trung ương. 
 
Sau cái đận Nguyễn Khoa Điềm nghiền sách của Bùi Ngọc Tấn, dễ đến mấy lần tôi nói với nhà thơ Xuân Sách: “Ông Sách ơi, thế này thì chân dung Nguyễn Khoa Điềm phải có phần 2 chứ?”. Ông Sách chỉ tủm tỉm cười: “Ờ…Ờ…”. Chờ mãi không thấy ông viết phần 2. 
 
Hóa ra tôi ngu quá. Đọc kỹ lại cái chân dung Nguyễn Khoa Điềm, tôi mới thấy tôi ngu quá. Mà ông Sách thâm thật. 
 
Xin dẫn ra đây cái chân dung ấy: 
\begin{blockquote}
        
\textit{Một mặt đường khát vọng}        
\textit{Cuộc chiến tranh đi qua}        
\textit{Rồi trở lại ngôi nhà}        
\textit{Đốt lên ngọn lửa ấm}       
\textit{Ngủ ngon a cay ơi} 
\textit{Ngủ ngon a cay à} 

\end{blockquote}
 
Và càng thấy ông Sách thâm khi đọc mấy câu nhại của Hà Sĩ Phu: 
\begin{blockquote}
        
\textit{Ngủ cho ngoan Mẹ ơi }        
\textit{Ngủ cho ngoan Mẹ hỡi!}        
\textit{Ngự trên lưng Mẹ}        
\textit{Con vô Thiên đường}        
\textit{Mẹ mà thức dậy}        
\textit{Con cho “lên… phường!”}        
\textit{....}        
        
\textit{Ai mà thức dậy} 
\textit{Vung chày là xong} 

\end{blockquote}
 
Ông Sách thâm. Và thơ rất thiêng. Văn chương chữ nghĩa rất thiêng. Cái thiêng bắt nguồn từ một đòi hỏi hết sức nghiệt ngã của cái nghiệp này: lúc nào cũng phải xuất phát từ tấm lòng thành tuyệt đối – một sự \textit{“chân thành không biết sợ”} (chữ của Stefan Zweig nói về Lev Tolstoy. Sống và viết, viết và sống phải trung thực, đơn giản có vậy thôi. \textit{“Chảy từ mạch máu ra là máu, từ cái vòi nước ra thì chỉ là nước lã”}, tưởng không thể không nhắc lại luôn cả lời ấy của Lỗ Tấn. 
 
Đã có nhiều bài bình luận phân tích về \textit{Thơ chân dung} của Xuân Sách. Nhưng tôi thích nhất mấy lời bình ngắn gọn sau đây của Hà Sĩ Phu: 
\begin{blockquote}
        
\textit{“Vị nghệ thuật nửa cuộc đời}        
\textit{Nửa đời sau lại vị người ngồi trên” \footnote{
Hai câu trong bài thơ của Xuân Sách vẽ chân dung Hoài Thanh.} }        
\textit{Nét này vẽ bác Lan Viên?}       
\textit{Bác Hữu?}        
\textit{Bác Cận?}        
\textit{Hay riêng bác Hoài?}        
\textit{Chân dung các bác ngời ngời}        
\textit{Chém cha riêng cái nửa đời phía sau}        
\textit{Một đời, hai nửa, vì đâu?} 
\textit{Nửa say quỷ kế, nửa đau nhân tình.} 

\end{blockquote}
 
Quả thật 
\begin{blockquote}
        
\textit{Thơ thiêng lắm, người ơi}        
\textit{Phản thơ thì phải chết}        
\textit{Chẳng ai giết mình mà mình tự giết} 
\textit{Treo nỗi nhục muôn đời.} 

\end{blockquote}
 
 
\textit{Đà Lạt 28.06.2008} 
 
© 2008 talawas 




\end{multicols}
\end{document}