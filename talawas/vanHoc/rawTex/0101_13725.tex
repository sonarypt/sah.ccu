\documentclass[../main.tex]{subfiles}

\begin{document}

\chapter{Nguyễn Du nghĩ gì về “thơ"? Thử tìm một lý giải}

\begin{metadata}

\begin{flushright}15.7.2008\end{flushright}

Thái Kim Lan



\end{metadata}

\begin{multicols}{2}

\begin{blockquote}

\textit{“Của tin, gọi một chút này làm ghi"}        
(Nguyễn Du, \textit{Đoạn trường tân thanh}) 

\end{blockquote}
 
Trước tác thơ Nôm \textit{Truyện Kiều} trong nội một đêm (theo truyền thuyết), Nguyễn Du kết thúc \textit{Đoạn trường tân thanh }bằng 6 chữ ngắn ngủi, - để nói theo cách hiện đại - "nhận định" về chính thơ ông:   
 
\textbf{“Lời quê chắp nhặt dông dài"…} 
 
Đem tất cả hồn và xác ký thác vào thơ đến nỗi sáng hôm sau tóc xanh trở nên bạc, thi nhân "phản tỉnh" - lại dùng một chữ khác có tính lý luận văn học -  về vai trò của thơ mình khi hạ bút chấm hết:  
 
"\textbf{Mua vui cũng được một vài trống canh.}" 
 
So với hơn ba nghìn lẻ (2/3254) câu thơ \textit{Truyện Kiều} mà mỗi chữ là mỗi động tâm đứt ruột, phê vào đó chữ "lời quê" và "mua vui", thì chẳng khác nào như thổi một sợi tơ phất qua núi Thái Sơn. Chúng nhạt nhất so với thi vị đậm quánh xúc cảm của những câu thơ đi trước. 
 
Tuy nhiên nếu đọc kỹ hơn và biết rằng thi hào của chúng ta trân trọng từng chữ "thơ" trong toàn bộ thi nghiệp của ông, cả thơ chữ Hán lẫn chữ Nôm, không cẩu thả trong lối dụng từ, đã ký thác mạng sống cho thơ "bạc mệnh hữu duyên lưu giản tịch" (mệnh bạc có duyên lưu lại nơi sách vở \footnote{
\textit{Điệp tử thư trung}, Tố Như thi, Quánh Tấn dịch, tr. 28, bản An Tiêm, Sài Gòn 1973, tr. 28, tham khảo thêm \textit{249 bài thơ chữ Hán Nguyễn Du}, Duy Phi, tr. 136, Hà Nội 2000} ), thì hai câu thơ cuối cùng cần được trân trọng không kém.  
 
Và đọc lại, bỗng quí: đó là hai câu thơ duy nhất đứng bên ngoài tác phẩm, chúng nhìn về thơ trong toàn diện tác phẩm. Trên bình diện phê phán hay lý luận văn học, có rất ít câu thơ trong tác phẩm của Nguyễn Du nói về thơ - điều ấy rõ - Nguyễn Du là nhà thơ, ông làm thơ, say đắm thơ, không mất giờ lý luận về thơ. Cho nên hai câu thơ cuối và đoạn thơ 1404 – 1466 nằm trong truyện Kiều là những suy nghiệm hiếm hoi mà Nguyễn Du bộc lộ về thơ. Truy nguyên ý nghĩa của những đoạn thơ này có thể khái lược quan niệm về thi ca làm nền tảng cho niềm tin và sức sáng tạo của nhà thơ.  
 
Tại sao thơ hay như thế, tiếng nào cũng não nùng xôn xao mà gọi là "lời quê"? Có phải đó chỉ là lời khiêm nhường cũ kỹ so với sự ồn ào đời nay? Có thật thơ Kiều là "quê"? Câu trả lời đã rõ là không. Vậy thì nói "quê" là "quê" kiểu gì? 
 
 
\textbf{I.} "\textbf{Lời quê chắp nhặt dông dài}" có thể có ba cách giải thích: 
 
\textbf{1.} Việt Nam đầu thế kỷ 19, Nho học vẫn còn đóng vai trò chủ đạo trên các lãnh vực tư tưởng, giáo dục, văn hoá. Việc sử dụng chữ Nôm trong các sáng tác văn học đang ở trong giai đoạn phát triển, nhưng thói quen trọng Hán khinh Nôm vẫn còn chế ngự tầm nhìn của mọi tầng lớp quần chúng. Câu "Nôm na là cha mách qué" phổ biến trong dân gian mãi đến tận thế kỷ sau. Đối với Hán ngữ sang trọng bác học, thì chữ Nôm là quê mùa. Nguyễn Du không thể không cảm thấy ngọn gió thời thượng thổi giạt qua hồn, nhưng sự thách đố sáng tạo trong ngôn ngữ mới đối với nhà thơ trở thành thúc giục đam mê. 
 
"Lời quê" là một thú nhận "quê", nhưng là một thú nhận ngẩng cao đầu đứng trên mảnh đất "quê" ấy: mảnh đất của thơ Nôm. Bằng chứng cho sự ngẩng cao là vẻ đẹp vô song của 3252 câu thơ đi trước. Vậy thì muốn được là "quê", nhà thơ trước hết phải là một kẻ sáng tạo đích thực trong ngôn từ. Nguyễn Du đã chứng tỏ được điều ấy. Dưới ngọn bút của ông, ngôn ngữ, Hán hay Nôm, và chính ngay với Nôm, đã hóa xác. Những gì là "quê" (hồn quê, nỗi quê, lòng quê) như củi, rơm, lá đã được nhà thơ biến thành chất lửa tinh ròng thơ Nôm, rực sáng trong \textit{Truyện Kiều}. Có thể lấy ngọn lửa của Nietzsche chiếu sáng tâm hồn thơ Nôm Nguyễn Du: 
\begin{blockquote}
        
\textit{Biết mình đến từ đâu}        
\textit{Khát khao như ngọn lửa}        
\textit{Bốc cháy và tự thiêu}        
\textit{Ta sờ, hừng ánh sáng}        
\textit{Ta đi, rụi tàn tro} 
\textit{Chính ta là lửa đỏ} 
         
(F. Nietzsche, Thơ)  

\end{blockquote}
 
\textbf{2.} Thú nhận thứ hai của "lời quê" là  sự lựa chọn khiêm tốn được là "quê", đối nghịch với hào nhoáng khoe khoang ồn ào, đánh bóng. Một sự nhún mình của người cầm bút, không dán bích chương, đẩy lùi cá nhân ra phía sau, nhường cho cảm nhận thực chứng "thơ" ra đàng trước, chỉ có "thơ" đến với người, đến với nhân gian. Đạo đức thi ca ở đây là sự tỏ lời, không mang chủ thể tự kiêu trên mình cho nên vẻ "quê" của "lời" gần với cái cúi đầu nhỏ nhẹ mà ngát hương của hoa tím.  
 
\textbf{3.} Nhưng có lẽ hai yếu tố trên chỉ là phản ứng về một quan điểm thời thượng của thi ca trong bối cảnh lịch sử của một nền văn học chữ Hán coi trọng tầm chương trích cú để tiến thân, lập công danh, mà Nguyễn Du chán chường. Chúng chưa phải là cơ sở căn bản của thi ca mà "lời quê" muốn bày tỏ. Chúng mới chỉ là sự phủ định từ bỏ mọi giả tạo bề ngoài. Tự thân của "lời quê" cần được hiểu trên bình diện chân mỹ của thi ca.  
 
Thơ, như Nietzsche phân tích, là hóa kiếp cái tất định (Notwendigkeit) đóng cứng của thế giới hiện tượng trở thành vẻ đẹp khả thể của chân lý chưa khai mở \footnote{
F. Nietzsche, \textit{Über Wahrheit und Lüge im außermoralischen Sinne}, 1873, và \textit{Götzen-Dämmerung}: \textit{Die }"\textit{scheinbar}"\textit{ Welt ist die einzige: }"\textit{die wahre Welt}"\textit{ ist nur hinzugelogen…}: Thế giới ảo là thế giới duy nhất, thế giới thật chỉ được dối trá thêm vào“, KSA 6 (\textit{Der Fall Wagner}; \textit{Götzen-Dämmerung}; \textit{Der Antichrist}…), tr. 75. 
\textit{Die }"\textit{wahre Welt}"\textit{ (...) war immer die scheinbare noch einmal.: }"\textit{Thế giới thật}"\textit{ luôn luôn đã là thế giới }"\textit{ảo}"\textit{ thêm một lần nữa. Đd  }KSA 1, tr. 24 
Die Wahrheit ist häßlich: wir haben die Kunst, damit wir nicht an der Wahrheit zu Grunde gehn: "Chân lý là xấu xí: chúng ta có nghệ thuật, để đừng bị khánh tận vì chân lý." Friedrich Nietzsche, Toàn tập, \textit{Sämtliche Werke}. Kritische Studienausgabe in 15 Einzelbänden,  KSA 13 (Nachgelassene Fragmente 1887-1889), tr. 500, 16[40]} .     
 
Sự bày tỏ, biểu lộ của thơ  là "quê" trong nghĩa hiện thực của nó là "\textbf{chân}"\textbf{,} mộc, đơn sơ, chân thành, \textbf{chân tình} của chân lý chưa khai mở ấy. Khóc là khóc, cười là cười, đau là đau. Đó là sự \textbf{trung thực}. Chính sự trung thực này là yếu tố nguyên thủy nhất  của thơ. Nó không phải là chân lý đóng khung, qui ước của đúng sai, mà là khả thể làm điều kiện cho một tác tạo mới như "đoạn trường tân thanh", tiếng kêu mới xảy ra như một chấn động ngôn ngữ quặn lên cùng nỗi đau xót chân tình.  
 
Cho nên trung thực là nền tảng của sáng tạo thi ca. Trung thực thuộc bình diện thể tính (Sein) chứ không thuộc bình diện chân lý (Wahrheit). Thi ca, theo Heidegger, "là một biến cố cội nguồn (Grundereignis) của thể tính (Sein, being) như là thể tính. Thi ca ban phát thể tính và phải ban phát nó, bởi vì trong ban phát thi ca không gì khác hơn là thể tính, thể tính ấy tự mang nó lại cho chính nó trong lời nói… Chỉ qua sự gọi tên của thi ca, hiện sinh mới được gọi tên, cái mà nó là". \footnote{
Heidegger, \textit{Toàn tập} (Gesamtausgabe), q. 39, Hölderlins Hymnen \textit{Germanien} und \textit{Der Rhein} (Wintersemester 1934/35), S. 257. 
Tham khảo thêm: Johannes Pfeiffer, Zu Heideggers Deutung der Dichtung, S. 60.}  
 
Lời đó là "quê", chân tình, là không sao chép rập mẫu từ chương, là lời quê "chắp nhặt" dưới ngọn bút của thi nhân. Chữ chắp nhặt ở đây đưọc hiểu như những tình cờ (zufällig), đối nghịch với đóng khung khuôn sáo (notwendige Konvention) tất định. "Chắp nhặt" chính vì thơ là sự đột biến, xảy ra trong từng giây từng phút, như viên ngọc hay viên sỏi rơi trong tâm, được lượm kết thành lời.    
 
Căn nhà thể tính – lời quê - không chứa gì khác hơn trong quá trình sáng tạo, xảy ra ấy, ngoài tấm lòng bộc bạch gọi tên chân phương, không dối trá của thi nhân. Cho nên "lời quê" là ngôn ngữ sáng tạo chân như, cội nguồn hiện hữu thi ca, từ đó tuôn trào, Nguyễn Du khiêm tốn gọi là "dông dài", "đoạn trường tân thanh". 
 
 
\textbf{II }"\textbf{Mua vui cũng được một vài trống canh}"\textbf{ hay khả thể sáng tạo tự do} 
 
Nếu "lời quê" Nguyễn Du xác định bản lai ngôn ngữ thi ca, trên phương diện thể tính, chúng ta có thể tìm hiểu "mua vui" trên bình diện "dụng" hay vai trò của thi ca. 
 
Có phải vai trò của thơ là để "mua vui" trong nghĩa cuộc vui chơi đổ một trận cười? Đọc câu "mua vui cũng được một vài trống canh, thoạt tiên người đọc ngẩn người ngạc nhiên, sau một trận đọc thơ long trời lở đất, đổ bao nhiêu nước mắt, "những điều trông thấy mà đau đớn lòng" như thế mà thi nhân bảo "mua vui cũng được" thì cũng lạ.  
 
Dường như nhà thơ muốn bỡn cợt với chính thơ ông? Có và không, có lẽ! Nguyễn Du bỡn cợt với thơ của ông, cũng chính vì thơ là huyết mạch, là tim gan, hồn vía của ông, bỡn cợt trên sự nghiêm trọng đến chết của chính mình. "Tàn hồn vô lệ khấp văn chương" \footnote{
\textit{Điệp tử thư trung}, đã dẫn, so sánh: \textit{Nguyễn Du toàn tập}, Mai Quốc Liên và nhiều người khác, tr. 183, nxb Văn học, Trung tâm Nghiên cứu Quốc học}  (hồn tàn không nước mắt khóc văn chương) 
 
"Mua vui" như thế cần phải hiểu trên một cấp bực khác hơn lãnh vực tương quan đối đãi buồn - vui, nước mắt đêm chầy và trận cười thâu đêm. Trên cấp bực ấy thú vui thường tình chỉ là thứ yếu. Trên thực tế, văn chương hầu như là món nợ mà thi nhân phải trả cho đời. Chống kiếm ngạo nghễ ngắm trời hay lăn lộn trong chốn bùn dơ thì trước sau "văn tự hà tằng vi ngã dụng" \footnote{
Khất thực, Nguyễn Du, Thơ chữ Hán Nguyễn Du, bản Lê Thước, Văn học, tr. 76} , chữ nghĩa văn chương nào đã dùng được việc  gì cho mình, chỉ có đói rét đến nỗi người phải mủi lòng thương... Vậy thì vui nỗi gì?  
 
Ấy thế mà thi nhân vẫn chọn con đường "Dâm thư do thắng vị hoa mang" \footnote{
\textit{Điệp tử thư trung}, đã dẫn} , yêu thơ, say đắm sách vở vẫn còn hơn đa mang vì hoa, đắm đuối trong phồn hoa, trong hoan lạc hay bi thảm đời thường. Cánh bướm thi nhân không lụy vì hương hoa trần tục, mà say hương sách vở, nhập hồn văn chương.  
 
Hình ảnh "bướm chết trong sách" (Điệp tử thư trung \footnote{
đd} ), một trong những bài thơ chữ Hán hay nhất của Nguyễn Du, có thể cho ta hình dung được chiều kích vượt đối đãi buồn vui thường tình của khái niệm "mua vui" theo Nguyễn Du. Cái duyên được chết trong sách - cũng là bạc mệnh đấy – là một cái chết hóa thân, khai phóng những chân trời mới cho năng lực tưởng tượng của cánh bướm sáng tạo. Cho nên chiều kích của „mua vui“ là tiếng đập cánh của con bướm bay vào khả thể tự do. Tự do sáng tạo, tự do tưởng tượng với cánh bay của ngôn từ trung thực, đó là niềm vui thi ca, diệu dụng của thể tính "nhà quê", để kết hợp "lời quê" với khái niệm căn nhà thể tính của Heidegger.  
 
Có thể nói không phải ngẫu nhiên mà "mua vui" đi liền với "cũng được" - ít thôi không nhiều, niềm vui lung linh giải thoát – trong  thoáng chốc, „một vài trống canh“, thời khắc mà mộng mị và tưởng tượng linh hoạt nhất là đêm, khi thế giới thường nhật trở nên ảo và mộng trở nên thật.   
 
Ở đó khả  thể tự do lên tiếng: thi nhân làm sống nàng Kiều, người đọc sống thế giới Nguyễn Du, không cần phải là thực như thời Minh, nhưng là thực tinh ròng trong xúc cảm sống động, “vui" trên phi lộ cảm thông, trong mộng mị giữa cõi trần "trần thế bách niên khai nhãn mộng" \footnote{
Nguyễn Du, \textit{La phù giang thủy các độc tọa}, tr. 91, Tố Như thi, Quánh Tấn dịch, bản An Tiêm} , ở đó tưởng tượng, khả thể của những tầng trời sáng tạo tự do bay lượn.    
 
Ngược lại với Platon đã xem thi ca là dối trá so với cuộc đời thực, cho nên không chấp nhận thi ca, Nietzsche cho rằng chính thi ca, ngay trong khi diễn tả ảo mộng hay ảo tưởng cuộc đời, lại có khả năng  đưa con người đến gần với chân lý toàn diện.  
 
Với một chút động tâm trước nghìn dặm sơn khê trong nỗi nhớ nhà, thi nhân có thể bứng rễ mọi tồn tại (Dasein) đã bị đóng chặt trong tất định (Notwendigkeit) khư khư bất dịch, ngay cả núi sông vật thể, chuyên chở trên mình mọi ảo tưởng dối trá mà dấn thấn vào khả thể sáng tạo. Chính trong quá trình ấy cơ hội đến gần thực tại tuyệt đối nẩy mầm. Bởi vì theo Nietzsche, chân lý luôn luôn ở trong sự trở thành, có nghĩa trong sáng tạo.  
 
Trong ý nghĩa ấy, có thể Nguyễn Du là thi nhân tuyệt hảo mà Nieztsche mong đợi dù chưa một lần tri giao: 
\begin{blockquote}
        
"\textit{Tri giao quái ngã sầu đa mộng}        
\textit{Thiên hạ hà nhân bất mộng trung}" \footnote{
Nguyễn Du, Ngẫu đề, tr. 131, \textit{Tố Như thi}, bản An Tiêm}  
        
(Bạn bè quen biết thường lấy làm lạ sao ta hay sầu mộng,        
Thiên hạ ai là người không ở trong mộng?) 

\end{blockquote}
 
Trong câu hỏi vặn lại ấy đã thấy hửng một nụ cười "tỉnh mộng" của thi nhân. Nhưng đừng mơ, đừng lầm ta tỉnh mộng, bởi vì chiều kích "tỉnh mộng" ấy  là chân trời của cuộc lang thang vô hạn, trong cơn mộng triền miên, mỗi giật mình "tỉnh mộng" là mỗi hạt lưu ly trong cơn triền miên ấy:  
\begin{blockquote}
        
\textit{Hành cước vô căn nhiệm chuyển bồng}        
\textit{Giang nam giang bắc nhất nang không}        
\textit{Bách niên cùng tử văn chương lý}        
\textit{Lực xích phù sinh thiên điạ trung}        
\textit{Vạn lý hoàng quan tương mộ ảnh}        
\textit{Nhất đầu bạch phát tấn tây phong}        
\textit{Vô cùng kim cổ thương tâm xứ}        
\textit{Y cựu thanh sơn tịch chiếu hồng \footnote{
Mạn hứng II, \textit{Thơ chữ Hán Nguyễn Du}, bản Lê Thước, tr.  64, 66} } 

\end{blockquote}
 
Chân không bén rễ, hành trang là túi không, trời đất một kiếp bồng bềnh, trong nỗi thương tâm vô hạn. Đối với Nguyễn Du, "cùng tử với văn chương" (chết tiệt với văn chương) trở nên khả thể mang ý nghĩa của cuộc đời, bởi vì con người trong trời đất không có gì ngoài văn chương, ngoài thơ. Cho nên thơ là viễn tượng cứu rỗi, giải thoát của thi nhân. Thơ có khả năng đưa con người vượt lên trên gió bạc đầu và muôn dặm công danh, vượt trên sầu vạn kiếp bởi vì thơ tự do lên tiếng. 
 
Và niềm vui như viễn tượng của tự do – trong chính mâu thuẫn của khái niệm tự do: chính là kiện tính "y cựu" của giọt nắng hồng chiếu trên núi xanh gây nỗi nhớ quê vô bờ. Tự do chính là sự trở về với một chút bền bĩ hơn tự do, hơn phù du: hồn quê… trong mộng thực của thi nhân.  
\begin{blockquote}
        
\textit{Trần thế bách niên khai nhãn mộng}        
\textit{Hồng sơn thiên lý ỷ lan tâm \footnote{
La phù giang thủy các độc tọa, tr. 91, \textit{Tố Như thi}, bản An Tiêm, Sài Gòn 1973} } 
        
(Trăm năm trần thế chỉ là giấc mơ mở mắt,        
Nghìn dặm Hồng sơn, luống chạnh lòng khi tựa lan can) 

\end{blockquote}
 
Trong "mua vui cũng được…" - hai chữ "cũng được" làm dấu trừ, phủ định hai chữ "mua vui" đi trước – cho nên không là mua vui, trong nghĩa tuyệt đối của thi ca: chối bỏ mọi ý nghĩa thực dụng mới là vui thơ. 
 
Nhưng thơ còn một nghĩa "ban phát" thực dụng, mà Nietzsche gọi là "tính thực dụng mê tín" (abergläubige Nützlichkeit) của quan niệm cổ điển Tây phương khá lỗi thời, trong lúc Nguyễn Du tin vào khả năng thực sự của nó trong cuộc đời: thơ có thể làm "trắng án", "xóa tội" như trường hợp của nàng Kiều: 
 
 
\textbf{III. Thi ca làm }"\textbf{trắng án}"\textbf{… gần bằng Thượng đế } 
 
Quan niệm này không mới trong lịch sử văn chương Đông Tây, nhưng Nguyễn Du đã làm sống lại trong đoạn thơ kể chuyện Kiều bị Thúc ông kiện trước cửa quan về tội đã quyến rũ Thúc Sinh. Đoạn thơ này cho ta có thể thấy được quan niệm tiêu biểu cổ điển về vai trò của thơ: 
 
Quan cho trát đòi Kiều và Thúc sinh lên phân xử. Cảnh kiện cáo thật sinh động dưới ngòi bút Nguyễn Du: 
\begin{blockquote}
        
\textit{Đất bằng nổi sóng đùng đùng,}        
\textit{Phủ đường sai lá phiếu hồng thôi tra} 

\end{blockquote}
 
Thật là hãi hùng, dễ sợ, nhất là cái ông quan "mặt sắt đen sì" với giọng oai nghiêm như lệnh vỡ phán: 
\begin{blockquote}
        
\textit{Suy trong tình trạng nguyên đơn,} 
\textit{Bề nào thì cũng chưa xong bề nào} 

\end{blockquote}
 
Thế thì chỉ còn một nước bị gia hình, oan mà không kêu được, "ba cây chập lại" làm cho "đào hoen quẹn má, liễu tan tác mày". 
 
Trong cái cảnh thảm khốc ấy, bỗng tình cờ vị quan nghe nói Kiều biết làm thơ. Như có một phép mầu, "mặt sắt đen sì" bỗng tươi lên, ông ra đề thơ "cái gông", Kiều làm một bài thơ. Ông đọc và ông cười… tha bổng! Trắng án!    
\begin{blockquote}
        
\textit{Cười rằng: "Đã thế thì nên}        
\textit{Mộc già hãy thử một thiên trình nghề"}        
\textit{Nàng vâng cất bút tay đề,}        
\textit{Tiên hoa trình trước án phê, xem tường.}        
\textit{Khen rằng: "Giá đáng Thịnh Đường} 
\textit{Tài này sắc ấy nghìn vàng chưa cân"} 

\end{blockquote}
 
Khi không, vâng, khi không, đang là hai kẻ tội đồ dưới mắt quan toà, hai tên không đầu đường xó chợ thì cũng mèo mả gà đồng, hư thân mất nết, nhờ một bài thơ bỗng thành: 
\begin{blockquote}
        
\textit{Thực là tài tử giai nhân} 
\textit{Châu Trần còn có Châu Trần nào hơn!} 

\end{blockquote}
 
Và ông quan nói như bị đồng ốp, vì ông quan này đã thoát xác, ra khỏi ông quan kia, từ lý đổi sang tình, giận dữ hờn oán tiêu ma: 
\begin{blockquote}
        
\textit{Thôi đừng rước dữ, cưu hờn,}        
\textit{Làm chi lỡ nhịp cho đàn ngang cung.}        
\textit{Đã đưa đến trước cửa công} 
\textit{Ngoài thì là lý song trong là tình.} 

\end{blockquote}
 
Bài thơ như một phép lạ, nó có khả năng biến đổi cõi ta bà hơn thiệt từ dữ sang lành, nhất là từ có tội sang vô tội. Nó có thể làm cho một người ghen tuông tột độ là Hoạn Thư thốt ra một câu cảm thông "Rằng tài nên trọng mà tình nên thương" khi đọc thơ Kiều. 
 
Quan điểm "lấy thơ chuộc lỗi" không hiếm trong lịch sử văn chương Việt Nam. Truyền tụng nhất là chuyện Lê Quí Đôn thuở bé nghịch ngợm bị phạt quì, nhưng lại được tha vì đã ứng khẩu làm bài thơ "Rắn đầu biếng học" nổi tiếng. Không nói chi danh nhân, thường dân thế hệ trước cũng lấy thơ ra để "làm hòa" như là nếp nhà, bà nội tôi hay giảng hòa những cãi vã giữa chú bác bằng mấy câu thơ ngâm lên lơ lửng giữa không gian, đổi giận làm vui.   
 
Cái gì trong thơ đã có phép biến hóa, tẩy sạch nghiệp trần tội lỗi? 
 
\textbf{1.} Từ một kẻ đang bị trừng phạt, là nạn nhân dưới trận đòn phủ phàng, Kiều, ngay ở thời điểm "cất bút tay đề" đã thôi không là nạn - mà trở nên chủ nhân sáng tạo, là kẻ tác tạo. Sự tẩy tội xảy ra trong chính giây phút làm thơ, năng lực thơ đã đưa thể cách con người Kiều vào trong một tầng phẩm chất khác với thế giới ngục tù. Ở đó, gông cùm xiềng xích, đề tài bài thơ phải làm, được cởi bỏ dù nó có xuất hiện thực trong thơ thuyết phục tuyệt đối đi nữa, nó trở nên sản phẩm thơ, hết là thực tại tàn nhẫn trong chính lúc ấy. 
 
\textbf{2.} Mặt khác, sức thuyết phục của bài thơ nằm ở phía người đọc. Ông quan khi bắt đầu đọc bài thơ, cũng đã trở nên một con người khác: không còn là con người của ý niệm công lý mà là người thưởng thức thơ. Hết ràng buộc hơn thua, so đo cán cân công lý, người thưởng thức thơ thong dong trong thế giới của tưởng tượng, ở đó chữ nghĩa không nặng nề xiềng xích mà ngược lại có thể cởi trói mọi định kiến bao vây. Trong tư thế "thưởng thức", ông quan dần dà thay đổi tư duy, mở rộng lòng để cảm thông. Hơn cả cảm thông, với tâm trạng của người thưởng thức rong chơi, ông trực nhận được chân lý không chỉ nằm trong đúng sai, buộc tội, mà phải nhìn thẳng vào tự tính con người: trong bản chất, \textbf{một kẻ có khả năng làm thơ không thể hành động gian ác, người làm thơ là kẻ nuôi dưỡng mầm thiện trong tâm.} 
 
(Ngược với quan niệm hiện nay, một người có thể làm nghìn bài thơ, nhưng cũng có thể giết người hàng loạt sau hay trước đó.) 
 
\textbf{3.} Tuy nhiên, nhà thơ sẽ không phải là Nguyễn Du nếu chỉ chuyển tải một quan điểm truyền thống về vai trò của thơ xuyên qua \textit{Đoạn trường tân thanh}. Đối với Nguyễn Du, thơ không cậy quyền thế, ngược lại quyền thế phải phục tòng thơ, được hoán cải bởi thơ. Không phải Kiều được tha tội mà chính ông quan được nữ thi nhân Thúy Kiều xá tội nhờ khả năng biết thưởng thức thơ của ông.  
 
Nàng Kiều trong bản chất là vô tội, không có gì để xử cả. Chính quan tòa mới là kẻ được giác ngộ bởi thơ. Trên bình diện tương thông, thơ đã hóa giải được tương quan bề trên - kẻ dưới, buộc tội và bị buộc tội, tương quan gay gắt giữa ta và người biến thành tương quan hòa khí của "chúng ta". 
 
Miêu tả trong đoạn thơ xử án quả thật có pha lẫn một chút chế diễu bi hài của một vở kịch trên sân khấu. Đọc ông quan "mặt sắt đen sì" là thấy được nét mỉm cười của nhà thơ khi chấm phá cá tính của người cầm cán cân công lý thời ấy (và cả thời nay): mặt thì giống Bao Công rồi đấy, thị oai thì… cũng gần bằng, rỗn rảng nặng lời e hơn, nhưng từ trường hợp Kiều mà suy diễn, thì ông quan này, cũng như những ông quan khác từ cổ chí kim, suốt đời tưởng mình là quang minh mà thật ra thường xuyên lầm lạc trong ngộ nhận phán xét, chỉ dựa vào quyền uy roi vọt. Nhưng ông quan này đã thuộc vào loại khá, vì biết thưởng thức thơ. 
 
Với một chút hóm hĩnh, Nguyễn Du tả sự hoán cải vị quan ngay ở lời khuyên của ông liên hệ đến nhạc thơ chui lọt vào tai, bỗng dưng vị quan thấy trong đầu rung chuông nhịp điệu: 
 
"Làm chi lỡ nhịp cho đàn ngang cung": thơ có khả năng sửa nhịp hòa lại cung đàn ngay chính trong hai lỗ tai thường chỉ nghe tiếng mình quát tháo thiên hạ của vị quan ấy. 
 
Dưới con mắt nhà thơ, cuộc xử án dựa trên một sự hiểu lầm ngay từ đầu. Một oan khốc cho Kiều, bỗng dưng vì yêu mà bị trừng phạt, vô lý chưa? Cho nên thơ giải oan, thơ có sức mạnh cải thiện –   thiện ý đã đánh tan ngộ nhận của vị quan, đã giải thể mọi nghiệp chướng, trả lại sự vô tội vốn là nguồn gốc hiện sinh nguyên thủy. Hiểu lầm, ngộ nhận đến từ thân khẩu ý, trong đó ngôn từ đóng vai chính, ngôn từ trung thực (thơ) hóa giải ngôn từ, thanh lọc ngộ nhận, đó là diệu dụng của thơ: 
\begin{blockquote}
        
\textit{Vị hữu văn chương sinh nghiệp chướng}        
\textit{Bất dung trần cấu tạp thanh hư \footnote{
Ngoạ bệnh, \textit{Thơ chữ Hán Nguyễn Du}, bản Lê Thước, nxb Văn học,  tr. 135} } 
        
(Chưa bao giờ có văn chương gây ra nghiệp chướng        
Không để cho bụi bặm lẫn vào nơi trong sạch) 

\end{blockquote}
 
\textbf{4.} Diệu dụng ấy nhà thơ trao cho một sinh linh yếu đuối nhất, đang lầm lạc "vì mấy đường tơ", bị đày đoạ tận cùng thân phận nhưng thiện căn còn hơn những "ông quan". Chính các quan, không những chỉ thời xưa mà đời nay trong mắt Nguyễn Du là những kẻ cần được thanh lọc bằng thơ.  
 
Sự thi hóa quan niệm minh triết về thơ trong đoạn "thơ trong thơ" dưới dạng thức liên hoàn đến vô cực chứng tỏ sức mạnh cảm hứng của nhà thơ: 
 
\textbf{Nguyễn Du làm thơ Kiều làm thơ… đến }"\textbf{cùng tử văn chương}"\textbf{ để hiện thực một nàng thơ - đó là mỹ nhân Kiều - đang làm thơ – và mỹ nhân làm thơ cũng đồng nghĩa với thi ca. }Bức họa thi ca mà Nguyễn Du vẽ ra trong khoảnh khắc ấy là một nàng Kiều không lộng lâỹ áo xiêm, kiêu sa son phấn, mà “đào hoen quẹn má, liễu tan tác mày”, nàng thơ, mỹ nhân “mai gầy vóc sương” chính là hóa thân vẻ đẹp cùng cực trần ai và siêu việt trần ai. 
 
Trùng điệp thi hứng như thế để minh bạch một điều: trong tất cả những dối trá đa chiều, thơ là năng lực có thể đưa con người gần sự thật tuyệt đối, thơ tác tạo cái tuyệt đối: 
  
"\textbf{Không thơ con người là hư vô, với thơ con người hầu như là thượng đế}". \footnote{
"Ohne den Vers war man nicht, mit dem Vers wurde man beinahe ein Gott", F. Nietzsche, \textit{Die Fröhliche Wissenschaft }(La gaya scienza) 84: Về nguồn gốc của thơ, đd.}  
 
Thi ca là sự hoàn thiện tuyệt đối của thần linh. 
 
Nietzsche nhận xét như thế khi phân tích nguồn gốc của thi ca theo quan điểm cổ đại, hầu như tương đồng với tư tưởng Đông phương: 
 
"Người ta mong rằng nhờ vào nhịp điệu (thơ) sự cầu xin của con người được những vị thần linh chiếu cố, sau khi nhận ra rằng con người dễ nhớ câu thơ hơn là lời nói rời rạc… Nhịp điệu trong thơ nhạc có sức cuốn hút đôi chân và cả linh hồn… ngay cả linh hồn của thần thánh… Như thế con người tìm cách ép buộc các vị thần bằng nhịp điệu và gây áp lực lên họ: người ta tròng cổ họ bằng thi ca như một sợi dây thòng lọng ảo thuật” \footnote{
Các bài thơ chiêu hồn thuộc lại xuất quỉ nhập thần của Nguyễn Du như "Văn tế thập loại chúng sinh" (thơ Nôm), "Văn tế trường lưu nhị nữ" (thơ Nôm), "Phản chiêu hồn" (thơ chữ Hán) đều nằm trong ý nghĩa này. Mong có dịp sẽ đi sâu hơn.} … Với nhịp điệu "người ta có thể làm tất cả: phù phép làm nên một công việc, bắt một vị Thượng đế xuất hiện ở bên cạnh hay lắng tai nghe; tác thành tương lai đúng theo ý muốn, giải tỏa linh hồn của mình ra khỏi bất cứ một quá tải nào đó (sợ hãi, cuồng bệnh, thương tâm, căm thù), và không những linh hồn riêng mà cả linh hồn của quỉ sứ độc ác nhất…." \footnote{
F. Nietzsche, Vom Ursprung der Poesie, trong \textit{Die Fröhliche Wisenschaft} (La gaya scienza), § 84}  
 
 
\textbf{IV. }"\textbf{Văn chương tàn tích nhược như ti}"\textbf{ và người trong cuộc }"\textbf{phong vận kỳ oan}" 
 
Tin vào tính thiện trong thơ, nhưng Nguyễn Du không mê tín đồng hóa thơ với thần linh, có linh thiêng chăng thì đó là chữ "tâm" (linh sơn chỉ tại nhữ tâm đầu \footnote{
Nguyễn Du, Lương Chiêu Minh Thái Tử phân kinh Phật đài, Nguyễn Du, \textit{Thơ chữ Hán}, q. 2, tr. 233  
Chi Điển Hoàng Duy Từ, Hoa Kỳ 1986} ) luôn luôn hiện diện đầy nhân ái, chân thực, rung động dạt dào của một kẻ tự nguyện là \textbf{người trong cuộc}. Chính vì văn chương không dấy nghiệp mà vẫn lụy với đời, vì đời. Không thể chờ được thương đến sau khi chết mà cũng không thể hỏi trời cho ra lẽ, thi nhân là kẻ đến với con người trước nhất, tự coi mình là người trong cuộc với sinh linh yếu đuối nhất, oan khiên nhất: 
\begin{blockquote}
        
Chỉ phấn hữu thần liên tử hậu        
Văn chương vô mệnh lụy phần dư        
Cổ kim hận sự thiên nan vấn        
Phong vận kỳ oan ngã tự cưu \footnote{
Nguyễn Du, Độc Tiểu Thanh ký, tr. 105, \textit{Tố Như thi}, bản An Tiêm}  
        
(Son phấn như có thần nên sau khi chết làm cho thương xót \footnote{
Cách dịch của nhóm Bùi Kỷ theo Trần Đình Sử, tham luận "Độc Tiểu Thanh ký của Nguyễn Du" đăng trong \textit{Nguyễn Du - Về tác phẩm và tác giả}, Trịnh Bá Dĩnh-Nguyễn Hữu Sơn-Vũ Thanh, nxb Giáo Dục 1996, tr. 82}         
Văn chương không có số mệnh mà bị hoạ đốt dở        
Những mối hận xưa nay khó ai mà hỏi trời đưọc        
Nỗi oan lạ lùng của kiếp phong nhã, ta tự coi như người trong cuộc…) 

\end{blockquote}
 
Làm người trong cuộc đối với Nguyễn Du có nghĩa thấm đượm, trải nghiệm hiện hữu giới hạn của con người trong trời đất vô cùng bằng hơi thở văn chương: 
\begin{blockquote}
        
\textit{Thiên địa biên chu phù tự điệp}        
\textit{Văn chương tàn tức nhược như ti \footnote{
Nguyễn Du, Chu hành tức sự, \textit{Thơ chữ Hán Nguyễn Du}, bản Lê Thước, tr. 266, Văn học} } 
        
(Giữa khoảng trời đất, chiếc thuyền con trôi nổi như chiếc lá        
Hơi tàn văn chương mảnh như sợi tơ) 

\end{blockquote}
 
Thi tứ mênh mông như bao trùm cả vũ trụ thoát ra từ hình ảnh "phù tự điệp" và "nhược như ti". Những mộng huyển bào ảnh của kiếp phù sinh "chiếc bách sóng đào", "mặt nước cánh bèo", "lênh đênh đâu nữa cũng là lênh đênh" (Kiều) được thu gọn lại trong hình ảnh chiếc thuyền bồng bềnh như chiếc lá trong thiên điạ bao la. Thi ca là tiếng vọng "trong cuộc" của nỗi bơ vơ ấy, trung thực đến trở thành một với kiếp phù sinh "nhược như ti", mỏng manh như sợi tơ mà xuyên suốt thời gian và không gian, cùng khắp nhân sinh. Tiếng vọng ấy không ồn ào, huyênh hoang, cường điệu mà nhỏ như tơ, nhỏ đến mức không thể nhỏ hơn, nhưng từ nó, từ mạch tơ ấy sống dậy hiện hữu con người. Như Heidegger đã nhận định, chính thi ca ban phát hiện sinh, gọi tên hiện sinh trong sự trung thực của nó. 
 
Mặt khác liễu tri hay chứng ngộ sâu thẳm nhất của Nguyễn Du về bản chất thi ca với câu "văn chương tàn tức nhược như ti" còn cho ta nhận rõ cảm nghiệm tự thân của thi nhân. Nếu thi ca là sự ban phát thể tính, làm nên hiện hữu sống thực, thì thi nhân, người ban phát thi ca chỉ là "tàn tức", nhỏ hơn một hơi thở sinh khí, một tăm hơi, mỗi giây phút có thể mất tăm, chìm lỉm trong mênh mang kim cổ. Trực giác về tính phù du của tự ngã khởi nguồn cho rung động sáng tạo. Thi nhân là kẻ trực nhận tính vô ngã ấy, đã từng "tắc hơi" (tàn tức \footnote{
Trong một số bài viết đăng trên các mạng (Vĩnh Phúc, Cái chi chi… là thơ, phần 1, Hội Luận Văn Học, Lê Văn Như  Ý, Văn chương vô dụng, Văn Nghệ sông Cửu Long.org) câu thơ "văn chương tàn tức nhược như ti" đã được các tác giả trích dẫn nhầm: "Văn chương tàn tích nhược như ti". Xem lại tất cả các bản mà tôi được sử dụng, thì thấy chính văn như đã dẫn ở trên, "tàn tức" chứ không phải "tàn tích". 
Bài thơ "Chu hành tức sự" trong \textit{Bắc hành tạp lục} của Nguyễn Du. Xin xem: \textit{Thơ chữ Hán Nguyễn Du}, Lê Thước, Trương Chính v.v. soạn dịch, nxb Văn Học 1978, tr. Tr. 265/266, \textit{Nguyễn Du toàn tập}, q. 1, Mai Quốc Liên soạn, nxb Văn Hoá, Trung tâm nghiên cứu quốc học, tr. 313, \textit{192 bài thơ chữ Hán của Nguyễn Du}, Bùi Hạnh Cẩn, nxb Văn Hóa –Thông Tin, Hà Nội 1996, tr. 26, \textit{249 bài thơ chữ Hán Nguyễn Du}, nxb Văn Hóa Dân Tộc, Duy Phi biên soạn-dịch thơ, Hà Nội 2000, tr. 240, \textit{Nguyễn Du –Thơ chữ  Hán}, Chi Điền Hoàng Duy Từ, Hoa Kỳ 1986, q. 2, tr. 51. Tôi xin cám ơn bác sĩ Trần Văn Tích đã có nhã ý cho tôi mượn những quyển sách nói trên để tham khảo.} ) trong biến động trở thành. Hay nói theo Thiền học, mỗi giác ngộ thể tính đều bao hàm sự chết của ngã tất định làm điều kiện cho giải phóng tự do, bay vào hiện hữu (Sein). Bởi vì nói như kinh Kim Cương, sự ban phát của tâm – mà đó là thi ca - chỉ thật sự toàn hảo khi người ban phát không khởi lên ý niệm "tôi ban phát", ban phát trong giờ phút ấy mới thật sự là biến động của thể tính.  
 
Sáng tạo thi ca nằm trong khoảnh khắc chết - sống của trở thành, tái sinh, trong đó thi nhân vắng mặt, dám chết như con tằm – đã bao lần "đến thác" - vẫn nằm trong kén, cựa mình nhả tơ.    
 
Ý nghĩa của "lời quê" chính là tiếng tơ ấy, 300 năm hay 300 năm nữa, mai sau "dẫu lìa ngó ý còn vương tơ lòng". 
 
\textit{München, Vu Lan sớm 2008} 
 
© 2008 talawas



\end{multicols}
\end{document}