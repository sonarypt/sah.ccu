\documentclass[../main.tex]{subfiles}

\begin{document}

\chapter{Ðọc một bài thơ như thế nào}

\begin{metadata}

\begin{flushright}23.7.2008\end{flushright}

Nguyễn Đức Tùng



\end{metadata}

\begin{multicols}{2}

\textbf{Bài ba: Bài thơ và tác giả }

Một nhà thơ Canada kể rằng hồi còn học trung học, ông có viết một bài thơ khóc mẹ rất hay, được giải thưởng. Hôm phát giải, khi được mời lên đọc bài thơ, ông đã trân trọng giới thiệu mẹ của ông có mặt trong hàng ghế cử tọa danh dự. Ai cũng kinh ngạc, vài người đứng lên chất vấn, một số khác tức giận bỏ ra về, có lẽ vì thấy mình bị chơi khăm. 

Câu hỏi ở đây là: một nhà thơ còn mẹ thì có thể viết một bài thơ khóc mẹ được không? 

Tôi đi xa hơn nữa: nếu trong đời thực một người không hề yêu mẹ anh ta, đối xử với bà rất tệ hại, mà viết một bài thơ thương mẹ thì có giả dối không? Nếu một người không có tình yêu nước, thậm chí “phản bội tổ quốc”, mà viết một bài thơ yêu nước, thì có được không? 

Lại đi xa hơn nữa: thế nào là sự giả dối trong văn chương? 

Thắc mắc của Phan Nhiên Hạo\footnote{\url{http://www.talawas.org/talaDB/showFile.php?res=12536&rb=12}} và những người khác, gián tiếp hơn, như Nguyễn Đăng Thường\footnote{\url{http://www.talawas.org/talaDB/showFile.php?res=12633&rb=12}}, trong mục ý kiến ngắn trên talawas sau khi đọc chùm thơ của tôi ngày 9.3.2008, tuy không đặt ra những câu hỏi rõ ràng và trực tiếp như thế, nhưng nếu ta theo đuổi đến cùng, chúng sẽ dẫn đến những câu hỏi tế nhị hơn, tổng quát hơn, về mối quan hệ giữa tác giả và bài thơ, giữa bài thơ và nhân vật, giữa ý định của người viết và ý nghĩa của tác phẩm. Trong bài này, tôi xin phân tích về mối quan hệ giữa tác giả và bài thơ và sẽ đề cập đến các vấn đề khác ở những bài sau. 

Trong cuốn \textit{Chân dung và đối thoại} của Trần Đăng Khoa, có kể một câu chuyện, nếu tôi nhớ không lầm, về việc Tố Hữu đã từng làm thơ ca ngợi chiến thắng Điện Biên Phủ trong khi thật ra ở ngoài đời, vào lúc đó, ông ta chưa hề đặt chân đến Điện Biên bao giờ. Tôi đoán rằng sự tiết lộ này sẽ làm cho nhiều người ngạc nhiên, một số sẽ có cảm giác như mình bị lừa, và một số khác, mến mộ Tố Hữu hơn, sẽ cảm thấy bối rối, không biết giải thích như thế nào, đành lờ đi. 
\begin{blockquote}


\textit{Chín năm làm một Điện Biên} 
\textit{Nên vành hoa đỏ nên thiên sử vàng} 

\end{blockquote}


Hai câu thơ này trở nên có vẻ trang điểm (decoration). Nếu bạn vốn thích thơ của nhà thơ nổi tiếng, tiếng tốt và tiếng xấu, thì sau khi đã biết câu chuyện kể trên của Trần Đăng Khoa, sự thích thú của bạn có giảm đi phần nào chăng? 

Chỉ có bạn mới trả lời được cho chính mình. Có những câu thơ mà ấn tượng về sự thành thật là cần thiết cho việc gợi lên những đồng cảm ở người đọc, như hai câu trên đây, nhưng cũng có trường hợp mà sự cần thiết này chỉ ở mức tối thiểu: 
\begin{blockquote}


\textit{Là sóng đó rồi tan thành bọt đó} 
\textit{Đổ qua có sắc màu mà đổ lại hóa hư không} 
(Chế Lan Viên) 

\end{blockquote}


Một dân tộc mệt mỏi vì binh đao thường chán những câu thơ hào nhoáng. Họ lại lắng nghe lời thì thầm thủ thỉ: 
\begin{blockquote}


\textit{Xuân này em có về không} 
\textit{Cành mai cố quận nở bông dịu dàng} 
(Bùi Giáng) 

\end{blockquote}


Các nhạc sĩ cũng chia sẻ số phận với các nhà thơ. Tôi nhận xét rằng sau những tuyên bố nào đó của nhạc sĩ Phạm Duy khi ông về Việt Nam, lòng mến mộ của một số người nghe ở hải ngoại, ít nhất là xung quanh tôi, dành cho ông trước đây đã thuyên giảm nhiều. Mới đây, gặp lại một người bạn cũ ở Mỹ, thường hát nhạc Phạm Duy, khi tôi yêu cầu anh hát một bài của ông thì anh ấy từ chối và chọn một tác giả khác. Trái lại với trường hợp Văn Cao hay Trịnh Công Sơn. Những người yêu mến họ không hề giảm đi. Chúng ta cần phân biệt rằng giá trị của một tác phẩm văn học nghệ thuật, trong khi vẫn phải lệ thuộc vào các hệ quy chiếu thẩm mĩ, thì vẫn có tính lâu dài hơn và khách quan hơn nhiều so với sự mến mộ của người thưởng ngoạn, vốn ngắn ngủi. Như vậy mối quan hệ giữa tác giả và bài thơ chỉ có thể trở thành mối quan hệ tương đối bền vững, một khi tác giả “đã xong” và hoàn cảnh làm phát sinh bài thơ đó đã thay đổi. \textit{Khi một tác giả còn đang sống sờ sờ bên người đọc và hoàn cảnh làm phát sinh bài thơ hoặc là chưa thay đổi hoặc là vẫn tiếp tục tác động một cách nào đó lên đời sống của người đọc đương thời, thì mối quan hệ này lúc nào cũng biến đổi.} Vì vậy, trong phê bình văn học, những cố gắng khách quan hóa sự thẩm định (appreciation) quả thật là khó khăn nhưng bao giờ cũng cần thiết. 

Nhà phê bình trẻ tuổi trong nước Trần Thiện Khanh, một người mới viết nhưng đầy tài năng, trong một trao đổi cá nhân gần đây, có hỏi tôi rằng: anh có những mô hình đọc thơ nào không. 

Thưa anh, tôi không có một mô hình đọc thơ nào cả. Nhưng tôi cho rằng đối với người đọc, \textit{việc hiểu thơ (understanding) là việc quan trọng bậc nhất}. Tôi biết rằng điều này có thể gây tranh cãi. Nhưng tôi vững tin rằng nhờ hiểu một bài thơ mà người đọc có thể thưởng thức đầy đủ vẻ đẹp thẩm mỹ của nó. Như vậy giá trị của một bài thơ là nằm trong tự thân nó, trong cấu trúc ngôn ngữ và hình ảnh của bài thơ, vốn vĩnh viễn theo thời gian. Nếu thế thì việc tìm hiểu mối quan hệ giữa bài thơ và tác giả, mối quan hệ giữa tác giả và hoàn cảnh lịch sử, cũng như các chi tiết tiểu sử của tác giả hay các sự kiện lịch sử liên quan đến đối tượng được mô tả, tức là các nhân vật trong thơ, có còn quan trọng nữa không? 

Có những sự kiện xảy ra trong đời sống của nhà thơ và xã hội ghi dấu ấn rõ ràng lên tác phẩm. 
\begin{blockquote}


\textit{Hãy cho anh khóc bằng mắt em} 
\textit{Ôi những cuộc tình duyên Budapest} 
\textit{Anh một trái tim, em một trái tim} 
\textit{Chúng kéo đầy đường chiến xa đại bác} 

\end{blockquote}


Những câu thơ của Thanh Tâm Tuyền có thể nào tách khỏi cuộc nổi dậy ngây thơ hào hùng tươi trẻ Hungary 1958 và sự kiện Liên Xô đưa quân vào thủ đô của nước này để trấn áp một chính quyền dám đi trước thời đại ba mươi năm? Không có sự kiện Budapest thì ý nghĩa của bài thơ là ý nghĩa nào? 

Có một người sĩ quan của bộ đội miền Bắc đã viết như sau: 
\begin{blockquote}


\textit{Thật sung sướng khi được chết} 
\textit{Như một thường dân} 
(Nguyễn Thụy Kha) 

\end{blockquote}


Đây có lẽ là một trong những câu thơ hay nhất về chiến tranh Việt Nam. Những người cầm súng phía bên này chắc sẽ trở nên thông cảm hơn một chút với những người cầm súng phía bên kia khi được đọc những câu thơ như thế. Chúng không nhiều. Văn chương có khả năng đem con người lại gần nhau. Tôi ít được đọc những câu viết giản dị mà có tầm suy nghĩ như thế ngay cả trong những tập thơ hay văn xuôi về chiến tranh trong ba mươi năm qua được quảng cáo khắp nơi. Văn chương hay bao giờ cũng cần sự dũng cảm, nâng người đọc lên, giúp họ vượt thoát qua các bức tường của chính mình. Nhưng nếu tác giả Nguyễn Thụy Kha không phải là một người lính thực thụ, liệu tôi có thấy câu thơ của anh rung động sâu xa như tôi đã thưởng thức không? 

Tuy vậy, hầu hết những bài thơ khác không khởi đi từ những kinh nghiệm trực tiếp hay hoàn cảnh cụ thể. Chúng được sinh ra như thế nào? Có thể từ những gợi ý rất gián tiếp, một câu chuyện được kể lại bởi một người bạn, một cảnh diễn ra ngoài đường phố, kí ức mơ hồ của tuổi trẻ, một cảm giác bùng cháy dữ dội khi xem một cuốn phim, cảm giác bâng khuâng bồi hồi trên trang sách cuối cùng của cuốn tiểu thuyết hay, vân vân. 

Tôi cho rằng đối với văn xuôi, câu chuyện có thật ngoài đời và câu chuyện trong tiểu thuyết thường rất khác nhau, trong khi đó đối với thơ, chúng rất gần nhau. Những kinh nghiệm trong đời của một nhà văn thường cách xa những kinh nghiệm của nhân vật của anh ta, trong khi những kinh nghiệm của nhà thơ thì lại gần gũi hơn với các kinh nghiệm của nhân vật. Điều kì lạ là ở chỗ càng gần gũi nhau, chúng càng trở nên mơ hồ, hư hư ảo ảo, biến đổi khôn lường. 
Thì cũng như tình yêu. 

Xuân về, tuyết tan, đi bộ sau giờ làm việc dưới những rặng anh đào trổ bông mê muội, tôi thường nhớ đến đoạn mở đầu bài thơ của e.e. cummings. 
\begin{blockquote}


\textit{in Just-} 
\textit{spring when the world is mud-} 
\textit{luscious the little} 
\textit{lame balloonman} 

\textit{whistles far and wee} 

\end{blockquote}


Rất khó dịch e.e cummings. Đó là ký ức của nhà thơ về những ngày cắp sách đi học. Không, đó chính là kinh nghiệm hiện tại của cậu học sinh khi mùa xuân về trên sân trường. Nhưng hình như chúng ta cũng bắt đầu nhận ra trong đoạn thơ trên một điều gì khác nữa, xa hơn, tinh tế hơn (\textit{far and wee}) những điều tôi vừa nói, phải chăng vì những âm \textit{s, l} rì rào như gió thổi trong lá cây? 

Muốn xem xét mối quan hệ giữa bài thơ và người làm ra nó không thể bỏ qua quá trình sáng tạo của nhà thơ đối với từng bài cụ thể. Một số nhà thơ có một ý tưởng rạch ròi, là họ định viết cái gì, viết như thế nào, bắt đầu ra sao, kết thúc với những chữ nào. Trong trường hợp này ý định của nhà thơ rất sáng sủa, như thể có một thứ linh hồn bất diệt, mà công việc của nhà thơ là gọi nó ra từ cõi sâu thẳm nào, rồi anh ta chỉ việc lấy xương thịt đắp lên để thành một con người thật. Như thế “cái hồn” là điều có trước, quyết định. Tôi gọi đó là \textit{ý định} của nhà thơ. 

Một số những nhà thơ có ý định như thế và quả thật họ đã hoàn tất công trình làm thơ của mình như thế, đúng với bản vẽ đầu tiên như một người thợ xây căn nhà trên bản vẽ của kiến trúc sư, đúng từng viên gạch. Những nhà thơ này theo tôi thật là may mắn, nhưng điều đáng tiếc là cái mà họ xây nên hầu hết chỉ có thể là những căn nhà chứ không phải là những bài thơ. Tôi ít thấy các ngoại lệ. 

Trong những trường hợp khác, nhà thơ có thể có những ý tưởng ban đầu, mà ta gọi là ý định, nhưng khi quá trình sáng tác xảy ra, chính ngôn ngữ và hình ảnh của bài thơ lại kéo họ đi. Nhà thơ không còn là chủ thể sáng tạo duy nhất nữa. Chữ của anh ta cũng không phải là chữ của anh ta. Căn nhà có thể được xây xong nhưng bản vẽ đầu tiên đã bị vứt bỏ. 

Cảm hứng nghệ thuật lại là một vấn đề phức tạp khác. Một số tác giả làm việc dựa trên cảm hứng, khi nó đến, họ làm việc rất nhanh một mạch, gần như không sửa chữa. Một số khác làm việc bằng kỉ luật hàng ngày, lao động miệt mài trên trang giấy, sửa đi sửa lại một câu thơ hàng trăm lần. Không có phương pháp nào ưu tiên hơn phương pháp nào, vì đối với người đọc, chỉ có thành tựu cuối cùng mới là quan trọng. 

Nói vậy hóa ra nhà thơ chẳng có trách nhiệm gì đối với tác phẩm của mình hay sao? 

Không phải thế. Quá trình sáng tạo rất phức tạp và việc nhà thơ không làm chủ được quá trình này hoàn toàn không gỡ bỏ được khỏi đôi vai của anh ta trách nhiệm của mình như một tác giả “toàn năng” trước người đọc và dư luận xã hội. Nhà thơ có thể không hoàn toàn làm chủ việc chọn lọc các chữ để đưa vào một câu thơ nào đó, nhưng anh ta lại \textit{hoàn toàn làm chủ việc lấy} \textit{chúng ra} khỏi câu thơ của mình. 

Trong những hoàn cảnh khó khăn khắc nghiệt, một nhà thơ có thể làm bất cứ việc gì để sống và có thể hoàn toàn giữ im lặng, nhưng một khi anh ta đã viết thì thơ phải nhất thiết phản ảnh những suy nghĩ và cảm nghiệm nghệ thuật của nhà thơ. Bài thơ là tiếng nói xã hội, tiếng nói công dân, dù đó là một bài thơ tình, hay là một bài thơ hoàn toàn không có đề tài thời sự. Nhà thơ có thể đốt tác phẩm của mình đi trước khi đưa nó đến nhà xuất bản. Theo nghĩa đen. Nếu điều đó là cần thiết. 

Tôi không có ý định so sánh điều này với việc các nhà thơ “cách mạng” như Chế Lan Viên, Xuân Diệu, Huy Cận… đã từng cho các tập thơ \textit{Điêu tàn}, \textit{Gửi hương cho gió}, \textit{Lửa thiêng}… của họ vào lửa, vì đó là hành động có tính cách tượng trưng, có thể là dưới sức ép, cũng có thể là do các nhà thơ thành thật tin như thế, ít nhất là vào lúc ấy. Nhưng đó là các tác phẩm đã được xuất bản và đã được công chúng biết đến rồi: quan hệ của tác phẩm và tác giả về căn bản đã xác lập xong. 

Có một khuynh hướng diễn dịch thơ ca nông cạn và rẻ tiền đang phổ biến khắp mọi nơi, nhưng tôi cho rằng, một cách đáng ngạc nhiên, ở hải ngoại cũng chẳng kém gì ở giới hàn lâm chính thống trong nước, vì một số lý do lịch sử. Đó là chính trị hóa việc diễn dịch thơ ca, việc đọc các thông điệp thơ ở bề mặt, việc quy cho nó những trách nhiệm chính trị mà thơ không nhất thiết phải có, việc từ chối nhìn con người như một toàn thể, nói chung là việc dung tục hóa thơ ca. 

Sự hiểu biết của chúng ta đối với nhà thơ có ảnh hưởng như thế nào đối với việc diễn dịch và thưởng thức một bài thơ? 

Nếu câu thơ: 
\begin{blockquote}


\textit{Tôi ở phố Sinh Từ} 
\textit{Hai người} 
\textit{Một gian nhà chật} 
\textit{Rất yêu nhau sao cuộc sống không vui?} 
\textit{Tổ quốc hôm nay…} 

\end{blockquote}


Của một tác giả nào khác, viết một thời điểm khác, không phải của Trần Dần, viết năm 1955, thì chúng ta có thấy còn hay nữa không, nếu còn hay thì có thấy hay khác đi không? 
\begin{blockquote}


\textit{Đừng phun thuốc muỗi. Chết thuyền quyên} 
\textit{Dân chủ khú mất rồi. Hỏng cả vại dưa ngon} 
(Trần Dần) 

\end{blockquote}


Người đọc rất muốn biết tác giả là ai, viết câu thơ này vào thời điểm nào. 
Nhưng đọc: 
\begin{blockquote}


\textit{Ngắc ngứ chiêm bao} 
\textit{Li tán cả vùng sao} 
(Trần Dần) 

\end{blockquote}


Hay: 
\begin{blockquote}


\textit{Những con đường sao mọc lúc ta đi} 
\textit{Những chiều sương mây phủ lối ta về} 

\end{blockquote}


của một nhà thơ mà Trần Dần hình như rất ngưỡng mộ, thì có cần biết tác giả mới thấy là hay, hay không? 

Như thế, có những bài thơ hay mà đứng độc lập một mình, vì có tính phổ quát, và có những bài thơ hay mà phải đứng dựa vào những hoàn cảnh khác, vì có những đặc tính riêng biệt. Những bài thơ yêu nước, chẳng hạn, thường bao giờ cũng gắn với một cuộc cách mạng hay chiến tranh cụ thể. Thật ra trong quá khứ, với văn chương truyền khẩu, vai trò của tác giả không quan trọng lắm, hầu hết là vô danh. Không có chữ viết, thơ có tính cách trình diễn, kể lại, hát lại. Vì kể lại và hát lại như thế, nên lời thơ không thể nào cố định do trí nhớ mỗi người mỗi khác, mỗi lúc mỗi khác. Tác phẩm của một làng xóm hay một tập thể không chỉ là tiếng nói mà \textit{tác giả dành cho cộng đồng }đó mà còn, thông qua động tác biểu diễn của người ngâm, người hát, người kể mà trở thành \textit{tiếng nói của cộng đồng đó}. Thời kỳ in ấn đã mau chóng lấy mất ảnh hưởng của văn chương truyền khẩu và đem lại nền văn học chữ viết. Sự hiện diện của tác giả ngày một rõ ràng và quan trọng. Bài thơ không còn là một tác phẩm tập thể, có thể được biến đổi qua các làng xóm hay theo thời gian, không còn là tiếng nói của cộng đồng mà là tiếng nói của cá nhân người viết. 

Bài thơ trở thành một phương tiện để tìm hiểu tác giả. 
\begin{blockquote}


\textit{Vườn ai mướt quá xanh như ngọc} 
\textit{Lá trúc che ngang mặt chữ điền} 

\end{blockquote}


Thích câu thơ quá, nhiều người tò mò muốn biết Hàn Mặc Tử có quen người đẹp nào ở Vĩ Dạ hay không? Mộng Cầm chăng? Cô gái nào mà tốt số, hay xấu số thế? Anh Nguyễn Đặng Mừng, trên talawas 9.4.2008\footnote{\url{http://www.talawas.org/talaDB/showFile.php?res=12819&rb=12}}, than thở rằng sau khi được giải thích cho nghe \textit{chữ điền} tức là bức bình phông bằng chè tàu thường gặp ở Huế, đâm ra bớt thích câu thơ đến một nửa. Trước đó anh tưởng rằng chữ điền là người phụ nữ có khuôn mặt chữ điền. 

Nếu chúng ta dừng lại ở đây, ấn tượng mà một số người đọc có thể có là: đúng rồi, thế ra mình cũng nghĩ nhầm, cứ tưởng chữ điền là mặt của một cô gái đẹp, ngờ đâu đó chỉ là bức bình phông. 

Nếu tôi dừng lại như thế, thật là không lương thiện. 

Tôi không muốn nói rằng việc hiểu \textit{chữ điền} của Hàn Mặc Tử như thế nào là không quan trọng. Một chữ có thể có nhiều nghĩa. Tìm hiểu về đời sống riêng của tác giả, nhất là một tác giả kì lạ như Hàn Mặc Tử, sẽ giúp hiểu thêm nhiều điều về thơ của ông. Nếu biết rằng có một cô gái ở Vĩ Dạ có khuôn mặt chữ điền, và đem lòng yêu nhà thơ của chúng ta, cũng là điều thú vị lắm chứ? Mặc dù tôi không chắc phụ nữ có mặt chữ điền thì có đẹp hay không. 

Những câu hỏi mà bạn có thể đặt ra nữa là: tác giả có ý dùng \textit{chữ điền} để diễn tả mặt người phụ nữ, nhưng trong đời thực ông không hề quen biết một cô gái nào ở Vĩ Dạ chi cả, thì có được không? Thế nếu mới đầu ông dùng \textit{chữ điền} để chỉ khuôn mặt, khi viết được nửa bài thì ông đổi ý, “muốn” nó là cái bình phông trước sân nhà ở Huế, thì có được không? Nếu một nhà phê bình hay khảo cổ bỗng nhiên một hôm tìm được một tài liệu bí mật chép bằng tay của chính Hàn Mặc Tử, trong một cái tráp chôn dưới hòn đá ngoài sân, nói rằng \textit{chữ điền} không phải là cái bình phông \textit{chè tàu} hay là khuôn mặt \textit{o tê} mà là một vật thứ ba nào đó, thì sao? 

Vấn đề tôi muốn nói ở đây là: sự chú ý thẩm mĩ của người đọc đã bị đánh lạc hướng. 

Chúng ta đang chạm đến một khái niệm quan trọng trong việc hiểu thơ, đó là sự thành thật. Ngược lại với sự thành thật là: giả dối. Không ít người quả quyết tin rằng họ biết được thế nào là thành thật, thế nào là giả dối. 

Tôi thì không chắc chắn như thế. 

Trong đời sống, chúng ta đánh giá cao sự thành thật. Thành thật là gì? Là ý nghĩ đi đôi với lời nói, lời nói đi đôi với việc làm. Nghĩ một đằng nói một nẻo, hay nói một đằng làm một nẻo là một lời chê trách. Nhưng trong văn chương thì sao? Muốn đánh giá sự thành thật của tác giả thì phải liên hệ đến các sự vật nằm ngoài văn bản. Vả lại, giá trị của một tác phẩm văn học có nhất thiết phải giới hạn trong việc tác phẩm đó phản ảnh chính xác những suy tư và tình cảm của tác giả hay không? 

Đọc một bài thơ hay một tác phẩm văn chương, người đọc có thể nhận xét: “bài thơ xúc động quá”. Khi nói như thế anh ta có thể hàm ý rằng người viết rất thành thật trong khi viết. 

Tôi cho rằng cần phân biệt hai thể loại thơ khác nhau khi chạm đến khái niệm thành thật và giả dối, đến vấn đề vai trò của tác giả. Một là thể thơ trữ tình và hai là thể thơ tự sự. Trong thơ trữ tình tác giả là (một) nhân vật, trong thơ tự sự tác giả không nhất thiết phải là nhân vật. 

Những người đọc thơ nhấn mạnh đến mối quan hệ giữa tác giả và bài thơ, và xem nó là một trong những tiêu chuẩn để đánh giá tác phẩm, mặc nhiên thừa nhận rằng bài thơ là một phần cuộc đời của tác giả, là bản mô tả các sự việc đã xảy ra, tường thuật về đời sống riêng tư hoặc là các vấn đề xã hội chính trị “đã thực sự xảy ra”. Trong khi đó cần hiểu rằng thơ cũng như các thể loại văn học khác, tiểu thuyết chẳng hạn, tạo ra các phiên bản của hiện thực cuộc đời thông thường: hiện thực của văn chương là một \textit{version} khác, khác với hiện thực cuộc đời. 

Có những tác phẩm dựa trên các sự thật “có thật”, các sự kiện lịch sử khách quan hay tiểu sử một cá nhân cụ thể chẳng hạn. Khi chuyển thành các trang viết thì chúng đã được biến đổi, kể lại, chuyển thể, bất chấp tác giả có ý thức hay không ý thức về điều này. Trong thơ trữ tình, các sự kiện kiểu này thường mờ nhạt, và hình ảnh của tác giả trở nên quan trọng. Trong thơ tự sự, ngược lại, các sự kiện trở nên rõ rệt hơn, nhưng không vì thế mà thơ trở thành một bản mô tả, Trong thơ tự sự, tiếng nói và chân dung tác giả được giấu đi rất kĩ. Tôi cho rằng, và điều này là quan trọng, chúng càng được giấu đi, bài thơ càng thành công. 

Cố gắng đi tìm tiếng nói của tác giả trong một bài thơ tự sự là một công việc thú vị nhưng hết sức khó khăn, nhiều khi không làm được. 

Mặc dù vậy, cũng phải thừa nhận một thực tế là nếu chúng ta phát hiện giữa bài thơ và tác giả có một khoảng cách, thì bài thơ trở nên bớt hấp dẫn đi. Ví dụ, đối với những người không thích Tố Hữu, thì các câu thơ vốn rất được khen, nhiều khi quá đáng, của ông, trở nên không còn hay nữa. Nhưng thực ra thì, chúng vẫn hay. 

Quan niệm cho rằng văn chương là sự tự biểu hiện (self expression) được cụ thể hóa trong một câu nói nổi tiếng của một nhà văn Pháp: văn là người. Tôi cho rằng quan niệm này có thể đúng, mặc dù chỉ phần nào thôi, khi áp dụng vào thơ trữ tình, nhưng không thể áp dụng trong thơ tự sự và thơ có tính kịch (dramatic). 

Có một điều khó khăn là sự phân biệt giữa thơ trữ tình và thơ tự sự hay các thể loại tương tự, đặc biệt trong trào lưu hậu hiện đại, không rạch ròi và dễ được mọi người đồng ý. 

Các thứ biên giới bao giờ cũng mịt mờ sương khói hơn là chúng ta nghĩ về chúng. Nhưng đó lại là điều may mắn của một nhân loại tường minh. 


\textbf{Các sách tham khảo cho bài viết:} 
<ul style="list-style-type: none">
item{Thi Vũ, \textit{Bốn mươi năm thơ Việt Nam}, NXB Quê Mẹ, 1993; }

item{
\textit{Tuyển tập Sông Hương ( 1983- 2003)}, Ban tuyển chọn: Nguyễn Khắc Thạch, Hồ Thế Hà, Hồng Nhu, Ngô Minh, Trương Thị Cúc, NXB Văn Hoá Thông Tin, 2003; }

item{
\textit{Tạp chí Thơ}, Hoa Kỳ, 2006-hiện nay; }

item{
\textit{Thơ}, tạp chí, Hội nhà văn Việt Nam, Hà Nội, 2008-hiện nay; }

item{Hoàng Ngọc Hiến, \textit{Những ngả đường vào văn học}, NXB Giáo dục, 2006; }

item{Nguyễn Hưng Quốc, \textit{Thơ con cóc và những vấn đề khác}, NXB Văn mới, 2006; }

item{
\textit{Tuyển tập Tiền Vệ I}, NXB Tiền Vệ, 2007; }

item{Đặng Tiến, \textit{Vũ trụ thơ II}, Thư Ấn Quán 2008; }

item{Trần Dần, \textit{Thơ}, NXB Đà nẵng, 2008; }

item{Gary Geddes, \textit{20th Century poetry &amp; poetics}, Oxford University Press, 1996; }

item{G.G. Sedgewick,\textit{ Of Irony, Especially in drama}, Ronsdale Press, 2003; }

item{Dana Gioia, \textit{Twentieth century – American Poetry}, McGraw Hill, 2004; }

item{David Lehman,\textit{The Oxford book of Poetry}, 2006; }

\end{itemize}


© 2008 talawas 
\end{multicols}
\end{document}