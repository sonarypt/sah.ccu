\documentclass[../main.tex]{subfiles}

\begin{document}

\chapter{Phát minh của thi sĩ}

\begin{metadata}

\begin{flushright}24.7.2008\end{flushright}

Bùi Chí Vinh



\end{metadata}

\begin{multicols}{2}

\textbf{Về trí nhớ của tôn sư Lê Quý Ðôn} 
 
Hồi nhỏ tôi rất phục Lê Quý Ðôn, thần đồng về mặt trí nhớ. Theo truyền thuyết ông có thể đọc vanh vách những gì mình chỉ nhìn thoáng qua hoặc nghe qua một lần, ông đã từng cứu bồ một bà chủ quán rượu bị mất sổ ghi nợ chép trên vách vì cháy nhà. Ông cũng đã từng thoát một trận đòn của thân phụ nhờ ứng đối trước mặt khách của cha một bài thơ Ðường Luật về rắn mà câu nào cũng đề cập đến từng chủng loài bò sát. 
 
Tôi không “xịn” như tiên sinh Lê Quý Ðôn, nhưng theo lời bạn bè thì trí nhớ cũng thuộc hàng cao thủ. Từ lúc 11 tuổi đến nay tôi đã làm trên 1000 bài thơ đủ nội dung thể loại và tự hào thuộc tối thiểu cũng hơn 800 bài mình ưa thích. Khác với bậc trí giả Lê Quý Ðôn, tôi bắt buộc phải thuộc thơ mình vì yếu tố thời thế. Nói hú họa, chẳng may tôi bị bọn cường quyền bạo chúa nào đó bịt miệng thì với trí nhớ trời cho, ít ra tôi cũng để dành một số lượng thơ cần thiết để lại cho thế nhân qua ghi chép hoặc khạc thơ truyền khẩu trong bàn rượu thân hữu. 
 
Năm nay bước qua tuổi ngũ thập tri thiên mệnh đáng lẽ trí nhớ kém dần, nhưng cũng may roi vọt cuộc đời không ngừng quất vào nên trí nhớ của thi sĩ lần lượt lại khôi phục. Nếu thiên tài Lê Quý Ðôn có bài thơ thất ngôn bát cú câu nào cũng nói về RẮN thì tôi cũng nối chí cha ông bằng bài thơ năm chữ câu nào cũng nói về CHÓ, vừa mua vui trong bàn rượu, vừa được chủ quán cho “xù” trong việc trả tiền. Nhân đây, bằng trí nhớ cá nhân, tôi chép lại bài thơ RẮN của Lê Quý Ðôn và bài thơ CHÓ của tôi để mọi người tủm tỉm cười chơi. 
\begin{blockquote}
 
\textbf{Bài thơ Rắn đầu biếng học} 
        
\textit{Chẳng phải LIU ÐIU cũng giống nhà}        
\textit{RẮN đầu biếng học lẽ không tha}        
\textit{Thẹn đèn, HỔ LỬA đau lòng mẹ}        
\textit{Nay thét, MAI GẦM rát cổ cha}        
\textit{RÁO mép chỉ quen lời lếu láo}        
\textit{LẰN lưng cam chịu vệt năm ba}        
\textit{Từ nay TRÂU lỗ siêng năng học} 
\textit{Kẻo HỔ MANG danh tiếng thế gia} 
 
\textbf{Bài thơ thịt chó} 
        
\textit{Sáng sớm nghe tiếng KHUYỂN}        
\textit{Giữa trưa bén mùi CẦY}        
\textit{Chiều bước vào quán CẨU} 
\textit{Chú TUẤT nằm đâu đây} 
        
\textit{Tiếng ÐỒNG QUÊ là NAI}        
\textit{Tiếng giang hồ là CỚM}        
\textit{Gần MỰC thì chú đen} 
\textit{Gần đèn thì chú ÐỐM} 
        
\textit{Thăm chú nhớ BÁNH TRÁNG}        
\textit{Mới nhất BẠCH nhì VÀNG}        
\textit{Nhâm nhi dăm XỊ ÐẾ} 
\textit{Mới tứ VỆN tam KHOANG} 

\end{blockquote}
 
 
\textbf{Về bà Hồ Xuân Hương, thuỷ tổ thơ tiếng lái} 
 
Ngoài trường hợp độc đáo của tôn sư Lê Quý Ðôn về mặt trí nhớ như đã nói ở trên, còn một nhân vật nữa trong văn học sử Việt Nam mà tôi cực kỳ ngưỡng mộ. Ðó là nữ sĩ Hồ Xuân Hương, từng được mệnh danh là “Bà Chúa thơ Nôm”, từng được coi là một trong những người phụ nữ hiếm hoi trên trái đất dám thực hiện nữ quyền một cách triệt để nhất trong thời đại phong kiến trên đe dưới búa bất chấp có thể bị nguy hại đến tính mạng. Tuy nhiên, những danh xưng ấy vẫn chưa đủ để nói về sự mở mang ngôn ngữ của bà. Theo tôi, thiên tài Hồ Xuân Hương còn là thủy tổ làm thơ về tiếng lái, là chưởng môn nhân đầu tiên của môn phái “đảo ngữ” một cách kỳ ảo tạo nên tứ thơ đối nghịch khôn lường mà những người đi sau như Bùi Giáng tha hồ kế thừa để phát huy nghệ thuật chơi chữ. 
 
Hồ Xuân Hương là một nhà thơ ngôn ngữ hai mặt, người đời thường truyền tụng là thơ “đố tục giảng thanh”. Nhưng thơ tiếng lái của bà lại chơi đòn tréo ngoe là “chuyển thanh sang tục”. 
 
Một minh chứng trong bài “Kiếp tu hành” như sau: 
\begin{blockquote}
        
\textit{Cái kiếp tu hành nặng đá đeo}        
\textit{Vị gì một chút tẻo tèo teo}        
\textit{Thuyền từ cũng muốn về Tây Trúc} 
\textit{Trái gió cho nên phải lộn lèo} 

\end{blockquote}
 
Nếu chịu khó đọc kỹ và có máu giang hồ một chút, ai nấy phải bật cười bởi “đá đeo” tức là “đéo đa”, “trái gió” tức là “chó dái”, “lộn lèo” tức là “lẹo… l” 
 
Một thí dụ khác trong bốn câu đầu bài thơ “Chùa Quán Sứ”: 
\begin{blockquote}
        
\textit{Quán Sứ sao mà cảnh vắng teo}        
\textit{Hỏi thăm sư cụ đáo nơi neo?}        
\textit{Chày kinh, tiểu để suông không đấm} 
\textit{Tràng hạt, vãi lần đếm lại đeo…} 

\end{blockquote}
 
Tương tự bài trước, nếu chiết tự ba chữ “đáo nơi neo” tức là “đéo nơi nao”, “suông không đấm” tức là “đâm không sướng”, “đếm lại đeo” tức là “đéo lại đêm”. 
 
Sự tài hoa của nữ sĩ họ Hồ biến thành phát minh mở đường cho các thi sĩ. Tôi không có máu ngông cuồng như Bùi Giáng khi xài tiếng lái bắt chước bà, tôi cũng không sử dụng tiếng lái trong chốn phòng the, tôi “thảy” tiếng lái của thế kỷ 21 vào những nỗi đau thế thái nhân tình, những trận phong ba cơm áo tạo nên tiếng cười cay đắng cho những ai đang bị áp bức. Ít nhất tôi cũng thu thập từ truyền khẩu, giai thoại của nhân dân để viết hơn 10 bài thơ tiếng lái, xin chép ra đôi bài sau để mọi người thưởng lãm. Ðó là 2 bài thơ mang tựa “Quốc kỳ” và “Ðảo ngữ hành”. 
\begin{blockquote}
 
\textbf{Quốc kỳ} 
        
\textit{CỜ VÀNG thì tình CÀNG VỜ}        
\textit{CỜ XANH  sao rụng CÀNH XƠ xác cành}        
\textit{CỜ ĐỎ ông CÒ ĐỠ anh}        
\textit{CỜ HỒNG cái CÒNG HỜ nhanh lắm bồ}        
\textit{Treo CỜ GÌ đỡ KỲ GIỜ?} 
\textit{Ê, CỜ TÂY hạ CẦY TƠ  ra đời!} 
 
 
\textbf{Ðảo ngữ hành} 
        
\textit{Hành đảo ngữ kể từ GIẢI PHÓNG }        
\textit{Thi ca làm PHỎNG DÁI niêm vần}        
\textit{Muốn in báo phải làm đầy tớ} 
\textit{Nhưng ta nào phải kẻ lòn trôn} 
        
\textit{Ta nào phải là ông Hàn Tín}        
\textit{Phò Lưu Bang phản bạn lừa thầy}        
\textit{KỸ SƯ vì thế thành CƯ SĨ} 
\textit{THẦY GIÁO từ đây chịu THÁO GIÀY} 
        
\textit{Họp ĐỒNG CHÍ thấy toàn ĐÌ, CHỐNG}        
\textit{XÔ VIẾT ngày nay khoái XIẾT VÔ}        
\textit{Hình treo LỘNG KIẾNG như LIỆNG CỐNG} 
\textit{Ðể thằng TO DỰ hét TỰ DO } 
        
\textit{Chú đeo BẢNG ĐỎ  mà BỎ ĐẢNG}        
\textit{Mượn SAO VÀNG che đậy SANG GIÀU}        
\textit{CĂNG BỒNG nhờ nói CÔNG BẰNG nhỉ} 
\textit{LƯU MANH nào lại chẳng LANH MƯU?} 
        
\textit{Theo CHÍNH PHỦ ai ngờ CHÚ PHỈNH}        
\textit{Vào CHIẾN KHU thì bị CHÚ KHIÊNG}        
\textit{Mồm ĐÁNH MỸ mà tâm ĐĨ MÁNH} 
\textit{TIỀN ĐÂU? chú chặn họng ĐẦU TIÊN} 
        
\textit{GIÁO CHỨC đói meo đành DỨT CHÁO}        
\textit{Làm NHÀ THƠ vô bót NHỜ THA}        
\textit{THIÊN TÀI không đủ THAI TIỀN hả?} 
\textit{CẤT ĐUỐC về quê CUỐC ĐẤT à!} 
        
\textit{KHIẾN CHÁN ta làm thơ KHÁNG CHIẾN}        
\textit{Gào THI ĐUA chú bịp THUA ĐI}        
\textit{LÀM THƠ mà LỜ THAM mới nhục} 
\textit{THÌ CẤY cày mất đất THẤY KỲ} 
        
\textit{LÃNH TỤ sạch nhờ ôm TỦ LẠNH}        
\textit{BẨN NGƯỜI DO bác BỎ NGƯỜI DÂN}        
\textit{BÁC ĐI quá sớm thành BI ĐÁT} 
\textit{NGHỆ SĨ tụi con NGHĨ XỆ quần…} 

\end{blockquote}

\textit{		} 
 
\textbf{Về loại thơ “bỏ dấu xuống chữ” của chính tôi} 
 
Sau những cơn stress nặng về cơm áo gạo tiền công danh địa vị phù phiếm hư ảo, con người ta luôn cần có sự thư giãn. Nối tiếp tiền nhân, tôi tự thể nghiệm mình qua kiểu thơ chơi chữ đời mới cho thanh thản tâm hồn. Cụ thể tôi bỏ các dấu gồm “dấu huyền, dấu sắc, dấu hỏi, dấu ngã, dấu nặng” xuống các mẫu tự của 24 chữ cái hoặc các từ để chúng thành thơ. 
 
Xin giới thiệu cùng các bạn một số bài thơ về chữ A, B, C, các từ bão, quậy, thớ như sau: 
\begin{blockquote}
 
\textbf{A} 
        
\textit{Con gái ta thường kêu bằng: ả}        
\textit{Ðôi khi đụng “xẩm” đổi thành: a}        
\textit{Các em xinh đẹp thì ta: á}        
\textit{Giống Chung Vô Diệm thì ta: ạ} 
\textit{Á ạ gặp ta cũng phải: à} 
 
\textbf{B} 
        
\textit{Nhìn em, ta muốn: bế}        
\textit{Muốn bế thì phải: bê}        
\textit{Bê em như bê: bệ}        
\textit{Bê bệ dễ rớt: bể} 
\textit{Ðặt xuống giường, ta: bề} 
 
\textbf{C} 
        
\textit{Trong tình yêu không xài: xể}        
\textit{Làm như thế hai đứa: xệ}        
\textit{Thà rằng giận nhau, ta: xê}        
\textit{Ta đi bộ, em đạp: xế} 
\textit{Lúc gối chăn, bụng sẽ: xề} 
 
\textbf{Bão} 
        
\textit{Giữa cuộc đời giông: bão}        
\textit{Ta ruột xé gan: bào}        
\textit{Văn miếu nuôi cường: bạo}        
\textit{Triều đình nuôi hổ: báo} 
\textit{Mình ta nuôi chiêm: bao} 
 
\textbf{Quậy} 
        
\textit{Ở biển ta là cá: quẫy}        
\textit{Sao ngươi đem bỏ mặt: quầy}        
\textit{Giang hồ có câu phải: quấy}        
\textit{Lẽ nào ta chịu lăn: quay} 
\textit{Lẽ nào ta không dám: quậy} 
 
\textbf{Thớ} 
        
\textit{Có con ong: thợ}        
\textit{Không thèm hít: thở}        
\textit{Khí hậu đền: thờ}        
\textit{Cho nên có: thớ} 
\textit{Mật thành ra: thơ} 

\end{blockquote}
 
19.7.2008 
 
© 2008 talawas 
\end{multicols}
\end{document}