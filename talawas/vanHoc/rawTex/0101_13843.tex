\documentclass[../main.tex]{subfiles}

\begin{document}

\chapter{Thơ Việt trên đường hội nhập}

\begin{metadata}

\begin{flushright}26.7.2008\end{flushright}

Yến Nhi



\end{metadata}

\begin{multicols}{2}

Lịch sử Thơ ca Việt Nam trải qua nhiều chặng cam go nhưng nhìn chung không thoát khỏi quy luật chung là luôn bứt phá tìm tòi trên con đường đổi mới để tiến kịp nhân loại, nhưng không đánh mất bản sắc riêng! Vấn đề càng bức thiết trong những thập kỷ cuối thế kỷ XX sang đầu XXI khi mà sự gián cách các biên giới địa lý – xã hội trở nên quá mỏng manh trước nhu cầu hội nhập mãnh liệt của nhiều tầng lớp độc giả. 
 
Làm ra Cái Mới trong Thi ca là cả một số đông gồm nhiều thế hệ nhà thơ. Thời gian đã minh xác cho ta những tác giả tiêu biểu của Thơ Việt thời hiện đại đã vượt khỏi ảnh hưởng Phong trào Thơ Mới, xây dựng và khẳng định một thời kỳ thơ ca rực rỡ từ sau 1945 mà các bộ Văn học sử đã điểm qua. 
 
Trong hơn nửa thế kỷ các nhà thơ lớn trên nhiều phương diện đã góp phần tạo dựng chân dung nền thơ ca  hiện đại Việt Nam, hầu hết ngày nay đã khuất, số còn lại nhìn chung vẫn còn sung sức, nhiều tác giả cố gắng không mệt mỏi với nhiều tìm tòi nhằm đổi mới phong cách sáng tạo Thơ và đã có những thành công nhất định, nhưng thật vui mừng vì một thế hệ tác giả mới đã xuất hiện, góp phần làm thay đổi bộ mặt thi ca Việt Nam. Đi tìm\textit{ Cái Mới của Thơ,} người ta thường chú ý vào giai đoạn \textit{Thơ đương đại}, đó là cái phần đậm được chú ý nhiều nhất dẫu nó chỉ là một quãng ngắn của thơ hiện đại, nhưng là quãng mới nhất có nhiều đặc trưng nhất! 

 
\textbf{1. Cảm xúc mới là động lực của cách tân} 
 
\textit{Cái Mới của Thơ} trước hết đó là cách cảm xúc mới của con người Việt Nam đương đại, nổi bật là xúc cảm của nhà thơ truớc cuộc sống hiện tại, một cuộc sống hội nhập trong một thế giới không còn chia cách,cuộc sống với mục tiêu dân giàu nước mạnh, xã hội hiện đại văn minh.. Từ cái xúc cảm mới mẻ này nó kéo theo những mới mẻ khác của hình thức nghệ thuật, của thi pháp mà truyền thống chưa có. Cảm xúc của con người Việt Nam thời nay quả có khác thời kỳ Cách mạng, kháng chiến. Nếu cảm hứng thời ấy là lý tưởng cứu nước và Chủ nghĩa xã hội, thì cảm xúc con người thời nay mở rộng hơn nhiều. Cái Mới của Thơ bắt nguồn từ sự mở rộng biên độ cảm xúc này. Không có gì thuộc về con người xa lạ với Thơ! Hay nói như cụ Nguyễn Du hai trăm năm trước: Trên mặt đất nơi nào chẳng có văn chương! (Đại địa văn chương tùy xứ kiến). Từ \textit{Chủ nghĩa anh hùng } mở rộng sang quỹ đạo \textit{Chủ nghĩa nhân đạo}. Hình như đó là con đường đi chung của văn học Việt trong lịch sử qua nhiều thế kỷ khi đất nước từ thời chiến chuyển sang thời bình. Con đường đi từ “Hịch tướng sĩ” (Hưng đạo vương Trần Quốc Tuấn), “Bình Ngô đại cáo” sang \textit{Quốc âm thi tập} (Nguyễn Trãi), \textit{Hồng Đức quốc âm thi tập} (Lê Thánh Tông)…, đến \textit{Đoạn trường tân thanh} (Nguyễn Du), \textit{Mai đình mộng ký} (Nguyễn Huy Hổ), \textit{Truyện Hoa tiên} (Nguyễn Huy Tự)…; từ “Văn tế nghĩa sĩ Cần Giuộc” (Nguyễn Đình Chiểu) đến thơ văn Nguyễn Khuyến, Tú Xương v.v… 
 
Lòng yêu nước vẫn là một nội dung quan trọng, nhưng những chủ đề đạo lý, thế sự cũng như những vấn đề riêng tư cá nhân ngày càng được lưu ý. Nói một cách khác Thơ thời nay thể hiện đầy đủ, toàn vẹn tình cảm con người, không có địa hạt \textit{bỏ qua} hoặc \textit{né tránh}, vẻ đẹp của Cái Tôi nhân bản trong Thơ hiện đại cũng chính là vẻ đẹp của Cái Ta nhân loại! 
 
Nhiều người chỉ quan tâm đến Cái Mới của Thơ trên phương diện sự tân kỳ của các yếu tố hình thức nghệ thuật. Thực ra cái mới của sự xúc cảm, của chiều sâu trí tuệ, của cách nhìn cuộc sống tạo nên Cái Mới của Thơ chứ không phải là những mô - típ, những hình ảnh mô phỏng hoặc  những biểu hiện có vẻ là lạ của hình thức nghệ thuật, cuả ngôn ngữ thi ca… Bộ phận Thơ giữa thế kỷ trước được gọi là Thơ Mới trước hết vì nội dung đòi hỏi tự do cá nhân thoát khỏi sự ràng buộc của lễ giáo phong kiến chứ không phải là ở thể thơ tám chữ tự do học theo thơ Pháp. 
 
Từ những năm giữa thế kỷ trước đã có một vài tác giả sử dụng lẻ tẻ các thể thơ tự do, thơ không vần, thơ văn xuôi, thơ leo thang, thơ vắt dòng…, những thi phẩm này mới so với trước đó nhưng không thể đưa vào hệ thống cái mới ngày nay, vì các thi sĩ bây giờ tuy có sử dụng một số yếu tố nghệ thuật thời ấy nhưng căn bản họ xây dựng các mô chuẩn thi pháp đương đại trên cơ sở một cách nhìn, cách cảm mới về đời sống và một quan điểm thẩm mỹ mới về nghệ thuật, về thi ca! 
 
Cái Mới không bắt đầu từ số không mà nó là sự tích hợp vẻ đẹp truyền thống cùng với sự trao đổi học tập những thành tựu của bè bạn nhiều nơi, nhiều khu vực cả trên hai bình diện lịch đại lẫn đồng đại. 
 
Dẫu vậy Cái Mới của thơ luôn nằm trong quỹ đạo của sự thể hiện Cái Đẹp chân chính chứ không là sự ham thanh chuộng lạ cực đoan, mà đồng điệu và cổ võ cho sự cực đoan này là việc tân trang Thơ, lắp ghép, đưa ra vô số những kiểu thơ với nhiều quy tắc ngữ pháp xa lạ với ngôn ngữ Việt, những tín niệm triết, mỹ học ngược với tâm lý, với đời sống Việt rồi khái quát thành tính hiện đại của Thơ! Nhưng cũng quyết không thể vì cái gọi là truyền thống mà đóng khung vào một mạch tư duy chật hẹp, vì tương lai không phải là “quá khứ kéo dài” mà phải cách tân và có những bước nhảy đột biến. Ngày xưa thơ Đường cũng phải Việt Nam hóa thông qua cái tâm trạng của các cụ nhà ta mới tồn tại được, bây giờ cái mới học được từ xứ người cần qua cái bộ lọc tâm hồn Việt Nam mới gọi là “đẹp”,”hiện đại”, có chỗ đứng bền lâu. Mạch chính của Thơ Việt hiện đại vẫn là loại thơ bắt rễ vào đời sống dân tộc vào thân phận "con người số đông" với bao chìm nỗi cay cực nhưng luôn biết vượt lên làm chủ số phận, đồng thời hàm chứa chiều sâu tư tưởng, triết lý và mỹ cảm thời đại.  
 
Giọng thơ ca ngợi hào sảng - cái giọng thơ chủ lưu một thời với nhiều vẻ đẹp từ trí tuệ đến dân giả từ cao sang đến bình dân hiện nay phát triển hòa vào dòng chảy nhân ái về thế sự, biểu lộ lòng trắc ẩn đến những thân phận không như ý, đến những suy nghĩ về tự do, về hạnh phúc đích thực của con người được giải phóng trong một “thế giới phẳng “, không chia cách cường nhược giàu nghèo! Khi xã hội tiêu thụ đang biến tất cả thành hàng hóa thì thơ ca hướng con người đến vẻ đẹp tinh thần vẻ đẹp đạo lý. Thơ không phải là lời giáo huấn suông nhưng cũng không là trò chơi ngôn ngữ mà là những vấn nạn nhân sinh ăn sâu vào ý thức đến tiềm thức… 
 
Một nội dung nhân bản không thể không nói đến là vấn đề tình yêu-tình dục, vấn đề trước đây tuy không phải hoàn toàn cấm kỵ nhưng vì các yêu cầu khác của đời sống bức thiết hơn nên các tác giả không tiện nói đến nhiều, thì nay đã là một mảng khá đậm trong thơ, khen cũng như chê có rất nhiếu ý kiến trái ngược. 
 
Thơ xưa nói về thân thể người phụ nữ, việc ân ái nam nữ  hay úp mở, giờ thì mạnh dạn và táo bạo hơn. Hãy so sánh những câu thơ của Nguyễn Gia Thiều, Bích Khê, của Vũ Hoàng Chương… rồi của Cầm Vĩnh Ui, từng được nhiều người biết với những câu thơ đầy những cảm xúc nhục thể của thi sĩ trẻ ngày nay ta sẽ thấy xung quanh vấn đề sex đã có những thay đổi lớn. 
 
Có thể nghĩ là các vẻ đẹp thân thể, những khao khát yêu đương, những “nhục cảm trần thế” của các cuộc tình là những thứ mà ngày xưa do những giới hạn cả khách quan lẫn chủ quan mà người ta e ấp che đậy, chỉ nói cái phần một nửa hoặc dấu kín cho riêng mình và người mình yêu thì ngày nay họ tự hào nói to lên đủ đầy, trọn vẹn… như là một hạnh phúc, một niềm tự hào, hân hoan muốn chia sẻ cùng bạn bè. Một tác giả đã bộc bạch: Bây giờ nhớ người yêu bên nhà hàng xóm chẳng ai còn e lệ “lặng lẽ giấu chùm hoa trong chiếc khăn tay”, mà sẽ chạy ngay sang hôn đến ngạt thở”. (Trả lời phỏng vấn của một nữ tác giả) 
 
Có người cho rằng họ ảnh hưởng thuyết này thuyết nọ,hoặc cho là  biểu hiện của tiến trình dân chủ hóa, bình đẳng giới trong xã hội. Chúng tôi nghĩ rằng, sự xích lại gần cuộc sống của Thơ khiến Thơ thể hiện tình yêu một cách trọn vẹn. Con người hằng ngày cần hấp thụ nhiều tri thức, nhiều tư tưởng, cũng đòi hỏi hưởng thụ một tình yêu đầy đủ, nhục cảm là một yếu tố không thể thiếu, miễn đó là một nhục cảm  khỏe khoắn, lành mạnh. Xã hội đi lên, các tập quán lễ giáo phong kiến phương Đông, tuy không phải  không còn ảnh hưởng ở một số người nhưng đã bộc lộ nhiều mâu thuẫn mà tầng lớp trẻ không muốn chấp nhận. Các nhà thơ trẻ  trong những tác phẩm thành công đã làm  được điều đó, thể hiện một tình yêu trọn vẹn đầy đủ tình cảm và bản năng , và thực tế những tác phẩm đó được người đọc chấp nhận. 
 
Từ cái sex hơi mờ ảo đạo lý qua cái sex bay bướm của cảm xúc tự do cho đến cái sex mạnh mẽ hài hòa tình cảm và bản năng, đó là con đường đi của yếu tố nhục cảm trong thơ ca Việt Nam từ trung đại qua đương đại! 

 
\textbf{2. Thiên về hướng nội} 
 
Chúng tôi nói Thơ ca đương đại Việt Nam “thiên về hướng nội”, không có nghĩa là nó chỉ toàn  “hướng nội” mà muốn giới thuyết ở phương diện nó “mạnh” hơn, thành công hơn phía kia - thơ hướng ngoại, thơ thông tấn, thơ tự sự… - vẫn đang tồn tại và không phải không có những thi phẩm được người đọc chấp nhận! 
 
Quả thật trong thơ đương đại, thủ pháp xây dựng hình tượng chủ thể trữ tình và nhân vật trữ  tình thường tích hợp làm một. Tác giả thường bộc lộ trực tiếp suy cảm của mình hoặc hóa thân vào nhân vật. Trước đây ta hay bắt gặp trong thơ hình tượng các nhân vật đứng độc lập với đầy đủ các sự tích như là đối tượng thẩm mỹ chính: hoặc một cô lái đò, một bà mẹ chiến sĩ, một chú liên lạc, một người lính chiến, một cô gái chân quê, một anh hùng… Những mẫu nguời cao cả ấy nay chỉ thoáng hiện trong những trang hồi ký, những mẫu người ngày nay mà xã hội đề cao như các doanh nhân, các nhà kỹ thuật số, các nông dân kiêm sáng chế, các chính trị gia cấp tiến… không phải không có, nhưng họ ít được đưa vào thơ như là một đối tượng thẩm mỹ có thể vì nhiều lý do, nhưng một sự thật dễ thấy là các tác giả ưa bộc lộ thẳng tâm tình, suy nghĩ của mình đối với thế sự hơn. Trong các tập thơ tiêu biểu của các tác giả trẻ hoặc trên các trang thơ Văn Nghệ thi thoảng có một vài nhân vật xuất hiện thì đó cũng chỉ là những nhân vật viết theo bút pháp ẩn dụ, tượng trưng thể hiện một thông điệp ngầm của tác giả chứ không phải là một nhân vật thực tế cụ thể-lịch sử. Phải chăng vì thơ đương đại muốn đi sâu khám phá những bí ẩn tâm hồn, những miền tâm tưởng sâu kín, trong những giây phút thoáng qua mà các thể loại khác bất lực. Phải chăng cuộc sống đương đại quá nhiều biến thiên, tấm lòng nhà thơ quá nhiều trắc ẩn mà cách miêu tả như trước đây khó thể hiện được như ý (vì dẫu nhân vật trữ tình trong thơ dù có được miêu tả kỹ lưỡng đến đâu cũng chỉ là những bức ký họa mà thôi khác với các nhân vật trong các loại hình tự sự). Thơ đương đại gần với sự tâm tình, lòng mong mỏi, sự sám hối, lời nguyện cầu hoặc nhiều ra thì cũng là lời tự thán cho vơi nỗi thế nhân, nó không muốn nêu gương mà chỉ mong đồng cảm, nó không ưa phản ánh mà chỉ thích suy ngẫm. 
 
Đọc các  bài thơ bây giờ, ta ít gặp các nhân vật mà chỉ thấy những thoáng tâm tình. Sự đổi thay đó về đối tượng kéo theo một cách thức thể hiện phóng khoáng tự do về hình thức nghệ thuật, không có điều kiện cho một “cách luật” nào được tồn tại và thi thố trên trang chữ của nhà thơ. Nào là thể loại, hình ảnh, ngôn từ… với bao nhiêu quy chuẩn một thời bỗng tan đi rất nhanh nhường chỗ cho những ngữ pháp, thi pháp mới mà chỉ có tâm trạng cá biệt của nhà thơ trong từng khoảnh khắc xúc động quy định. Khi người viết hướng ngoại thì những quy chuẩn khách quan còn tác động lên cách thức miêu tả, còn khi đã hướng nội hoàn toàn thí hình thức thể hiện cũng tự do hoàn toàn. Nó chỉ là cái dạng vật chất của tâm trạng nhà thơ lúc đó  mà thôi. Nó là hình thức những cũng chính là nội dung vậy. 

 
\textbf{3. Cách tân về thủ pháp xây dựng hình tượng, thể tài và ngôn ngữ} 
 
Nói về sự đổi mới Thơ, chúng tôi nghiêng về phía chủ trương trước hết đổi mới cách nhìn, cách cảm “hãy nhìn đời bằng cặp mắt xanh non”. Về nghệ thuật thơ, chúng tôi thấy thơ đương đại đang đổi mới trên hai phương diện, một về việc \textit{đổi mới thủ pháp xây dựng hình tượng, }hai \textit{về sự  đổi mới thể tài cùng kỹ thuật tạo tác câu chữ.} 
  
Thơ đương đại, bên cạnh các tác giả vẫn quen cách viết tùy hứng, các khổ các phần liên kết theo mạch tình cảm, âm hưởng trữ tình bàng bạc suốt bài thơ, nhiều tác giả bây giờ sáng tạo tập trung vào hình tượng chính\textit{, hình tượng tổng thể}, yếu tố trí tuệ chi phối nhiều trí tưởng tượng, cái tứ hình thành từ trước trong đầu nhà thơ sau đó mới hiện trên mặt giấy và được tô điểm thêm. Sáng tác một bài thơ, tác giả ít chú ý các biện pháp đơn lẻ mà định hướng vào “hình tượng tổng thể” bao trùm toàn bài thơ. Từng câu thơ dung dị, từ ngữ đời thường, vần điệu tự do, đọc từng dòng thơ có khi chưa thấy diễn đạt một ý gì, chẳng có hình ảnh gì mới lạ, nhưng đọc xong toàn bài, suy ngẫm  độc giả mới lĩnh hội được cái thông điệp mà tác giả ngầm gửi. Câu cuối trong kết cấu bài thơ nhiều khi trở thành “điểm sáng” tụ kết tư tưởng toàn bài! Nói vậy không phải các thi sĩ đương đại bỏ qua việc xây dựng hình ảnh, các biện pháp tu từ, nhưng nó chỉ như những nét hoa văn, những màu sơn ô cửa, sắc men tường nhà…, các tác giả chú trọng nhiều đến toàn bộ kiến trúc lâu đài cơ!… Trong trào lưu hội nhập, việc chú ý vào thủ pháp xây dựng hình tượng sẽ giúp cho thơ khi phải xa rời cái áo tu từ giàu lợi thế về nhạc điệu và hình ảnh của ngôn ngữ Việt, để khi chuyển sang một ngôn ngữ khác vẫn giữ được vẻ đẹp cơ bản của mình, thu hút được sự đồng cảm của độc giả nhiều khu vực! 
 
Hình tượng trong Thơ đương đại ngoài việc vượt khỏi lối kết cấu tuyến tính duy cảm với những chi tiết nặng tính cụ thể - lịch sử truyền thống, để biểu đạt trọn vẹn sâu sắc các chiêm nghiệm, các suy tư, tình cảm về đời sống, thông qua các thông điệp mà tác giả gửi tới bạn đọc, hình tượng thơ đương đại thường là \textit{một phức thể kết hợp nhiều yếu tố, nhiều biện pháp cả thực lẫn ảo, từ  ý thức đến vô thức, từ khả giải đến bất khả giải}…, những thủ pháp nghệ thuật mà trước đây rất dè dặt. Những dạng thức tồn tại của thế giới được thể hiện có thể chỉ thuần là sự tưởng tượng, suy cảm của tác giả. Thế giới tâm linh trước đây bị bỏ qua, nay nhà thơ có thể đi sâu khai phá. Bằng những thủ pháp này bài thơ tạo được những hiệu ứng thẩm mỹ phong phú đa dạng và mới mẻ ở người đọc! 
 
Xin minh chứng bằng một đoạn thơ gần đây của một tác giả đang rất được giới trẻ hâm mộ, anh viết về Tố Như: 
\begin{blockquote}
 
	\textit{trước mùa trăng sinh nở } 
\textit{	Nguyễn Du là người mộng du ân ái cùng trăng} 
\textit{	nhưng chưa đến nửa đêm thì Truyện Kiều đã viết xong} 
\textit{	và Nguyễn Du đạp mây trở về sông Tiền Đường.} 
\textit{    	để lại một bông trăng thức trong chiếc bình đêm} 
\textit{	thức chầm chậm} 
\textit{	đến sáng thì nở} 
\textit{	nở thành một nàng Kiều trắng trong} 
\textit{	giữa vẫn đục cõi người. } 
\textit{    	khi Nguyễn Du về} 
\textit{	bụi giang hồ } 
\textit{	trần thế vẫn như xưa} 
\textit{	ông lại gặp trăng đêm} 
\textit{	nở một đóa sũng sờ} 
\textit{    	nở chầm chậm đến sáng thì tắt } 
\textit{                      	nở chầm chậm đến sáng rồi chết.} 
	 
		(Nguyễn Việt Chiến – “Trăng Nguyễn Du”, \textit{Tạp chí Thơ}, Xuân 2008) 
 \end{blockquote}
      
Mối liên đới Trăng-Thi sĩ thì đã có nhiều nhà thơ viết và có nhiều thi phẩm bất hủ, song cái “tứ” đầy mộng mị, bất khả giải, vô thức “người mộng du ân ái cùng trăng”, nhà thơ về để lại “một bông trăng” đến sáng thì “nở thành một nàng Kiều trắng trong”… và từ đấy giữa cõi thế nhà thơ  hằng đêm lại gặp một đóa trăng “nở sững sờ, nở chầm chậm đến sáng rồi chết”, thì quả thật độc nhất vô nhị. Cái hình tượng thơ đẹp sững sờ đó được phô diễn cũng bằng một ngôn ngữ, một thể tài rất giản dị, rất tự nhiên, mê hoặc người đọc một cách liêu trai, mở hướng cho một bút pháp thơ nhiều triển vọng! 
 
Về sự đổi mới thể tài và kỹ thuật tạo tác câu chữ, Thơ đương đại nói chung lấy việc hòa nhập đời thường làm tiêu chí, sử dụng vốn từ vựng không câu nệ, lời ăn tiếng nói đời thường kể cả từ địa phương, từ tục, cũng ùa vào thơ rất “nhuyễn”, nhưng để phục vụ cho các thủ pháp xây dựng hình tượng kiểu mới, nếu thơ thời trước chú ý đến cú pháp của dòng thơ, câu thơ, thì  thơ đương đại chú ý cú pháp toàn bài. Để mở rộng trường liên tưởng của người đọc tô đậm cái hình tượng tổng thể, các tác giả sử dụng các dấu chấm, phẩy, viết hoa, sang dòng… rất phóng khoáng. Có khi cả bài là một câu, lại có khi nhiều câu trong một dòng. Lối vắt dòng tạo đột biến trong cảm xúc, các khoảng lặng gây sự chú ý kéo dài. Một điều nữa cũng cần lưu ý đó là sự cách tân các thể thơ. Để xây dựng các hình tượng tổng thể thoáng đạt, giàu sức biểu hiện giàu cá tính, như trên đã nói, các tác giả sử dụng một ngôn ngữ đa dạng, thích hợp với mạch tư duy, mạch xúc cảm của tác giả, các kiểu ngôn ngữ này khó kết hợp bó mình trong các thể thơ cách luật cũ, luôn tìm cách bứt phá, tạo một kiểu kết hợp mới theo khuynh hướng mở rộng , từ đó hình thành các thể thơ tự do, thơ văn xuôi mà sự ràng buộc nhạc điệu chỉ thể hiện trong các kết cấu nội tại tùy biến.	 
  
 
\textbf{4.  Về Thơ trình diễn} 
 
Gần đây một số tác giả tìm cách đổi mới thơ bằng cách kết hợp Thơ với các loại hình khác như âm nhạc, điện ảnh, vũ đạo…  
 
Nói về cái sự “trình diễn thơ” trước số đông thì từ trước đến nay đã có nhiều thể nghiệm. Có người ngâm thơ, người đọc thơ trên nền nhạc. Hát ca trù là một hình thức diễn xướng các bài hát nói của thi sĩ. Thi thoảng có nơi còn có múa phụ họa cho việc đọc thơ hoặc minh họa thơ bằng một đoạn phim... Đó là các hình thức trình bày thơ truyền thống để Thơ đến với độc giả  ngoài cách thức in trên giấy, với nhiều hiệu ứng thẩm mỹ, tựu trung không ngoài mục đích giúp người đọc cảm thụ hình tượng thơ tốt hơn. Dẫu cách thức nào, phối hợp nào thì cũng bám sát và  tô đậm làm cộng hưởng thêm vẻ đẹp và sức lan tỏa của ngôn ngữ thơ! 
 
Còn “Thơ trình diễn” (có người còn gọi là “Thơ đa phương tiện”) thì sao? Nó có cái giống với việc trình diễn thơ ở việc phối kết các thể loại nghệ thuật, nhưng có cái sự khác. Trong nhiều thể nghiệm sự liên kết các loại hình nghệ thuật mà Chủ nghĩa hậu hiện đại nêu lên thì sự pha trộn nhiều thể loại vào một tác phẩm, sự kết hợp đồng thời văn bản ngôn ngữ với màu sắc, âm thanh, hình khối… trong việc sáng tạo được đề cao. Đấy là sự liên kết nội tại trong kết cấu hình tượng nghệ thuật mà không làm mất bản chất đặc trưng của loại thể\textbf{.} Thơ vẫn cứ phải là thơ, văn vẫn là văn, kịch, vẫn là kịch, họa vẫn là họa, nhạc vẫn là nhạc… chứ không phải làm một phép cộng tất cả để thở thành một loại nghệ thuật đa nguyên, nói là gì cũng được! 
 
Chúng tôi may mắn có được thưởng thức một số bài thơ graphic, một số họa phẩm có gắn phù điêu, một vài vở kịch mà nhân vật diễn bằng động tác múa, xen vài hình chiếu trên nền phông vải, một số truyện thiên về ảo giác, tâm thức, chuyển đổi cảnh liên tục như phim ảnh… nhưng tất cả các sự phối hợp đó xét về liều lượng cũng như đặc trưng không bao giờ hòa đồng, đánh mất đặc trưng của loại hình.Thơ thì để đọc( thầm hoặc thành tiếng ), họa thì để xem, nhạc thì để nghe… những phối hợp khác chỉ là để cộng hưởng thêm cho cái hiệu ứng thẩm mỹ chính mà thể loại tác phẩm tạo nên. Bản chất của thơ, văn là nghệ thuật ngôn ngữ. Những phụ gia nào biến chúng thành thứ nghệ thuật khác chắc hẳn thất bại. Thơ chỉ ở thế thượng phong khi bám vào sức mạnh của ngôn ngữ khai thác trí tưởng tượng của người đọc, đi sâu vào địa hạt thầm kín của tâm tư tình cảm của con người mà các thể loại khác không có được lợi thế như nó. Khi nó nhờ vả nhiều hoặc muốn lấn sân thứ nghệ thuật khác thì nó sẽ mất dần sức mạnh cố hữu cùng  nhiệm vụ chân chính, mà chỉ còn là thứ nghệ thuật mua vui hạng hai, thành một thứ tạp kỹ tầm thường, vì tác động vào công chúng bằng âm thanh thì kém xa âm nhạc, bằng ánh sáng và màu sắc không thể sánh với hội họa, điện ảnh… Có chăng nó chỉ đánh vào thói ham thanh chuộng lạ của một bộ phận công chúng nào đó nhưng rồi cũng sẽ qua nhanh. 
 
Thơ không in lên giấy thì anh cứ đọc cho thính giả nghe, đọc có ngữ điệu, công chúng sẽ thưởng thức, sẽ lĩnh hội qua ngôn từ, từ chính ngôn từ làm lan tỏa khơi dậy bao tưởng tượng trong tâm tư người đoc, người nghe. Những thứ phụ gia khác không thể thay sức mạnh của ngôn từ. Bát phở ngon là do nước cốt xương, bánh gạo dẻo,miếng thịt tái ngọt, chứ không phải bày thêm vào thịt gà, trứng tráng hoặc tôm nõn… 
 
Trình diễn Thơ không nên biến thành tạp kỹ thơ, không nên làm công chúng quên mất lời thơ, chỉ nhớ các động tác uốn éo, những âm thành cuồng nộ và mấy đoạn phim gợi tò mò… 
 
\textit{Chủ nghĩa hậu hiện đại}, có những gợi ý hay cho nghệ thuật, nhưng dần cũng bộc lộ nhiều lỗ hổng khi xã hội thay đổi, rồi \textit{Chủ nghĩa kinh điển mới} cũng vậy, tham chiếu nhiều cách nhìn về  xã hội về thế giới về con người mong tái hợp nghệ thuật và kỹ thuật, có lý nơi này và thời này nhưng chưa hẳn đã hợp lý nơi kia và thời kia. Rất trân trọng các tìm tòi, đổi mới, sáng tạo của các tác giả mong tìm một con đường không nhàm chán để đến với  công chúng một cách ấn tượng truyền cảm. Tuy nhiên, đừng vì cái sự lạ  mà quên mất cái sự đẹp, cao nhã và thanh khiết  mà ngôn ngữ thơ tiềm ẩn, đừng đánh mất bản chất Thơ. Các thủ pháp âm thanh, màu sắc, hình khối, ánh sáng... của các thể loại kịch, múa, điện ảnh… khi liên kết với Thơ là một con dao hai lưỡi, cần chú ý đến liều lượng và ý nghĩa thẩm mỹ khi sử dụng, cái khó là nó phải kết hợp từ bên trong kết cấu hình tượng nghệ thuật, bên trong tác phẩm, chứ không phải bên ngoài như một phép cọng tầm thường .  
 
 
\textbf{5. Từ thi bản đến thi phẩm} 
 
Có ý kiến cho rằng thơ ca đang khủng hoảng, cả về phương diện người sáng tác, người phê bình, lẫn người đọc. Người viết thì lai căng, nhà phê bình thì thờ ơ, người đọc thì trượt dài theo lối quen cũ… Chúng tôi cho rằng, thơ Việt không đáng bi quan như vậy! Quả tình có hiện tượng nhiều lúc, nhiều nơi  lạm phát thơ. Quá nhiều mà không hay! Tuy nhiên, suy nhìn cho kỹ, cân nhắc ta sẽ thấy điều này: Nếu đem sánh với thời kỳ trước thì thời nay tuy chưa có tác giả lớn, chưa có những bài thơ tương xứng với thời đại, được công chúng hâm mộ rộng rãi,nhưng thơ hay, được từng bộ phận độc giả yêu thích thì không ít. Bộ phận thích thơ kiểu này, bộ phận thích thơ kiểu kia… Dẫu khó tính đến đâu nếu không cầu toàn và thiên lệch thì sẽ thấy số thơ được người đọc ưa thích, thơ mang cái đẹp mới bây giờ so với thời Thơ Mới hay thời kháng chiến tỷ lệ người đọc ở các bộ phận độc giả cộng lại quả không sút kém, mà chắc trội hơn vì số người có điều kiện tiếp cận, đọc thơ phát triển hơn, điều kiện ấn loát cũng tốt hơn. Đó là về số lựợng, còn chất lượng, chúng tôi cho rằng rất nhiều bài thơ hay bây giờ, nếu bằng một cách đọc không trượt theo lối mòn sẽ cảm thấy thú vị không kém các thi phẩm một thời làm ta yêu thích. Hay nói như một tác giả: trên con đường làm mới thơ ca, mỗi thời kỳ, mỗi trào lưu có một \textit{mã số thẩm mỹ} riêng, người đọc muốn cảm thụ tốt cần biết cách \textit{giải mã} để tiếp cận tác phẩm. Chỉ nguyên một thể tài lục bát mà ta đã thấy biết bao mới mẻ, thích thú và biết bao tác giả thành danh chỉ với một thể tài này!  
 
Trong thời đại mở cửa và có nhiều phương tiện thông tin hiện đại, ngừời viết cũng như người đọc có điều kiện để tiếp cận nhiều khuynh hướng sáng tạo, hoc tập nhiều nền văn hóa tiên tiến. Các nhà văn, nhà thơ phải tiêu hóa tốt các thứ mới mẻ và hữu ích thì viết có nhiều cái hay, còn ai vội vã, sống sít thì viết ít cái được! Điều rút ra bài học là nhà văn trong đời sống hội nhập hiện nay phải biến thành máu thịt những quan điểm triết học, mỹ học tiên tiến, đồng thời cần thâm nhập sâu sắc vào đời sống cộng đồng mới có những sáng tạo giá trị. Nghe thì rất cũ nhưng làm được thì không dễ! Đó là một điều căn bản của quá trình sáng tạo nghệ thuật mà các nhà thơ cần tâm niệm vì rằng: Tác phẩm được viết ra với bao nhiêu mới lạ về câu chữ chỉ là những mới lạ của \textit{văn bản}, bao giờ những mới lạ đó đi vào được cảm xúc của độc giả để họ cộng hưởng \textit{đồng sáng tạo} thì văn bản mới trở thành \textit{tác phẩm văn học} đích thực. Quên điều này nhiều tác giả mới đi được nửa chặng đường sáng tạo, khi cố gắng chủ quan tạo cái lạ cho thơ mình, mà chưa hình dung được nửa thứ hai khi tác phẩm đến với độc giả. 
 
Thơ Việt đang “từ ngôi làng đi ra thế giới” \footnote{
Ý của Nguyễn Khoa Điềm} , nơi ta đến cũng phải là nơi nhân loại đến. Chúng ta cần đi theo cách của mình phù hợp đời sống Việt, tâm hồn Việt, ngôn ngữ Việt, không bảo thủ mà đi sau, không lai căng mà đi lạc! 
 
Chúng tôi tâm đắc với ý kiến của một nhà nghiên cứu \footnote{
Đỗ Lai Thúy: “Nguyễn Đình Thi - Một cánh én bay qua mùa xuân”. \textit{Tạp chí Thơ}, Hội Nhà văn Việt Nam, số 1-2007}  cho rằng năm mươi năm trước đã từng có một /vài cánh én đổi mới bay qua mùa xuân Thơ Việt nhưng rồi vì nhiều lý do đã rẽ ngang nửa chừng. Nay có rất nhiều cánh én trẻ trung chao liệng dẫu chưa có cánh chim đầu đàn, mùa xuân thi ca Việt đang dần hiện? 
 
Trên phương diện lý luận, các nhà thơ trẻ Việt Nam chỉ tự mày mò về lý luận rồi thử nghiệm. Những cây cao bóng cả thì quen với các công cụ mỹ học cũ, những cây bút lý luận mới chỉ dừng lại ở sự vỗ tay hoan hô giới thiệu, họ chưa cập nhật được một cách vững chắc hệ thống thi pháp mới để lập thuyết định hướng cho các tác giả trẻ, bước mạnh mẽ qua các rào cản bảo thủ nhân danh truyền thống. Kinh Thánh có câu: Lỗi tại tôi, lỗi tại tôi mọi điều!...        
 
\textit{Tháng lập thu 2006-2008} 
 
© 2008 talawas        




\end{multicols}
\end{document}