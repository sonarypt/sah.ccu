\documentclass[../main.tex]{subfiles}

\begin{document}

\chapter{Cách làm thơ sống dậy: Đấu Thơ và Thơ Quật}

\begin{metadata}

\begin{flushright}31.7.2008\end{flushright}

Trịnh Thanh Thủy



\end{metadata}

\begin{multicols}{2}

Gần đây trong vài bài viết về sinh hoạt thơ ca Việt Nam của Yến Nhi\footnote{\url{http://www.talawas.org/talaDB/showFile.php?res=12911&rb=0101}}, Hoàng Hưng\footnote{\url{http://www.talawas.org/talaDB/showFile.php?res=11324&rb=0101}}, Hà Linh, Lan Hương, Trịnh Lữ\footnote{\url{http://www.talawas.org/talaDB/showFile.php?res=9110&rb=0101}} v.v…, tôi thấy được vườn lan thơ ca Việt Nam đang ngập ngừng nở bung những đóa lạ, dịu dàng toả hương. Chúng đang dò dẫm những bước nở thử nghiệm. Công chúng yêu và đọc thơ bắt đầu làm quen với đóa lan mới giống ngoại quốc được đem về trồng thử ở vườn nhà với phương cách biến thiên theo địa phương. Đó là nghệ thuật trình diễn thơ hay “Thơ trình diễn” (Performance poetry). Cái mới lạ của “Thơ trình diễn” thu hút được nhiều khán giả đến xem vì phần lớn họ tò mò muốn tìm hiểu thơ ca được đưa lên sân khấu như thế nào, được đọc ra sao. Nhưng hình như nó chưa được phát triển đúng mức hay phổ biến rộng rãi lắm.  
 
Theo các ngòi bút điểm thơ phân tích, điểm mạnh của “Thơ trình diễn” là sự kết hợp đa nguyên giữa các hình thái nghệ thuật bao gồm thủ thuật trình diễn của sân khấu, ngôn ngữ văn bản viết của thơ, cộng thêm sự liên kết màu sắc, âm thanh, phối khí cũng như các tiểu xảo nghệ thuật khác. Việc diễn đạt thơ với nhiều nỗ lực sáng tạo như vậy có thể mang vị trí của “Thơ trình diễn” lên một vị trí nghệ thuật cao. Tuy nhiên, việc thực hiện và phổ biến nó trong một xã hội mới bắt đầu mở mang như của ta hiện nay là một việc không phải dễ và khá phức tạp. 
 
Để phong phú hoá vườn hoa thơ ca Việt Nam, tôi xin giới thiệu đến các bạn một giống hoa lạ, một phương cách bón phân mới làm hoa cũ tươi hơn, làm sinh hoạt thơ ca èo uột vươn mình sống dậy, nếu chúng ta biết cách biến chế và gọt tỉa nó cho hợp với khí hậu và thổ ngơi của địa phương mình.  Đó là Thơ Quật (Slam Poetry) và những cuộc Đấu Thơ (Poetry Slam). 
 
Khi đạo diễn kiêm nhà viết truyện phim Paul Edward ở Mỹ quyết định đem Thơ Quật (Slam Poetry) lên sàn đấu của nghệ thuật thứ bảy là điện ảnh, thì khán giả khắp nơi trên thế giới bắt đầu biết đến những cuộc Đấu Thơ (Poetry Slam) rộng rãi hơn. Với tựa đề “Fighting Words”, phim “Đấu chữ” đã bật sáng “những thi sĩ đường phố”, nâng cao phong trào “Đấu thơ” và đem thơ ca đến giới thưởng lãm nghệ thuật như một tặng vật ý nghĩa đầy hương sắc.  
 
Đấu thơ hay đấu thơ quật là một hiện tượng đương đại đang bung nở khắp nơi ở Hoa Kỳ, Âu Châu, Canada và vài nơi ở châu Á tựa một đoá hoa tư tưởng biết nói của thế kỷ. Thi sĩ từ những hẻm tối, góc phố, quán cóc, bar rượu bước lên sân khấu trình diễn thơ mình bằng cảm xúc sưng tấy trên khuôn mặt, bồng bế lên hai tay, phát sóng nơi giọng đọc, xuất thần trong ánh mắt và đôi lúc lâm vào trạng thái mê sảng như những kẻ ngồi đồng. Họ dự thi với những bài thơ tự làm đã soạn sẵn hay có thể phải tự làm ngay trong cuộc đấu. Thời gian hạn định chỉ có ba phút ngắn ngủi phù du để tự suy nghĩ, sắp chữ, cô đọng những tư duy, vui buồn, cảm xúc thành một bài thơ, để quyết đấu, giành thắng và giật giải.  
 
Vì thế Bob Kaufman đã nói “Mỗi cuộc đấu thơ là một kết thúc”.   
 
Gregory Corso lại hỏi “Tại sao bạn muốn kết bạn với những người già như chúng tôi làm gì, hỡi bạn trẻ? Nếu tôi còn trẻ, tôi đi dự đấu Thơ Quật.”  
 
Thơ Quật là thơ gì? Sao Gregory Corso lại nói trẻ nên đi đấu Thơ Quật? Đấu Thơ Quật có phải dành cho giới trẻ không? 
 
Chúng ta thử tìm hiểu xem Thơ Quật là gì và người ta đem Thơ Quật ra đấu ở các quán cà phê như thế nào. 
 
Khi dùng từ Thơ Quật (Slam Poetry) người ta hình dung tới một loại thơ hip hop được các thi sĩ đọc lớn lên, được diễn tả bằng khuôn mặt, cử chỉ, nhân dáng, bằng những ngôn từ chỉ trích cay độc và có khi pha đầy tục tằn. Phân tích đơn giản hơn, chúng ta có thể phân rõ sự khác biệt giữa Thơ Quật và thơ viết thường thấy là, \textit{Thơ Thường để đọc} và \textit{Thơ Quật để nghe. } 
 
Poetry Slam (Đấu Thơ Quật) theo định nghĩa của Wikipedia là một cuộc thi thơ tại chỗ, ở một nơi mà những thi sĩ có thể đọc to, trình diễn thơ mình trên một sân khấu, có khán giả xem và chấm điểm trong một thời gian hạn định. Yếu tố sống động của cuộc thi thơ này là sự khảo sát và chấm điểm của người dự thính. Giám khảo được chọn ra từ khán giả đến nghe thơ. Người dự thi có điểm cao nhất sẽ thắng. 
 
Cuộc thi nào cũng ràng buộc với những luật lệ. Đi sâu hơn vào các cuộc đấu Thơ Quật ta sẽ thấy nhiều điểm hay của nó. Mỗi nơi có một số những luật chơi riêng nhưng căn bản vẫn là:  
\begin{itemize}

item{Bài thơ phải do chính thí sinh làm, bất kể đề tài hay thể loại nào. Mỗi bài thơ chỉ có thể dùng một lần vào mỗi cuộc đấu mà thôi.


}

item{Các thi sĩ dự đấu không được hóa trang (mũ, mão, xiêm áo, kể cả tóc giả), không được dùng vật hỗ trợ (nhạc khí, đồ chơi... chẳng hạn). Thú vật và sự trợ lực của người khác cũng bị cấm. Họ có thể dùng những gì ban tổ chức cho sẵn như sân khấu, ghế, bàn, lối đi, micro v.v… Những gì mà các thí sinh khác có cũng được cung cấp đồng đều. Những cử chỉ và điệu bộ của thân hình cũng bị cấm.


}

item{Thời gian hạn định không vượt quá ba phút. Trước khi đồng hồ điểm, thí sinh có thể thử và điều chỉnh micro. Nhưng khi đồng hồ bắt đầu đếm, cuộc trình diễn chỉ có thể diễn ra trong vòng ba phút. Nếu vượt quá, điểm sẽ bị trừ vào tổng số điểm đạt được. Thành phần ban giám khảo gồm có 5 người được chọn ra từ khán giả và được chỉ dẫn để biết cách cho điểm. Thông thường họ được khuyên nên cho theo tiêu chuẩn được áp dụng là phân làm hai, một nửa cho bài thơ, nửa kia cho cách diễn đọc. Các giám khảo được hướng dẫn nên làm việc tích cực, cương quyết và công bằng đối với mọi thí sinh và mỗi vòng đấu. Điểm được cho từ 0 tới 10. Tất cả thí sinh, ai biết làm thơ cũng có thể ghi danh thi đấu và trình diễn thơ trong vòng đầu. Thường thì phân nửa tổng số thí sinh ở vòng đầu và vòng nhì được chọn vào vòng kế tiếp là vòng thứ ba, cuối cùng là chung kết. Khán giả có thể tương tác với thi sĩ hay ban giám khảo khi họ thấy cần. Họ được quyền phản đối hay hoan hô trong suốt buổi diễn hay sau buổi diễn. Tỷ như ở quán Green Mill ở Chicago, nơi xuất phát đầu tiên của Thơ Quật, khán giả được chỉ dẫn cách thức thế nào để biểu lộ sự phản đối một cách cấp tiến khi họ không thích thi sĩ, kể cả cách búng tay, dậm chân và những cách cổ võ, khích lệ khác nhau. Nếu khán giả diễn tả sự bực bội của họ với một thi sĩ đến một mực độ nào đó, người thi sĩ sẽ rời bỏ sân khấu, kể cả lúc họ chưa trình diễn xong bài thơ của họ.


}

item{Tùy thuộc vào mỗi thi sĩ và tùy vào cuộc đấu, có nhiều đề mục và nhiều loại thơ khác nhau được diễn đọc. Cái hay nhất của cuộc đấu là sự phong phú của sự chọn lựa, thi sĩ toàn quyền xử dụng mọi thể thơ. Bạn có thể tìm ra sự khác biệt trong thơ các thí sinh. Thơ tình yêu, thơ đối kháng, xã hội, hài hước, chỉ trích, phê bình, thể loại nào cũng có. }

\end{itemize}
 Làm thế nào để thắng? 
 
Thắng một cuộc đấu thơ thật cam go và đòi hỏi nhiều tài năng, kỹ thuật và kể cả thật nhiều số đỏ. Được sự tán thưởng và yêu thích của ban giám khảo, khán giả qua mấy vòng đấu là cả một kỳ công. Bạn có thể là người thắng cuộc trong cuộc đấu tuần này nhưng tuần khác là kẻ chiến bại. Không có một công thức nào để thắng một cuộc đấu thơ dù cho bạn là một thi sĩ giỏi và có kinh nghiệm diễn đọc thơ xuất sắc - chỉ nên thực tập, thực tập và tiếp tục thực tập.  
 
Những cuộc đấu thơ bắt đầu tự bao giờ? 
 
Đó là năm 1984, một nhà thơ xuất thân từ một người thợ xây dựng tên là Marc Smith đã đọc thơ mình lần đầu ở hộp đêm “Get Me High Lounge” chuyên chơi nhạc Jazz ở Chicago. Trong một hướng mới nhấn mạnh đến việc trình diễn thơ trên sân khấu, ông đã thổi luồng sinh khí lạ vào sinh hoạt thơ ca. Thơ đã được đặt xuống sàn diễn, nhảy lên mic uốn éo, nhập hồn trên khuôn mặt diễn xuất nhập thần của tác giả rồi thăng hoa thành làn khói tỏa hương tấp vào khán giả. Thi sĩ đã nhập cuộc chơi, tung hê, lên đồng với chính điệu thơ, cung chữ do tim óc mình vắt ra. Khán giả được mời vào sân chơi, thưởng thức, phê bình, cho điểm và khen thưởng bằng những giải thưởng tượng trưng nhưng đầy khích lệ tinh thần. Giống như những cuộc tranh tài thi thơ thời Hy Lạp cổ, Thơ Quật ra đời từ đấy. 
Nói đến các thi sĩ Thơ Quật, chúng ta phải nhắc tới  một thi sĩ Thơ Quật nổi tiếng là Bảo Phi đã 2 lần thắng vẻ vang ở quán cà phê Nuyorican Poet's Café\footnote{\url{http://www.talawas.org/talaDB/http://nuyorican.org/}}, New York và 2 lần đoạt giải The Minnesota Grand Poetry Slam. Bảo Phi là người Việt Nam duy nhất được xuất hiện trên đài truyền hình HBO trong chương trình Russell Simmons Presents Def Poetry\footnote{\url{http://www.talawas.org/talaDB/http://www.hbo.com/defpoetry}}. Đây là một chương trình được khởi chiếu liên tục hàng tuần theo mùa, mục đích để giới thiệu các thi sĩ thơ quật trình diễn thơ của mình trước đám đông khán giả. Ngoài ra anh cũng là người đứng thứ 6 trên 250 thi sĩ thơ quật toàn quốc trên sân khấu của giải Thơ Quật toàn quốc National Poetry Slam. 
Sinh ra tại Sài Gòn, rời Việt Nam tháng 4 năm 1975, Bảo Phi lớn lên tại tiểu bang Minneapolis, Hoa Kỳ. Tốt nghiệp tại đại học Macalester anh viết sách, làm thơ. Những tác phẩm của anh là: \textit{Surviving the Translation}, \textit{From Both Sides Now: The Vietnam WAr and Its Aftermath in Poetry} (Scribner, 1998), \textit{The Def Poetry Jam Anthology } (Three Rivers, 2001), \textit{Legacy to Liberation: Writings from Revolutionary Asian America} (Bid Red Media, 2000), and \textit{Screaming Monkeys: Critiques of Asian American Images} (Coffeehouse, 2003).  Anh có ra hai CD, \textit{Flares} và \textit{Refugeography}.  
 
Bài thơ "Race" của anh đã được Billy Collins chọn vào tuyển tập “Những bài thơ Mỹ hay nhất năm 2006” (\textit{Best American Poetry 2006}\footnote{\url{http://www.talawas.org/talaDB/http://bestamericanpoetry.com/archive/?id=20}}). Chúng ta có thể tạm nghe anh trình diễn bài thơ quật “Race” ở đây http://www.imeem.com/shadowofpoetry/music/C467X9eg/bao_phi_race/\footnote{\url{http://www.talawas.org/talaDB/http://www.imeem.com/shadowofpoetry/music/C467X9eg/bao_phi_race/}}  
hoặc  http://www.youtube.com/watch?v=ui3yq2b9GV0\footnote{\url{http://www.talawas.org/talaDB/http://www.youtube.com/watch?v=ui3yq2b9GV0}} 
 
Đây là một bài thơ về tình yêu của Bảo Phi do tôi chuyển ngữ: 
\begin{blockquote}
 
\textbf{Ánh sáng} 
        
Vào đêm        
khi vạn vật chuyển hình ngàn mảnh vỡ        
của ánh sáng        
và vòng cung đen,        
xắt những lát cắt bóng tối        
thành những góc độ, 
        
Tôi hiểu rằng tình yêu của em chính là chiếc đồng hồ thái dương của tôi:        
mặt trời kéo về và vầng trăng đã phủ nhiễu 7 lần        
từ lần cuối em hôn tôi,        
sự rạng rỡ cũng biến thiên 
        
Tôi trườn xuống        
kiếm tìm ánh sáng vành cung        
góc tai em,        
tôi ẩn mình        
trong hang động ngon mềm nhỏ bé phía lưng em        
hạnh phúc đường cong ấy đủ chống đỡ nuôi nấng tôi 
        
Tôi thấy em khi ánh huỳnh quang tiếp tục toả sáng        
trần xe điện ngầm và ánh xanh nhấp nháy        
trong hầm tối vụt biến thành vạn vật, 
        
người đàn ông Trung Hoa        
đứng nơi rãnh đường ở trạm xe, quá gần        
với phố Tàu,        
lằn xếp trên khuôn mặt quen thuộc ông ta        
lưu giữ các mẩu chuyện nhiều hơn         
những ám hiệu rực rỡ        
từ những chiếc túi miễn cưỡng        
của người cho tiền, đầy ắp bao đựng vĩ cầm của ông        
Ông nhìn tôi, cả hai đều hiểu         
dù cho những đồng xu có tự chôn mình thành sở hữu của ông,        
ông cũng không bao giờ mua được sự đổi thay 
        
Tiếng thầm thì yêu em biết là bao         
văng vẳng trong bóng tối        
và khi tôi không chịu đựng nổi nữa         
tôi lái đi tìm em        
quầng sáng mắt đèn xe tôi nhắm hướng ngàn sao        
những dấu trừ gạch trên        
con đường cao tốc xén bớt khoảng trống giữa hai ta         
tôi đến để hiểu rằng ban đêm         
sự thiếu vắng ánh sáng         
là khoảng cách em        
ước muốn của tôi        
được quỳ trước em        
vầng sao xa trong khứu giác        
miệng khát        
ánh mặt trời em        
Tôi cũng xin chạm ánh rực rỡ        
bắt được từ vòm mun tóc em        
bằng những ngón tay tôi 
        
Nói với em tôi sẽ nói        
tôi cần tia xạ quang vừa nhô khỏi chân trời da em 
        
Tôi vẫn nói tôi cần niềm trong sáng         
cuồn cuộn từ hơi thở        
của em 
        
Và tôi sẽ bảo em rằng tôi cần niềm vui        
em tuyệt mỹ không thể tả bằng lời 
        
Lần nữa tôi bảo tôi cần        
khẩn xin thì thầm một ounce (gờ-ram) rực rỡ 
        
Ánh sáng duy nhất trên đầu những ngón tay em        
xoắn ốc trong tôi 
        
kể cả khi tôi nhắm mắt        
em nhuộm        
phía sau mí mắt tôi màu cam 
        
tôi chợt nhận ra rằng        
đôi mắt vụt trở thành ánh sáng cuối đường hầm        
chỗ tôi khởi đầu 
        
những nụ hôn hoá thành những đồng hồ thái dương 
        
đôi bàn tay trở nên kinh và vĩ tuyến        
đuổi bắt bóng mặt trời xuyên suốt vùng thời gian 
        
những đầu ngón tay đổi dạng các vì sao 
        
tình yêu em chuyển màu ánh sáng 
        
8/9/99-9/13/99 
        
Bảo Phi        
(Trịnh Thanh Thủy chuyển ngữ) 

\end{blockquote}
 
Ngày nay đấu Thơ Quật không những được tổ chức ở New York, Chicago, Los Angeles, San Francisco, North Dakota, Alaska, Fairbank v.v. mà nó còn xảy ra ở nhiều nơi trên thế giới như Anh Quốc, Thụy Điển, Phần Lan, Âu Châu, Canada và cả Á Châu như Singapore. Hằng năm, giải Thơ Quật Toàn Quốc (National Slam) hấp dẫn được rất nhiều thi sĩ và khán giả từ khắp nơi về tham dự. Các cuộc thi bao gồm cả thi đơn và thi đội. Con số các thành phố gởi người về thi đấu rất đông. Năm 1992 có 17 thành phố gởi đại diện về tranh giải cả đơn lẫn đội. Năm 1993, có tới 23 đội từ Victoria B.C ở Canada, Phần Lan (Âu Châu), Seattle, Portland, Los Angeles, Washington D. C. v.v. 
 
Năm 1996, 27 đội tìm về tranh giải, trên 3000 khán giả dự khán ngày chung kết. Số đội tăng lên 33 năm 1997. Năm 2000 có 56 đội.  
 
Các bạn có thể xem các cuộc đấu thơ quật toàn quốc hay quốc tế qua các link nêu ở dưới trong mục “Tài liệu tham khảo”. 
 
Đấu Thơ Quật ngày một cực thịnh. Ai bảo người ta đã quên thi ca? Thế giới vẫn biết đến nó đấy chứ.  Vấn đề là chúng ta phải biết cách làm nó sống dậy những nơi nó đang chết. Biết cách bày nó lên sân khấu bằng cách này hay cách khác miễn cách đó hữu hiệu. Thế giới đang sống lại với thi ca bằng các cuộc đấu Thơ Quật, tại sao Việt Nam ta không thử làm như vậy xem sao nhỉ? Bắt đầu từ các quán cóc, hè đường, góc phố,đọc thơ, đấu thơ. Những quán cà phê tao đàn, hội quán sinh viên, những buổi sinh hoạt trường lớp, ở đâu có sinh khí văn nghệ, ở đấy có thể tổ chức những cuộc đấu thơ. Miễn ở đó có khán giả, có thi sĩ, một cuộc đấu thơ có thể ra đời. Luật chơi có thể dựa theo luật Thơ Quật hoặc du di biến cải theo nhu cầu địa phương hay hạn chế theo điều kiện an sinh khu vực. Tôi những mong vườn lan thi ca Việt Nam được rộn ràng, bung nở đầy vườn những đóa lan xuân. 
 
\textbf{Tài liệu tham khảo}  
\begin{itemize}

item{Bao Phi 
http://www.baophi.com/\footnote{\url{http://www.talawas.org/talaDB/http://www.baophi.com/}} }

item{Beau Sia Asian Invasion on HBO Def Poetry Jam 
http://www.youtube.com/watch?v=diNLPGHZbGM&feature=related\footnote{\url{http://www.talawas.org/talaDB/http://www.youtube.com/watch?v=diNLPGHZbGM&feature=related}} }

item{National Youth Poetry Slam Finalists "This Is For You" 
http://www.youtube.com/watch?v=jOv47njeLHQ&feature=related\footnote{\url{http://www.talawas.org/talaDB/http://www.youtube.com/watch?v=jOv47njeLHQ&feature=related}} }

item{German International Slam 2007: Marc Uwe Kling rocks! 
http://www.youtube.com/watch?v=R5XHzFXATaE&feature=related\footnote{\url{http://www.talawas.org/talaDB/http://www.youtube.com/watch?v=R5XHzFXATaE&feature=related}} }

item{Light 
http://www.asianamericanpoetry.com/spotlight_poem_display2.php?poem=Bao_Phij7wkakms8v\footnote{\url{http://www.talawas.org/talaDB/http://www.asianamericanpoetry.com/spotlight_poem_display2.php?poem=Bao_Phij7wkakms8v}} }

item{Poetry Slam 
http://en.wikipedia.org/wiki/Poetry_slam\footnote{\url{http://www.talawas.org/talaDB/http://en.wikipedia.org/wiki/Poetry_slam}} }

\end{itemize}
 
 
© 2008 talawas 
\end{multicols}
\end{document}