\documentclass[../main.tex]{subfiles}

\begin{document}

\chapter{Giai thoại của thi sĩ}

\begin{metadata}

\begin{flushright}1.8.2008\end{flushright}

Bùi Chí Vinh



\end{metadata}

\begin{multicols}{2}

Vừa rồi, sau khi ra mắt hai tập \textit{Thơ tình Bùi Chí Vinh} và \textit{Thơ đời Bùi Chí Vinh} trong nước lẫn trên mạng, tình cờ tôi được nghe một số giai thoại hay hoặc không hay của thiên hạ bàn tán về mình. Ðối với tôi, hay hoặc không hay đều vẫn là giai thoại. Nhưng giai thoại phải có cơ sở xác đáng, có thực tế chứng minh, có những người trong cuộc chứng kiến thì giai thoại đó mới trường tồn, truyền khẩu hợp lý và khoa học được. Những giai thoại đồn đại chung quanh hình tích, sự đi đứng, năng khiếu làm thơ ứng khẩu của tôi xuất hiện ngay từ sau giải phóng, lúc tôi còn rất trẻ, đang làm việc tại một tờ báo và chỉ mới 21 tuổi đầu. Giai thoại mỗi ngày mỗi phát triển thêm lúc tôi đi bộ đội, rồi đi giang hồ, rồi làm đủ mọi thứ nghề để sống, thậm chí cả giai thoại lúc tôi bày tỏ chính kiến của mình… 
 
Trong phạm vi bài viết này tôi xin mở đầu bằng một giai thoại quái đản nhất vừa nghe được. Sự quái đản ở đây tuỳ nghi ai muốn hiểu sao thì hiểu trong khi câu thơ đồn đại về tôi lại khởi nguồn từ một tình bạn rất đẹp. Cụ thể từ hai câu thơ tam sao thất bổn sau đây: 
\begin{blockquote}
        
\textit{Trọc đầu BÙI làm sao CHÍ ở} 
\textit{Nhục còn chưa có lấy gì VINH} 

\end{blockquote}
 
Và họ nói rằng hai câu thơ trên là do Bùi Giáng ứng khẩu tặng tôi trong bàn nhậu lúc tôi đang múa may chữ nghĩa, khiến tôi hoàn toàn tâm phục khẩu phục. Suy nghĩ như thế không riêng gì tôi mà những người quen biết tôi đều phải phì cười. Bởi một lẽ đơn giản, tác giả hai câu thơ trên không phải là Bùi Giáng tiên sinh mà là ông anh Mặc Tuyền, một nhà thơ kiêm kịch tác gia bụi đời làm “chọc quê” tôi khi tôi mạt lộ đang ngồi ở vỉa hè phụ sửa xe cùng anh Phan Văn Bồng, tự Bế Văn Bồng mưu sinh kiếm sống vào thời điểm cuối thập niên 80 đói rách. Thời điểm ấy nạn dịch bo bo khoai mì hoành hành, mâm cơm không có gạo trắng mà ăn, Mặc Tuyền cố kiềm chế sự ngông cuồng của tôi nên làm hai câu khá cảm động. Vừa chơi chữ, vừa nói về chữ “nhục”, nhục ở đây có nghĩa là “thịt”, thi sĩ lớn cỡ nào mà đầu cạo trọc và thiếu thịt ăn thì bao tử cũng đói meo và chí khí lẫn chí mén cũng đi chơi chỗ khác. 
 
Còn Bùi Giáng tiên sinh đương nhiên thuộc về đẳng cấp khác. Ông và tôi không phải huynh đệ hoặc thân thiết tri kỷ, nhưng khi gặp nhau chưa bao giờ ông dèm pha biếm nhẽ thế hệ sau mình. Giai thoại giữa tôi và ông độc đáo hơn nhiều. Cách đây hơn 20 năm, tôi và Hồ Lê Thuần (con trai cố bí thư Thành Ðoàn trước 1975 là Hồ Hảo Hớn) vi hành xuống miệt Gò Vấp chợ Long Hoa lúc nửa đêm. Nhằm vào lúc Bùi Giáng rời chùa Long Huê gần đó ra chợ quậy tưng bừng khói lửa với một cây chổi rách tượng trưng cho ấn kiếm. Gọi là ấn kiếm vì Bùi Giáng luôn vỗ ngực xưng vương bất cứ lúc nào cao hứng. Ðêm đó chúng tôi ngồi uống rượu vỉa hè chứng kiến “vua cỏ” Bùi Giáng làm bà con chạy tán loạn và nhìn ông múa chổi tiến về phía chúng tôi. Ông vừa đi vừa khạc thơ rồi dòm trừng trừng vào mặt tôi. Trong cơn say xỉn ngất trời, Hồ Lê Thuần xúi tôi đọc thơ đáp lễ. Thế là người ngồi người đứng xuất khẩu thành thi qua lại liên tục. Không biết Bùi tiên sinh “phê” thơ tôi ra sao, chỉ biết Người tự động quỳ xuống bàn chúng tôi dâng cây chổi rách lên và tuyên bố “\textit{Ðêm nay Trẫm thay mặt cựu hoàng Bảo Ðại giao ấn kiếm cho thế hệ Hồ Chí Minh”. }Câu nói đầy tính “chính trị” và đối phó của Bùi Giáng bắt buộc tôi phải nhận cây chổi và làm một bài thơ tặng ông tại chỗ, có chép lại nhét túi ông đàng hoàng, xin mạn phép ghi ra đây để khép lại lời đồn về sự “tâm phục khẩu phục” của tôi trước Bùi Giáng: 
\begin{blockquote}
 
\textbf{Cách lạy của Bùi Giáng} 
        
\textit{Liên tồn, l… tiên, liền tôn}        
\textit{Bác Bùi chưa gặp đồng môn đây mà}        
\textit{Ta hăm bảy tuổi đăng khoa}        
\textit{Bác hơn sáu chục mới là Trạng Nguyên}        
“\textit{Bác đi, bi đát}”\textit{ cơn điên}        
\textit{Ðể mua trí tuệ }“\textit{l… tiên, liên tồn}”        
“\textit{Riêng ta}”\textit{ thành }“\textit{ra tiên}”\textit{ con}        
\textit{Lúc say xỉn vỗ hậu môn cười khà}        
“\textit{Bán dùi Bùi Giáng}”\textit{ xót xa}        
“\textit{Bình Chí Vui}” \textit{ta vốn là }“\textit{Bùi Vinh}”        
\textit{Bác không màng nhắc triều đình}        
\textit{Có đâu ta nỡ cố tình làm vua}        
\textit{Chi bằng giữa chợ say sưa}        
\textit{Bùi to Bùi nhỏ đi lùa các em}        
\textit{Kìa sao bác lạy như điên} 
\textit{Ðợi ta đỡ dậy chiêu hiền nữa sao?!?} 

\end{blockquote}
 
Ngoài ra tôi còn chép cho Bùi tiên sinh bài thơ BÌNH CHÍ VUI khi ông muốn tôi bình tĩnh chí nam nhi trở lại để có thể tồn tại trước bọn sâu bọ làm người. Tôi đã làm bài thơ này theo “môđen” tiếng lái và chơi chữ của ông: 
\begin{blockquote}
 
\textbf{Bình Chí Vui} 
        
“\textit{Bùi Chí Vinh, Bình Chí Vui}”        
\textit{Không bình chí, chắc tiếng cười mất tiêu}        
\textit{Chí trong bình, chí mốc meo}        
\textit{Chui ra bình, chí mới nhiều nhục vinh}        
\textit{Bùi làm thiên hạ giật mình}        
\textit{Sờ ngay }“\textit{cái đó}” \textit{kẻo em mếch lòng}        
“\textit{Bùi như lạc}” \textit{nhậu sướng không?}        
“\textit{Trần như nhộng}”\textit{ Bùi tồng ngồng đái chơi}        
\textit{Bất bình nên chí chưa vui} 
\textit{Các em nên gọi ông Bùi Chí Vinh} 

\end{blockquote}
 
Chuyện gặp Nguyễn Ðức Sơn giang hồ hơn. Trước đó khi mạn đàm về thi ca cùng đồng nghiệp, tôi luôn luôn khẳng định miền Nam trước đây có 4 chưởng môn nhân đại diện cho 4 trường phái thi ca tiêu biểu. Ðó là Bùi Giáng thơ trên trời, Nguyễn Ðức Sơn thơ dưới đất, Thanh Tâm Tuyền thơ tự do kiểu Tây phương, Tô Thuỳ Yên thơ hành cổ điển kiểu Ðông phương. Vì thế lần hạnh ngộ Nguyễn Ðức Sơn trên cao nguyên Ðại Lào sơn lam chướng khí, tôi đã ăn những gì ông tự trồng tự hái và đã đấu khẩu những gì ông muốn. 
 
Nguyễn Ðức Sơn vốn sở trường thơ lục bát và nổi tiếng trước những bài thơ tinh gọn đến mức độ tối thiểu về chữ mà vẫn dào dạt ý tứ. Có bài thơ chỉ hai câu, mỗi câu hai chữ như “Cái lỗ – Tối cổ” đủ nói hết về chế độ mẫu hệ, về nơi khai sinh ra loài người. Có bài thơ chỉ ba câu, mỗi câu một chữ như “Hột – Thì – Le” đủ nói hết về bản chất sinh tồn thiện ác của nhân loại. Và tôi đã mượn những ý thơ độc đáo đó để đưa vô bài thơ làm tặng ông như một thứ giai thoại truyền khẩu: 
\begin{blockquote}
 
\textbf{Ðụng độ Nguyễn Ðức Sơn} 
        
“\textit{Hột thì le}” \textit{thật đó sao?}        
\textit{Ta dân }“\textit{thảy lỗ}” \textit{đến chào đồng môn}        
\textit{Xưa nay hai kẻ du côn}        
\textit{Ít khi đời sống cô hồn như nhau}        
\textit{Như miếng trầu khác miếng cau}        
\textit{Nhưng có cau, chẳng có trầu, như không}        
\textit{Như không sinh chuyện động phòng}        
\textit{Hột sao le được }“\textit{nụ hồng thi ca}”        
\textit{Như không sinh nở đàn bà} 
“\textit{Cái lỗ tối cổ}” \textit{thành ra tầm thường} 
        
\textit{Ta thừa văn, bác dư chương}        
\textit{Hôm nay một chén Hồ Trường chao nghiêng}        
“\textit{Thiên tài}”\textit{ nhờ lỗ }“\textit{tai thiền}”        
\textit{Buồn lên núi hú chẳng phiền Tarzan}        
\textit{Buồn hái nấm luyện thành sâm}        
\textit{Buồn quay vào vách thương thầm Ðạt Ma}        
\textit{Buồn hơn xuống động bẻ hoa} 
\textit{Buồn hơn chút nữa kiếm ta đỡ buồn} 
        
\textit{Kiếm ta ta cứ ngông cuồng}        
\textit{Sánh vai với Nguyễn Ðức Sơn cũng kỳ}        
“\textit{Kỳ}”\textit{thì theo }“\textit{Thiệu}”\textit{ mà đi}        
\textit{Ta theo bác đã chắc gì tịnh tâm}        
\textit{Chẳng thà bút vẩy thơ đâm}        
\textit{Rong chơi đợi trận cát lầm đi qua}        
\textit{Ðừng khen chê trước mặt ta}        
\textit{Sợ e tiếng gáy làm gà ghét nhau}        
\textit{Chẳng thà trong cuộc bể dâu} 
\textit{Cưa nhau chén rượu cho sầu chia hai…} 

\end{blockquote}
 
Riêng đối với Phạm Thiên Thư thì tôi “quậy” theo kiểu bụi đời hơn. Năm 1980 tôi được nghỉ phép, mặc đồ bộ đội rách xác xơ đi lang thang cùng Hoàng Linh qua đường Lý Chính Thắng (tức Yên Ðỗ cũ). Hoàng Linh là bạn giang hồ của tôi, anh là con trai nhà văn Hoàng Ly và là em vợ Phạm Thiên Thư lúc đó. Anh giới thiệu tôi với Phạm tiên sinh đang mở tiệm hớt tóc và bỏ mối rượu ngay trên đường này. 
 
Cuộc hội ngộ diễn ra y chang truyện kiếm hiệp của Kim Dung. Ngoài trời mưa tầm tã, bên trong tiệm Phạm tiên sinh ngừng hớt tóc cho vị thân chủ mặt mũi kỳ dị và kéo vị ấy ngồi xuống rót chai rượu màu xanh tiếp tôi và Hoàng Linh. Sau tuần rượu đầu đàm đạo về thơ, vị khách lạ đứng dậy chỉ vào mặt tôi và phán “\textit{tuổi Giáp ngọ phải không, sanh vào cuối tháng 9 âm lịch phải không, chào đời nửa đêm phải không?}” Rồi ông ta đứng dậy bỏ đi một mạch. Lời phán của kẻ dị nhân khiến tôi hoang mang nhưng hai anh em Phạm Thiên Thư, Hoàng Linh chỉ khẽ gật gù khoái trá. Ðến giờ này tôi vẫn chưa hiểu dị nhân đó là ai và tại sao chỉ sau một quẻ Dịch ông ta lại biết ngày giờ năm sinh tháng đẻ của tôi trong khi cả bàn không ai biết. Hôm đó trong lúc cụng ly nghe Phạm tiên sinh thố lộ về cuộc đời trôi nổi lên voi xuống chó của ông làm tôi ngậm ngùi vô tận. Tôi thừa biết họ Phạm sở trường về thơ bốn chữ nên ứng khẩu tặng ông bài thơ cùng thể loại mà ông ưa thích. Bài thơ như sau: 
\begin{blockquote}
 
\textbf{Ghẹo Phạm Thiên Thư} 
        
\textit{Rượu Phạm Thiên Thư}        
\textit{Thơ Bùi hiền sĩ}        
\textit{Một chén càn khôn} 
\textit{Ðất trời tuý luý} 
        
\textit{Tưởng huynh tên }“\textit{Thị}”        
\textit{Nên mới vào chùa}        
\textit{Dè đâu tửu sắc} 
\textit{Cũng ghiền nam mô} 
        
\textit{Huynh giữ một bồ}        
\textit{Chứa toàn thịt chó}        
\textit{Ta giữ bồ kia} 
\textit{Chứa toàn tín nữ} 
        
\textit{Vì huynh quân tử}        
\textit{Như Nhạc Bất Quần}        
\textit{Ta đành tiểu tử} 
\textit{Như Ðiền Bá Quang} 
        
\textit{Tiếu Ngạo cung đàn}        
\textit{Một gian lều cỏ}        
\textit{Huynh mới bẻ gươm} 
\textit{Ta còn mãi võ} 
        
“\textit{Ðoạn Trường}” \textit{hai chữ}        
\textit{Huynh ngâm nát lòng}        
“\textit{Vô Thanh}” \textit{đâu chứ} 
\textit{Cửa thiền huynh trông} 
        
\textit{Ta con nhà tông}        
\textit{Giống lông giống cánh}        
\textit{Quen ngủ chiếu rơm} 
\textit{Dùng cơm khổ hạnh} 
        
\textit{Gặp chiều mưa lạnh}        
\textit{Chén tạc chén thù}        
\textit{Ðem thơ tặng Phạm} 
\textit{Ðếch cần Thiên Thư!} 

\end{blockquote}
 
Cũng trong thời gian đó, tôi lang bạt rất nhiều nơi, làm quen với nhiều người, trong đó có thi sĩ Nguyễn Bắc Sơn từ Phan Thiết vô là tác giả tập thơ \textit{Chiến tranh Việt Nam và tôi}\footnote{\url{http://www.talawas.org/talaDB/showFile.php?res=6057&rb=0106}} nổi tiếng.  
 
Nguyễn Bắc Sơn có hẹn hò đâu với Trần Mạnh Hảo nên rủ tôi lên chung cư Hội Văn Nghệ số 190 Nam Kỳ Khởi Nghĩa lai rai ba sợi chơi. Khi đi, tôi có rủ thêm Trần Hữu Dũng, Vũ Ngọc Giao là hai chiến hữu giang hồ cùng cạn chén tang bồng hồ thỉ. 
 
Rượu vào lời ra. Tôi và Nguyễn Bắc Sơn thay phiên nhau khạc thơ chan chát. Có lẽ Nguyễn Bắc Sơn không ngờ tôi là một kẻ hậu sinh chưa hề có tên tuổi trước giải phóng mà khạc thơ quá đã, nên anh “bốc” liền một câu: “\textit{Thằng cha Bùi Chí Vinh này làm bài thơ nào cũng hay hết, nhưng thơ họ Bùi là Ðồ Long Ðao, còn thơ Nguyễn Bắc Sơn ta mới là Ỷ Thiên Kiếm}”. Nguyễn Bắc Sơn đâu biết câu phát biểu đó vô tình làm “mồi” cho một bài thơ giai thoại về anh và tôi sau này. Bài thơ được tôi ứng khẩu tại chỗ như sau: 
\begin{blockquote}
 
\textbf{Cách nhậu với Nguyễn Bắc Sơn} 
        
“\textit{Ta làm thơ bài nào cũng hay}”        
\textit{Nghe gã Nguyễn Bắc Sơn nói thế}        
\textit{Té ra gừng già ngươi chưa cay}        
\textit{Ta chỉ hạt tiêu mà rơi lệ}        
\textit{Làm thơ ta làm từ bụng mẹ}        
\textit{Ðợi ngươi nổi tiếng là ta sinh}        
\textit{Sinh sau đẻ muộn giống Hạng Thác}        
\textit{Cho người Khổng Tử đỡ hợm mình}        
\textit{Sinh sau đẻ muộn giống chim hạc} 
\textit{Cho đàn cò đói đỡ ăn đêm} 
        
\textit{Nhà ngươi bốc ta cứ như chưởng:}        
\textit{Rằng thơ ta ngông như Tạ Tốn}        
\textit{Câu trước câu sau Ðồ Long Ðao}        
\textit{Vần dưới vần trên Ỷ Thiên Kiếm}        
\textit{Ðao kiếm dành cho bọn cường hào}        
\textit{Có đâu đưa vào thơ bố trận}        
\textit{Tại đời lắm muối nên thơ mặn}        
\textit{Chứ thiết gì ta nghiệp võ công}        
\textit{Kìa coi hoàng đế Quang Trung đó} 
\textit{Ðến chết còn ghê chữ má hồng} 
        
\textit{Tiếc rằng ngươi không là thiếu nữ}        
\textit{Thiếu nữ bốc, ta thành vua Trụ}        
\textit{Nhà ngươi bốc, ta thành bia hơi} 
\textit{Uống say bọt bay hết lên trời…} 

\end{blockquote}
 
Ðối ẩm với Nguyễn Bắc Sơn tại nhà Trần Mạnh Hảo mà quên nhắc đến họ Trần thì quả là điều không phải phép. Khi tôi 15 tuổi tham gia phong trào sinh viên học sinh đấu tranh đô thị ở Sài Gòn thì bộ đội Trần Mạnh Hảo đã siết cò AK ở trong rừng. Ngay giải văn học Thành phố Hồ Chí Minh đầu tiên sau giải phóng 1976 – 1977, tôi và Trần Mạnh Hảo đã biết nhau khi anh đoạt giải thơ với tập \textit{Tiếng chim gõ cửa}, còn tôi đoạt giải thơ với tập \textit{Hạnh phúc có thật}. Tôi với anh còn thân nhau bởi cùng đi lưu diễn đọc thơ các trường đại học cùng với Nguyễn Duy, Văn Lê, Nguyễn Nhật Ánh. Phải nói thật, tôi thân với Trần Mạnh Hảo hơn những nhà thơ ngoài Bắc khác một phần vì quê quán cha tôi thuộc tỉnh Nam Ðịnh, đồng hương với anh. 
 
Trần Mạnh Hảo và tôi mỗi người đều tạo ra những sóng gió và dư luận riêng bởi cá tính và thơ của mình. Trong bàn nhậu đám đông, tôi và anh luôn luôn giữ vai trò chủ lực trong việc đọc thơ phục vụ bè bạn bằng thơ trí nhớ hoặc thơ ứng khẩu. Ai cũng khẳng định rằng tôi và anh đều có trí nhớ đặc biệt, thuộc lòng bất kỳ bài nào của mình viết ra, cho dù là viết giỡn chơi. Thậm chí giới giang hồ mỗi lần nghe tôi và anh đấu khẩu bằng thơ đều gọi là “Nam Chinh, Bắc Chiến”. Một lần ngồi dưới chân cầu Công Lý trước nhà chị Phương Huệ, có mặt khá đông bá tánh tín đồ Phật Giáo, Trần Mạnh Hảo đã cao hứng đọc oang oang bài thơ chinh phục thiên hạ. Bằng trí nhớ tôi chép ra đây sau một thời gian quá lâu hơn 20 năm, nếu có sơ xuất hoặc thiếu câu nào đoạn nào mong Trần Mạnh Hảo thông cảm: 
\begin{blockquote}
 
\textbf{Phùng Phật, sát Phật} 
        
\textit{Phùng Phật phải sát Phật}        
\textit{Sát Phật, Phật quay về}        
\textit{Ngộ rồi mà chưa ngộ} 
\textit{Tỉnh tỉnh mà mê mê} 
        
\textit{Thuý Kiều vừa thành Phật}        
\textit{Mười lăm năm tu hành}        
\textit{Cõi tâm là cõi Phật} 
\textit{Lầu không lầu không xanh} 
        
\textit{Phật tự thân người đẹp}        
\textit{Không dưng, sao Phật Bà}        
\textit{A Di Ðà sát Phật} 
\textit{Phật hoá thành đôi ta!} 

\end{blockquote}
 
Tôi thấy tình hình căng quá bèn giải thoát cho các tín đồ Phật Giáo bằng bài thơ thức ngộ sau đây: 
\begin{blockquote}
 
\textbf{Phùng Phật, cứu Phật} 
        
\textit{Trần Mạnh Hảo sát Phật}        
\textit{Giữ lại mình Quan Âm}        
\textit{Nói theo kiểu phàm tục}        
\textit{Diệt dục mà sinh dâm}        
\textit{Nói theo kiểu cờ bạc}        
\textit{Úp Tây mà lật Ðầm}        
\textit{Nói theo Bùi hiền sĩ} 
\textit{Muốn vậy chìa hai trăm!} 

\end{blockquote}
 
Bao giờ cũng vậy, những cuộc đấu khẩu thơ giữa tôi và Trần Mạnh Hảo đều làm thiên hạ bật cười nhẹ nhõm tới bến sau khi thần kinh căng thẳng cũng tới bến. Anh em văn nghệ mà. Những người có khả năng khuấy đảo thiên hạ chỉ đếm trên đầu ngón tay, tại sao lại không biết thương nhau bảo vệ nhau trước những cặp mắt cú vọ của đám tiểu nhân rình mò tâu hót ám hại. 
 
Một giai thoại nữa có liên quan tới Trần Mạnh Hảo khi họ Trần dẫn theo hai vị chức sắc thuộc tỉnh Hà Nam Ninh đến Hội Văn Nghệ 81 Trần Quốc Thảo ăn nhậu và ra câu đối thách thức. Chuyện đó đã hơn 15 năm. Hôm đó tôi đang ngồi uống bia dưới gốc cây đa cùng với Nguyễn Quốc Chánh, Ðoàn Vị Thượng… và nhiều anh em văn nghệ khác. Hai bên chào nhau và ráp bàn. Trần Mạnh Hảo tuyên bố: “\textit{Có một câu đối chúng tôi ra vế mà từ Bắc vô Nam chưa ai đối được hoàn hảo về mặt nghĩa đen lẫn nghĩa bóng.} \textit{Nếu các bạn trong bàn giải được, chúng tôi cá độ một chầu nhậu thả giàn}”. 
 
Vế đối ra như sau:        
\begin{blockquote}
        
“BATA đi giày vải” 

\end{blockquote}
 
Phải thú thật là vế ra quá độc. Bởi băng “Hà Nam Ninh” của Trần Mạnh Hảo gồm đúng 3 người, mà cả 3 đều đi giày vải, và giày vải đều mang hiệu BATA. Thế là anh em chiến hữu đều hướng mắt về phía tôi. Trong tình thế chỉ mành treo chuông, tôi gật đầu cái rụp. 
 
Sau 15 phút động não nhằm xác minh một đơn vị tiểu thủ công nghiệp mang tên “Ðại Chúng” chuyên sản xuất dép râu ở Chợ Lớn, tôi hùng hồn đứng dậy đối lại như sau: 
\begin{blockquote}
        
“ÐẠI CHÚNG lết dép râu” 

\end{blockquote}
 
Câu đối lại đã quá rõ ràng. Khi ba cán bộ đi giày Bata thì đám đông đại chúng nghèo khổ đành phải mang dép râu lết bánh. Thế là sau một hồi tranh cãi gọi điện thoại bàn xác minh cơ sở sản xuất dép lốp Ðại Chúng có thật hay không thì băng Trần Mạnh Hảo đành phải chung độ chứ còn phải hỏi. 
 
Cũng trong giai thoại về câu đối, nhân đây tôi nhắc chuyện này như một nén nhang thắp tặng linh hồn hai vị thuộc giới văn nghệ đã khuất. Ðó là hai nhà thơ trào phúng Tú Rua và bác Cử Tạ, vốn là hai nhân vật nằm trong hai câu đối của tôi. Chuyện xảy ra vào cuối thập niên 80 khi tôi và Lê Dụng (con trai cố nhạc sĩ Hoàng Việt) đến nhà Tú Rua chơi. Nhà thơ trào phúng Tú Rua vừa là chủ tiệm may đắt khách, vừa là một cộng tác viên đắc lực của báo \textit{Văn nghệ Thành phố}, nơi Lê Dụng công tác. Trong lúc trà dư tửu hậu chén tạc chén thù, ông chủ tiệm may Tú Rua cao hứng phán một câu “\textit{Nghe đồn Bùi Chí Vinh có khả năng ứng tác về ca từ thi phú cổ điển.} \textit{Vậy ông có ngon làm hai câu đối nói về chí khí của Tú Rua tôi trong sáng như sao Tua Rua trên bầu trời đêm thì tôi sẵn sàng đãi ông và Lê Dụng suốt một ngày khắp các quán Sài Gòn}”. Lời phán của Tú Rua như một tiếng sét đánh ngang mày. Mà đã là sét đánh thì nháng lửa và tung tóe như sao. Bất giác tôi liên tưởng đến bác Cử Tạ phụ trách mục “Ôn Cố Tri Tân” trên báo \textit{Long An cuối tuần} thường hay bốc thuốc Ðông Y ở khu Ông Tạ. Tôi nháy mắt với Lê Dụng như một nhân chứng vàxuất khẩu thành… hai câu đối như sau: 
\begin{blockquote}
        
TÚ RUA “\textit{rua}” \textit{SAO RUA} 
\textit{CỬ TẠ tạ ÔNG TẠ} 

\end{blockquote}
 
Tôi thấy Lê Dụng khoái trá, còn Tú Rua lặng người. Trong ba từ “rua” của vế trên thì chữ “rua” thứ nhì là tiếng Pháp có nghĩa là “bắt tay”. Tương tự trong ba từ “tạ” của vế đáp thì chữ “tạ” thứ nhì thuộc tiếng Hán có nghĩa là “vái chào”. Và kết quả là chúng tôi say xỉn quắc cần câu như thế nào có lẽ các bạn cũng hình dung ra được. 
 
Cũng trong thập niên 80, tôi thường xuống khu Ông Tạ giao du với gia đình nhà văn Lưu Ngũ và các hảo hớn anh chị sống ngoài vòng pháp luật ở khu vực đó. Lưu Ngũ xuất thân là cựu trung uý Biệt động quân của quân đội Sài Gòn, sau giải phóng đi học tập cải tạo và trở thành nhà văn bất đắc dĩ nhờ đoạt giải văn học thành phố năm 1976 – 1977 với truyện dài \textit{Vũng lầy}. Anh chán ghét chiến tranh đến mức độ chỉ muốn làm con người, nhưng làm con người giữa thời đại “bo bo” và “xuyên tâm liên” thì quả khó làm sao. Trong một đêm nhậu đã đời với những kẻ “Ðảng nghi ngờ, nhân dân chú ý” chúng tôi đã đi lang thang trên đường phố chỉ toàn xe đò chạy bằng than, nhìn thấy những chiếc xích lô kiếp ngựa thồ mà phu xe toàn là cựu chiến binh bộ đội lẫn quân đội chế độ cũ, chúng tôi tiếp tục nhìn thấy những giọt lệ ứa ra từ các mệnh phụ phu nhân, các tiểu thư nghèo khổ phải “đứng đường” đón khách kiếm tiền độ nhật. Bài thơ “Sinh nghi hành” mở đầu một giai thoại truyền khẩu sau này ra đời từ đó: 
\begin{blockquote}
 
\textbf{Sinh nghi hành} 
        
\textit{Sinh nghi ta viết một bài hành}        
\textit{Vợ nghi chồng, em út nghi anh}        
\textit{Cha nghi con cái, bè nghi bạn}        
\textit{Thủ trưởng thì nghi hết ban ngành}        
\textit{Láng giềng dòm ngó nghi hàng xóm}        
\textit{Ngoài đường nghi phố chứa lưu manh}        
\textit{Ngay ta khi viết bài in báo}        
\textit{Cũng nghi mình kiếm chác công danh}        
\textit{Trời ơi, mọi chuyện sinh nghi thật} 
\textit{Chén kiểu thường nghi kỵ chén sành} 
        
\textit{Thời buổi công hầu như chén cứt}        
\textit{Thiếu chó, mèo ăn cũng rất nhanh}        
\textit{Mèo ăn cho chó leo bàn độc}        
\textit{Vừa sủa vừa nhai riết cũng rành}        
\textit{Trẻ con khát sữa, ai cho bú}        
\textit{Vú mẹ gầy, sâu rúc nồi canh}        
\textit{Quang Trung bỏ núi Tây Sơn xuống}        
\textit{Hoảng hốt vì gương vỡ chẳng lành}        
\textit{Nguyễn Du chỉ một đêm dạo phố}        
\textit{Ðoạn Trường ngồi viết lại Tân Thanh}        
\textit{Thuý Kiều phát triển nhiều như thế}        
\textit{Thảo nào đất nước hoá lầu xanh}        
\textit{Nhà tù phát triển nhiều như thế} 
\textit{Kẻ sĩ làm sao dám học hành} 
        
\textit{Ta làm thơ mà lòng đứt ruột}        
\textit{Suốt đời bao tử chạy loanh quanh}        
\textit{Lãnh tụ nói }“\textit{đói quên nghi kỵ}” 
\textit{Ơn ấy ngàn năm sáng sử xanh} 

\end{blockquote}
 
Trước khi tạm ngưng phần đầu bài viết “Giai thoại của thi sĩ” này, tôi thiết tưởng không có gì ý vị hơn khi nhắc đến một ông bạn phương xa là Nguyễn Lương Vị. Hồi còn ở trong nước chưa định cư ở Mỹ, Nguyễn Lương Vị sống cùng địa phương với tôi, và những lúc buồn bã cô độc, anh thường ghé nhà rủ tôi nhâm nhi chén rượu quên sầu. Anh buồn vì một lý do cực kỳ giản dị: anh là một con người chứ không phải một con thú hoặc một cỗ máy. Thậm chí anh còn rung động nhanh hơn con người bình thường một bậc, bởi anh là… thi sĩ. 
 
Nguyễn Lương Vị thường ngồi bứt tóc trong lúc đánh cờ tướng. Ðánh xong bàn cờ là tóc anh rụng như mưa. Anh sống nửa dại nửa khôn nửa tỉnh nửa điên như thế nên phải tự giải thoát mình trong triết lý Phật Giáo. Mỗi lần say xỉn anh thường thuyết giảng cho tôi nghe về tiểu thừa đại thừa, về sắc sắc không không, về cõi luân hồi sát na sát khí… để cho tôi “choáng” mà bớt quậy. Nào dè tôi quậy còn tưng bừng hơn. Tôi có tặng anh bài thơ sau đây trước khi anh sum họp gia đình bên Mỹ: 
\begin{blockquote}
 
\textbf{Phật sống} 
        
\textit{Chư huynh bàn về tu luyện}        
\textit{Ðứa đại thừa, đứa tiểu thừa}        
\textit{Ðứa nào cũng sắp thành Phật} 
\textit{Chỉ mình ta còn gươm khua} 
        
\textit{Ðời này nói đến hơn thua}        
\textit{Biết bao giờ cho hết chuyện}        
\textit{Ta thấy chư huynh yêu chùa} 
\textit{Cũng là tự thân bảo hiểm} 
        
\textit{Nhưng tu như vậy còn kém}        
\textit{Biết khôn lựa gốc bồ đề}        
\textit{Có người tu hang Pắc Bó} 
\textit{Sau này thành Phật sướng ghê!} 

\end{blockquote}
 
7. 2008 
 
© 2008 talawas 
\end{multicols}
\end{document}