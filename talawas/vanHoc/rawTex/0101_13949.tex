\documentclass[../main.tex]{subfiles}

\begin{document}

\chapter{Về vốn thực tế trong việc đọc thơ ca}

\begin{subtitle}

(Nhân đọc ý kiến ngắn của Tâm Đàm)

\end{subtitle}

\begin{metadata}

\begin{flushright}7.8.2008\end{flushright}

Huỳnh Phan



\end{metadata}

\begin{multicols}{2}

Trong ý kiến ngắn của mình, khi bàn về việc hiểu câu hát “\textit{Đại bác đêm đêm \textbf{dội} về thành phố, người phu quét đường \textbf{dựng} chổi đứng nghe” }(theo cách viết của Tâm Đàm) trong bài “Đại bác ru đêm” của Trịnh Công Sơn, Tâm Đàm đã viết: “Nghe tiếng đại bác thì phải tìm nơi ẩn nấp, sao lại \textit{dựng chổi đứng nghe}? Gần đây trong một lần nói chuyện, anh bạn Nguyễn Đặng Trí Tín của tôi kể: “Ngày xưa mỗi lần nghe tiếng đại bác hoặc tiếng súng, ba anh ấy nghiêng tai lắng nghe xem đó là tiếng nổ của phía bên nào”... Dân Việt mấy đời sống chung với súng đạn, họ phân biệt được tiếng nổ của các loại vũ khí cả hai bên. \textit{Người phu quét đường} của Trịnh Công Sơn cũng lắng nghe như thế chăng?” 
 
Thật ra khi đặt lời cho một bài hát người nhạc sĩ hoàn toàn có toàn quyền viết theo ý riêng mình và sau đó người đọc / nghe tuỳ theo kinh nghiệm kiến thức khả năng cảm thụ của mình cũng hoàn toàn có quyền hiểu theo cách riêng của mình. Nhưng trong trường hợp này, theo tôi lời nhạc của Trịnh Công Sơn đã phản ánh đúng thực tế cuộc chiến lúc đó ở miền Nam (từ vĩ tuyến 17 trở vào), ít nhất về tình huống tiếng đại bác từng đêm \textit{vọng} về thành phố (lưu ý là \textit{vọng} chớ khộng phải \textit{dội} như Tâm Đàm đã nghe nhầm). Giai đoạn đó hầu như ngày cũng như đêm, lúc nào cũng có tiếng đại bác nả về nông thôn - điạ bàn hoạt động chủ yếu của phe công sản. Không có việc đại bác đêm đêm nả về các thành phố. Có chăng chỉ một vài quả súng cối đơn lẻ loại 60, 81 hay 82 li (mm) hay quả hoả tiển 122 li (là những loại vũ khí tương đối cơ động, dễ di chuyển của phía cộng sản) hoặc chất nổ (do họ lén lút mang vào) thi thoảng nổ trong thành phố chứ không ở mức độ từng đêm. Do hoạt động chủ yếu trên địa bàn nông thôn, mọi việc vận chuyển, tiếp tế hoàn toàn dựa vào sức người, phe cộng sản hầu như không có khả năng sử dụng đại bác. Đó là loại súng kình càng, nặng nề, dễ bị phát hiện, huỷ diệt bởi quân đội “quốc gia” / Mĩ nhờ họ có nhiều phương tiện chiến tranh hơn, nhất là máy bay đủ loại (phe cộng sản không có được lợi thế này).  Cuộc chiến không phải thuộc dạng chiến tranh quy uớc, dàn trận đánh nhau, hễ bên này có vũ khí, phương tiện gì thì bên kia cũng có tương tự như thế. Đây là loại chiến tranh chủ yếu mang tính du kích, phe cộng sản phải thường trực bị đe doạ bởi nguy cơ bị phát hiện, huỷ diệt nên phải lẫn tránh, che dấu mình ở vùng nông thôn, rừng núi để bảo toàn lực lượng. Vì thế họ không phải an nhiên, dễ dàng bắn đại bác lúc nào cũng được. Phe “quốc gia” thì hoàn toàn chủ động trong việc này, họ có thể nả pháo về nông thôn bất cứ lúc nào theo tin tình báo nhận được hoặc bắn hú hoạ vào những nơi tình nghi có phe cộng sản đang ẩn nấp (lúc đó ở miền Nam có nhiều vùng nông thôn “được” phe “quốc gia” xếp vào loại cho phép “oanh kích tự do”). 
  
Trong việc hiểu phần đầu câu hát, trước nhất có lẽ do Tâm Đàm đã nghe nhầm từ \textit{vọng} thành từ \textit{dội}, lại không có được thực tế vừa nêu (sau này có thêm thông tin từ bạn bè, dù đúng nhưng chẳng ích gì thêm) nên đã có cách hiểu không phù hợp. Theo tôi, Tâm Đàm đã hiểu từ \textit{dội} theo nét nghĩa \textit{trút xuống} nên đã hiểu câu hát theo ý sai lệch quá xa. Thật ra, nếu Tâm Đàm hiểu từ \textit{dội} với nét nghĩa \textit{bật trở lại do gặp vật cản} thì việc hiểu câu hát có lẽ không sai lệch xa như vậy. Tuy nhiên, theo cảm nhận của tôi từ \textit{dội} có lẽ không được tinh tế, xác đáng bằng từ \textit{vọng}. \textit{Dội} là bật trở lại nên có thể có hàm ý \textit{rất chát chúa, đinh tai} (đối với tiếng động lớn như tiếng đại bác), và không hề chuyên chở trong nó hàm ý \textit{mờ nhạt} và rất ít hàm ý \textit{từ xa đưa lại} như từ \textit{vọng}.   
 
Chính do tiếng đại bác từ xa vọng lại nên cường độ âm thanh có thể có phần mờ nhạt, thêm vào đó có thể bị tiếng chổi quét đường kêu loạt xoạt và các tiếng động khác lấn áp. Vì thế muốn nghe tiếng đại bác rõ hơn người phu quét đường trong phần kế của câu hát tất phải dừng chổi lại (lưu ý là \textit{dừng} chớ khộng phải \textit{dựng} như Tâm Đàm đã nghe nhầm). Nhưng vì sao họ lại quan tâm đến tiếng đại bác từ xa vọng lại để dừng chổi đứng nghe như vậy? Cũng do sự nghe nhầm này cùng với việc không trải qua thực tế chiến tranh lúc đó nên cách hiểu của Tâm Đàm đối với phần kế này cũng lệch hướng nốt.  Người phu quét đường hay những người lao động bình thường nói chung khác ở các thành phố thường là những dân quê bị chiến tranh đẩy khỏi nông thôn. Do đó, việc \textit{người phu quét đường \textbf{dừng} chổi đứng nghe} là một điều khá dễ hiểu. Không phải do họ không sợ vì biết chắc là đại bác không nả về phía mình mà khi nghe tiếng vọng của đại bác, với nỗi lo âu canh cánh hướng về quê cũ, nơi có ruộng vườn, mồ mả cha ông và bà con, hàng xóm (đang còn bán mạng bám lấy ruộng vườn ở đó để sống), họ dừng chổi lại để lắng nghe cho rõ hơn xem đại bác đang nả về phía quê cũ của mình hay một nơi nào khác. Người phu quét đường dừng chổi lắng nghe để mà lo âu cho người thân, xóm giềng… khi cảm nhận đại bác đang nả về nơi đó, hoặc xót xa, thương cảm người cùng cảnh khi đại bác bắn về vùng quê khác. Dĩ nhiên họ cũng có thể ngừng quét, \textit{dựng chổi} ở đâu đó để lắng nghe tiếng đại bác, nghỉ ngơi giây lát nhưng nói chung động tác \textit{dựng} có vẻ không phù hợp và hơi thừa trong điều kiện quét rác trên đường phố trong đêm. Họ dừng chổi lại để nghe cho rõ hơn rồi lại phải tiếp tục ngay cộng việc của mình, với con đường trước mắt phải quét và nhiều con đường khác phải làm sạch trong đêm. Vì thế từ \textit{dựng} rõ ràng là không sát hợp ở đây. Không thấy được tình cảm của người phu quét rác với làng xóm, quê hương cũ, không thấy sự lam lủ của họ trong công việc thì việc dùng từ \textit{dừng} hay \textit{dựng}  có lẽ không khác biệt nhiều lắm (muốn dựng chổi trước hết phải dừng quét đã). 
 
Thật ra nếu muốn không gây nhầm lẫn và cũng logic hơn thì có thể viết “\textbf{tiếng }\textit{đại bác} đêm đêm vọng về thành phố” thay vì “\textbf{Đại bác} đêm đêm vọng về”, nếu nhịp điệu bài hát cho phép. Tuy nhiên, theo tôi viết như vậy là không cô đọng và không biểu cảm bằng cách viết của tác giả (tuy không thật logic vì chỉ có \textit{tiếng đại bác} mới có thể \textit{vọng }được nhưng cũng không gây ra sự hiểu lầm nào ở đây). Khi nghe từ \textit{đại bác} người ta liên tưởng tới sự gay gắt, khốc liệt của cuộc chiến vì đại bác là thứ trực tiếp gây ra chết chóc, huỷ diệt. Còn \textit{tiếng đại bác} có thể chỉ làm người ta liên tưởng tới sự đinh tai, nhức óc nhiêu hơn là đến sự chết chóc, huỷ diệt đang cận kề. Câu hát này làm tôi nhớ tới quyển tiểu thuyết \textit{Đêm nghe tiếng đại bác} của nhà văn Nhã Ca. Ở đây, mặc dù tác giả, do sự cô đọng của tựa đề một quyển sách, không nêu rõ việc nghe tiếng đại bác xảy ra trong hoàn cảnh nào nhưng đối với những ai có vốn sống đều đoán được là nó xảy ra ở thành phố và việc đại bác nổ là xảy ra hằng đêm. Nếu ở nông thôn, đang bị đạn đại bác nổ trên đầu thì người ta chỉ lo đến chuyện chết sống chứ không thể nào thảnh thơi để suy tư những điều làm ra nội dung câu chuyện và nếu chỉ nghe đại bác nả trong một đêm nào đó thôi có lẽ không đủ để có những suy tư như vậy. 
 
Đến đây tôi có thể đoan chắc rằng Tâm Đàm chưa từng thực sự \textit{đọc} lời ca gốc của tác giả mà chỉ \textit{nghe} hát thôi nên đã nhầm hai từ \textit{vọng} và \textit{dừng} mà tôi cho là hai từ quan trọng trong câu hát trên. Việc nhầm lẫn này cũng là điều bình thường vì lời ca có thể bị biến dạng đôi chút khi ca sĩ phải phát âm cho phù hợp với nhạc điệu, có thể bị tiếng nhạc lấn áp và nhất là khi người nghe có thể định hướng sai lệch trong việc hiểu lời ca do thiếu vốn thực tế. Tôi cho rằng điều sau này là nguyên nhân chính yếu, ít ra là trong trường hợp này. Việc hiểu các câu khác trong bài hát này cũng thế. Nếu không nắm được / trải nghiệm thực tế cuộc chiến thì có lẽ cũng khó hiểu đúng và khó lòng thấy hết được tính phản chiến của bài hát này. 
  
Nhận việc nghe nhầm lời ca như trên, tôi nhớ đến một số trường hợp lời ca bị hát nhầm / sửa đổi mà cho tôi là đã làm mất đi sự tinh tế, đẹp đẽ của lời gốc trong một số chương trình ca nhạc do ca sĩ Việt Nam ở nước ngoài trình bày. 
 
Chẳng hạn, trong bài \textit{Tiếng sông Hương} của Phạm Đình Chương có câu hát, theo trí nhớ riêng (tôi không tìm được bài nhạc gốc) là “ngậm ngùi hân hoan \textit{suối lệ} đoàn viên” thì ca sĩ lại hát là “ngậm ngùi hân hoan \textit{tiếng cười} đoàn viên”.  Rõ ràng việc đổi \textit{suối lệ} thành \textit{tiếng cười} đã làm giảm đi một mức độ to lớn sự biểu cảm của câu hát. \textit{Tiếng cười}, theo tôi không lột tả được đúng mức độ của sự vui mừng của cảnh vợ chồng đoàn tụ sau mùa chinh chiến bằng \textit{suối lệ}. Nghe có vẻ trái khoái nhưng thực tế là người ta lại tuôn rơi nước mắt khi có được một niềm vui tột đỉnh.  Ngay trong chính âm nhạc cũng như vậy, nhiều bản nhạc vui nổi tiếng thường viết ở cung \textit{thứ} (dùng cho nhạc buồn) và ngược lại những bản nhạc buồn tha thiết lại viết ở cung \textit{trưởng} (dùng cho nhạc vui). 
 
Trong bài \textit{Quê mẹ} của Thu Hồ có câu “ chiều chiều mắt \textit{hoen mờ} vì con” thì ca sĩ lại hát là “chiều chiều mắt \textit{ngấn lệ} vì con”.  Ở đây, cũng vậy khi đổi \textit{hoen mờ} thành \textit{ngấn lệ} câu hát rất biểu cảm và hình tượng trở thành một câu đơn sơ, trần trụi.  Đương nhiên, khi nói tới \textit{mắt hoen mờ mỗi chiều} trong văn cảnh bài hát chắc rằng không ai nghĩ đó là chỉ do bởi tuổi già mà đều nghĩ đó còn do những giọt nước mắt nhớ thương con và từ \textit{mờ} làm tăng thêm tình cảm của mẹ thương con (đến mức hao gầy thân thể của mình). 
 
Hoặc trong bài \textit{Một mình thôi} cuả Anh Việt Thu có câu “thà quên đi như chúng ta không \textit{hề} quen”, ca sĩ chỉnh lại là “thà quên đi như chúng ta chưa \textit{làm} quen”.  Đôi trai gái \textit{chưa hề quen} là hai người hoàn toàn xa lạ nhau, do đó chẳng có kỉ niệm buồn vui gì chung với nhau để mà nhớ lại cả. Còn cặp trai gái \textit{chưa làm quen}, tức là đã ở giai đoạn đã biết nhau, có thể đã có tình ý với nhau, đã (trộm) nhìn nhau hoặc cười (duyên) với nhau… Do đó, mỗi người vẫn có thể nhớ lại những “kỉ niệm” này. Câu hát nguyên gốc tỏ ý hết sức dứt khoát (quên) trong khi câu hát sửa đổi làm nhẹ tênh ý này. 
 
Theo tôi, mấy trường hợp nêu trên xảy ra, chủ yếu là do người sửa không huy động đúng mức vốn thực tế mà hầu như ai cũng có để hiểu / hát đúng lời hát. Ở đây, hầu như không đòi hỏi vốn thực tế mà người đọc / nghe có thể chưa có hay chưa từng trải nghiệm như trong trường hợp câu hát của Trịnh Công Sơn. 
 
Mấy trường hợp đó chỉ là một vài trong số rất nhiều trường hợp sửa lời bài hát xảy ra trong thực tế mà tôi còn nhớ. Thực tế hiện nay cho thấy có rất nhiều trường hợp sửa lời ca mà việc sửa lời thường làm câu hát từ kém giá trị hơn, ít đẹp đẽ hơn cho đến hoàn toàn lệch xa ý nghĩa như trường hợp Tâm Đàm. Hiếm thấy trường hợp sửa lời ca làm giá trị, tính thẩm mĩ của câu nhạc tăng lên. Trong việc này, như tôi đã nêu trên, tôi chỉ dựa theo trí nhớ riêng nên không chắc mình đúng hoàn toàn. Đối với những lời ca thay đổi mà tôi được nghe, tôi không rõ do chính tác giả, hay ca sĩ thực hiện. Tôi cũng không rõ do trình độ cảm thụ của ca sĩ hoăc do nhạc sĩ / ca sĩ / người tổ chức muốn nương theo trình độ tiếng Việt ngày càng thui chột của nhiều người Việt đang sống ở nước ngoài mà lời ca đã được thay đổi như thế. 
 
Cuối cùng, xin được nói thêm rằng tôi viết bài này để góp thêm một cách hiểu riêng của mình về một số câu nhạc cụ thể nhân đọc bài “Đọc bài thơ như thế nào”\footnote{\url{http://www.talawas.org/talaDB/showFile.php?res=13805&rb=0101}} của Nguyễn Đức Tùng và ý kiến ngắn của Tâm Đàm, và cũng với mong mỏi những ca sĩ khi hát sẽ cẩn thận hơn đối với từng câu, từng từ trong các bài hát mình ca. Tôi hoàn toàn không có ý tranh luận với Nguyễn Đức Tùng mà chỉ minh hoạ thêm là vốn thực tế là một thành tố không thể không xem xét tới trong cảm thụ thơ ca như anh có đề cập, ít ra là trong những trường hợp như thế này. Tôi cũng không có ý phê phán điều gì đối với Tâm Đàm mà trái lại rất cảm thông với tác giả. Kinh nghiệm thực tế xưa cũng như nay cho thấy có không quá ít trường hợp hiểu văn thơ không đúng, kể bậc thầy như trong giai thoại nổi tiếng\footnote{\url{http://www.talawas.org/talaDB/http://vi.wikipedia.org/wiki/V%C6%B0%C6%A1ng_An_Th%E1%BA%A1ch}} Tô Đông Pha sửa… sai câu thơ “\textit{Minh nguyệt sơn đâu \textbf{khiếu}, Hoàng khuyển ngoạ hoa \textbf{tâm}}” của Vương An Thạch, dù rất tài tình thành “\textit{Minh nguyệt sơn đâu \textbf{chiếu}, Hoàng khuyển ngoạ hoa \textbf{âm}}”, như trường hợp mới đây về đề văn tuyển sinh đại học\footnote{\url{http://www.talawas.org/talaDB/http://phongdiep.net/default.asp?action=article&ID=4883}} trong nước, hoặc như nhiều trường hợp hiểu không đúng thơ Hàn Mặc Tử mà Nguyễn Đức Tùng đã nêu… Cũng hi vọng Tâm Đàm sẽ hiểu ý và không trách tôi khi đọc xong bài viết này và hai đường link mới. 
 
© 2008 talawas 
\end{multicols}
\end{document}