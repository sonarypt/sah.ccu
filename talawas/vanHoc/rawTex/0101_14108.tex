\documentclass[../main.tex]{subfiles}

\begin{document}

\chapter{Ai biểu không làm thơ như Tố Hữu!}

\begin{metadata}

\begin{flushright}1.9.2008\end{flushright}

Cao Trần



\end{metadata}

\begin{multicols}{2}

Ở đời, có người vô tù, làm thơ và trở thành thi sĩ nổi tiếng, như ông Nguyễn Chí Thiện, chẳng hạn, với thi tập \textit{Hoa địa ngục}. Có người, cũng nằm ấp như ông Thiện, nhưng không hề (biết) làm thơ, vậy mà khi ra tù, lại có nguyên một tập thơ (dù chẳng lấy gì làm hay ho) lận lưng, chẳng những để làm thi sĩ, mà còn để làm… cha (và làm bác) thiên hạ. Xem ra thơ và tù, không biết ở những xứ sở khác thì sao, chứ ở Việt Nam, có mối quan hệ vô cùng khăng khít. Nói cách khác, làm thơ, dù hay hay dở, nếu không khéo, có thể vô tù như chơi, nhất là làm thơ ở Việt Nam và làm thơ không giống… Tố Hữu. 
 
Ông Tố Hữu làm thơ từ hồi Việt Nam còn bị Tây đô hộ, nghĩa là lúc đó chưa có “độc lập, tự do và hạnh phúc”; cho nên, ông đã không bị Tây bắt bỏ tù vì tội “lợi dụng quyền tự do dân chủ” như trường hợp thầy giáo Nguyễn Đình Phương của huyện Nam Đàn, tỉnh Nghệ An. Báo \textit{Tiền phong} ngày 25/8/2008 thuật rằng, xin trích nguyên văn: 
\begin{blockquote}
 
Tháng 11/1992, xã Nam Tân và Nam Thượng thuộc huyện Nam Đàn, Nghệ An xẩy ra tranh chấp đất đai. Để ổn định tình hình, chính quyền địa phương đã cho đóng cột mốc, phân chia ranh giới hai xã. Việc làm của chính quyền địa phương hồi đó đã không được người dân đồng tình. Thời gian đó, thầy giáo Nguyễn Đình Phương đã sáng tác 5 bài thơ, trong đó có bài “Cột mốc hay là cột ngốc\footnote{\url{http://www.talawas.org/talaDB/http://www.tienphong.vn/Tianyon/Index.aspx?ArticleID=134778&ChannelID=12}}”. 

\end{blockquote}
 
Cũng vì bài thơ này mà ngày 28/7/1993, công an huyện Nam Đàn đã bắt giam thầy Phương 4 tháng, đồng thời Viện Kiểm sát huyện Nam Đàn đã ra quyết định truy tố thầy về tội “lợi dụng quyền tự do dân chủ xâm phạm lợi ích nhà nước, tổ chức xã hội hoặc của công dân”. 
 
“Cột mốc hay là cột ngốc” của thầy Phương, một bài lục bát gồm 10 câu, thực tình mà nói, không phải là một bài thơ hay, và cũng chẳng phải là một bài thơ “dữ” như mấy bài thơ trong tập \textit{Hoa địa ngục} của thi sĩ Nguyễn Chí Thiện. Theo thiển ý, có thể xếp nó vào dạng vè cỡ như bài “Hòn đá to, hòn đá nặng, chỉ một người, nhấc không đặng” của Hồ Chí Minh. Tuy nhiên, vì thầy Phương làm bài “Cột mốc hay là cột ngốc” để chửi xéo giới chức địa phương là một lũ ngốc, nên mới có chuyện.  
Và chuyện đã xảy ra một cách hết sức khôi hài như sau, cũng theo \textit{Tiền phong}: \begin{blockquote}
 
Ngày 28/8/1993, công an huyện Nam Đàn có Quyết định 07, trưng cầu giám định tác phẩm văn học bài thơ “Cột mốc hay là cột ngốc”. Ngày 6/9/1993, Sở Văn hóa Thông tin Nghệ An đã thành lập Hội đồng giám định gồm 3 người, do ông Đặng Khắc Thắng làm tổ trưởng. Các ông đã phân tích và đánh giá như sau: Bài thơ hô hào cổ động, kích động người nghe cản trở chủ trương đóng cột mốc. Tác phẩm châm biếm, đả kích, coi thường tổ chức chủ trương đóng cột mốc. Bộc lộ quan điểm gán ghép, quy chụp trong việc xử lý nguyên nhân dẫn đến xung đột đất đai ở địa phương. Tạo cho người nghe lầm tưởng, bài xích tổ chức và cá nhân...\end{blockquote}
 

Như đã nói, bài thơ của ông giáo Phương chẳng hay, cũng chẳng “dữ,” và chắc chắn cũng chẳng có gì là khó hiểu. Còn ông giáo Phương, xin lỗi, chỉ là một ông giáo “quèn” của một tỉnh lẻ, không hề có tên tuổi trên văn đàn, thi đàn hay bất cứ loại… đàn nào khác. Vậy mà nhà chức trách tỉnh Nghệ An đã phải yêu cầu Sở Văn hóa Thông tin lập hẳn một “Hội đồng giám định” để tìm hiểu tác phẩm của ông Phương, rồi nhân đó, cậy nhờ công an dọn sẵn một xà-lim, đưa ông vô suốt bốn tháng trời để tiện việc tìm hiểu cuộc đời và sự nghiệp văn chương, nếu có, của ông. 

Sau khi tìm hiểu xong, công an cảm thấy văn nghiệp của ông Phương thuộc vào loại quá… xoàng, còn tự do thì ông làm gì có để mà lợi dụng vào mục đích “xâm phạm lợi ích nhà nước (tự do làm thơ ông còn không có nữa kia mà!), cho nên họ đành lặng lẽ thả ông ra.  

Có lẽ nhờ vụ này, ông giáo Phương bất giác phát hiện ra rằng từ hồi cách mạng về đến giờ, ông không hề có tự do gì ráo. Thế là ông nổi giận, vác đơn đi kiện. Từ đó đến nay, ông đã mòn mỏi kiện tụng suốt mười lăm năm. Trong suốt mười lăm năm đó, thứ tự do duy nhất mà ông có là… tự do đi kiện. Ngày 24/8/2008 vừa qua, ông cho báo \textit{Tiền phong} biết ông sẽ tiếp tục kiện nữa, và nhà cầm quyền dường như rất tôn trọng quyền tự do duy nhất đó của ông: họ để mặc ông muốn kiện gì thì kiện và nhất định không chịu giải quyết vụ kiện của ông. 

Ai biểu ông không làm thơ như Tố Hữu, và quan trọng hơn nữa, ai biểu lưng ông thẳng hơn lưng Tố Hữu! 

© 2008 talawas   
\end{multicols}
\end{document}