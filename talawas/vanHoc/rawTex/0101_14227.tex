\documentclass[../main.tex]{subfiles}

\begin{document}

\chapter{Tháng tư gãy súng – trò chơi ráp chữ mang tính biểu niệm}

\begin{metadata}

\begin{flushright}16.9.2008\end{flushright}

Liêu Thái



\end{metadata}

\begin{multicols}{2}

Nếu như ở \textit{Xáo chộn chong ngày}\footnote{\url{http://www.talawas.org/talaDB/http://www.talachu.org/tho.php?bai=19}}, thử nghiệm trước đó, hay trong \textit{Xin lỗi hổng chiu nổi}, thử nghiệm gần đây nhất, ta được chứng kiến và tham dự vào trò chơi thơ của Bùi Chát: đảo chiều cấu trúc, tạo độ vòng của nghĩa, giễu nhại, cắt dán… nhằm tạo ra một cấu trúc “đặc dị” mang đậm dấu ấn cá nhân, thì với \textit{Tháng tư gãy súng}\footnote{\url{http://www.talawas.org/talaDB/http://www.talachu.org/tho.php?bai=17}}\textit{ \footnote{
\textit{Tháng tư gãy súng}, tác phẩm Bùi Chát, nxb Giấy Vụn 12/2005 

\end{blockquote}} }, ta lại thấy Bùi Chát thử một trò chơi mới - trò ráp chữ mang tính biểu niệm thông qua nghệ thuật xếp đặt (trên văn bản) nhằm tạo độ căng và những nếp gấp của ngôn ngữ, tạo không gian xoay chiều, đa hình tuyến, đa thoại, biến những đơn vị riêng lẻ trong tập thơ thành một tổng thể có tính nhất quán, liên tục và đưa những giới hạn ý niệm trong biểu tính của cái không hạn định nâng lên một tầng bậc mới – tầng bậc của trình diễn ngôn từ. 
 
Ta hãy bắt đầu bằng một ví dụ: 
\begin{blockquote}
 
.... con khốn nạn kia mầy lại đây cho tau biểu cái nầy... từ ngày mày quen  
 
. . . . . . . . . . . . . . . . . . . . . . . \textit{fatimah lại cho ba mầy dạy . . . . ông phải dạy cho } 
 
. . . . . . . . . . . . . . . . . . . fatimah mầy còn chần chừ giì nữa . . . ba phải dạy  
 
cái thằng người việt chó đẻ đó trông mầy không còn ra cái giì cả . . .  
 
\textit{nó nge điều phải trái . . . . . con nầy càng lớn càng khó dạy . . . ông ngậm cái ống}  
 
cho nó biêt con gái muslim phải sống ra sao . . . . . . . . . . . còn cái thằng người  
 
mầy muốn đi luc nào thời đi về luc nào . . . . . . . . . thời về . . . .  
 
\textit{tẩu để tui châm lửa cho . . . . con gái như thế còn mặt mũi nào mà sống với ummah}  
 
việt chó đẻ đó ba cứ để con trừng trị . . . . . . . . . . hôm thứ sáu tuần trươc con lên trên mosque  
 
mầy gầm mặt xuống mà nge tau nói . . . đừng có trố măt ra như thế . . . nhà  
 
. . \textit{tui đã nói to nói nhỏ biêt bao nhiêu mà nó cũng không nge . . . . . mấy bà } 
 
mấy đứa bạn hỏi chừng nào thì em gái mầy biêt vò spring roll . . . nge mà muốn tự tử . . . .  
nầy là nhà có danh giá trong ummah . . . là nhà có iman . . . mầy có nge rõ  
 
\textit{nhà ibrahim băt đầu xầm xì con nhỏ nhà nầy săp bỏ nhà theo trai} . . . . . . . .  
 
. . . . con phải bẻ cổ cái thằng người việt chó đẻ đó . . . . . . . . . .  
 
không . . . mẹ mầy nói với tau người ta đồn mầy ốm ngén . . . mặt mũi mầy  
 
. . .\textit{ mầy có nge rõ không . . . người ta nói mầy là đồ qadhf . . . mần sao mà tui sống}  
 
. . . . còn mặt mũi nào mà sống với ummah được chớ . . . . . . . . con người có học mà ngu như  
 
dạo này xanh xao phờ phạc . . . tại vì học thi à . . . mầy mà còn học hành giì  
 
\textit{được hử trời . . . nhục ơi là nhục . . . học cái giì mà học . . . bao nhiêu chỗ khá giả mầy}  
 
vậy . . . đi theo cái thằng người việt chó đẻ . . . ba cho con hut một hơi . . . .  
nữa . . . . mầy nói thiệt cho tau nge mầy đã quan hệ với thằng chó đẻ đó tới  
 
\textit{có chịu đâu . . . lại đâm đầu đi theo cái thằng người việt đó} . . . . . . . . . .  
 
. . . phải giiêt cái thằng chó nầy . . . người mình chêt hêt rồi hay sao mà đi theo cái thằng đó . .  
mưc nào rồi . . . mầy nói đi . . . bà để cho nó nói . . . mầy im đi để cho nó nói  
 
. . . . . . . . . . . . \textit{mầy nói thiệt với ba mầy đi} . . . . . . . . . . . . . . . . .  
 
. . . mầy nói thiệt đi . . . . . . . .. . . . . . . . . . mầy nói thiệt đi . . . . . . . . . . . . . .  
 
. . . mầy nói đi . . . hử . . . mầy nói sao . . . mầy muốn cưới nó hử . . . . . bà có  
 
\textit{. . . . . . . . . hử . . . . . . . . . .mầy nói cái giì . . . . cưới thằng đó hử . . . mầy có}  
 
. . . . . .. . hử . . . . . .. . . . . . mầy nói cái giì . . . . . . cưới thằng chó đó . . . trời  
nge nó nói giì không . . . nó muốn mần vợ thằng chó đẻ đó . . . mầy giỡn với  
 
\textit{điên không . . . . ông có nge nó nói giì không . . . trời ơi là} \textit{trời con ơi là} \textit{con } 
 
ơi . . . ba có nge nó nói giì không . . . tat cho nó vỡ mặt ra . . . . mầy dám nói giỡn hử con kia . .  
tau hử . . . sao . . . cái giì . . . mầy nói thiệt hử con đĩ kia . . . . . tau căt cổ mầy  
 
\textit{nhục ơi là nhục . . . sao . . . cái giì . . . mầy nói thiệt sao . . . hử . . . cái con đĩ kia } 
 
. . . mầy dám nói như thế hử . . . cái giì . . . mầy nói thiệt hử . . . tau đánh vỡ mặt mầy ra bây giừ  
. . . thằng sadi bế bà nội mầy vào phòng ngủ để tau giiêt con đĩ mât dạy này  
 
\textit{. . . . . . . mẹ nín đi chớ việc giì mẹ phải khoc . . . để tụi tui dạy nó chớ . . . sadi bế bà nội vào } 
 
. . bà nội . . . sao bà khoc . . . sao bà lại binh nó . . . để ba tui dạy nó chớ . . . bà để tôi bế vào  
. . . bà khoc loc như thế thì tui dạy nó thế nào . . . cái con đĩ này con đĩ này  
 
\textit{phòng ngủ . . bà cứ binh nó như thế thì còn dạy dỗ giì được . . . cái con đĩ mât dạy } 
 
. . . . bà đừng có giãy giụa như thế . . . luc nào đụng đến nó thời bà lại khoc loc  
tau chôn sống mầy mầy dám nói như thế hử con đĩ con đĩ con đĩ con đĩ con  
 
\textit{mầy dám nói như thế sao . . . ông chôn sống nó đi . . . đồ đĩ đồ đĩ đồ đĩ đồ đĩ đồ đĩ}  
 
mầy dám nói như vậy à . . . chôn sống nó đi . . . quá nhục nhã . . . tau phải giiêt cái thằng chó  
đĩ đồ đứng đường đồ suc vật tau đánh mày nat xương con đĩ con đĩ con đĩ . .  
 
\textit{. . trời ơi . . . trời ơi . . . giiêt cái thằng chó đẻ đó đi giiêt cái thằng chó đẻ đó đi giiêt . . } 
 
đẻ đó cái thằng chó đẻ tau căt cổ nó tau lột da nó tau bằm xương nó tau chặt đầu nó chẻ sọ nó . .  
saigon 28.2.2005 
 
\textit{Nguồn}\textbf{:} trich phần “jihad” trong truyện ngắn “bên ngoài kinh qur’an”\footnote{\url{http://www.talawas.org/talaDB/http://www.tienve.org/home/activities/viewTopics.do;jsessionid=F5051336CB569C5343DD21032F9BE4BB?action=viewArtwork&artworkId=244}} của hoàng ngọc-tuấn                                                                          
        
“câu chuyện đã được viêt, bên ngoài kinh qur’an &amp; kiều” (Tr.38. Mặt B) 

\end{blockquote}
 
Ngay từ trang bìa, \textit{Tháng tư gãy súng} đã có những khác biệt (nếu không nói là lập dị so với các tác phẩm thông thường) bởi cách trình bày quái [thai] lạ từ hình ảnh một nữ thổ dân trong trang phục truyền thống đứng vòng tay, cầm chiếc khăn che ngực (một bộ ngực khá đẹp và cuốn hút!), rốn hơi lồi… Nhưng người đọc chỉ thấy được phần mặt khi xoay theo chiều úp ngược vào mình vì dường như cô đã nằm vắt qua hai bìa sách. Tư thế “hiện hữu” của cô ít nhiều đã phản ánh một sự gián đoạn của thực tại bởi một cái gì đó có tính khúc xạ liên tục, nhân quả đến từ một hiện hữu khác bí ẩn và bất khả… Và cũng từ đó phát sinh một ý niệm, dự cảm về những trục trặc, đảo lộn, mất trật tự, hoặc hơn nữa là sự rối mù của hệ thống bên trong. 
\begin{blockquote}
 
\textit{la lợi na &amp; lưu lực vĩ học cùng lơp trong một trường đại học ở thành đô &amp; là người yêu của nhau. trong một buổi tự học trên lơp, sau khi đọc sach được hơn tiếng đồng hồ, đôi bạn trẻ băt đầu có những động tac thân mật: ôm &amp; hun nhau; rồi... nằm xuống đât. họ không biêt rằng, mọi hành động đều bị máy quay của trường gi lại. &amp; họ càng không biêt rằng, hành động "nằm xuống ngịch thơ" của mình trở thành một tình tiêt quan trọng để nhà trường khai trừ...} 
\textit{	} 
\textit{ba ngày sau, la lợi na &amp; lưu lực vĩ bị mời lên văn phòng nhà trường làm kiểm điểm. hai sv này đã phải viêt bản tường trình. 10 ngày sau, nhà trường gọi phụ huynh của 2 sv đến thông báo sự việc. sau đó 2 hôm, trường chánh thưc ra quyêt định đuổi học 2 sv. } 
 
\textit{sau khi xảy ra sự việc, mẹ của la lợi na đã đưa con mình đến bịnh viện để kiểm tra. bịnh viện kêt luận: màng trinh của la lợi na vẫn còn nguyên vẹn. điều đó có ngĩa: na &amp; vĩ vẫn chưa đi quá giới hạn. } 
 
\textit{nhà trường thời nhưt định cho rằng chứng cứ ngịch thơ quá rõ ràng; nên vẫn giữ nguyên quyêt định đuổi học. } 
 
chú: \small{[1]} xem thêm \textit{xáo chộn chong ngày}\footnote{\url{http://www.talawas.org/talaDB/http://www.talachu.org/tho.php?bai=19}} – nxb giấy vụn, 2003 
 
\textit{Nguồn:} http://www.tuoitre.com.vn/tianyon/index.aspx?articleid=72404&channelid=7\footnote{\url{http://www.talawas.org/talaDB/http://www.tuoitre.com.vn/tianyon/index.aspx?articleid=72404&channelid=7}}      
        
(“chỉ vì động tác nằm xuống... ngịch[1] thơ “ - Tr. 17. Mặt A) 

\end{blockquote}
 
Những mẩu tin (tập thơ đầy rẫy những thông tin lấy từ báo chí, có vẻ như chẳng có chút tính thơ nào. Đó là loại thông tin vặt, thông tin lá cải được trích ra từ những mẩu quảng cáo, báo lá cải và một vài bài báo nghiêm túc [một cách khôi hài]…) được dán vào tập thơ sẽ chẳng có ý nghĩa nào nếu nó không mang trên nó một cái tựa “có tính thơ” và được nhào - nặn - hậu - hiện - đại. Chính cái tựa đã chuyển hoá, biến nội dung bên dưới thành một loại thơ đặc dị [hợm] không giống ai. Tính biểu niệm, sự hoán đổi mục đích và ý niệm trong ngôn ngữ diễn ra tức thì sau thao tác dán - đầu - lân này. Gọi đó là thao tác dán đầu lân nghe ra cũng có vẻ hơi gượng và ép chữ nhưng thật ra cũng không còn cách gọi nào hợp lý hơn. Và không ngẫu nhiên chút nào, giữa bản tin và cái tựa có một mối tương quan khởi niệm, tạo ra một hồi nghĩa bổ sung để phát sinh một cái gì đó rất mới lạ, vượt ngoài những biểu nghĩa tự thân của từng “thành phần” trước nó, sinh ra nó… như một cuộc cộng hôn.  
\begin{blockquote}
 
\textit{một vị mục sư chẳng phiền hà việc tới lui thăm viếng cac hộp đêm "super sexe": vài năm trươc ở gatineau [gần ottawa]. trong cái "lapdancing" này, nếu trả 10$ cho một bổn nhạc sẽ được cùng một vũ nữ tuyệt đẹp nhảy khoả thân trong phòng riêng, ánh sáng mờ nhạt, &amp; chỉ riêng hai người theo luật định là có quyền sờ mó tự nhiên thân thể của người đẹp. câu chuyện kêt thuc: vị khach mục sư chêt trong căn phòng tối như thiên đường đó. sự kiện này được báo chí phản ảnh nóng bỏng, có một tờ ở montréal viêt trong những dòng cuối như sau: "có thể ông ta thấy giống cái của vợ mình, hoặc chưa hề thấy bao giờ..." } 
 
\textit{Nguồn:} trich "làm cái giì đó trươc bàn thờ" của la toàn vinh showFile.php?res=4219&rb=0102\footnote{\url{http://www.talawas.org/talaDB/showFile.php?res=4219&rb=0102}}                                 
 
(“bí mật ở thiên đường" -Tr. 11. Mặt A) 

\end{blockquote}
 
Tôi còn nhớ lúc nhỏ rất mê chơi trò lấy giấy vở học năm trước, rủ bạn bè cùng chẻ tre, làm sườn, dán thành chiếc đầu lân mỗi dịp trung thu về. Hồi đó không có nhiều loại màu, không có màu nước hay màu Agrilic như bây giờ nên cả bọn dùng sơn, vôi màu quét làm nền và vẽ râu, vẽ răng, lông mày, sừng, mắt… trông rất buồn cười và ngộ nghĩnh. Xong, lén lấy tấm bạc phơi lúa của bà ngoại tôi (nói là “lén” chứ thực ra bà biết nhưng vờ không hay!) đem ra gấp lại làm tư cho có chiều dài và dùng chỉ khâu cẩn thận vào chiếc đầu lân quai quái kia. Cuối cùng, chúng tôi cũng có một con lân rất ư là ngộ nghĩnh và rồi cũng “dắt” nó đi múa quanh xóm, cũng kiếm được tiền mua đậu nấu chè liên hoan, chia lãi… Đó là chưa nói đến chiếc trống được tạo bằng thùng đựng nước, bịt miếng vải dù lên miệng thùng và quấn dây thun, kéo thật căng buộc lại, dùng hai cây dùi vót bằng tre có đầu phình to để gõ, mỗi khi đánh lên mặt vải dù, âm thanh phát ra nghe tủng… tủng… tủng … rất vui tai! Cũng xin nói thêm, vải dù  là một loại vải cắt ra từ những chiếc dù của lính trong chiến tranh. Vào những năm 80, thứ hàng này được xem như của báu ở thôn quê Việt Nam, phần lớn người dân giỏi lắm cũng được mặc thứ vải tám do hợp tác xã phân phối và muốn có được thứ vải đó đôi khi phải biết hối lộ cán bộ lương thực. Thời đó, người ta có câu cửa miệng “nhất thuế vụ, nhì lương thực…” 
 
Giả sử không có chiếc đầu lân thì tấm bạc chỉ là đồ dùng phơi lúa, và ngược lại, nếu không có tấm bạc thì sẽ chẳng bao giờ có một con lân xinh xắn, dễ thương đang nằm sâu bền trong trí nhớ để tôi kể ra bây giờ. Vấn đề là lúc đó, với cách nghĩ trẻ nít, thấy vui là làm, làm không cần suy tính, không theo một công thức hoặc một cách thức cố hữu nào nên “bọn trẻ” mới dám làm, mới tạo ra được tác phẩm lắp ghép hồn nhiên và dễ thương đến vậy! Tôi cho rằng đó cũng là một thứ nghệ thuật của tuổi thơ, một thứ nghệ thuật thuần túy chưa biết đến luận giải, học thuật và hàn lâm – những thứ vốn làm con người trưởng thành hơn, già dặn hơn trong nghệ thuật, đời sống nhưng đồng thời cũng có thể khuôn đúc người nghệ sĩ chết trơ trong những khái niệm định sẵn.  
 
Đương nhiên không thể so sánh một cách sống sượng chiếc đầu lân với \textit{Tháng tư gãy súng}, ở đây, tôi chỉ muốn nêu lên một mối liên hệ. 
\begin{blockquote}
 
\textit{có rât nhiều kiểu hôn thơ mà có thể bạn chưa từng mường tượng ra. mỗi kiểu hôn thơ lại có một thông điệp riêng. dưới đây là một số kiểu hôn thơ mà bạn có thể tham khảo hòng mang lại cảm giac tuyệt vời nhưt cho n[ch]àng thơ của mình } 
 
\textit{\textbf{1. hôn mơn trớn}  
chạm sat mặt của bạn vào mặt thơ, chơp nhẹ hai hàng mi &amp; cọ nhẹ vào má thơ. nếu bạn mần đúng thao tac, bạn sẽ mang lại cho thơ cảm giac thiệt gần gũi, ấm ap. } 
 
\textit{\textbf{2. hôn má } 
một nụ hôn thân thiện kiểu như là tôi thich thơ lắm đấy. thường thì cach hôn này cho lần đầu còn bỡ ngỡ. đặt bàn tay lên vai rồi nhẹ nhàng lướt lên má thơ. } 
 
\textit{\textbf{3. hôn tai } 
nhẹ nhàng nhấm nhap tai thơ. tránh những tiếng động quá lộ liễu, sẽ đánh mât cảm giac của thơ. } 
 
\textit{\textbf{4. hôn kiểu... người inuit}  
thay vì dùng môi, bạn hãy dùng mũi mình cọ nhẹ vào mũi thơ. kiểu hôn này cũng mang lại cảm giac rât đặc biệt. } 
 
\textit{\textbf{5. hôn lên măt } 
hai tay của bạn giữ lấy đầu thơ &amp; chầm chậm ngiêng đầu thơ theo hướng mà bạn muốn nụ hôn sẽ tới. từ từ hôn ngược lên phía đôi măt nhắm hờ của thơ. } 
 
\textit{\textbf{6. hôn ngón tay } 
khi nằm bên nhau, cắn nhẹ mơn man ngón tay thơ. } 
 
\textit{\textbf{7. hôn chơn } 
đây là một cử chỉ rât gợi tình &amp; lãng mạn. nó có thể mần thơ nhột nhưng hãy cứ tiêp tục. đầu tiên cắn nhẹ ngón chơn cái, rồi hôn lươt cả bàn chơn. } 
 
\textit{\textbf{8. hôn trán } 
nụ hôn của những người thân. đặt nhẹ nhàng môi lên trán thơ. thơ sẽ cảm thấy nhỏ bé &amp; cần được che chở. } 
 
\textit{\textbf{9. hôn tan chảy}  
có một thí ngiệm vui với nụ hôn này như sau. đặt một miếng đá nhỏ trong miệng sau đó há miệng ra &amp; hôn thơ, dùng lưỡi chuyển viên đá sang miệng thơ. đây thực sự là một nụ hôn quyến rũ kiểu phap. } 
 
\textit{\textbf{10. hôn kiểu phap } 
là nụ hôn phải dùng đến lưỡi. một số người cho là nụ hôn tâm hồn. nó tạo nên sự đồng điệu giữa hai tâm hồn bằng cach truyền hơi thở của bạn &amp; thơ qua lưỡi nhau. thiệt ngạc nhiên, người phap lại gọi đó là nụ hôn kiểu anh } 
 
\textit{\textbf{11. nụ hôn có hương vị huê quả } 
} 
\textit{lấy một miếng huê quả rồi đặt giữa hai môi của bạn (như nho, dâu, miếng dứa nhỏ hoặc xoài đều lí tưởng ). hôn thơ &amp; liếm một nửa miếng huê quả, trong khi thơ cũng mần như thế, cho đến khi miếng huê quả bị cắt đôi. cứ để miếng huê quả chạy thẳng vào miệng bạn. } 
 
\textit{\textbf{12. nụ hôn làm tin } 
} 
\textit{chẳng hạn như phủ lên môi bạn một dải băng để thu hut sự chú ý của thơ. làm ra vẻ như bạn đang cố nói một điều giì nhưng không sao gỡ được dải băng ra. khi thơ tháo dải băng giup bạn hãy nói với thơ rằng “tôi đã cât giữ đôi môi cả ngày chỉ để dành riêng cho thơ”. ngay sau đó là một nụ hôn say đắm. } 
 
\textit{\textbf{13. nụ hôn nóng lạnh } 
} 
\textit{liếm môi thơ cho đến khi nóng dần rồi đột ngột thổi phù lên đó. làn hơi mat sẽ tạo nên sự bùng nổ cảm giac. chăc chắn thơ sẽ mần lợi đặng cả hai cùng đạt đến đam mê. } 
 
\textit{\textbf{14. hôn qua thư } 
} 
\textit{gởi cho thơ những nụ hôn qua thư bằng cach gi các kí tự x vài lần trong cùng một hàng ở cuối thư, như kiểu xxxxxxx. } 
 
\textit{\textbf{15. hôn liếm}  
} 
\textit{trươc khi hôn đưa lưỡi của bạn dọc theo môi thơ, bât kể ở đầu hay cuối môi. bạn sẽ có những cảm giac mê đắm thiệt lạ thường. } 
 
\textit{\textbf{16. hôn cổ } 
l} 
\textit{ươt nhẹ từ môi xuống cổ thơ rồi quay lại hôn môi. } 
 
\textit{\textbf{17. hôn cắn môi } 
} 
\textit{nụ hôn này có thể tạo ra cảm giac gợi tình. khi đương hôn, hãy cắn nhẹ vào môi thơ. chú ý đừng quá mạnh nếu không bạn sẽ làm đau thơ đấy. } 
 
\textit{\textbf{18. hôn môi đảo ngược } 
} 
\textit{ngĩa là bạn đứng bên trên thơ &amp; hôn thơ từ trên đầu. theo cach này hai người sẽ cảm nhận được sự nhạy cảm của môi dưới đối phương bằng cach cắn &amp; mut. } 
 
\textit{\textbf{19. hôn rún } 
} 
\textit{sử dụng môi &amp; lưỡi để làm nhột rồi hôn rún của thơ. } 
 
\textit{\textbf{20. hôn vai } 
} 
\textit{đến từ phía sau, ôm choàng lấy thơ rồi hôn từ vai xuống. nụ hôn này chứa rât nhiều cảm xuc &amp; thương yêu. } 
 
\textit{\textbf{21. hôn nhấm nhap } 
} 
\textit{băt đầu từ trán, rồi hôn lươt trên môi, sau đó chuyển xuống cánh tay, đến bàn tay, rồi quay trở về cánh tay,lên mặt &amp; nhẹ nhàng hôn lên môi cho tới khi thơ đòi hỏi một nụ hôn mê đắm. } 
 
\textit{\textbf{22. hôn nói } 
} 
\textit{thì thầm vào miệng thơ. nếu bị băt quả tang, hãy nói đơn giản như chico mã, “tôi không định hôn thơ. tôi chỉ thì thầm vào miệng thơ đấy chứ”. } 
 
\textit{\textbf{23. cắn lưỡi } 
} 
\textit{một hình thưc của hôn kiểu phap. khi thơ đương hôn \textbf{mở miệng} hãy cắn nhẹ vào lưỡi thơ (nhưng đừng cắn quá mạnh bởi có thể làm đau thơ) } 
 
\textit{\textbf{24. hôn vội vã } 
} 
\textit{khi bạn đương bận, có việc gâp phải đi ngay, hãy cứ hôn nhẹ lên mũi thơ thay vì môi. } 
 
\textit{\textbf{25. nụ hôn chơn không}  
} 
\textit{trong khi hôn \textbf{mở miệng}, mut nhẹ như thể bạn đương lấy không khí từ miệng thơ. đây là một cach hôn rât ngộ ngĩnh. } 
 
\textit{\textbf{26. hôn đánh thưc } 
} 
\textit{trươc khi thơ ngủ dậy, hãy hôn nhẹ lên má rồi chuyển thành những nụ hôn mềm mại cho tới khi bạn hôn đến môi n[ch]àng ta. chăc chắn không có cach đánh thưc nào ngọt ngào hơn thế. } 
 
\textit{\textbf{27. hôn ảo } 
} 
\textit{đối với những người yêu thơ qua internet, hãy gởi một tấm thiệp điện tử hoặc một nụ hôn qua email bằng biểu tượng.} 
 
\textit{liên hoàn cươc chú:} 
 
\textit{\textbf{ý ngĩa của những nụ hôn }}  
\begin{itemize}

item{
\textit{hôn lên tay: tôi ngưỡng mộ, sùng bái thơ } }

item{
\textit{hôn lên má: tôi thich thơ rồi đấy } }

item{
\textit{hôn lên cằm: thơ thiệt đáng yêu } }

item{
\textit{hôn lên cổ: tôi muốn thơ! } }

item{
\textit{hôn lên môi: tôi yêu thơ } }

item{
\textit{hôn lên tai: đùa chut nào } }

item{
\textit{hôn bât cứ chỗ nào trên cơ thể: thơ thiệt tuyệt vời } }

\end{itemize}
 \textit{\textbf{những nụ hôn mà thơ thich nhưt}}  
\begin{enumerate}

item{
\textit{ phía sau tai } }

item{
\textit{ chop mũi } }

item{
\textit{ sau gáy } }

item{
\textit{ mặt dưới cánh tay trươc } }

item{
\textit{ thăt lưng, eo } }

item{
\textit{ lòng bàn tay } }

item{
\textit{ cổ tay } }

item{
\textit{ mí măt } }

item{
\textit{ măt cá chơn } }

item{
\textit{ đầu ngón tay } }

item{
\textit{ xương bả vai } }

item{
\textit{xương sống } }

item{
\textit{ đằng sau đầu gối } }

item{
\textit{ hôn bụng… &amp; từ từ xuống dưới… } }

\end{enumerate}
 \textit{Nguồn:} http://netmode.vietnamnet.vn/namgioi/anhtieudiem/2005/04/409907/\footnote{\url{http://www.talawas.org/talaDB/http://netmode.vietnamnet.vn/namgioi/anhtieudiem/2005/04/409907/}}  
 
(“thơ. 27 kiểu hôn” Tr. 2. Mặt A) 
 
Có thể nói rằng hành động dán những mẩu tin quảng cáo, rao vặt, biến chúng thành tác phẩm thơ của Bùi Chát có mối liên hệ nào đó với việc kết đầu lân vào tấm bạc. Cả hai thứ “vật liệu” trên đều không có mối tương quan nào với cái nó tạo ra. Có chăng thì cũng ở một xác suất rất thấp, gần như là zero! Nhưng một khi người nghệ sĩ phả sức sống của mình lên sự vật nhằm thực thi ý niệm, bản ngã thông qua nỗ lực, biến hoá, đương đầu với cái đã định danh, định tính (của anh ta và thời đại anh ta sống) thì một khả thể “ra đời” với diện mạo mới mẻ, sinh động của nó, sẽ tạo ra không ít ngạc nhiên và khó chịu (nói theo ngôn ngữ mà các nhà phê bình nghệ thuật Việt hay dùng là phản cảm – một khái niệm cũng mù mờ và không kém phần… phản cảm!).  
\begin{blockquote}
 
\textit{[không sợ ráng chịu: theo khuc duy &amp; lê ôn] } 
 
\textit{oc: rái cá[1]  
oedipe: khoảng rào kín[2]  
ogre: rắn[3]  
om: rắn vipe[4]  
ondines: rắn uraeus[5]  
ong: răng[6]  
ong vò vẽ: râu cằm[7]  
orphée: rây[8]  
osiris: rebis[9]  
ouranos: rìu[10]  
ouroboros: rìu mõm chồn[11]  
ô: cái rom[12]  
ô dù: rún[13]  
ôliu: rồng[14]  
ôrô: rùa [15]  
ôtô: ruồi[16]  
ôc sên: ruột[17]  
ống bễ, ống thổi: rửa, gội[18]  
ống điếu thần: rừng[19]  
ống thông hơi: rương[20]  
ống xì: rượu mật ong[21] } 
 
\textit{\textbf{đối &amp; chiếu:} } 
 
\small{\textit{[1]trèo lên cây ổi [vô tư]  
[2]hái huê [hớn hở]  
[3]bươc xuống vườn huệ [huế]  
[4]ngăt nụ xì tin [mực tím/tuổi mới lớn]  
[5] nụ tì xin [mực xanh/tuổi mới ốm]  
[6] thè ra mim mím [nhẹ nhàng]  
[7] em bán mình rùi [cho ai, cho ai]  
[8] anh… [?]  
[9] tím tái thay! [dễ sợ]  
[10] [ô hay] !!!!  
[11] bao tiền một chiêc máy bay [du thuyền cũng đặng]  
[12] sao anh không sắm [tậu]  
[13] luc/thời còn nguyên [luyến tiếc]  
[14] bi giừ em đã có tuyền [tiền]  
[15] như con chim chuyền [khó nắm]  
[16] như cá dưới sông [khó lội]  
[17] cá dưới sông có khi còn băt [nói thăt]  
[18] chớ cá dưới bể băt mần răng đây [nói thiệt]  
[19] lại về trèo lên ngọn cây [hớn hở lần nữa]  
[20] ngồi chờ huê nở [mong đợi]  
[21] bàn tay rờ cành/&amp; lá [ngậm ngùi] }} 
 
\textit{Nguồn:} [trong bảng tra mục từ: o &amp; r]. từ điển biểu tượng văn hoá thế giới (phụ lục) – nhà xuât bản đà nẵng, trường viêt văn nguyễn du – tháng 10.1997  
(“bài tập liên kêt ngữ vựng/sự im lặng đáng sợ” - Tr.31. Mặt B) 

\end{blockquote}
 
Nhưng với một tập thơ, nếu chỉ dừng lại ở những đặc điểm vừa nêu thì e rằng chưa thể gọi là hấp dẫn, mới lạ, sáng tạo. Bởi lẽ độc giả không thể chấp nhận buông cảm xúc, cảm giác và lý trí của mình trôi theo một loại nghệ thuật chỉ thuần tô vẽ hình thức và trí trá (nếu không có nội dung tốt đương nhiên những thao tác nhằm tạo mới lạ bên ngoài chỉ là hành vi lừa bịp), rỗng tuếch… \textit{Tháng tư gãy súng} chứa đựng ý đồ tác giả bên trong tác phẩm nhiều hơn là phần lắp ráp bên ngoài. Từ kĩ thuật trích nguồn, dán tựa, dán đuôi chú thích… cho đến triệt tiêu thanh sắc trong trường hợp không cần thiết bởi yêu cầu của từ vị đã khiến những bài thơ có độ nén, độ hàm súc của ngôn từ và độ căng ngữ nghĩa. Những cụm từ, từ riêng lẻ, giới từ như: \textit{rất, tất, mất, đất, nấc, khất… nhức, đứt, mức, đức… giúp, cúp, đúc, múc, quốc, xác suất, khúc khích, rúc rích, xúc xích, mít ướt…} được lược bỏ dấu, thành: \textit{rât, tât, nâc, khât…nhưc, đưt, mưc, đưc… giup, cup, đuc, muc, quôc, xac suât, khuc khich, ruc rich, xuc xich, mit ươt…} Ý nghĩa được mở rộng bởi những mắt chữ, mắt âm có sức nhốt, sức rướn của ngữ vựng, ngữ nghĩa… Những cái rất đỗi quen thuộc bỗng mới mẻ, bí ẩn bởi mạch phức cảm nào đó đang phôi sinh, hình thành, kích hoạt vào cảm giác người đọc. Đến lượt mình, người đọc lại tìm thấy một mạch khác được khai mở nội tại (trong chính cách tiếp cận mang dấu ấn cá nhân của họ thông qua những biến ảo của hệ từ mở…). Và điều quan trọng nhất là khả năng tạo ra tiếng cười sảng khoái cho người đọc xuyên suốt tập thơ, tiếng cười không phải do cù lét mà có mà là cảm giác hài hước nấp sau sự suy ngẫm. 
\begin{blockquote}
 
\textit{[tham luận gởi đại hội nhà văn tp hcm 3/2005]}  
\begin{enumerate}

item{
\textit{thỉnh thoảng kiểm kê đồ đạc xem có món nào quên chưa dùng tới. phat hiện nó bạn sẽ cảm thấy vui cũng như mua đồ mới - &amp; tránh tràng hạp mua những thứ đã mua rồi! } 
 }

item{
\textit{mua đồ theo giá trị sử dụng &amp; luôn coi trọng sự lâu bền, đừng để bị cuốn hut bởi mã ngoài hoặc chạy theo trào lưu.} 
 }

item{
\textit{tìm mọi lý do trì hoãn hoặc không mua nếu món đồ chưa thiệt sự cần thiêt. } 
 }

item{
\textit{luôn tự nhủ hầu hêt hàng hoá đều được sản xuât hàng loạt, luôn được thay thế bởi những xêri tối tân hơn chứ không phải thứ giì quí hiếm mà phải mua ngay bằng mọi giá. } 
 }

item{
\textit{hạn chế cho đồ cũ (làm từ thiện thì không tánh) vì việc này sẽ khiến bạn cảm thấy cần phải đi mua đồ mới ngay để... tự thưởng cho sự hào phóng của mình. } 
 }

item{
\textit{tránh để bị tac động bởi những ảo tưởng về giá trị. đại loại như phải mua thơ đời mới thời bạn mới là người năng động, hay văn học phải “xoay” thời mới kiêu hãnh. } 
 }

item{
\textit{thỉnh thoảng về các vùng quê hoặc du lịch đến những nơi hoang dã để thấy người ta có thể sống thoải mái bên ngoài thế giới vật chât. còn bạn sẽ thấy vừa lòng hơn với chánh mình. } 
 }

item{
\textit{tránh chơi với những người hợm hĩnh, khoe của. mỗi người có lối sống &amp; có quyêt định của riêng mình. } }

\end{enumerate}
     
(\textit{Nguồn}: http://www.tuoitre.com.vn/tianyon/index.aspx/articleid=69151&channelid=7\footnote{\url{http://www.talawas.org/talaDB/http://www.tuoitre.com.vn/tianyon/index.aspx/articleid=69151&channelid=7}} 
                            
(“8 cach để nữ sĩ @ hạn chế mua sắm thơ “- Tr. 8. Mặt A) 

\end{blockquote}
 
Điểm đặc biệt cần lưu ý khi đọc \textbf{Tháng tư gãy súng} là bố - cục – xoay - chiều của tập thơ. Thay vào cách đọc thông thường đi theo thứ tự từ trang đầu đến trang cuối, mặt sau của trang trước sẽ mang nội dung/ số thứ tự tiếp theo cho trang sau cho đến hết tập trong một chỉnh thể đã được mặc định dựa vào ý đồ của tác giả và nhà xuất bản (vốn trở thành cách xử lý thông thường trong in ấn) thì ở đây, bạn đọc khi lật sang mặt sau sẽ gặp một trang khác, một trục ý nghĩa khác, hoàn toàn xa lạ với trang trước và nghịch chiều (trong mọi nghĩa). Hiệu ứng thị giác, hiệu ứng installation được nâng lên một tầm mức đáng kinh ngạc. Dường như tác phẩm đã có sự cộng hưởng giữa sách điện tử (nơi bạn đọc có thể khai thác một cách triệt để khả năng tạo tác của người đọc để tái tạo tác phẩm theo ý muốn…) với sách in văn bản (âm trên giấy). Đây cũng là nét đặc dị của \textit{Tháng tư gãy súng}. Từ kiểu trình bày theo qui chuẩn ở mặt này chuyển sang kiểu sắp xếp thưa thớt, thả dấu ngoặc, dấu phẩy, dấu ngang một cách hỗn mang, quái dị ở mặt kia; từ nội dung bỡn cợt, giễu nhại ở trang trước, lật sang trang sau lại là một nội dung triết lí tưng tửng, nói như khóc mướn… (trong trạng huống quay ngược đầu), tất cả khiến cho tập thơ vừa buồn cười, rắc rối nhưng đồng thời cũng là lạ, bí hiểm và sinh động. Đương nhiên, những yếu tố này mang tính thử nghiệm nhưng là một cuộc thử nghiệm sáng tạo, dũng cảm, chấp nhận trả giá bằng sự cô đơn của cái mới chưa được chấp nhận, chưa có tiền lệ trong “mỹ cảm truyền thống”. 
\begin{blockquote}
   
\textit{[quảng cáo giùm lý đợi về gian hàng thơ việt, tại hội chợ thơ toàn thế giới (dự định 2012)]  
 
\textbf{sự tin cậy}  
là sản phẩm đã tồn tại cả ngàn năm qua  
 
\textbf{sự tín nhậm}  
đã được khoa học chứng minh là thưc ăn có nhiều lợi ich dinh dưỡng  
 
\textbf{chứng nhận}  
đã được câp giấy chứng nhận ISO 9001: 2000, ISO 14001, GMP &amp; HACCP  
 
\textbf{cam kêt}  
không có cholesterol, không hoá chât biểu quản, không màu nhơn tạo  
 
\textbf{cần thiêt cho cuộc sống năng động}  
vì thơ việt là thưc ăn bổ dưỡng:}  
\begin{itemize}

item{
\textit{cung câp năng lượng tưc thời cho cơ thể } }

item{
\textit{ giup tăng cường sưc đề kháng &amp; sưc mạnh tinh thần } }

item{
\textit{ tham gia thuc đẩy nhanh quá trình phục hồi sưc khỏe } }

item{
\textit{ cải thiện trí nhớ &amp; năng lực trí tuệ } }

item{
\textit{  giup giảm căng thẳng thần kinh } }

item{
\textit{ tăng quá trình tiêt sữa non ở phụ nữ [có mần thơ] nuôi con } }

item{
\textit{ bổ máu nhờ khả năng tăng hâp thụ &amp; sử dụng chât săt } }

\end{itemize}
 \textit{\textbf{lưu ý:} thực phẩm này không phải là thuôc, không có tac dụng thay thế thuôc chữa bịnh} 
 
\textit{Nguồn:} \textit{trang quảng cáo 15 – báo tuổi trẻ - ra ngày thứ sáu 11-03-2005}  
 
(“cho tinh thần. cho cơ thể. cho cuộc sống. hay là 5 lý do để bạn chọn thơ việt) 

\end{blockquote}
 
Sự lạ lùng trong thủ pháp, sự quái dị trong bố cục, sự rắc rối, khúc khuỷu trong cấu trúc, sự lan man trong ngữ âm, sự ngông ngếch trong ngữ nghĩa tưởng như là một thứ tạp-pí-lù vô hình trung đã tạo ra “món ăn mới” mới, giàu sức sống cùng những biến tấu giữa phông nội dung với cấu trúc, mô thức văn hoá, chiều kích âm vang từ lượng… hàm ẩn bên trong phản xạ tự nhiên của tâm lí, của ý hướng sáng tạo, tìm tòi đã khiến \textit{Tháng tư gãy súng} xác lập được một giá trị, gợi ra những khả thể tồn tại mới của thơ.  
\begin{blockquote}
 
\textit{[từ gãy đến rât gãy] } 
 
\textit{nguyên nhơn chủ yếu dẫn đến tình trạng này là: người đờn ông tự bẻ súng luc đương gay cấn, do bị kich thich quá độ, cũng có thể do một tai nạn bât ngờ, hoặc do đối tac thô bạo khi quan hệ[1]... trong chiến tranh, đã có nhiều tràng hạp phải tạo hình cho cac chiến sĩ. đây là việc mần cần thiêt để họ trở về cuộc sống bằng thường, được mần cha, mần chồng. } 
 
\textit{tuy nhiên, hiện nay trong thời bằng còn có nhiều lý do hơn để người ta đi tạo hình vũ khí đặc biệt này. } 
 
\textit{\textbf{1.}  
đã mần cha của 3 đứa con nhưng anh lê thanh d. vẫn không thể nào quên đặng quãng thời gian hơn 10 năm sống trong đau khổ. năm anh d. vừa tròn 2 tuổi, trong một lần đi tiểu tiện chú chó hàng xóm đã chén mât mẩu "thịt thừa" của anh. luc đó, anh còn quá bé để biêt hậu quả của việc thiếu mẩu thịt cho cuộc sống sau này. năm anh d. 18 tuổi, cái tuổi đã biêt thế nào là bổn năng đờn ông, anh băt đầu cảm thấy bưc bach. } 
 
\textit{''nếu không mần thế nào để con lấy được vợ con sẽ tử tự'', anh d. tuyên bố. chỉ còn một cach duy nhưt là đưa anh đi tạo súng giả. } 
 
\textit{\textbf{2.}  
công việc ổn định, là kỹ sư của một cơ quan nhà nươc, chưa cùng ai bao giờ nhưng anh đinh xuân t. say đắm một cô gái đã có chồng. yêu gái có chồng nhưng đã ly hôn thì không sao, đằng này chồng cô gái lợi đương ở tù. mối tình đẹp kéo dài được thời gian thì chồng cô gái mãn hạn tù. tưởng rằng từ nay đàng ai nấy đi nhưng hai con người ấy không xa được nhau. anh chồng biêt chuyện đã không để an. } 
 
\textit{''một là cả hai cùng chêt. hai là mang được cái đó của thằng đó về đây'', anh chồng tuyên bố dưt khoat sau khi biêt vợ ngoại tình. biêt chồng không nói chơi cô gái đành phải theo phương án thứ hai. vẫn những lời tình tứ ngọt ngào như không có chuyện giì xảy ra, nhưng đúng vào luc cả hai đương ở chín tầng mây thì "xoẹt", nòng súng đã lìa xa, máu vọt lên bắn tứ tung, cô gái sợ quá không còn bụng dạ thu chiến lợi phẩm về báo cáo. } 
 
\textit{\textbf{3.}  
anh đ.t.p (ở quận thanh xuân, hà nội) sanh sống trong một gia đình có tới 3 thế hệ, lại không có điều kiện tach riêng cac phòng. cả nhà chỉ vẻn vẹn 40m2 khiến sanh hoạt gia đình khá bât tiện. đương ở tuổi xuân sưc lại phải kìm hãm khiến anh p. luôn cảm thấy khó chịu, bưc bối. anh thường xuyên trong tình trạng trên bảo dưới không nge. } 
 
\textit{anh p. đã tâm sự với bac sĩ trong lần đến điều trị rằng, anh luôn trong tình trạng không thể mần được cái thiên chưc ông trời ban cho đờn ông, ngĩa là nổ súng. có luc gần được thì nó lại ỉu xìu. oái oăm thay, nhiều luc không gần gũi vợ, chỉ cần kich thich nhẹ súng đã sẵn sàng &amp; có thể xuât chiêu. những luc này, nhà đông người anh không biêt xử trí ra sao &amp; trong một lần quá tay đã bẻ gãy nòng súng. } 
 
\textit{\textbf{kêt } 
sau lần đó, anh p. đã phải phẫu thuật bằng cach khâu cầm máu chỗ vỡ thể hang, dẫn lưu máu tụ &amp; khâu lại bao xơ. mọi chuyện lại trở về như xưa nhưng anh p. không dám mạnh tay lần nữa. tràng hạp anh t. sau khi bị người tình căt bỏ vật quý giá nhưt, đã vội vã vượt hơn 100 cây số về hà nội nhờ các chuyên gia bv việt - đưc nối gep sau hơn 3 giờ đã hoàn thành. anh trở về cuộc sống bằng thường, lấy vợ &amp; có 2 con } 
 
\textit{còn cuộc phẫu thuật giup anh d. lấy được vợ kéo dài gần 6 tiếng đồng hồ. sau này, anh d. đã mần cha của 3 người con, trong đó một cháu đã tham gia cuộc thi huê hậu. } 
 
\textit{khuyến cáo } 
\textit{nếu gãy nòng súng mà bịnh nhơn điều trị kịp thời có thể tự lành (tràng hạp nhẹ) nhưng nòng súng sẽ bị ngoẹo, việc tiểu tiện khó khăn, thậm chí không tiểu tiện được. quan hệ tình dục khó khăn &amp; dễ biến chứng như viêm đàng tiêt niệu, sanh dục, đặc biệt có thể bị liệt dương hoàn toàn. nếu bịnh nhơn bị gãy mà để quá lâu mới phẫu thuật nòng sẽ ngắn hơn. } 
 
\textit{chú: [1] theo ngôn ngữ phạm hoàng quân gọi là: mỹ} \textit{nhơn hí dương vật…. quá chớn}  
 
\textit{Nguồn:} http://netmode.vietnamnet.vn/namgioi/canhdanong/2005/04/405337/\footnote{\url{http://www.talawas.org/talaDB/http://netmode.vietnamnet.vn/namgioi/canhdanong/2005/04/405337/}}  
 
(“tháng tư gãy súng”  - Tr. 27. Mặt B) 

\end{blockquote}
 
Có thể nói, \textit{Tháng tư gãy súng} sẽ không mang lại cho bạn đọc một ý nghĩa hay một triết lý nhất định nào đó; nó còn khiến người đọc lan man và rối mù nếu không có một tâm thế mở rộng và tĩnh tại để đón nhận những gì khác lạ so với mỹ cảm (quen thuộc) của mình. Nhưng chí ít, tác phẩm này cũng mang đến một thể cộng hưởng cho người đọc nó, tạo ra mối liên hệ sáng tạo giữa tác giả và độc giả, cách đọc và số lần đọc, cách lựa chọn trang và kết nối với trang khác… Dường như đó là điều Bùi Chát muốn gửi đến bạn đọc. Và người nghệ sĩ có quyền lựa chọn thái độ sáng tạo theo những gì ý niệm gợi mở họ. Mọi khả thể đều hình thành từ/ bởi một cái gì đó rất ư bất khả và phi lý! Vấn đề đọng lại trong độc giả khi đọc \textit{Tháng tư gãy súng} có lẽ là câu hỏi: \textit{Nó} có phải là thơ?! 
 
©talawas 2008  




\end{multicols}
\end{document}