\documentclass[../main.tex]{subfiles}

\begin{document}

\chapter{Đọc một bài thơ như thế nào}

\begin{metadata}

\begin{flushright}22.9.2008\end{flushright}

Nguyễn Đức Tùng



\end{metadata}

\begin{multicols}{2}

\textbf{Bài bốn: Điểm nhìn và nhân vật} 
 
\textit{Hầu hết những điều mà thơ nói với chúng ta, chúng ta đều biết cả rồi, nhưng biết không đầy đủ, biết không sâu sắc, không đủ mãnh liệt để thay đổi.} 
David Constantine 
 
Thơ đương đại ngày càng có khuynh hướng nghiêng về văn xuôi, không chỉ thể hiện ở bình diện ngôn ngữ, mà còn cả ở phong cách tự sự. Ở các nền thơ phương Tây, thơ tự sự và thơ có tính kịch có truyền thống lâu đời và vẫn tiếp tục tạo ảnh hưởng; trong khi đó, thơ Việt Nam, ít nhất là kể từ Thơ Mới đến nay, không có truyền thống đó. Vai trò của người nói, người kể chuyện, các nhân vật, và mối quan hệ giữa họ với nhau, có thể được xem xét rõ ràng hơn trong bối cảnh của các câu chuyện kể; nhưng các câu chuyện như thế, đáng tiếc, lại ít khi được tìm thấy trọn vẹn trong thơ trữ tình.  
 
Do nhiều nguyên nhân, mà trực tiếp là ảnh hưởng của chủ nghĩa lãng mạn vốn lấy cái tôi làm trung tâm, những thành tựu lớn nhất của thơ Việt Nam mấy chục năm qua, xuyên suốt các giai đoạn lịch sử và các không gian văn học, từ Thơ Mới đến thơ kháng chiến, trong thơ miền Nam lẫn thơ miền Bắc… chủ yếu vẫn là thơ trữ tình. Ngay cả các trường ca cũng đậm chất trữ tình. 
 
Thơ trữ tình truyền thống là tiếng nói đơn độc từ một cá nhân này đến một cá nhân khác, từ một tâm hồn riêng lẻ đến một tâm hồn riêng lẻ thứ hai. Nó được viết ra trong im lặng, được đọc lên trong im lặng, được nhớ lại trong im lặng. Phương pháp hiện thực xã hội chủ nghĩa, mà hiện nay các nhà thơ đang xa lánh nó như một bóng ma, thực ra đã góp phần phá vỡ truyền thống này, tạo ra dòng thơ được gọi là trữ tình cách mạng. Nhưng ngay cả trong dòng chảy này, thơ trữ tình vẫn là tiếng nói của cái tôi, mặc dù không còn hoàn toàn đơn độc nữa.  
 
Tôi lấy làm ngạc nhiên là khi gặp các bài thơ mới hiện nay có khuynh hướng tự sự, người đọc, nhiều người trong số họ là những nhà thơ, truyền thống hay hậu hiện đại, tỏ ra lúng túng trước các mối quan hệ tác giả - người nói – nhân vật. Tôi tin rằng trong việc hiểu thơ, khái niệm \textit{điểm nhìn} có tầm quan trọng đặc biệt. Điểm nhìn chính là người quan sát và là mối liên hệ giữa anh ta/ chị ta và sự vật được quan sát. Có những câu hỏi sau đây cần đặt ra:   
\begin{itemize}

item{Ai là người nói (speaker) hay người kể chuyện (narrator)?  }

item{Ai là nhân vật (character)?  }

item{Ai là tác giả (author)? }

item{Mối quan hệ giữa người nói và nhân vật là gì?  }

item{Mối quan hệ giữa người nói và hành động là gì? }

\end{itemize}
 Cũng như đối với truyện ngắn và tiểu thuyết, bài thơ chỉ có thể có một trong hai điểm nhìn, (từ) ngôi thứ nhất và (từ) ngôi thứ ba. Trong một số trường hợp, ngôi thứ hai cũng được sử dụng, nhưng chúng ta sẽ không bàn ở đây.  
\begin{blockquote}
        
\textit{Nàng có ba người anh đi bộ đội}        
\textit{Những em nàng}        
\textit{Có em chưa biết nói} 
\textit{Khi tóc nàng đang xanh.} 
        
\textit{Tôi người Vệ quốc quân}        
\textit{Xa gia đình} 
\textit{Yêu nàng như tình yêu em gái} 
        
(“Màu tím hoa sim”, Hữu Loan) 

\end{blockquote}
 
Trong bài thơ tự sự- trữ tình thuộc vào loại hay nhất của thi ca Việt Nam, người kể chuyện ở ngôi thứ nhất, số ít. Chúng ta cũng dễ dàng cho rằng người kể chuyện chính là tác giả. Sở dĩ như thế là vì tiểu sử của nhà thơ Hữu Loan thống nhất với bài thơ, ông cũng là người đi kháng chiến chống Pháp, và vì chính ông cũng kể lại câu chuyện có thật của đời mình trong một số bài viết hay phỏng vấn. Nhưng điều đó không xảy ra đối với nhiều bài thơ khác. Đây là trường hợp người kể chuyện tham gia trực tiếp vào câu chuyện, là một nhân vật của câu chuyện ấy. Cũng có khi người kể chuyện chỉ là người quan sát đứng bên lề, không phải là kẻ trực tiếp tham gia. Giữa hai thái cực này, bao giờ cũng có những trường hợp trong đó người kể vừa phần nào là kẻ quan sát vừa phần nào là người tham dự vào các hành động hoặc các sự kiện.  
\begin{blockquote}
        
\textit{When my mother died I was very young}        
\textit{And my father sold me while yet my tongue}        
\textit{Could scarcely cry weep weep weep weep}        
\textit{So your chimneys I sweep &amp; in soot I sleep.}        
        
(“The Chimney Sweeper”, William Blake) 
        
\textit{Mẹ tôi mất khi tôi còn bé dại}        
\textit{Cha bán tôi đi khi tôi vừa tập nói}        
\textit{Tôi rao khóc khóc khóc khóc} 
\textit{Ngày tôi cạo ống khói, đêm tôi ngủ trong muội} 

\end{blockquote}
 
Tôi không thể dịch được lối chơi chữ của Blake, với các âm \textit{weep, sweep, sleep}, nửa líu ríu nửa bi bô của trẻ con.  
 
Người kể chuyện trong bài thơ đứng ở ngôi thứ nhất nhưng ai cũng thấy rằng đó không phải là tác giả. Không có gì cấm một đứa trẻ mố côi cạo ống khói trở thành một nhà thơ lớn, nhưng đó không phải là tiểu sử của Blake. Bài thơ “Người thợ cạo ống khói” được viết vào khoảng 1789, trong bão táp của cuộc cách mạng Pháp. Thoát ra khỏi ảnh hưởng của chủ nghĩa lãng mạn vốn có cái nhìn đẹp đẽ về con người và thiên nhiên, Blake rọi ánh sáng vào các góc tối đen của xã hội và của tuổi thơ nghèo đói, bằng cách mang một mặt nạ khác: mặt nạ của một đứa trẻ. Chỉ bằng chiếc mặt nạ, nhà thơ mới giữ được thái độ thản nhiên, lạnh lùng, thậm chí hài hước, một thái độ mang lại cho bài thơ sức mạnh mà thể trữ tình không làm được.  
 
Người kể chuyện cũng có thể ở ngôi thứ ba, như trong bài thơ sau đây:         
\begin{blockquote}
        
\textit{Mama, please brush off my coat}        
\textit{I’m going down the street}        
Where’re you going, daughter?        
\textit{To see my sugar-sweet}        
        
(“Mama and Daughter”, Langston Hughes) 
        
\textit{Mẹ ơi, chải cho con áo khoác }        
\textit{Con phải đi ra phố bây giờ}        
Con đi đâu, hở con yêu đồ ngốc 
\textit{Con đi gặp chàng trai trong mơ} 

\end{blockquote}
 
Trong bài thơ của Hughes trên đây, không có danh xưng tôi tham dự. Câu chuyện của hai mẹ con, thông qua các đối thoại linh hoạt và thú vị, được tác giả viết lại một cách phi danh xưng. Người đọc có thể tưởng tượng như chúng ta đang ngồi nghe lén hai mẹ con trò chuyện. Như vậy tác giả có mối liên hệ gì với câu chuyện? Trong trường hợp này, anh ta chỉ là người quan sát thuần tuý, không dự phần vào các xúc cảm và hành động. Cũng như trường hợp đối với ngôi thứ nhất, có những hoàn cảnh khác nhau nối tiếp từ thái cực này đến thái cực kia của ngôi thứ ba. Người quan sát –tác giả có thể biết được tất cả mọi chuyện, như một đấng toàn năng, và kể lại cho chúng ta nghe, nhưng anh ta cũng có thể là một người quan sát bị giới hạn, như thể một người đang đứng trong góc tủ, nhìn qua khe hở ánh sáng chỉ thấy được cái lưng của cô con gái với chiếc áo khoác lông thú mịn màng. Anh ta không nhìn được mặt cô ấy mà chỉ nghe giọng nói, không biết cô đang nghĩ gì mà chỉ đoán sự việc xảy ra qua những câu trao đổi ngắn. Langston Hughes đã chọn một lối mô tả giới hạn (narrow focus). Mặc dù nổi tiếng, được nhiều người yêu mến cho đến tận ngày nay và bản thân đã từng viết hai cuốn tiểu sử tự thuật, Hughes vẫn là một tác giả khó nhận diện. Những người thân quen của ông chỉ biết những điều mà ông muốn cho người khác biết, ví dụ như chuyện ông chưa từng lập gia đình. Ông là một trong số những nhà thơ thường chọn các mặt nạ nhân vật và các giọng điệu rất khác nhau để thể hiện điều mà cái tôi - ngôi thứ nhất không thể hiện được.  
 
Như vậy, có một nhu cầu được mang mặt nạ.  
 
Có một nhu cầu của con người được chọn điểm nhìn từ vị trí khác, được đọc khác đi, được viết khác đi, và được suy nghĩ khác đi. 
 
Trừ những người nô lệ hạnh phúc, tức là những kẻ tự nguyện giam mình trong quá khứ của họ. 
Hãy nhìn vào một ngày của một người đàn ông mà bạn quen biết: buổi sáng ở sở làm, anh là viên chức; giờ ăn trưa, anh là người bạn tâm tình; buổi chiều về nhà, anh là cha; buổi tối, anh là chồng; trước mặt cấp trên, anh là đầy tớ nhịn nhục; trước mặt người đẹp, anh là anh hùng, v.v… Anh ấy có phải là bạn không? Thì chỉ có bạn mới biết.  
 
Người ta có thể đi đến sự thật bằng những con đường khác nhau, cắt đứt hoàn toàn sợi dây của các nguyên khởi. Điều kỳ diệu của nghệ thuật là ở chỗ nói điều không nói, chỉ ra cái không chỉ ra, quyến rũ vì không kết luận. 
 
Ngôi thứ nhất - tôi là một nhân vật phức tạp. Anh ta hay chị ta có thể vừa là người kể chuyện, vừa là tác giả, vừa là nhân vật tham gia trực tiếp vào câu chuyện, hay là đóng một hoặc hai trong các vai trò.  
 
Trong tiếng Việt chữ \textit{tôi} ngôi thứ nhất còn được thay thế bằng các tiếng xưng hô khác, tuỳ theo mối quan hệ: 
\begin{blockquote}
        
\textit{Mai chị về em gửi gì không} 
\textit{Mai chị về nhớ má em hồng} 
        
(Nguyễn Đình Tiên?) 

\end{blockquote}
 
\textit{Chị} là ngôi thứ nhất, là người nói.  
\begin{blockquote}
        
\textit{Lối ta đi giữa hai sườn núi}        
\textit{Đôi ngọn nên làng gọi núi đôi}        
\textit{Em vẫn đùa anh sao khéo thế} 
\textit{Núi chồng núi vợ đứng song đôi} 
        
(Vũ Cao) 

\end{blockquote}
 
\textit{Anh} đây là ngôi thứ nhất, người kể chuyện. \textit{Ta} là chúng ta, đôi ta. \textit{Em} là nhân vật, nhưng vì em có mặt trong chúng ta, nên em cũng có thể trở thành người kể chuyện. Anh lại chuyển thành nhân vật, tham gia vào hành động. 
 
Trong thơ của các nhà thơ hiện thực xã hội chủ nghĩa trước đây hoặc hiện nay, \textit{ta }thường nghiêng về \textit{chúng ta, }số nhiều.  
\begin{blockquote}
        
\textit{Đường ta rộng thênh thang tám thước}        
        
(Tố Hữu) 

\end{blockquote}
 
Tôi nghe nói rằng Tố Hữu đã từng nghe theo lời khuyên của Trần Đăng Khoa mà đổi thành: 
\begin{blockquote}
 
\textit{Đường ta rộng thênh thang ta bước} 

\end{blockquote}
 
Hình như nhiều người cũng tán thưởng sự sửa đổi này. Theo tôi, câu thơ nguyên thuỷ của tác giả là một câu thơ hay, còn câu sửa lại không được như vậy. Nó là một khẩu hiệu. Bốn chữ \textit{thênh thang tám thước }gồm các âm \textit{t }và \textit{th} đi liền nhau, gây ra cộng hưởng, vừa xinh xắn vừa vang dội, bốn chữ \textit{thênh thang ta bước }không có được yếu tố này, chữ \textit{ta }lập lại chữ \textit{ta }trước nó, đọc lên ngắn, cụt, chữ \textit{bước }kém hơn chữ \textit{thước. }Điều quan trọng hơn là hình ảnh tám thước gây cảm giác chi tiết, rõ rệt, còn \textit{ta bước }trong trường hợp này không phải là một hình ảnh. Nó là một ý niệm. 
 
Thế mới biết càng cụ thể, khiêm tốn, càng mênh mông, càng mơ hồ, khoác lác, lại càng nhỏ bé.  
 
Đây là bài học không chỉ dành cho thơ.  
 
Ở một nhà thơ khác, cũng nói về \textit{ta }và \textit{đường}: 
\begin{blockquote}
        
\textit{Ta về một bóng trên đường lớn} 
\textit{Thơ chẳng ai đề vạt áo phai } 
        
(Tô Thuỳ Yên) 

\end{blockquote}
 
\textit{Ta} nghiêng hẳn về \textit{tôi}. Đó là một cái \textit{ta }cá nhân, thân phận, nhưng lại là một cái tôi siêu ngã, hay siêu cá thể (superego). Thật ra, nó đứng giữa số ít và số nhiều. Tính siêu ngã này bộc lộ rõ hơn ở một nhà thơ khác, một tu sĩ ẩn cư trên thượng nguồn sông Hương, nơi tôi có dịp đến thăm mùa hè vừa qua cùng với các nhà văn Nguyễn Văn Dũng, Nguyễn Quang Hà. Để nghe giọng đọc thơ man mác của ông trong buổi chiều sương khói.  
\begin{blockquote}
        
\textit{Ta đứng giữa sơn khê} 
\textit{Hư không rơi hạt muối} 
        
(Minh Đức - Triều Tâm Ảnh)  

\end{blockquote}
 
Hoặc hoàn toàn là số ít, khiêm cung nhưng cứng rắn:         
\begin{blockquote}
        
\textit{Ta không trẻ không già}        
\textit{Không li ti không vĩ đại}        
        
(Nguyễn Trọng Tạo) 

\end{blockquote}
 
Chữ \textit{tôi} vẫn không thay nó được. Bạn thử xem. 
 
Hoặc gọi hẳn ra bằng số nhiều, như một đám đông: 
\begin{blockquote}
        
\textit{Nước chúng ta} 
\textit{Nước những người chưa bao giờ khuất} 
        
(Nguyễn Đình Thi) 

\end{blockquote}
 
Hẹp hơn, là \textit{chúng tôi}, nhưng mạnh mẽ, cay đắng hơn: 
\begin{blockquote}
 
\textit{Chúng tôi cực kì thính mũi, nhất là đánh hơi các loại mùi thúi.} 
        
(Nguyễn Quốc Chánh) 

\end{blockquote}
 
Người kể ở ngôi thứ ba, một cách gián tiếp thông qua các đối thoại có tính kịch như trong bài thơ của Langston Hughes nói trên, hoặc trực tiếp chỉ ra: 
\begin{blockquote}
        
\textit{Nhiều khi hắn thấy dương vật hắn đang ở Sàigòn,} 
\textit{Đầu hắn ở Hà Nội } 
\textit{Và tay chân thì rơi rụng đâu đó ở Sóc Trăng} 
 
(“Lỗ thủng lịch sử”\footnote{\url{http://www.talawas.org/talaDB/http://www.tienve.org/home/literature/viewLiterature.do;jsessionid=1EB8C20A672489005887E75FCFDF1A3A?action=viewArtwork&artworkId=1512}}, Nguyễn Hữu Hồng Minh) 

\end{blockquote}
 
Sự phân biệt tác giả và người nói, người nói và nhân vật vốn đã cần thiết lại càng trở nên cần thiết khi đọc bài thơ trên đây của Nguyễn Hữu Hồng Minh! Thơ mộng hơn, ngày xưa người ta thường dùng các đại từ \textit{chàng} và\textit{ nàng }trong nhiều trường hợp. Không hiểu sao ngày nay, trong thơ cũng như trong tiểu thuyết hay truyện ngắn, các nhân vật này ngày càng ít xuất hiện, có lẽ vì chiến tranh, hay vì xã hội đã trở nên không còn lãng mạn?  
\begin{blockquote}
 
\textit{Chàng vẫn tin rằng cứ đến khung cảnh ấy mà an tọa, một mình, rồi lặng lẽ suy tưởng thì chẳng mấy chốc sẽ có chim chóc đến làm tổ trên đầu} 
        
(“Một chỗ thật tịch mịch\footnote{\url{http://www.talawas.org/talaDB/http://vnthuquan.net/truyen/truyentext.aspx?tid=2qtqv3m3237nnnqnqn0n31n343tq83a3q3m3237n1n}}”, Võ Phiến) 

\end{blockquote}
 
Thì ra \textit{chàng} vẫn còn sống tử tế đâu đó trong cuộc đời, lặng lẽ tịch mịch, nhưng có vẻ sung sướng lắm, sung sướng hơn chúng ta.  
 
Về mặt sáng tạo, hành động có ý thức mang mặt nạ nhân vật và sử dụng các danh xưng khác nhau, tạo nên trung tâm của các câu chuyện, là sự bộc lộ nỗi xao xuyến, bồi hồi về cái tôi tâm linh, về \textit{ego}. Các vở kịch và các bài thơ có tính kịch biểu thị rõ nhất sự mâu thuẫn giữa một bên là tính chất phi nhân xưng của tác giả và một bên là câu hỏi không ngừng đối với sự tồn tại cá nhân. Cá nhân chỉ tồn tại và được thể hiện trong mối quan hệ với các cá nhân khác, hay với thiên nhiên, nhưng thiên nhiên đến lượt nó lại trở thành một nhân vật khác. Các “nhân vật” có thể không phải là người mà có thể là cây cỏ, động vật, các vật vô tri giác. Như trong bài “Thói quen của cơn đói” của Nguyễn Quang Thiều:        
\begin{blockquote}
        
\textit{Năm 14 tuổi}        
\textit{Tôi cùng chị tôi cắt tiết một con vịt} 
\textit{Trong chiếc bát sành màu đỏ ôm nhau} 
        
\textit{Khi tôi buông con vịt ra}        
\textit{Nó không chết}        
\textit{Đầu ngoẹo sang một bên} 
\textit{Đi liêu xiêu như người say rượu \footnote{
“The Habit of Hunger”, \textit{Language for a new century}, Tina Chang, Nathalie Handal, Ravi Shankar, NXB Norton, 2008. Bản dịch tiếng Việt của tác giả Nguyễn Quang Thiều.  
\begin{blockquote}
 
Thói quen của cơn đói 
 
\textit{Năm 14 tuổi} 
\textit{Tôi cùng chị tôi cắt tiết một con vịt} 
\textit{Trong chiếc bát sành màu đỏ ôm nhau} 
 
\textit{Khi tôi buông con vịt ra} 
\textit{Nó không chết} 
\textit{Đầu ngoẹo sang một bên} 
\textit{Đi liêu xiêu như người say rượu} 
 
\textit{ 	Nơi cổ họng bị cắt} 
\textit{Từng hạt máu tươi} 
\textit{Vương trên cổ lông trắng muốt} 
\textit{Như một chuỗi hạt cườm} 
\textit{Bị đứt} 
 
\textit{Nó vùi đầu vào chậu nước rửa bát} 
\textit{Mò từng hạt cơm thừa} 
\textit{Nhưng những hạt cơm không tìm thấy đường về tới dạ dày} 
\textit{Lọt qua nơi cổ họng bị đứt} 
\textit{Rơi…rơi…} 
 
\textit{Rồi nó theo con đường quen thuộc} 
\textit{Tìm xuống ao sâu} 
\textit{Tìm ra cánh đồng} 
\textit{Tìm ra sông, ra biển} 
\textit{Bắt cá, mò cua} 
\textit{Nó vùi đầu xuống bùn} 
\textit{Máu đỏ loang như dầu trên nước} 
 
\textit{Tôi run rẩy, buốt đau đi tìm con vịt} 
\textit{Với lưỡi dao vô hình} 
\textit{Tôi cắt thịt dọc đường đi} 

\end{blockquote}} } 

\end{blockquote}
 
Có hai hiện thực trong thơ, biểu hiện rõ hơn trong những bài thơ có khuynh hướng tự sự. Một hiện thực thuần tuý, tự nhiên, và một hiện thực của trí tưởng tượng. Ở các nhà thơ khác, đó là hai hiện thực nâng đỡ nhau; ở Nguyễn Quang Thiều, đó là hai thứ đối lập. Trong một thế giới ngày càng đảo lộn, những giá trị cũ ngày càng mất mát, những giá trị mới không thay thế được chúng, môi trường bị huỷ diệt, con người tìm đến với thiên nhiên, mơ ước khôi phục nó bằng nghệ thuật. Trong khi bước đi giữa hai bờ hiện thực, nhà thơ không ngớt hoài nghi chính mình. Thơ tự sự quan tâm đến các câu chuyện, tức là các mối quan hệ giữa người và người, và có vẻ như không quan tâm đến bản thân các nhân vật, nhưng thật ra số phận của họ là nguyên nhân và hậu quả của nhau. Con vịt là hình tượng thuộc về cái thiên nhiên bị làm đau đớn, nhưng là một lương tâm được đánh thức. Nhân vật tôi, sau đó, đi tìm con vịt để khôi phục lại trật tự đã mất, hay anh muốn đi tìm một trật tự mới?    
 
Thơ tự sự mặc dù kể chuyện, lại có khả năng bộc lộ tác giả một cách mạnh mẽ không kém thơ trữ tình. Có thể tìm thấy điều này dễ dàng trong các bài thơ nổi tiếng của Robert Lowell, Sylvia Plath…Các yếu tố hư cấu và các yếu tố tiểu sử xen kẽ nhau, rất khó phân biệt, làm cho một bài thơ tự sự vừa trình diện với người đọc một khuôn mặt tác giả trọn vẹn vừa đòi hỏi sự phát triển của tác giả như một nhân vật trong suốt tiến trình của bài thơ.  
 
Mối quan hệ giữa người nói, người kể, nhân vật, tác giả, mà tôi gọi chung là các nhân vật, bao giờ cũng được xem xét quanh trục của câu chuyện. Đó không phải là một trục tuyến tính mà là một cấu trúc dựng chuyện. Cấu trúc của chuyện kể trong văn xuôi và trong thơ đều giống nhau ở điểm: đó là hành động. Cũng như trong ngôn ngữ, không có chủ từ thì không có câu văn phạm, không có hành động thì không có câu chuyện, càng không có cốt truyện. Thuật ngữ “tự sự” là sự phô diễn hành động, hay sự chuyển dịch, theo thời gian. Bản thân thứ tự của các chuyển dịch theo thời gian này tự nó không đủ để tạo ra hành động.        
\begin{blockquote}
        
\textit{Mai chị về em gởi gì không} 
\textit{Mai chị về nhớ má em hồng} 

\end{blockquote}
 
Mặc dù có nhiều động tác như thế (gởi, nhớ), đoạn thơ trên đây chưa đủ để tạo ra hành động, mặc dù có một chuỗi liên tiếp các động thái. Các động thái này cần xảy ra trong một trật tự có tính nhân quả. Vì khuôn khổ giới hạn của bài thơ, (không kể các tiểu thuyết bằng thơ như \textit{Truyện Kiều} hay các trường ca sau này), các nhà thơ thường khôn ngoan chọn việc trình bày câu chuyện bằng các đối thoại của hai nhân vật, và bằng cách đó, xây dựng cách hành động một cách gián tiếp.  
 
Câu chuyện của các nhân vật cuối cùng sẽ dẫn chúng ta đến đâu? Một bài thơ tự sự được kết thúc như thế nào? Có hai phương pháp. Có thể kết luận bằng cách khép lại và nêu ra cách giải quyết vấn đề.  
\begin{blockquote}
        
\textit{Ngày mai cô sẽ từ trong tới ngoài}        
\textit{Thơm như hương nhuỵ hoa nhài}        
\textit{Sạch như nước suối ban mai giữa rừng}        
        
(“Cô gái sông Hương”, Tố Hữu) 

\end{blockquote}
 
Các nhà thơ xã hội chủ nghĩa thường có thói quen này, không những vì lý do chính trị mà còn vì niềm tin thẩm mỹ văn học. Hãy nghe Tố Hữu tâm sự: 
 
\textit{\textbf{Tố Hữu}: Viết Tiếng hát sông Hương tôi nhằm bày tỏ thái độ đối với một cảnh đời đau khổ và cũng là một tệ nạn xã hội… Phải chăng đó là một định mệnh? Tôi phản ứng với tác giả }Đời mưa gió\textit{ khi ông miêu tả thân phận người con gái giang hồ.} 
 
\textit{\textbf{Hà Minh Đức}: Nhưng Nhất Linh cũng không thi vị hoá cuộc đời của nhân vật, Tuyết đã phải trả giá về lối sống phóng túng, chạy đuổi theo lạc thú.} 
 
\textit{\textbf{Tố Hữu}: Đúng thế, nhưng tư tưởng tác giá }Đời mưa gió\textit{ là bế tắc, không tìm được lối thoát cho nhân vật khổ đau này. \footnote{
Hà Minh Đức, \textit{Nhà văn nói về tác phẩm}, trang 22, NXB Giáo Dục, 2004} } 
 
Tôi tin là ông thành thật nghĩ như thế. Thật ra nhiệm vụ của tác phẩm văn học, dù là thơ hay tiểu thuyết, không phải là, hoặc ít khi là, đưa ra phương pháp \textit{giải quyết} các bài toán xã hội. Công việc của chúng là \textit{trình bày} các số phận cá nhân, các vấn đề của ý thức xã hội và lương tâm con người.  
 
Bài thơ của William Blake trên đây không đưa đến một phương pháp giải quyết nào cả. Bài thơ này không bộc lộ sự căm phẫn xã hội, thái độ phản kháng của tác giả, cái tôi không bày tỏ quan điểm của mình, và cũng không chỉ ra phương pháp để giải quyềt vấn đề, đó là tệ nạn lao động ở trẻ em. 
 
Chúng ta biết rằng vào thời đó ở châu Âu những đứa trẻ thường được tuyển dụng vào các công ty cạo ống khói, vì chúng nhỏ người có thể chui qua được ống khói dễ dàng. Muội khói than gây ra các dị tật thần kinh, ung thư phổi, ung thư mũi họng, ung thư tinh hoàn, và làm trẻ con không lớn được.  
 
Đoạn kế tiếp tác giả kể về một nhân vật là chú bé tên Tom cũng làm nghề cạo ống khói, khóc vì bị cạo trọc đầu, có lẽ để cho chú dễ làm việc, và kết thúc bằng việc chú Tom và những đứa trẻ khác được xoa dịu bởi giấc mơ, bằng lòng với cuộc sống hiện tại và thức dậy khi trời chưa kịp sáng đi làm việc, lòng ấm áp dưới bầu trời lạnh lẽo. Chính thái độ bằng lòng và chấp nhận cuộc sống đáng thương, chứ không phải là ý thức phản kháng, của ngôi thứ nhất và của Tom, gây ra xúc động ở người đọc: đó là hiệu quả nghệ thuật lớn nhất của bài thơ. 
 
Nếu William Brake chọn một phương pháp kết thúc khác, ví dụ như nhân vật đứa bé tự ý thức thân phận mình và đi đến những hành động để làm thay đổi số phận của nó, thì sức thuyết phục thẩm mỹ sẽ như thế nào? Cách kết thúc như vốn có của bài thơ đặt người đọc vào một trạng thái không những xúc động mà còn lay chuyển sâu xa ở họ ý thức xã hội.  
 
\textit{Hình thức} là món quà lớn nhất là người nghệ sĩ có thể mang lại cho cuộc đời. Đáng tiếc có hai thái cực đang xảy ra với thơ Việt Nam. Một là, chỉ chú ý đến nội dung và tư tưởng, trong sáng tác và phê bình. Hai là, sự kém hiểu biết về hình thức và các thể thơ. Tôi chưa thấy một nhà thơ lớn nào lại không từng viết một bài thơ có vần, như lục bát hay bảy chữ. 
 
Phần kết thúc và câu cuối cùng có vai trò đặc biệt quan trọng. Như trong bài thơ sau đây, viết những năm sáu mươi, dữ dội mà sâu lắng, mô tả một khuôn mặt khác của sự thật.  
\begin{blockquote}
        
\textit{Người ta gọi tôi là địa chủ}        
\textit{Đây một lũ người tự xưng là cùng đinh}        
\textit{Đem bắt trói tôi vào cột đình}        
\textit{Đã hai ngày qua tôi vẫn làm thinh}        
\textit{Nhưng đến trưa nay tôi bỗng hoảng kinh}        
\textit{Số là tôi khát nước lắm rồi}        
\textit{Ôi chao tôi ước ao tôi ao ước}        
\textit{Và không thể cầm lòng tự cao}        
\textit{Tôi kêu hãy cho tôi nước nước nước}        
\textit{Tôi bỗng nghe một tiếng trả lời: được}        
\textit{Rồi một kẻ đi đến rất chậm bước}        
\textit{Lúc đứng gần sau lưng tôi nó nói thỏ thẻ}        
\textit{Hãy hả họng cho tao đổ tội nghiệp đồ chết khát}        
\textit{Tôi cảm động nhắm mắt run run hả họng khô rát} 
\textit{Nó hắt ngay vào một nắm cát} 
        
(“Nước”, toàn bài, Quách Thoại)  

\end{blockquote}
 
Sức mạnh của bài thơ dồn vào câu cuối, thản nhiên, tàn nhẫn. Đúng như hiện thực.  
 
Điểm nhìn quan trọng đối với người đọc lẫn người viết. Nó giúp người đọc bước vào thế giới của câu chuyện, nhập vai, cảm nhận, suy tư và xúc động cùng với nhân vật, như một nhân vật. Những hài kịch và bi kịch của cuộc đời, những xung đột của tình huống, sự cam chịu hay cưỡng chống lại số phận của cá nhân được nhìn qua một góc nhìn hay một lăng kính. 
 
Góc nhìn ấy, lăng kính ấy là do người viết chọn lựa “giúp” cho người đọc. Người đọc “vô tình” nhận lấy. Vì thế, quyền năng của tác giả rất lớn: anh ta ảnh hưởng đến người khác. Câu chuyện “Màu tím hoa sim” có thể được kể lại bởi một nhân vật không phải là người lính Vệ quốc quân, người mẹ của cô gái chẳng hạn. Hay chính là cô ấy. Thì sao? thì cảm nhận của người đọc sẽ khác đi. Bài thơ của Quách Thoại sẽ còn khác hơn nhiều, cả về cốt truyện lẫn cảm xúc, cả về ngôn ngữ lẫn giọng điệu, nếu nhân vật xưng tôi không phải là nạn nhân, mà là thủ phạm. Hay tòng phạm. 
 
Như thế, không có một nền văn học nào là tuyệt đối khách quan. 
 
Thơ tự sự có vai trò gì? Tôi được nghe một câu chuyện: ngày trước, trong một cuộc họp với thổ dân (da đỏ) phía Bắc tỉnh BC, Canada, đại diện chính quyền tuyên bố rằng các vùng đất ở đó là thuộc quyền sở hữu của nhà nước. Mọi người ngơ ngác một hồi. Cuối cùng, một vị trưởng lão lên tiếng hỏi: “Nếu đất đai là của quý ngài, thì quý ngài có câu chuyện (stories) nào kể lại cho chúng tôi nghe không?” Các đại diện chính quyền không có \textit{stories} nào để kể lại cả. Người thổ dân này liền đọc một câu chuyện bằng thơ liên quan đến những sinh hoạt của người da đỏ hàng trăm năm trước, bằng tiếng địa phương, để chứng minh rằng vùng đất đó là của họ.  
 
Dù bằng chữ viết hay truyền miệng, những câu chuyện bao giờ cũng sống lâu hơn cả trong lòng chúng ta, lưu giữ ở đó ánh sáng của mặt trời đã tắt. Đó là nơi mà các nhân vật trở về, đi lại, nói năng, với các suy nghĩ, cảm xúc mang dấu ấn ngôn ngữ của thời đại họ. Khi nghe một câu chuyện cảm động, có phải là tất cả đều im lặng, và rồi chúng ta cùng cười phá lên, hay cùng chảy nước mắt? Nếu bạn biết cách kể chuyện, nhất là lại bằng thơ, thì không một thứ lý luận nào có thể thay thế được, vì người nghe xúc động cùng với người kể, kẻ giận dữ trở nên bình tĩnh, kẻ tự mãn trở nên khiêm tốn, những lầm lỗi được khoan thứ, và lịch sử được viết lại trong lòng chúng ta một lần nữa.  
 
Nhiều người nghĩ rằng lịch sử không thay đổi. Nhưng lịch sử không phải là quá khứ. Lịch sử là \textit{câu chuyện} về quá khứ. Vì vậy, tôi tin rằng nó có thể được viết lại, đọc lại, nghe lại. Thật ra nó đang được viết lại mỗi ngày, bởi từng cá nhân chúng ta, để cho con người đối với nhau ngày càng khôn ngoan hơn và do đó, chất phác hơn. 
 
 
\textbf{Tài liệu tham khảo} 
 
Thi Vũ, \textit{Bốn mươi năm thơ Việt Nam}, NXB Quê Mẹ, 1993 
 
Võ Phiến, \textit{Tuyển tập}, NXB Văn Mới, 2001  
 
\textit{Tuyển tập Sông Hương ( 1983- 2003)}, Ban tuyển chọn: Nguyễn Khắc Thạch, Hồ Thế Hà, Hồng Nhu, Ngô Minh, Trương Thị Cúc, NXB Văn Hoá Thông Tin, 2003 
 
Tạp chí\textit{ Thơ}, Hoa Kỳ, 2006- hiện nay 
 
Tạp chí \textit{Thơ}, Hội nhà văn Việt Nam, Hà Nội, 2008- hiện nay 
 
\textit{Tuyển tập Tiền Vệ I}, NXB Tiền Vệ, 2007  
 
Gary Geddes,\textit{ 20th Century Poetry &amp; Poetics}, Oxford University Press, 1996  
 
G.G. Sedgewick, \textit{Of Irony, Especially in Drama}, NXB Ronsdale Press, 2003 
 
Dana Gioia, \textit{Twentieth Century – American Poetry}, McGraw Hill, 2004 
 
Camille Paglia, \textit{Break, Blow, Burn,} NXB Vintage, 2005  
 
David Lehman, \textit{The Oxford Book of Poetry}, 2006 
 
 
© 2008 talawas 




\end{multicols}
\end{document}