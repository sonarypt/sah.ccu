\documentclass[../main.tex]{subfiles}

\begin{document}

\chapter{Mấy ý nghĩ về thơ }

\begin{metadata}

\begin{flushright}11.10.2008\end{flushright}

Quang Dũng

Nguồn: Báo Văn nghệ, Hà Nội, số 137 (6.9.1956), 138 (13.9.1956), 139 (20.9.1956). LạI Nguyên Ân sưu tầm và biên soạn. 

\end{metadata}

\begin{multicols}{2}

Theo ý tôi, đã có những triệu chứng đáng mừng cho thơ: Người đọc đã muốn đòi hỏi một cái gì khác trong tiếng nói của thơ chứ không muốn đọc dễ dãi một ít sự việc ghi chép thành vần, một ít tình cảm đã gặp nhiều lần sắp xếp thành vần điệu và, bao giờ cũng thế, có cái kết hướng đi lên một cách đã quá quen thuộc. Người làm thơ cũng tự ngấy mình, nếu cứ đi mãi vào cái nếp dễ dãi ấy; làm xong một bài thơ mà thấy như chưa làm gì cả vì bản thân cũng thấy tẻ nhạt, không hào hứng không rung động. Hay có hào hứng rung động thì cũng chỉ là cái ảo ảnh chốc lát do vần điệu gây ra chứ cái vui lớn như của một người phát minh ra một cái gì cống hiến được cho cuộc sống thì không có. Tìm tòi công phu và sáng tạo, đó là cái băn khoăn chính của người làm thơ bấy nay; bây giờ đã có nhiều dấu hiệu của cuộc khủng hoảng trưởng thành trong thơ. Nhiều quan niệm khác nhau về kỹ thuật và nội dung đã dần dần lên tiếng, và cũng như đang có một sức chuyển mình chưa biết đến thế nào nhưng chắc chắn là sẽ làm phong phú cho thơ của chúng ta. 
 
Từ cách mạng, qua kháng chiến mười năm, cũng như ở mọi địa hạt khác, ở thơ người ta thấy không có được nhiều tác phẩm lớn. Đòi hỏi như thế thì cũng có thể là vội vàng, nhưng không phải là không chính đáng. Dân tộc làm một cuộc cách mạng. Cuộc sống bao nhiêu cái vĩ đại đang diễn ra, cái mừng vui cũng như cái đau khổ, mười năm qua, ai mà chẳng thấy là những cái mà con người thấy có cái vinh dự được trải qua. Cái hiện tại thật là to lớn, ai mà không thấy tự hào có mặt. Thế mà nguồn thơ chưa bắt được mạch sống, chưa có chất máu đỏ tươi, hơi thở khổng lồ của cuộc đời hiện tại, thực cũng là vấn đề đáng suy nghĩ. 
 
Tôi hình dung lại lúc bước vào kháng chiến, lúc người thi sĩ bước vào cách mạng. “Cách mạng”, chỉ hai tiếng đó với bao nhiêu âm hưởng và hình ảnh nó gợi lên, đủ lôi cuốn người làm thơ, − vốn tìm cái đẹp chân chính − không mặc cả, lao cả cuộc đời mình theo nó. Nhưng thật là bỡ ngỡ. Ta như những người chán một bờ nhỏ hẹp thèm một đất đai khác, mê biển rộng, ra đi mà thiếu một địa bàn vững chắc. Nên có cái hào hứng, cái nhiệt thành của người đi mà chưa có cái phong thái vững vàng và trong sáng của người đến.  
 
Những người làm thơ lớp trước, khi bước vào Cách mạng đều muốn đem hết cái nhiệt tình của mình để làm trách nhiệm hướng dẫn − và tưởng như hướng dẫn được − cái sức thơ măng trẻ của dân tộc. Nhưng có một điều là chính những người làm thơ ấy lúc làm cái việc tìm đường cho thơ thì con người thơ trong họ biến mất. Chỉ còn có con người “đường lối”, “lập trường”, con người “đi lên” hay “đi xuống” nó choán hết. Và cứ như thế, những người thi sĩ ấy chạy theo miết những biện pháp cụ thể của chính sách, những công việc sự vụ ngập đầu. Những người ấy đã một thời được quần chúng ưa thích, Cách mạng cũng tin tưởng ở những người ấy và sở cậy họ ở trong lĩnh vực thơ, kỳ vọng họ sẽ đem cái hồn thơ làm phong phú cho Cách mạng, mở thêm những chân trời bao la về cảm xúc, cái mà cách mạng bản thân nó cũng mong đợi ở người làm thơ. Để tiến lên lên cùng cách mạng, hồn con người cần phải phơi phới, bay lộng như lá cờ anh dũng của nhân dân đang tự giải phóng mình; hồn con người trong cuộc đấu tranh quyết liệt này, va chạm cũng vĩ đại, đau xót cũng trên mức bình thường, vui sướng hào hứng cũng đến cùng độ; thế mà trước hết, những người thi sĩ đứng tiêu biểu cho cái truyền thống thơ của dân tộc lúc bước vào cái con đường vinh quang của lịch sử ấy của dân tộc, đã tự trói chân mình, cắt cánh mình…, bỏ ngay cả chính cái phơi phới phóng khoáng của tâm hồn nó là một điều nằm trong mục đích của cách mạng: giải phóng cho con người, được hoàn toàn là con người, tự do lớn lên, tự do vươn tới những chân trời tình cảm mà từ lâu đời, phong kiến và tư bản đã đứng lù lù chặn lối… 
 
Dưới cái ảnh hưởng ấy, những người làm thơ, những người có cái cốt cách thi sĩ, có cá tính phong phú ngay cả ở chế độ cũ, tưởng bước vào ánh sáng của cách mạng thì hoa sẽ nở muôn cành, chim sẽ bay vạn dặm, cái khí thế người thi nhân bao giờ cũng dũng cảm bảo về cái Đẹp và cái Thật, tưởng sẽ được khuyến khích, thì trái lại, đã chịu bước đi từng bước một, mang nặng những cái mâu thuẫn trong tâm hồn, e dè với cách mạng, cách bức với cách mạng, quên hẳn chính mình là hơi thở của cuộc đời vĩ đại mai này mà cách mạng đang tạo nên. Làm thế nào mà nói lên được cái đẹp tốt, cái viễn ảnh làm phấn khởi lòng người đi cùng trong một chuyến thuyền tới nơi đất đai mà chính mình chưa nắm được nó ra thế nào? 
 
Chưa ý thức được cách mạng một cách rõ rệt nên tiếng nói của thơ đối với cuộc đời mới đang xây dựng này thiếu hào sảng, thiếu cái khí thế vững mạnh của nó. Người làm thơ bấy nay chưa làm chủ vấn đề, nói trắng là chưa thấy được rõ cái nhiệm vụ cách mạng của mình, nên tiếng nói chỉ là phụ hoạ mà không có được cái giá trị sáng tạo đóng góp cho cách mạng. 
 
Tôi muốn trình bày ra đây những dẫn chứng cụ thể về cái ý “tiếng nói của thơ mới chỉ là phụ hoạ chứ không có được cái giá trị sáng tạo đóng góp cho cách mạng”. 
 
Từ lúc dân tộc làm cuộc khởi nghĩa tháng Tám cho đến lúc bước vào cuộc kháng chiến, người văn nghệ nào cũng chờ đợi những đường lối sáng tác mà Cách mạng sẽ mang tới cho mình, những đường lối ấy họ ước mong là sẽ tới để giải phóng cho ngòi bút họ. Bởi thế cho nên Cách mạng đòi hỏi gì, họ đều đứng dậy “có mặt” ngay. Đường lối của cách mạng, họ tuyệt đối tuân theo và triệt để ủng hộ. Những con người văn nghệ − nói riêng ở đây là những người làm thơ − vừa hôm qua đây còn mang bao nhiêu bản chất ngang tàng, khinh mạn với cuộc sống (cũ), kiêu bạc với cuộc đời (cũ), mỗi người như một hòn núi lửa, mỗi người có cái tự hào riêng với cái thế giới riêng của mình, mỗi người sống theo một nhân sinh quan đặc biệt của mình, ấy thế mà nghe tiếng hô của Cách mạng, họ tập trung được ngay, cùng một ý chí chiến đấu và cùng một lòng cố vượt, hy sinh cái đời riêng với những thói quen của mình − cái hy sinh ấy đã thật là đau xót đối với họ − để đi theo Cách mạng. Là vì họ đều hăm hở muốn được Cách mạng mở cho họ những chân trời mới về cảm xúc và có cái sung sướng của những người được đi vào một xứ sở mới lạ và phong phú về đề tài. 
 
Nhưng những quan niệm giới hạn về văn nghệ của những người có trách nhiệm hướng dẫn họ đã làm tắt cái lửa đó. Người ta quan niệm người làm thơ là những người mơ mộng không sát được thực tế cách mạng, là những người hoàn toàn ngây thơ mà cách mạng phải dạy dỗ uốn nắn mới đi vào được những vấn đề mới. Những người chỉ đạo đường lối − là những cán bộ chính trị chứ không phải là những nhà thơ, hoặc là nhà thơ nhưng đã không còn mang tính chất thi sĩ nữa − lại cứ lo lắng và tự thấy mình như phải có cái trách nhiệm hướng dẫn thơ đi cho đúng với công tác trước mắt của cách mạng, áp dụng nó vào những công việc có tính chất thực dụng ngay, dựa theo một ý thức không rộng rãi về “phục vụ” mà đưa ra những nguyên tắc, những lệ luật máy móc để đánh giá thơ … do đó đã gò thơ đi vào công tác cách mạng một cách bị động. 
 
Xuất phát từ những điểm ấy, người lãnh đạo đã xem người làm thơ rất nhỏ bé, rất phụ thuộc, coi như họ chỉ là những người có khả năng diễn ca mọi vấn đề của cách mạng, coi như họ là một bộ môn cần thiết cũng có cho đủ ở trong một nước và cũng có một tác dụng nào trong công tác tuyên truyền cho cách mạng mà thôi. Trên lý luận, Đảng coi trọng vấn đề văn học, nhưng trên thực tế công tác, những người lãnh đạo đã không tin tưởng ở những người làm thơ và không nhìn họ đúng với cái tầm quan trọng của vấn đề. 
 
Thành ra chính đi theo cách mạng mà cái tri thức về cách mạng, người làm thơ không được trau dồi cho đến nơi đến chốn; người làm thơ lúc nào cũng bỡ ngỡ với những vấn đề mới do công việc cách mạng biến diễn ra; lúc nào cũng làm một người đuổi theo vấn đề. Không thấy mình được chủ động gì hết trong lĩnh vực của tư tưởng nghệ thuật. Theo ý tôi, cách mạng đâu có muốn như thế! Cách mạng đâu có cứng nhắc muốn tạo ra những người chỉ biết phục tòng mình mà  không sáng tạo gì thêm được cho cách mạng. 
 
Người lãnh đạo cách mạng tưởng không gì chán hơn là lại được đọc những tác phẩm văn thơ đúng như trong đường lối mình vạch ra, tư tưởng của tác phẩm chỉ ở trong cái phạm vi tư tưởng mà mình đã gợi ra. Chính những người lãnh đạo cách mạng cũng đòi hỏi ở Văn nghệ − riêng là thơ − nói lên cho mình những tình cảm lớn lao, cổ vũ được cả cho mình, bồi dưỡng cho mình bằng những cái xúc cảm lạ lùng vĩ đại mà người thi sĩ tiên giác được trên cơ sở của cách mạng. Nhưng làm thế nào được? Đã có người thi sĩ nào gần đây có thể nói lớn được rằng: “Cách mạng đã hoàn toàn tin cậy ở tôi. Tôi tự hào được đứng ở hàng ngũ tiền phong của cách mạng, vì cách mạng đã giúp cho tôi ý thức nổi cái trách nhiệm và đã cho tôi đầy đủ cái hiểu biết về cách mạng rồi”.    
 
Do cái tình trạng ấy mà người làm thơ cũng cứ vô tình thấy mình nhỏ lại, rồi bằng lòng theo những lối đã vẽ ra, trăm người cùng đi một kiểu, cùng tới một chỗ, cùng gặp những cây cỏ và sông núi giống nhau, kể lại cho nhau nghe thì ai cũng biết cả rồi… chỉ một con đường ấy, cây cỏ ấy, núi và sông ấy… Cả một nếp suy nghĩ, vui buồn của dân tộc bỗng nhiên đều bị ảnh hưởng của sự hướng dẫn một chiều đó. 
 
Công chúng đọc thơ cũng bị gò vào cái nếp nhận xét rất khắc nghiệt của công thức, nên chi, cái hiện tượng trước khi đọc một bài thơ, người ta đã mai phục sẵn những nguyên tắc để mổ xẻ nó, phê phán nó. Óc nhận xét máy móc đã choán mất cái óc thưởng thức hồn nhiên và bình tĩnh của người đọc. Người làm thơ giống nhau, người phê bình cũng phê bình bằng những lý luận, những nguyên tắc giống nhau. Rồi chính người làm thơ không dám đi sai cái công thức mà mình đã đưa tràn ngập trong thơ, chỉ vì lại sợ công chúng − đã được hướng dẫn theo nếp ấy − lại vận dụng chính cái tinh thần đó mà phê phán thơ mình.  
 
Cái vòng luẩn quẩn đó đã kéo dài.  
 
Người làm thơ tự thấy như mình phải giả tạo đi phần nào. Đã thấy hai con người trong một mình anh. Cũng như Tartarin Quichotte và Tartarin Sancho trong một truyện của Daudet. Lúc Tartarin Quichotte hét đòi phải đưa cho mình giáo, mác, đao, chuỳ, để xông pha vào chốn nguy hiểm, thì Tartarin Sancho hét cô hàng cà-phê phải đưa cho mình súc-cù-là, bánh, thịt và áo len. Mà chỉ là một Tartarin. Chỉ là một con người chứa hai con người xung đột nhau. Người thi sĩ có chứa đựng những cảm xúc rào rạt chân thành với cuộc sống − mà cuộc sống thì mênh mông rộng lớn, cuộc sống có chứa đựng chính sách chứ đâu phải là chính sách chứa đựng nổi cuộc sống − mà nói ra thì e ngại những sự sai lạc này nọ, nên không thể nào thấy thoả mãn được. 
 
Có cái gì đè nặng lên đường lối văn học nghệ thuật chúng ta mà văn học nghệ thuật lại từ chối những sự phản ảnh chân thành của mọi khía cạnh tâm tình! Cái gì đè nặng lên tư tưởng mọi người để hễ nói ra là phải nói những điều đã đúng từ trước khi nói? Sao lại phải gò nhau vào một cái khuôn rồi bảo cứ tự do đi lại trong cái khuôn đó? Hễ không chịu được thì lại chụp mũ là tư tưởng còn lạc hậu! chưa ý thức được cái tự do mới! cái tự do trong những khuôn khổ!...  
 
Thơ của chúng ta, chính trong mười năm qua, bản thân nó, dầu cho công thức đi nữa, thì cũng đã lớn, cũng đã là một bước dài so với cái đời chưa cách mạng. Nhưng giá không vướng vào những khuyết điểm trên kia, ai biết nó đã đi xa đến thế nào? Ai đoán được nó đã cao lớn như thế nào? Và sự nghiệp của thơ đối với dân tộc của chúng ta đang cách mạng đã có những đóng góp to tát và phong phú đến thế nào! Cái sống của dân tộc làm cách mạng thì to quá, mà cái tiếng nói mới của dân tộc, cái nghĩ mới của dân tộc, cái tứ hào hùng để khích lệ thêm cho dân tộc thì lại như rụt rè, tụt lại sau, bị bỏ rơi một quãng đường không phải là ngắn.  
 
Không riêng gì trong địa hạt thơ, các ngành khác của văn học nghệ thuật cũng đều cảm thấy như vậy (như những bài đã phát biểu trên báo \textit{Văn nghệ} hoặc những ý phát biểu trong cuộc học tập lý luận vừa qua) cho nên có cái hiện tượng băn khoăn chung, dấu hiệu của những sự thay đổi sắp tới, − và tất cả những hy vọng đó muốn chờ mong ở Đại hội Văn nghệ Toàn quốc lần thứ hai này giải quyết. Sự trưởng thành ngày nay, theo ý tôi, chính là do mười năm kháng chiến và cách mạng vừa qua. Chính ở trên miếng đất đã nuôi dưỡng chúng ta ấy mà chúng ta đã lớn lên và nhận thấy mình cần thay đổi. Cũng như mỗi người nhớ lại những nét vẽ của mình ngày nhỏ: người thì là một cái gậy, đầu tròn như một cái vung, tay là một cái que có năm ngón toè ra như năm cái tăm. Ngày nhỏ ta cho thế là đúng hình dáng của một người rồi. Lớn lên, ta không cần phải biết luật viễn cận hay hiểu về khoa học tạo hình mà cũng cứ nhận ngay là sai và buồn cười về những nét đơn sơ nguệch ngoạc ấy của thời thơ ấu. Qua một chặng đường dài, ta đã nhiều kinh nghiệm. Ánh sáng của Đại hội thứ hai mươi Đảng Cộng sản Liên Xô, ánh sáng của những lý luận văn học của các nước bạn mới tràn vào, ùa vào như ánh sáng ùa qua cửa sổ mới mở, đến với chúng ta; ánh sáng của những buổi kiểm điểm văn học nghệ thuật theo một đường lối dân chủ thực sự do Đảng chủ trương; ánh sáng của một nền nhân văn chân chính lúc nào cũng dắt người ta trở lại đi đúng đường; tất cả đã giúp chúng ta đứng dậy, nhận định nổi cái yếu ớt, cái gò bó, cái nông cạn nó làm cản trở đà lên của chúng ta từ trước đến nay. Chính ta đã lớn lên, lớn lên vì cả những băn khoăn mà ta đã ấp ủ hàng bao nhiêu ngày nay. Bây giờ đã đến lúc cởi mở. Bây giờ đã đến lúc người làm thơ vun đắp, đề cao cái truyền thống thi sĩ của mình. Và Cách mạng thiết tha phát huy cho cái truyền thống đó. 
 
Theo ý tôi, cái truyền thống vinh quang của người thi sĩ là ở chỗ lúc nào họ cũng thấy có trách nhiệm gắn bó với cuộc sống. Họ là những người biết yêu ghét vui mừng hay căm giận ở một mức độ cao hơn mọi người. Họ là những người lúc nào cũng tha thiết với chính nghĩa và tự thấy như mình phải có một sứ mệnh: làm cái tiếng nói trong sáng nhất, can đảm nhất và tha thiết nhất của con người. 
 
Cũng chính vì có mang được cái truyền thống đó trong mình nên người làm thơ mới được xã hội cần đến.  
 
Nếu anh cũng rung động bình thường như mọi người, đau xót bình thường như mọi người, dũng cảm bình thường như mọi người dũng cảm được, thì nói lên hay không nói lên, lời anh cũng vậy mà thôi, vì nó chỉ ở mức bình thường mà mọi người đều có cả. Làm sao mà nói ra những điều xúc động lòng người? Chính cái truyền thống ấy nó đã giúp cho người thi sĩ vượt được lên trên mọi vấn đề của thời đại, mọi trói buộc tầm thường của cuộc đời để nhìn thấy những khía cạnh mà người ta không nhìn thấy rõ được. Rồi họ thấy tự họ phải có cái trách nhiệm nói lên với cuộc sống những ưu tư của họ, những điều họ đau xót, những điều họ ước vọng hay những niềm tin tưởng của họ đối với cuộc sống.  
 
Không có người thi sĩ lớn nào đã được cuộc đời yêu mến kính trọng mà lại không từ chỗ cảm thông sâu sắc với thời đại của mình vượt ra khỏi cái khuôn sáo của thời đại mình; bởi thế tiếng nói của họ nhiều khi lạ lùng làm người ta sửng sốt − nhưng tin tưởng ở trong con người họ vững chắc, họ cứ đi thẳng con đường sứ mệnh của họ và xã hội bao giờ cũng công bằng, chóng chầy sẽ biết đến họ và coi họ là những người bạn tâm tình thứ nhất của mình, vì chính họ đã nói được lên cái ước vọng chính đáng nhất của xã hội. 
 
Người thi sĩ thấy vinh dự được đi với Cách mạng, Cách mạng cũng thấy cái vinh dự được nhiều thi sĩ về “tụ nghĩa” ở dưới bóng cờ mình. Họ vốn là những người bất mãn lớn với cái cũ rồi. Họ đi với Cách mạng là để được làm Cách mạng chứ đâu phải để bất mãn với Cách mạng. Đôi điểm không thông đã làm cho họ phật ý, nhưng nghĩ đến Cách mạng, tin ở cái chân lý cao cả của Cách mạng, họ vẫn phấn đấu, kiên trì và trung thành với Cách mạng, dầu họ là trong Đảng hay ngoài Đảng. Đã đứng ở hàng ngũ phấn đấu cho Cộng sản chủ nghĩa là cái lý tưởng tuyệt đích của nhân loại bây giờ, họ cũng có cái kiêu hãnh mang một tâm hồn Cộng sản − và họ nhất định tin như thế − chỉ có khác là kết nạp hay chưa kết nạp mà thôi. 
 
\begin{center}
*\end{center}
  
Người ta quan niệm từ xưa tới nay rằng người làm cái “nghiệp” văn chương thường là lận đận. Cái đó không phải là không có lý. Người Văn nghệ sĩ, người làm thơ nói riêng ở đây, là người có nhiều khí khái, có nhiều tự trọng và có một cái ý thức rõ rệt về trách nhiệm của mình, tất nhiên là không thể dung hoà với những cái trạng thái hỗn loạn của xã hội cũ, nên họ thường là chống đối; họ là những hiệp sĩ mang cây bút vì con người mà xông pha mà hy sinh. Thân thế họ vì vậy mới “ba chìm bảy nổi”. Những chính thể đương thời không dung được họ và họ cũng không dung tha gì cho những chính thể đó. Họ đều là những người bất mãn xung thiên cả. Có người đã nói: “Bao giờ cũng vậy, văn chương kiệt tác vẫn là những thứ văn chương chống đối lại”. Cái đó theo tôi rất đúng đối với thời đại cũ. 
 
Nhưng ở chế độ của chúng ta, điều đó không còn là một quy luật nữa. Cách mạng là nơi “đất thánh” mà người thi sĩ hằng mơ ước đi tới, − dầu có gặp những bệnh ấu trĩ nó làm chậm bước đi của họ phần nào − thì họ vẫn cứ thẳng bước đi tới. Chống đối lại Cách mạng ư? Không bao giờ họ lại chống đối lại chính cái lý tưởng tha thiết nhất của đời họ mà họ đã quyết hiến dâng cả tim óc và máu nóng! Họ chỉ chống đối những cái sai lầm có phương hại đến Cách mạng chính vì cái ý thức thiết tha bảo vệ Cách mạng của họ mà thôi. 
 
Theo ý tôi, những người làm thơ là những người như vậy, thiết tha gắn bó với cái hiện tại của Cách mạng, sôi nổi muốn góp phần xương máu của mình vào cho sự nghiệp Cách mạng chóng thành công, thế mà họ chưa đạt được những điều mong ước. Vì sao? Chính bởi họ chưa được hoàn toàn tự do phát triển tài năng của họ. Chính vì họ chỉ mới làm vai trò phụ hoạ chứ không phải vai trò tiền phong của Cách mạng. 
 
Bản thân người thi sĩ cũng vì thế mà không thoải mái vững vàng trong những bước đi mới của mình. Nói cách khác tức là kém một tinh thần “chủ nhân ông” trong vấn đề.  
 
Giải phóng cho thơ không gì bằng trang bị cho người làm thơ đầy đủ cái hiểu biết về Cách mạng, đặt tin tưởng vào họ và để họ tự do tung hoành trên cái diện tích bao la của cuộc sống Cách mạng, chắc chắn là tiếng nói của thơ lại sẽ ngân vang sang sảng và phong phú vô cùng. 
 
Giải phóng cho thơ, không gì bằng trả lại cho người làm thơ cái giá trị xứng đáng với cái truyền thống cao quý mà họ đã mang ở trong máu họ.  
\end{multicols}
\end{document}