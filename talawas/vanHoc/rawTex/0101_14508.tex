\documentclass[../main.tex]{subfiles}

\begin{document}

\chapter{Đọc một số thơ gần đây}

\begin{metadata}

\begin{flushright}18.10.2008\end{flushright}

Tô Hoài

Nguồn: Báo Văn nghệ, Hà Nội, số 154 (4.1.1957), tr. 2, 10. Lại Nguyên Ân sưu tầm và biên soạn.

\end{metadata}

\begin{multicols}{2}

Một điều rõ ràng là lâu nay thơ của ta nhuốm một bản sắc khá đặc biệt. 
 
Phải trở ngược lại một chút, từ sau Cách mạng tháng Tám và kháng chiến. Những bài thơ hay nhất của thời kỳ đó đã xuất hiện. Những bài thơ ca ngợi sự sống và cuộc chiến đấu đã đánh dấu bước trưởng thành vượt bậc của thơ Việt Nam và đồng thời cũng là đánh dấu một trang sử lớn của dân tộc. Tôi nói: bước trưởng thành vượt bậc của thơ là vì, như chúng ta đều biết, trước ngày Tổng khởi nghĩa 1945, nếu như có một số nhà thơ và chiến sĩ Cách mạng đem thơ trực tiếp góp sức làm thành cuộc khởi nghĩa đánh Pháp đuổi Nhật, nếu như có một số nhà thơ gác bút, đau khổ suy nghĩ, bắt đầu thấy ánh sáng mới, thì đại bộ phận thơ ta bấy giờ đương từ giường bệnh xuống nhà xác, thơ ta đương hấp hối: kín mít và trụy lạc. Cho đến khi, cả dân tộc trỗi dậy, Cách mạng mở cửa cho một giai đoạn lịch sử thì cũng mở cửa luôn cho thơ và tâm hồn nhà thơ. Từ đấy thơ vượt lên, nó tha hồ đánh những cái nó muốn đánh, yêu những cái nó muốn yêu. Do đó, thơ ca Việt Nam hiện tại, lần đầu tiên trong lịch sử, mới thật là có những bài thơ ca ngợi sự sống, con người và cuộc chiến đấu, những thơ ca hay nhất của Tố Hữu, Văn Cao, Trần Hữu Thung, Tú Mỡ, Bảo Định Giang, Xuân Miễn và hàng trăm nhà thơ xuất sắc khác. 
 
Như vậy, tôi cho rằng mười năm qua đã có được những bài thơ nổi tiếng và giá trị nhất định của mười năm qua. 
 
Nhưng cũng ngay khi "thịnh trị" nhất đó, cũng vô số những cái bã rả tầm thường. Chính vì thế mà, nếu không khéo nhìn, người ta dễ không đánh giá thật đúng được giá trị thơ Cách mạng và kháng chiến, mà cũng không phân biệt được những cái kém cỏi ở ngay bên cạnh những châu báu ngọc ngà kia. Những kém cỏi đó cứ kéo dài rồi lây ra tràn lan. Đã nhiều lúc, người làm thơ và yêu thơ phải phàn nàn: thơ ta bế tắc, nhạt nhẽo, nghèo nàn, một chiều và rập khuôn. Làm sao mà nên nỗi thế, đó là một vấn đề tôi sẽ phân tích ở một dịp khác, đây tôi chỉ nêu một hiện tượng khách quan. Ví dụ như khi Tố Hữu viết những câu tuyệt hay: \textit{Anh ở Vĩnh Yên lên. Tôi trên Sơn Cốt xuống…} thì sau đấy, trang thơ báo nào cũng mọc ra hai người đứng đối đáp lải nhải. Khi Nguyễn Đình Thi phóng ra hình thức thơ tự do thì, thôi thì loạn! Bao nhiêu câu kệ lại rặt những: \textit{Kọt kẹt… Bàn tay… Bàn tay…} một cách tục tĩu và ngây ngô. Tất cả những cái đó, vì không ai xem xét, đánh đập gì cả, nó cứ đọng dần thành một chiều, rập khuôn, − củi rác trôi trên sông tụ lại nhiều quá cũng cản được nước chảy, và chính nó đã gây nên mọi bế tắc trong thơ gần đây. 
 
Nhưng, một sức lực mới đương làm thơ ta hồng hào khởi sắc lên rồi. Thật thế, cái thời bế tắc kia đã hết, nhờ sự cố gắng của người làm thơ và của nhiều điều kiện khách quan khác, từ khi hòa bình. Thơ ta đã vượt qua được một chặng đường gian khổ. Lúc này đây, thơ ta đương được nước, nó đương thúc sâu vào các ngọn nguồn ngách sông của nhiều vấn đề mới (hoặc cũ nhưng lúc nào cũng mới), những vấn đề tình yêu, vấn đề trách nhiệm, tình cảm đất nước, lòng tin yêu Đảng, những suy nghĩ và lao động kiến thiết hàng ngày rất bình thường mà rất vinh quang… Người làm thơ đương cố gắng đem thơ vào một triết lý về cuộc sống. Để cải tạo con người và thay đổi xã hội − dù biểu hiện cái xấu hay cái tốt gì gì đi nữa, cuối cùng cũng là nhất trí làm cho cuộc đời và chế độ cứ hớn hở mãi lên. 
 
Nhiều người đã thành công. Phải kể đến những tác giả trong tập \textit{Cửa biển }(thơ của Hoàng Cầm, Văn Cao, Lê Đạt, Trần Dần, − Nhà xuất bản Văn nghệ) và rải rác nhiều bài của Tế Hanh, Yến Lan, Quang Dũng, Hoàng Tố Nguyên, Phạm Hổ… Họ hăng hái mà tình tứ, táo bạo nhưng có trách nhiệm, và đẹp, và đáng yêu xiết bao. Những bài thơ đầy ánh sáng, thiết tha, rực rỡ úa vào tâm tưởng chúng ta, đập phủi các thứ nhọ nồi bồ hóng, sửa sang trang trí đầu óc và cuộc đời cho luôn được mới, được đẹp. 
 
Tôi vốn yêu thơ, tôi hết sức hoan nghênh. Họ đã giúp tôi suy nghĩ, thưởng thức, làm việc, và nhờ tinh thần thơ họ mà người đọc thơ thấy thêm những lớn lao sâu sắc của đời sống trên đất nước còn đau, còn khổ, nhưng rất yêu thương và tươi trẻ của chúng ta. Những loạt thơ đầu tiên ấy đã đem được một sắc thái riêng vào thơ của thời đại. 
 
Nhưng cũng lại trong loạt thơ đầu tiên ấy đã kéo theo nhiều cái rơm rếch. Nếu ví loạt thơ ấy là một phát đạn "khám phá mới", thì phát đạn bắn lên, bụi rác xung quanh cũng cuốn bay theo. Có người nói: quí ở những thơ hay, xá chi cái rác bay lên vì gió, bởi nếu quả nó là rơm rác, thì rồi rơm rác cũng cứ khắc rơi. Có thể có người cho là nói bây giờ là chặn họng, chột mất sự hăng hái của người ta. Nhưng ở trường hợp đây, tôi thấy nó rất mập mờ và lộn sòng nguy hiểm. Bởi vậy, tôi phản đối sự nhân nhượng trên và cứ xin nói, không hề tự thấy mình hẹp hòi. 
 
Trách nhiệm người làm thơ, cũng là trách nhiệm người công dân, người thi sĩ công dân đúng mức, luôn đi sâu tìm mình và xung quanh, đem góp cái mình chung với mọi người, dựng nên trí tuệ của giai cấp, sự thông minh của Đảng, của dân tộc, của loài người. Nhưng trong khi đấu tranh đào mình trong suy nghĩ, quan sát, sinh hoạt, sản xuất, công tác và hoàn cảnh thực tế xung quanh (những điều ấy cũng có nghĩa là luôn luôn tự cải tạo) ở mỗi người trình độ khác nhau về nhiều mặt, có tiến bộ, có đứng yên, có giật lùi, có những người thường kém tỉnh táo, say một cách quáng gà, bởi vậy mà cái tự ái trí thức cũ và nghệ sĩ cũ đã có đất sống lại, tự mình lại nịnh hót mình. Những biểu hiện đó là bề mặt của cuộc đấu tranh tư tưởng trong mỗi chúng ta. Quả nhiên là một cuộc gay go chống đối, hoặc tiến lên, hoặc lùi xuống, mà trong đó, tư tưởng vô sản đương đầu với những loại tư tưởng phi vô sản, hai bên đương giành giựt chỗ đứng, bằng cách này hay cách khác, − ở đây thì rõ ràng là tư tưởng vô sản phải đối chọi với cái trí thức nửa mùa và nghệ sĩ kiểu cổ đương khoe mẽ cái cá nhân ích kỷ đã lỗi thời lại vừa kiêu căng, vừa đồi trụy của nó. 
 
Một số bài thơ của Hoàng Cầm cũ (thật không? hay cũng lại là một kiểu rằng mình chống công thức đã lâu) và mới, ý và lời bâng khuâng cảm khái, nửa đẹp, nửa chua. Có thể Hoàng Cầm thích nó, vì tưởng nó là thật có cảm. Tôi cho là nó yếu − song nó có một sức dối trá khá mạnh, nhưng dù sao nó cũng không thể so được với những bài thơ khi sôi nổi, khi nhớ thương của Hoàng Cầm đã làm trong kháng chiến. Ẩn nấp trong những bài thơ này, là một thứ cái Tôi không đúng, không phải, đã được khai quật lên, nhà thơ đem kẻ môi son, bôi má hồng cho xác chết, phù phép cho xác chết khóc, cười, biết kêu và biết thở dài vân vân… 
 
Tôi cũng nghĩ tương tự như trên, khi thấy Huy Phương (“Nhật ký đêm hè”) đương đêm ra tựa cửa sổ nhìn xuống đường phố, nghĩ mình thì nghèo một cách vừa đẹp vừa buồn, trông lên trời thì thấy trời rộng và có nhiều sao, nghĩ xuống cuộc sống thì thấy đời còn nhiều bon chen và xấu xa, nhà thơ bèn làm bài thơ dở khóc dở cười gửi lại cho con mình mai sau lớn lên sẽ đọc. Cứ kể cái việc cửa sổ đề thơ hoài tình gửi mai sau như thế thì cách đây hàng chục năm, những Lưu Trọng Lư, Nguyễn Xuân Huy, Thái Can, Huy Cận đã làm nhiều lắm (chưa kể Nguyễn Du thì đã \textit{Chẳng biết ba trăm năm sau}…, nghĩa là Nguyễn Du đã làm thơ giống Huy Phương từ hàng trăm năm về trước kia) nhưng thôi, hãy tạm xá cho cái hình thức (Bởi vì nghĩ rằng nếu bây giờ cứ cửa sổ đề thơ gửi cho con mà nội dung thơ ấy là mới nhất, hiện đại nhất, thì thật ai cũng mong mỏi vô cùng), ở đây nói đến cái xác chỉ để nhìn vào cái hồn, tôi chỉ muốn xem sâu vào lòng bài thơ ấy, tôi tưởng trước khi Huy Phương kể tội cái bóng đen làm bẩn cuộc đời, nên hãy đưa triết lý thơ mình mà mổ xẻ, xem xét rồi xử bắn ngay cái bóng đen u ám ấy ở trong lòng mình trước nhất. Đừng nghĩ mình đã "thoát tục" mà chính bóng tối lòng mình cũng đương nhập vào bóng tối cuộc đời, làm hại cuộc đời không ít. Xin trách ta rồi hãy trách đời, \textit{nghĩ mình công ít tội nhiều}, thưa bạn. 
 
Đúng, phân biệt cho được chứng minh thư thật hay giả về cái Tôi sống, về cái Tôi chết trong lúc này thì khá gay go. Tôi thường trao đổi với Lê Đạt và chúng tôi cũng đồng ý thế. Ấy vậy mà khi làm thì chính nhà thơ bạn tôi kia cũng lại ngã lộn phộc vào cái tử tiệt mà mình hàng ngày chửi rủa không tiếc lời. Đó là trường hợp “Nhân câu chuyện mấy người tự tử\footnote{\url{http://www.talawas.org/talaDB/showFile.php?res=9370&rb=08}}”, cái bài thơ dài lằng nhằng, khá chua mà cũng khá hùng đã bắt đầu từ một ý muốn chủ quan của mình, nhân có một câu chuyện tự tử bi đát của thiên hạ. Lê Đạt khuyên người yêu hãy khoan khoan chớ vội đi thắt cổ, em hãy cùng ta can đảm bước vào con đường tình chông gai, hãy hất đi cái bục công an phong kiến đương chặn giữa trái tim ta. Nói chuyện với người yêu mà thành chuyện của đời, chuyện mình mà lại thâu tóm và hình dung được tất cả tư tưởng và vấn đề của thời đại, nếu làm được thế thì bài thơ nhất định bất tử, bài thơ có ích vô cùng và nhà thơ cũng cao tay vô cùng, nhưng nếu không làm nổi như thế thì nó chỉ là một cách gửi thư không tốn tem. Tôi nghĩ là bài thơ Lê Đạt nên xếp vào trường hợp thứ hai. Bởi nó chỉ do chủ quan không đúng của mình níu vào một sự thật không đâu (việc tự tử) nên câu chuyện không thể nào điển hình (hoặc chỉ điển hình cho riêng mình), không thể là vấn đề của thời đại. Càng không thể là tư tưởng thời đại, không thể "đứng mũi nhọn cuộc sống" như ý muốn của nhà thơ, mà nó chỉ lẻn lút ở sau lưng cuộc sống với một loạt tàn dư của thứ tư tưởng lạc loài (nếu không là lạc hậu lắm rồi) trong một xã hội đương vùn vụt biến đổi từ mười năm trở lại đây. Vì vậy mà người đọc dễ cho là Lê Đạt "ngậm mực phun đen chế độ". Nhưng thế cũng là làm ra to chuyện quá, hoặc là một kiểu truy nguyên lên đến cùng mà thôi. Tất nhiên không có giá trị, mặt khác, nghĩa là có hại, nhưng mà chỉ là ở một chừng mực nào. Bởi vì, tôi tưởng ý bắt đầu của Lê Đạt có thể là đúng, nhưng sai lầm là ở chỗ đã gán ghép, chắp vá mình vào thực tế (cố nhiên, cái mình đeo chứng minh thư giả) − nên những thực tế trông qua mình ấy đã méo xệch. Nếu cứ còn nhìn thực tế qua mình một cách bức bối, bực dọc như vậy, nhất định một sự thực tốt lành tới đâu cũng phải xám ngắt. Ở đây chưa phải là bàn cãi gì đến chuyện bôi đen hay vẽ đỏ, mà vấn đề còn ở mức giản đơn và ấu trĩ hơn là con người nhà thơ đã nhìn lạc mắt, từ cái đưa mắt đầu tiên, không khách quan phân biệt và khách quan tổng hợp được sự suy nghĩ của mình với sự sống quanh mình. 
 
Nói sang chuyện hình thức. Cũng nhiều cái tức cười. Ở Lê Đạt, hình thức và nội dung cái sai lầm còn son phấn và đội mũ mặc áo trà trộn tinh vi; ở nhiều nhà thơ khác, khuyết điểm này lại ngô nghê hơn. Đại để như Hồng Lực cả tiếng bênh vực sự hủ hóa, kêu gọi: \textit{Một trái tim và một tấm lòng }cứ việc mà đứng dậy văng tê. Hoặc lộn sòng như Phác Văn:         
\begin{blockquote}
        
\textit{Như đường ray ôm ghì cuộc sống. }        
\textit{Giữ người yêu giữ từ trong mộng}…  

\end{blockquote}
 
Đó là bạn ấy muốn đem những tiếng nói thời đại": đường ray, cuộc sống" đắp lên những hình ảnh "ôm ghì, người yêu, trong mộng, hơi thở", v.v… Không nên, sự thật bao giờ cũng tàn nhẫn với những cái gì giả dối, sự lộn sòng ấy chỉ làm thơ bạn càng trơ trẽn và trống rỗng hơn. Tôi rất yêu những bài thơ xinh xinh của Phác Văn. Tôi thiết tưởng bạn tôi cứ mũ áo bình thường như thế thì rất nên; đừng nghe ai xui mà cho thế là nhàm. 
 
Cũng vì hình thức, những bạn thơ mà tôi vừa kể trên thường đem mặc cho thơ mình một số quần áo khá lòe loẹt, cố làm ra vẻ lạ và oai. Phổ biến nhất là cái áo triết lý vừa dài vừa lụng thụng, cứ lải nhải, nhũng nhẵng, tha hồ cho người đọc phải khổ sở xem thơ (một số bài của Lê Đạt, Trần Dần). Hoặc họ lại có cái thú viết hoa những chữ mà mình cho là chữ thần, đại khái như Tình Yêu, Trái Tim, Vĩ Đại, Hùng Anh, Đớn Đau, Trụy Lạc, Cuộc Đời… Ngày trước thì một số nhà thơ ngồi trong vò của phái Xuân Thu và Hàn Mặc Tử cũng hay chơi thế, vài năm gần đây thì có Như Mai cũng tập lại lối chơi chữ ấy, nhưng rồi người đọc và thời gian và chính người viết ngày một tiến lên đã triệt được cái nọc lố lăng, cầu kỳ nọ. Bây giờ, một số bạn ngỡ là mới, lại đào lên và bệ về. Hoặc còn có bạn làm thơ kiểu cách cứ xuống dòng bạt mạng, vừa gieo thơ vừa dùng phép quỉ thuật giúp cho cách đặt câu, như Hữu Loan:        
\begin{blockquote}
        
\textit{Đầu người}        
\textit{Và tình yêu} 
\textit{Treo trên đầu sợi tóc…} 

\end{blockquote}
 
Đó là một trong rất nhiều câu khó nghe, đương lổn nhổn trên nhiều trang báo. Chưa hết. Có bạn lại mắc chứng hay kêu to: \textit{Trung ương Đảng ơi!}, rồi hô hào hùng hồn đao búa: \textit{Tuổi hai mươi đương làm gì? Hỡi tất cả}!... Rồi thì bài nào cũng kết luận rập khuôn: "bút thơ" tôi nhất định dũng cảm xông lên hàng đầu, "bút thơ" tôi quả quyết đi trừ khử những bóng đêm và ngoáo ộp trong cuộc đời. Rốt cuộc, những nhà thơ đeo nhãn hiệu "chống công thức" đó cũng lại ngã vào cái khuôn công thức của những bài thơ tầm thường trước kia chỉ kết luận toàn xác chữ, nào công sức nhân dân, nào vùng lên, vùng lên, v.v… cái khuôn cũ rích ấy mà họ vẫn không ngớt miệng sỉ vả nay họ lại đem ra quét nước sơn khác mà thôi. 
 
Tại sao, trong thời gian gần đây, thơ ta đã có phần tiến bộ như vậy mà lại còn mang những bệnh tật nặng nề khốn cùng nhường đó? Nói kiểu chính trị, ấy cũng là cái bệnh "sốt vỡ da" của người đương lớn. Nhưng mà tại sao lại như thế? 
 
Hiện nay đông đảo văn nghệ sĩ đương lao mình vào thực tế đấu tranh để thể hiện tác phẩm. Nhưng tôi thấy cũng còn lắm người trong lúc này từ chối đi vào thực tế, thực tế cuộc sống và thực tế văn học nước ta. Cố nhiên, họ từ chối bằng nhiều mẹo quyến rũ rất êm ái mà chính họ cũng bị mắc lừa, chứ không ai dám trắng trợn xua tay đâu. (Cũng như một số cây bút chính trị của báo \textit{Nhân} \textit{văn} đã ngạc nhiên khi thấy quần chúng bảo là báo họ bôi đen chế độ.). Đi làm Cải cách Ruộng đất, sửa sai, vận động sản xuất, đi công trường và mỏ bây giờ, họ quan niệm là một chuyến lấy tài liệu ăn xổi và họ xấu hổ, khi người ta bảo đó là một cuộc tự cải tạo. (Tự cải tạo chỉ là một cách nói sống sượng, vô ý, nhưng, đằm mình vào cuộc sống ấy, hỏi một người viết chân chính nào ở nước ta lại nỡ phản đối?) Huống chi, chúng ta, ở bất cứ lớp người nào, cái lực học nghề, học đời phỏng đã được là bao. Hãy đo sức mình thật, đừng mơ hồ ảo tưởng, đừng bốc. Thế mà còn có người cho rằng cứ phải đi tìm thực tế mãi ở đâu, có cái kho thực tế ngay trong mình, cái tâm hồn dễ nóng dễ nguội của mình đây cũng nhiều chất lắm, khai nó lên, hàng đời cũng chẳng hết, việc gì phải tìm kiếm đâu mất công. Cho nên mới ra đời những loại thơ gửi cho con, những “Nhất định thắng\footnote{\url{http://www.talawas.org/talaDB/showFile.php?res=7351&rb=08}}” ngông nghênh, những truyện ngắn Trần Lê Văn quanh quẩn tả vài cái tật hay hay cỏn con của văn nghệ sĩ. Họ không chịu khó đọc một chút, đọc văn học hiện đại nước ta thôi, cho nên tinh thần và lời lẽ đâm ra dông dài, leo thang xuống thang quái quỉ (tôi không phản đối hình thức thơ leo thang, nhưng tôi chán lối bắt chước dễ dãi "kéo xuống hay cho lên cũng được"), khi đọc cái tham luận mới nhất của Cholokov, họ rất khoái câu: “Với một nhà văn, không thể nói mỗi chuyến đi thực tế mà phải nói cả đời nhà văn ấy là đi vào thực tế.” Họ hiểu nghĩa cả đời vào thực tế ấy là ngồi moi móc đẽo gọt mình. Thật ra, tôi nghĩ rằng với nhà văn Việt Nam, tuổi đời cũng như tuổi nghề còn rất trẻ thì bất cứ chuyến đi dài ngắn nào cũng là cần thiết. Nếu Cholokov nói ở Mạc-tư-khoa có một số nhà văn sợ thực tế chỉ sống trong cái trục ba góc: nhà, câu lạc bộ, nơi nghỉ mát, thì ở Hà Nội, chúng ta chế nhạo cái lối sống lười lĩnh đó, nhưng chính ra có một số trong chúng ta cười người đấy trong khi không tự biết mình cũng dần dần hóa ra con kiến đậu cành đào, con kiến leo vào leo ra, hết ở nhà (hoặc cơ quan) lại đi bát phố, hết bát phố lại về nhà, v.v… 
 
Dần dần, họ xa với thực tế lớn lao xung quanh. Họ cứ quanh quẩn với mình. Một ý nghĩ tủn mủn của mình bèn cho là một vấn đề xã hội, một khám phá mới, một "cú" chống công thức dữ dội. Mới về Hà Nội thì nhớ đồng ruộng và rừng núi, rồi đi đến không muốn ra khỏi thành phố, thế mà vẫn cả tiếng, tráo mắt lên quát: đứa nào bảo tao bị tư tưởng tư sản tấn công là nói láo. 
 
Ấy vậy đó. Những nguyên nhân trên, tôi cho là chính yếu và tai hại nhất lúc này. Đặt tên nó là tư tưởng gì cũng được, nhưng hiện tượng nó là như vậy. 
 
Trở lại với một số bài thơ trên kia thì tôi thấy rằng trong sự cố gắng đặt trách nhiệm cho mình, người làm thơ đã đem mình hòa thành cái Tôi của thời đại. Sự suy nghĩ công phu và can đảm ấy rất đáng kính trọng, nhưng chỉ buồn là cái anh chàng trí thức nửa mùa trong người nhà thơ cũ hay nhân cơ hội này mà nó lại bò dậy, vác ra cái xác chết rồi, cái xác lâu nay nằm khô trong tâm hồn ta. Chính vì vậy mà bên cạnh những bài thơ quí giá, đã có một số thơ ốm yếu xanh xao, lại khoác một hình thức lòe loẹt huênh hoang trống rỗng, chết ngay đuôi ra mà cứ tưởng ta khỏe nhất thiên hạ. Nguy hiểm hơn là họ lại muốn nâng cao những cái tàn tạ ấy lên thành một triết lý, một quan niệm nghệ thuật, một nhân sinh quan mới, ngụy biện rằng đó là "người thơ đem trách nhiệm mình nâng xã hội lên". Tôi tưởng chẳng phải thế. Cái hơi thở tàn ấy thật ra chỉ hợp ở những con bệnh u tối lỗi thời. Nó chỉ có sức làm ngóc dậy được những cái gì loạn lạc nhất mà thôi. 
 
Cái gì đã chết rồi, xin khiêng trả ra bãi tha ma. Nếu người sống mà cứ nhìn đời bằng con mắt múi nhãn của xác chết thì nhất định một sự thực tốt đẹp nhất cũng xám xịt. Tôi mong các bạn thơ tôi mau chóng vượt qua được cái khúc bối rối của tâm tư mình, anh hãy cho chúng tôi đọc những bài thơ mới nhất về tình yêu và tuổi trẻ, về trách nhiệm và cuộc đời, bài thơ thật là của anh và cũng là của sự sống lớn lao hôm nay. 
								 
\textit{7-10-56} 
\end{multicols}
\end{document}