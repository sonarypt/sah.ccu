\documentclass[../main.tex]{subfiles}

\begin{document}

\chapter{Federico García Lorca: Họa mi & Trumpet}

\begin{subtitle}

(110 năm Federico García Lorca 1898-2008)

\end{subtitle}

\begin{metadata}

\begin{flushright}2.11.2008\end{flushright}

Hoàng Hưng

Nguồn: Phần chính in trong Thơ Federico García Lorca, bản dịch của Hoàng Hưng, Sở Văn hoá Lâm Đồng xuất bản 1988. Bản đăng trên talawas có bổ sung của tác giả.

\end{metadata}

\begin{multicols}{2}

"Con hoạ mi xứ Andalusia đã bị sát hại!". Tiếng kêu ấy truyền đi khắp các trung tâm văn hoá châu Âu một ngày mùa thu năm 1936 báo hiệu mở màn cuộc tàn phá nền văn minh nhân loại. Sau đó là Guernica, là Ba Lan,… là chiến tranh thế giới… 
 
Federico García Lorca, nhà thơ lỗi lạc của Tây Ban Nha, là một trong những nạn nhân đầu tiên của chủ nghĩa phát xít. Bị bọn tướng lĩnh phản bội nền Cộng hoà bắt giữ ngày 17-8-1936, thi thể anh được tìm thấy trong đống xác 15.000 người bị bắn ngày 19-8 trên miệng một vực sâu gần Granada. Granada của đời anh, của sự nghiệp anh, nơi anh sinh ra, nơi anh về để nhận cái chết thảm khốc. "Nếu có ngày, nhờ Trời, tôi được vinh quang, thì vinh quang ấy phân nửa thuộc về Granada, nơi đã tạc nặn nên cái tạo vật tôi: thi sĩ bẩm sinh không thể cải hồi". 
 
Granada là một trong bốn thành phố lớn \footnote{
Granada, Sevilla, Malaga, Cordoba}  của xứ Andalusia ở miền nam Tây Ban Nha, xứ sở của Carmen, của những điệu nhảy và bài hát mê cuồng, của những hội đấu bò tót làm máu đập thành tiếng trên vạn đôi môi, của những rặng ôliu ngăn ngắt, những vườn cam và hoa nhài ngát hương đêm hè khiến "những người đang ngủ bỗng khát thèm từ bao lơn nhảy xuống". Xứ sở đặc hữu sự giao hoà hai nền văn minh Đông – Tây: nơi đây đã từng là "một trong những vương quốc đẹp nhất của châu Phi" mà ngơời Ảrập xây dựng nên, còn để lại bao dấu tích trong kiến trúc, trong nghệ thuật, trong hồn người, để lại trong không gian một cái gì mơ hồ, xa xăm, huyền bí. \footnote{
Vào đầu thế kỷ VIII, đế quốc Ảrập xâm lăng chiếm đoạt Tây Ban Nha. Đến cuối thế kỷ đó, những thủ lĩnh Ảrập ở Cordoba tuyên bố độc lập với đế quốc năm 929, chính thức thành lập Vương quốc Ảrập Cordoba. Song đến 1031, vương quốc bị tan rã thành nhiều tiểu quốc, và do đó, bị những ngơời Kitô giáo phản công chiếm lại, dần dần cho đến 1492 thì người Ảrập bị quét sạch khỏi bán đảo.}  
 
Sinh ra ở một làng quê gần thành Granada \footnote{
García Lorca sinh ngày 5-6-1898 ở làng Fuente Vaqueros.} , trong một gia đình nông dân bậc trung thuộc một dòng họ lâu đời, nhà thơ thừa hưởng ở người cha tâm hồn gắn bó với đất đai, thiên nhiên, ở người mẹ trí thông minh và những năng khiếu nghệ thuật. \footnote{
García Lorca là tên của người mẹ nhà thơ, cũng là tên một thành phố có nhiều dấu tích pha trộn hai nền văn minh Do Thái và Ảrập. Có nhà nghiên cứu gợi ý: phải chăng do nơi mẹ, nhà thơ mang trong máu một nguồn gốc phương Đông xa xôi khiến thơ ông giàu ảnh hưởng của Kinh Thánh, chứa đựng một niềm sầu xứ kín thầm và tràn đầy tưởng tượng, nức hương thơm?}  Tuổi thơ anh hoàn toàn "thôn dã" với "những đàn cừu, đồng ruộng, bầu trời, sự cô tịch" như về sau anh kể lại. Đến năm 1909 gia đình anh mới dọn lên thành phố, và ở Granada thủ phủ xưa của xứ Andalusia, García Lorca đã trải qua thời cắp sách ở bậc trung học và đại học. Một sự việc rất có ý nghĩa: ở khoa luật của đại học Granada, anh sinh viên García Lorca gặp được một người thầy, một người anh tinh thần, một người bạn lớn, đó là giáo sư Fernando de los Ríos, một nhà lý luận về chủ nghĩa xã hội, một niềm vinh dự của nền đại học Tây Ban Nha lúc đó, và sau này là Bộ trưởng Giáo dục trong chính phủ Mặt trận Bình dân. 
 
Mùa Xuân năm 1929, theo lời khuyên của giáo sư, García Lorca lên Madrid trú ngụ ở cư xá sinh viên, nơi đang mở rộng cửa đón nhận những tư tưởng triết học và mỹ học mới mẻ nhất của thời đại. Chính ở nơi đây anh bắt đầu tình bạn với Salvador Dalí – hoạ sĩ, Bunuel – nhà điện ảnh, Rafael Alberti, Pedro Salinas – nhà thơ... Và chính ở nơi đây, thi tài của anh đã được khẳng định và chào đón trong nhiệt thành của những người bạn trẻ. Bạn bè anh kể lại: García Lorca có một sức quyến rũ lạ lùng, từ con người anh với phong độ thanh quý, vẻ vui hoạt, đôi mắt u tối nhưng lại tươi cười, nước da màu đồng, giọng nói như đồng, "một cái gì như chớp lóe trong thể chất, một năng lượng luôn luôn chuyển động, một niềm vui, một sự bộc phát mãnh liệt, một vẻ trìu mến hoàn toàn siêu việt. Con người anh kỳ diệu, màu nâu, kêu gọi sự toàn phúc" (Pablo Neruda), đến kỳ tài ngẫu hứng của anh về nhạc, về hoạ, về sân khấu, về thơ, cả sáng tác lẫn thể hiện (trước khi học văn và luật, García Lorca đã say mê âm nhạc, anh còn là một hoạ sĩ có nét vẽ duyên dáng, là một người chơi dương cầm đặc sắc). 
 
Từ những đêm thơ nhạc trong khuôn viên cư xá sinh viên, tiếng tăm nhà thơ trẻ Andalusia vang ra khắp thủ đô. Giữa làng thơ Madrid lúc đó đang ồn ào những khuynh hướng thâm nhập từ Paris, đặc biệt là trường phái siêu thực – mà biến dạng của nó tại các nước nói tiếng Tây Ban Nha có tên gọi là "sáng tạo chủ nghĩa" (créationnisme) hay "cực đoan chủ nghĩa" (ultraisme) – giữa lúc nhiều người ầm ĩ kêu gọi "Âu hoá Tây Ban Nha", mà để chống lại, nhà thơ lớp cũ Unamuno bèn xướng lên điều ngược lại "Phi hoá châu Âu" – thơ García Lorca nổi bật lên xu hướng trở về khai thác dân ca, tìm lại những truyện thơ trữ tình và lịch sử còn lưu truyền trên miệng người dân quê tỉnh lẻ. Thế hệ thơ anh đã tìm thấy ở anh người mang sứ mệnh đẹp đẽ: tìm lại tâm hồn Tây Ban Nha đang có nguy cơ bị quên lãng, nối kết cái truyền thống với cái thời đại. 
 
Tập thơ đầu tay của García Lorca ra đời năm 1921 đã báo hiệu sự hình thành thi tài, phong cách và hướng đi của anh. Nhưng phải đến những bài thơ sáng tác từ năm 1921 trở đi (sau này được tập hợp trong tập \textit{Thơ và bài hát}) mới định hình cái giọng hót riêng quyến rũ của con "hoạ mi Andalusia". Thời gian này anh say mê tìm tòi, ghi chép, thu thanh dân ca, đến nỗi có người bạn gọi anh là "chàng hát rong thời Trung cổ". Năm 1922, anh cùng với nhạc sĩ Manuel de Falla tổ chức hội Cante Hondo ở Alhambra (nơi có cung điện Quốc vương Ảrập của Vương quốc Granada xưa). Cante Hondo (có nghĩa là bài hát giọng trầm) là loại dân ca độc đáo của miền Andalusia. Qua hội này, anh đã khai thác được hàng trăm bài Cante Hondo với "lời ca say đắm", "giai điệu xưa", tha thiết và ám ảnh như tiếng "một con hoạ mi mù ca hót". (Người giật giải thưởng của hội là một ông cụ 73 tuổi!) Chính đây là khởi nguồn những bài thơ tuyệt vời mười năm sau sẽ ra mắt trong tập \textit{Thơ về những làn điệu Cante Hondo}, trong đó những thể dạng chủ yếu của loại dân ca này (Seguidilia, Solea, Saeta, Petenera) được nhân cách hoá, được diễn tả trong thế giới thích hợp với từng thể, có các nhân vật và phong cảnh khác nhau. 
 
Năm 1924 anh bắt tay viết những bài \textit{romance} hiện đại. \textit{Romance} là thể thơ có nguồn gốc từ lâu đời ở các nước dùng ngôn ngữ Latin, nhưng đặc biệt phát triển ở Tây Ban Nha, đó là những bài ca dân gian kể chuyện lịch sử, chuyện anh hùng hiệp sĩ hay tình yêu. Năm 1928 tập \textit{Romance gitan} ra đời đã thành công một cách phi thường. Tên tuổi nhà thơ trẻ vượt biên giới quốc gia (tập thơ được dịch ra 20 thứ tiếng), đồng thời nhiều bài \textit{romance} lại quay về thâm nhập các làng quê Tây Ban Nha và được lưu truyền như những bài dân ca, đặc biệt là trường hợp bài "Cô nàng ngoại tình" (Sau này trong khi đưa đoàn kịch La Barraca đi lưu diễn các miền quê, García Lorca đã có dịp được những cô gái làng đọc cho nghe từng đoạn trong những \textit{romance} của chính anh.) Tập \textit{Romance gitan} đơợc coi là tập thơ phổ biến rộng rãi nhất trong thơ ca hiện đại Tây Ban Nha. 
 
García Lorca đã giải thích tên tập thơ của mình: "Tôi đặt tên tập \textit{romance }này là \textit{gitan}, bởi vì trong đó tôi ca hát xứ Andalusia, mà chất \textit{gitan }là biểu hiện thuần tuý nhất, đích thực nhất của xứ ấy". 
 
Người \textit{gitan} ở Tây Ban Nha, cũng như người \textit{digan} ở Nga, người \textit{bohemien} ở Tiệp, Pháp... có gốc Ấn Độ, làm thành những cộng đồng du cơ độc đáo của châu Âu. Và có lẽ chính ở vùng Andalusia, họ đã tìm thấy quê hương, cây ghi ta và những vũ điệu của giống người lang bạt đầy quyến rũ đã làm nên linh hồn của xứ này. Sau tập \textit{Romance gitan}, người ta gọi García Lorca là "nhà thơ \textit{gitan}". Được gọi thế anh cảm thấy thích thú, có lúc anh còn nửa hư nửa thực gợi ra một giai thoại về "nguồn gốc \textit{gitan}" bí mật của mình. 
 
Xứ sở Andalusia đã cho García Lorca giọng điệu đích thực để hát về nó. Trong một bức thư gửi nhà thơ Guillen, Lorca viết: "Tôi chỉ muốn nói với anh rằng tôi ghét giọng đàn sáo réo rắt. Tôi yêu giọng con người, chỉ giọng con người mà tình yêu phơi trần, giọng con người nổi bật lên giữa những phong cảnh giết người". \footnote{
Andalusia đối với García Lorca chẳng khác gì Kinh Bắc đối với Hoàng Cầm sau này.}  
 
Những phong cảnh giết người có sức cuốn hút mãnh liệt. Phải chăng cái dữ dội của "cánh đồng dựng đứng dưới hai mươi mặt trời, những dòng sông chồm lên", cái đau đớn của "rặng Morena mạn sườn nhỏ máu", cái "hoang vu lượn sóng" hấp dẫn ta với vẻ đẹp nghiêm trầm, bạo liệt đầy nam tính? 
 
Và trên nền phong cảnh ấy bật lên, trần trụi, giọng con người. 
 
Con người thơ García Lorca "miệng đầy nắng và đá lửa", rên lên, kêu lên nỗi khao khát đốt cháy cơ thể "làn áo và thịt da/ hoá thành huyền đen thẫm", mê cuồng như một điệu ca không biết "đi về đâu/ với tiết tấu không đầu", nhức nhối như có "một mũi lao cắm xuống bật kêu thành tiếng" giữa hai hàm răng. Khát khao sự sống đến khắc khoải, vì con người thơ García Lorca luôn đối mặt với cái chết. Cái chết như hiện diện mọi lúc, mọi nơi. Nó "rình rập/ từ trên ngọn tháp Cordoba", nó ở trên mũi dao nhọn run rảy "giữa lòng ngã tư/ nơi phố phường rung lên/ như sợi dây", cái chết ám ảnh như định mệnh khắc nghiệt điểm giờ chàng đấu bò tót Ignacio: 
\begin{blockquote}
        
"Tất cả mọi đồng hồ đều chỉ năm  giờ        
Ôi năm giờ chiều tăm tối!" 

\end{blockquote}
 
Với bài thơ dài "Tang khúc cho Ignacio", García Lorca đã đạt đến mức bi tráng sâu thẳm và vang dội vào bậc nhất trong thơ ca nhân loại nói về cái chết: 
\begin{blockquote}
        
"Ignacio lên từng bậc thang        
Cõng trên lưng cái chết.        
Tìm kiếm bình minh        
Mà bình minh không có        
Tìm bóng đích thực mình        
Mà giấc mơ đánh lạc        
Tìm thân mình khoẻ đẹp        
Mà thấy máu mở tuôn..." 

\end{blockquote}
 
Từ cuộc tranh chấp vĩnh hằng không thể hoà giải giữa sự sống – cái chết, những khao khát không bao giờ thoả mãn, những cái đích không bao giờ đạt được... Sự bất lực của phận người sinh ra nỗi buồn chất chứa thơ anh, nỗi buồn có trăm biến dạng: thất vọng, ưu phiền, xa vắng, đắm chìm, cô tịch… Có điều nỗi buồn García Lorca không hề có sắc màu bi luỵ yếu hèn. Nó là tiếng kêu đau đớn của kiếp người vút lên như "cây cầu vồng đen" trước cái trơ trơ nhẫn tâm của trời xanh, của núi xa im lặng. Nó lành mạnh như "nỗi ưu phiền màu đen" của cô gái \textit{gitan} "chạy theo hạnh phúc". 
 
Nỗi buồn đầy cám dỗ và ám ảnh, hiệu quả của những nhịp điệu và ảnh tượng có màu sắc ma thuật phối hợp một cách kỳ tài vẻ duyên dáng bay bướm với sự sâu xa máu thịt của những năng lượng kín thầm. 
 
Trong một bài nói chuyện về nghệ thuật, García Lorca đưa ra khái niệm \textit{duende} để so sánh với vai trò của "nàng Thơ" và "thiên thần" trong sáng tạo nghệ thuật. Theo anh, "thiên thần" bay cao phía trên đầu người, ban ân sủng cho con người đón nhận một cách thụ động. "Nàng Thơ" thì mách bảo, gợi nguồn cảm hứng và nhà thơ như nghe thấy những tiếng nói mơ hồ của nàng. Song, cả "thiên thần" và "nàng thơ" đều ở bên ngoài nhà thơ, đem đến cho anh ta ánh sáng và hình thức. Còn \textit{duende}, đó là cái phải đánh thức từ trong tận cùng sâu thẳm của máu ta, nó đốt cháy máu ta, nó "vứt bỏ thứ hình học êm đềm ta học được, nó đập vỡ các bút pháp", nó là "quyền lực chứ không phải tác phẩm, cuộc chiến đấu chứ không phải tư duy", nó là cái mà Goethe đã nói đến: "quyền lực bí mật mà mọi người đều cảm thấy và không triết gia nào giải thích", nó là "tinh thần của đất". Và García Lorca cho rằng nghệ thuật của Tây Ban Nha là nghệ thuật của \textit{duende}. 
 
Thực ra thơ anh nhiều lúc đạt đến sự hoà hợp của cả thiên thần, nàng Thơ và \textit{duende}. Trong tập \textit{Romance gitan} và \textit{Tang khúc cho Ignacio}, sự thuần khiết của hình thức, những cấu trúc có trí tuệ thật hài hoà với cảm xúc cuồn cuộn, chất bi thương, chất nhục cảm, sức ám thị của từ ngữ, và cả một cái gì có tính cách linh thị, ảo giác. 
 
Dõi theo tiến trình thơ anh, ta thấy García Lorca có xu hướng ngày càng muốn đi xuống chiều sâu hồn người, như mũi dao nhọn vào sâu những lớp thịt đau đớn để tìm đến tận "gốc rễ của tiếng kêu". Nhà thơ đã từng tâm sự: "Bây giờ tôi làm một thứ thơ mở toang mạch máu". \footnote{
Thật ngẫu nhiên, gần 20 năm sau Allen Ginsberg cũng vào học trường này, ông là người mê García Lorca, chịu ảnh hưởng García Lorca và cũng có nhiều bài thơ về New York có cùng tinh thần phản kháng xã hội công nghiệp phi nhân. Xin đọc “Siêu thị ở California”, “America”, “Tỉnh giấc ở New York”…}  Thấy được tiến trình ấy ta dễ dàng đón nhận sự đột biến trong thơ anh vào những năm 1929-1930, đột biến khiến nhiều người ngỡ ngàng đến mức không nhận ra García Lorca hoặc có người – vô tình hay cố ý – còn không muốn nhắc đến khi nói tới García Lorca mà họ chỉ quen như "con hoạ mi của xứ Andalusia" và chỉ như thế mà thôi. Đó là trường hợp những bài thơ trong tập \textit{Nhà thơ ở New York}. 
 
Giữa năm 1929, Lorca theo giáo sư cũ của mình là Fernando de los Ríos sang New York, và sống như một sinh viên trong Đại học Columbia. \footnote{
Thư gửi một nhà thơ Colombia}  Thành phố "dây thép và bùn nhơ" gây chấn thương sâu xa cho con hoạ mi Andalusia. Chất nhân bản, chất bản năng của Đất phản ứng mạnh mẽ với nền văn minh công nghiệp của Thép – Ximăng. Nhưng khác với trường hợp Essenin, nỗi khắc khoải giết người không giết được García Lorca, mà lại làm bùng lên một hoả diệm sơn thơ đầy tinh thần phản kháng (sự phản kháng – tự vệ của anh mạnh đến nỗi ngay trong sinh hoạt ở đại học, anh từ chối nói tiếng Anh, và luôn tìm cơ hội để phổ biến những bài dân ca của quê mình). Sự phản kháng này không hề mang dấu mặc cảm tự ti của công dân một nước nhược tiểu trước bộ máy đồ sộ của cường quốc lớn nhất, mà là tiếng thét sang sảng của một công dân thế giới hiện đại, người vừa đặt chân tới New York đã chào Hudson là "dòng sông lớn của ta" y như một người đồng hương, một người bạn ngang hàng với Walt Whitman. Con người đó, chỉ sau vài tuần lễ, đã đủ sức dựng lên hình ảnh sừng sững ma quái của một nền văn minh bệnh hoạn, mất gốc, ngự trị bởi đồng tiền và máy móc. 
\begin{blockquote}
        
"Điệu nhảy những bức tường khuấy động miền đồng cỏ        
Và nước Mỹ ngạt thở vì máy móc với lệ tuôn". 
         
"Khi trăng lên        
Những ròng rọc sẽ quay làm rối bầu trời        
Một thế giới đầy kim sẽ vây bọc trí nhớ        
Và những quan tài sẽ chở đi những ai không việc làm" 

\end{blockquote}
 
Đó là nước Mỹ đang bước vào thời kỳ khủng hoảng kinh tế. Nạn nhân của nó là "những đứa trẻ" bị "những đồng bạc như đàn ong giận dữ cắn xé tan tành", những "phụ nữ chết chìm trong dầu mỡ", "những người loạng choạng vì chứng mất ngủ/ như thể vừa chìm trong máu ngoi lên". 
 
Chính cái xã hội phi nhân đó đã gây cho García Lorca cơn ác mộng triền miên, anh cảm thấy mình sống trong một thế giới ngột ngạt, bị ma ám, thế giới của những nghĩa địa, của những người chết rồi vẫn chưa yên, thịt da như chịu sự hành hình dai dẳng muôn đời: 
\begin{blockquote}
        
"Trong nghĩa địa xa vời có một người chết         
Than vãn suốt ba năm        
Vì đầu gối còn mang một phong cảnh khô cằn        
Và đứa trẻ sáng nay chôn khóc la dữ dội..." 

\end{blockquote}
 
Chủ đề cái chết trong tập thơ này được đào sâu triệt để, với một sự quằn quại tìm kiếm có tính chất một cuộc nổi loạn bản thể học, khiến tập thơ nhiều lúc mở ra những vực thẳm khôn dò, đe doạ dẫn nhà thơ đến bế tắc đen tối, hư vô. Song, điều đáng chú ý là, ở bất cứ bài thơ nào, sự nổi loạn bản thể học cũng gắn nhơ hình với bóng với sự phản kháng xã hội. 
 
Cái xã hội phi nhân khiến anh căm giận và anh bộc lộ thái độ rất dứt khoát: 
\begin{blockquote}
        
"Tôi biết làm gì đây: sắp xếp lại những phong cảnh? 
Sắp xếp lại những mối tình sau đó sẽ thành những tấm hình, những mẩu gỗ và những bụm máu?        
Không, không. Tôi tố cáo!"… 

\end{blockquote}
 
Anh phẫn nộ kêu gọi sự trừng phạt và mơ ước "một đứa trẻ da đen/ thông báo cho lũ người da trắng của thế giới vàng/ ngày đăng quang của lúa". 
 
García Lorca dành những tình cảm nồng thắm cho Người Đen, những con người của khu Harlem mà anh hằng lui tới, của điệu \textit{jazz} u uất, cuồng nhiệt mà anh thấy rất gần gũi điệu Cante Hondo của xứ sở anh: 
\begin{blockquote}
        
"Người Đen! Người Đen! Người Đen! Người Đen!        
Máu không lối thoát, trong đêm của anh đêm bị lật nhào        
Máu không sắc đỏ. Máu giận dữ dưới làn da,        
Mãnh liệt trong ngạnh dao găm và lòng cảnh vật". 
        
"Ôi! Harlem, bị cải trang!        
Ôi! Harlem, bị một đám y phục không đầu đe doạ!" 

\end{blockquote}
 
Có lẽ trong tập thơ này ta thấy nhà thơ đã hoàn toàn bị chi phối bởi \textit{duende}, những câu thơ vọt thẳng từ cõi thẳm sâu của tiềm thức thành luồng phún xuất, phá vỡ tiết điệu nhịp nhàng được trí tuệ kiểm soát trong thơ anh trơớc đó, những tiếng thét rợn gáy, những ảnh tượng hãi hùng và nhiều lúc phi lý, tối tăm như những gì đè nặng lên ta trong những cơn ác mộng. García Lorca ở đây là tiếng kèn \textit{trumpet} âm u và sang sảng giọng đồng. 
 
Cuộc Mỹ du đã ảnh hưởng quan trọng đến cuộc đời và sự nghiệp của García Lorca những năm sau đó. Trở về nước, anh còn bị ám ảnh bởi "ấn tượng của lạnh lùng và tàn bạo... Không ở đâu trên thế giới người ta cảm thấy mãnh liệt như ở đấy sự vắng mặt hoàn toàn của tinh thần... quang cảnh khủng khiếp, mà không có sự hùng vĩ". Có phải đó là một lý do khiến cho, khi nền cộng hoà được lập nên vào mùa Xuân 1931, García Lorca đã lao vào những hoạt động văn hoá sôi nổi với sự ủng hộ của chính quyền? Anh thành lập đoàn kịch mang tên "La Barraca" dưới sự bảo trợ của Bộ Giáo dục, đi lưu diễn khắp nơi với mục tiêu phổ biến cho đông đảo quần chúng những vở kịch hay nhất trong kho tàng văn hoá cổ truyền của đất nước. Anh say sưa viết kịch, và có những vở như \textit{Lễ cưới đẫm máu} ca ngợi tình yêu tự do đã thành công rực rỡ ở cả trong nước lẫn nước ngoài. Anh đi nói chuyện về nghệ thuật khắp nơi. Vài tháng trước những biến cố đau thương dẫn đến cái chết của nhà thơ cũng như của nền cộng hoà, trong một cuộc phỏng vấn báo chí, anh tuyên bố một dự định sáng tác những vở kịch có nội dung xã hội theo một cách nhìn xã hội chủ nghĩa. 
 
García Lorca bặt tiếng vào giữa tuổi 37, lúc tài năng qua nhiều thử thách, đang bước vào thời kỳ chín trái. Cả đất nước Tây Ban Nha sau đó cũng bặt tiếng dưới nền độc tài. Nhưng trong sự im lặng triền miên đó, những tiếng hát của con hoạ mi Andalusia lại vang lên ở khắp nơi trên thế giới, sự cám dỗ, ám ảnh của thơ anh như tăng thêm gấp bội bởi hào quang sự tuẫn tiết của anh. 
        
*    
Mùa Thu 2000, lần đầu tiên được phép xuất ngoại, tôi đã tranh thủ đáp tàu từ Paris xuống Granada để hành hương tới những thi tích García Lorca. Một chuyến đi nhiều bồi hồi, hầu như tất cả đều làm nhớ đến những câu thơ của ông. Tôi đã tìm đến tận ngôi nhà nhỏ của gia đình García Lorca, nơi nhà thơ trở về ở những ngày cuối cùng và bị lính Franco bắt đưa đi thủ tiêu. Bây giờ nó thành nhà lưu niệm García Lorca. Tôi đã tặng nhà này tập thơ Federico García Lorca chuyển ngữ tiếng Việt của mình in bằng giấy đen năm 1988 trước niềm vui bất ngờ của nhân viên trông nom và đám đông du khách đến thăm. Đó là một trong những giây phút hạnh phúc nhất của đời tôi. Sau chuyến đi, tôi đã ghi lại cảm xúc của mình trong bài thơ ngắn sau: 
\begin{blockquote}
        
Lorca 
        
Những đồi ô liu chạy trong trăng bạc        
Góc tối toa tàu con tim tôi đập 
        
Lorca 
        
Đồng mênh mông rực cháy và nứt toác        
Đâu rồi kị sĩ Cordoba?        
Chỉ một bóng cao bồi Far West        
Giữa phim trường bao la \footnote{
Một phim trường được dựng lên ở vùng này để quay pim cao bồi Viễn tây của Mỹ.}  
        
Lorca 
        
Nhịp chân dồn dập gitan        
Trên sàn diễn giả trang hang đá        
Trán nàng gịot giọt mồ hôi 
        
Biển tháng chín mình tôi        
Địa Trung Hải sóng chạy về tít tắp 
        
Lorca 
        
Đêm bập bùng ghia Granada        
Bom nổ sớm mai quảng trường tan tác \footnote{
Ngày tôi rời Granada, một quả bom của bọn khủng bố đã nổ giữa thành phố giết hại nhiều dân thường.}  

\end{blockquote}
 



\end{multicols}
\end{document}